\documentclass[a4paper,12pt]{article}
\usepackage[italian]{babel} %per documenti in italiano
\usepackage{fontspec} %permette di usare Times New Roman, ma solo con XeLaTeX (!!!)
	% Per compilazione locale con font installati:
	% \setmainfont{Times New Roman}
	% \newfontfamily\latinmodern{Latin Modern Roman}

  % Per compilazione su Overleaf con font in cartella locale ./fonts/
  % \setmainfont[Path = ./fonts/, Extension = .ttf, UprightFont = TIMES, BoldFont = TIMESBD, ItalicFont = TIMESI, BoldItalicFont = TIMESBI]{Times New Roman}
	% \newfontfamily\latinmodern{lmroman10-regular.otf}

  % Per compilazione locale senza font:
  \newcommand{\latinmodern}[1]{\text{#1}}

	\newcommand{\latinmath}[1]{\text{\latinmodern{#1}}} %definisce il comando \latinmath{} che applica font Latin Modern Roman al testo nell'argomento
	\addto\captionsitalian{\renewcommand{\contentsname}{\textbf{\scshape Indice}}}
\usepackage{microtype} %migliora spaziatura del font
\usepackage{amsmath, amssymb, amsfonts, amsthm} 
\usepackage{mathrsfs}
\usepackage{thmtools} %permette di definire ambienti personalizzati per teoremi, definizioni, ecc. (il font del testo tra parentesi va specificato manualmente) 
	\declaretheoremstyle[
	headfont=\bfseries, % Titolo in grassetto
	notefont=\normalfont, % Testo tra parentesi in corsivo
	headpunct={. $\,$}, % Separatore tra titolo e nota
	postheadspace=0pt, % Spazio dopo l'intestazione
	bodyfont=\normalfont
	]{stiledef}
	\declaretheorem[
	style=stiledef,
	name=Definizione,
	numbered=no,
	]{defin}
	\declaretheoremstyle[
	headfont=\bfseries, % Titolo in grassetto
	notefont=\normalfont, % Testo tra parentesi in corsivo
	headpunct={. $\,$}, % Separatore tra titolo e nota
	postheadspace=0pt, % Spazio dopo l'intestazione
	bodyfont=\itshape
	]{stileteo}
	\declaretheorem[
	style=stileteo,
	name=Teorema,
	numbered=no,
	]{theo}
	\declaretheoremstyle[
	headfont=\bfseries, % Titolo in grassetto
	notefont=\normalfont, % Testo tra parentesi in corsivo
	headpunct={. $\,$}, % Separatore tra titolo e nota
	postheadspace=0pt, % Spazio dopo l'intestazione
	bodyfont=\normalfont
	]{stileexm}
	\declaretheorem[
	style=stileexm,
	name=Esempio,
	numbered=no,
	]{exm}
	\declaretheoremstyle[
	headfont=\bfseries, % Titolo in grassetto
	notefont=\normalfont, % Testo tra parentesi in corsivo
	headpunct={. $\,$}, % Separatore tra titolo e nota
	postheadspace=0pt, % Spazio dopo l'intestazione
	bodyfont=\itshape
	]{stilelem}
	\declaretheorem[
	style=stilelem,
	name=Lemma,
	numbered=no,
	]{lem}
	\declaretheoremstyle[
	headfont=\itshape, % Titolo in grassetto
	notefont=\normalfont, % Testo tra parentesi in corsivo
	headpunct={. $\,$}, % Separatore tra titolo e nota
	postheadspace=0pt, % Spazio dopo l'intestazione
	bodyfont=\normalfont
	]{stiledimo}
	\declaretheorem[
	style=stiledimo,
	name=Dimostrazione,
	qed=$\spadesuit$,  % Simbolo QED personalizzato
	numbered=no
	]{dimo}
\usepackage[left=2cm,right=2cm,top=2.5cm,bottom=2cm]{geometry} %margini
\usepackage{changepage} %per cambiare margini di porzioni di testo localmente con \begin{adjustwidth}{sx}{dx}  
\usepackage{setspace} \onehalfspacing %interlinea (1,5)
\usepackage{parskip} %anche se sembra in conflitto con \parskip=0pt dopo, in realtà elimina lo spazio tra gli item degli ambienti come itemize, enumerate, description. comodo per non doverli modificare tutti uno a uno
	\parindent=1cm %indentazione dei capoversi; per non indentare il primo capoverso, serve usare \noindent ogni volta
	\parskip=0pt %spazio tra i paragrafi (nessuno)
\usepackage{titlesec} %titoli di sezione e sottosezione in maiuscoletto grassetto, con punto dopo il numero
	\titleformat{\section}{\normalfont\bfseries\scshape}{\thesection.}{0.5em}{}
	\titlespacing*{\section}{0pt}{3.5ex plus 1ex minus .2ex}{2.3ex plus .2ex}
	\titleformat{\subsection}{\normalfont\bfseries\scshape}{\thesubsection.}{0.5em}{}
	\titlespacing*{\subsection}{0pt}{3.5ex plus 1ex minus .2ex}{2.3ex plus .2ex}
	\titleformat{\subsubsection}{\normalfont\bfseries\scshape}{\thesubsubsection.}{0.5em}{}
	\titlespacing*{\subsubsection}{0pt}{3.5ex plus 1ex minus .2ex}{2.3ex plus .2ex}
	\newcounter{fourth}[subsubsection]
	\renewcommand{\thefourth}{\thesubsubsection.\arabic{fourth}}
	\newcommand{\subsubsubs}[1]{
		\refstepcounter{fourth}
		\phantomsection
		\addcontentsline{toc}{paragraph}{\thefourth\quad#1}}
\usepackage{etoolbox} %pacchetto per scrivere comandi
	\appto{\footnotelayout}{ %impostazione grandezza testo e interlinea delle note
		\fontsize{10pt}\selectfont
		\setstretch{1}}
	\let\originalFootnoterule\footnoterule %definisce spazio sotto la linea che si trova sopra alle note
	\renewcommand{\footnoterule}{ 
		\originalFootnoterule %richiama le impostazioni native della linea sopra alle note
		\vspace{4pt}} %valore che determina lo spazio tra la linea e il testo sotto
	\makeatletter
	\renewcommand{\@makefnmark}{ %posizione dei numeri di nota rispetto alla parola nel corpo del testo 
		\hbox{\kern-0.6ex\raisebox{-0.15ex}
			{\textsuperscript{\normalfont\@thefnmark}}}}
	\renewcommand{\@makefntext}{ %posizione del numerino e del testo delle note 
		\noindent\@makefnmark\hspace{0.2em}}
	\makeatother
	\AtBeginEnvironment{quote}{ %impostazioni per citazioni fuori testo
		\singlespacing %interlinea singola
		\small %grandezza del testo
		\leftskip=1cm \rightskip=1cm} %margini della citazione
\usepackage[style=english]{csquotes} %con questo si può usare \enquote{} per inserire una parola tra virgolette alte
\usepackage{enumitem}
	\setlist[description]{leftmargin=1.2cm, itemindent=-0.3cm} %indenta gli elenchi fatti in description
	\makeatletter
	\renewenvironment{thebibliography}[1] %modifica la visualizzazione della bibliografia (non utilizzata in questo documento)
	{\section*{\refname}
		\begin{list}{}%
			{\setlength{\leftmargin}{1cm} %sposta tutte le righe del valore desiderato
				\setlength{\itemindent}{-1cm} %annulla/modifica l’indentazione della prima riga
				\setlength{\itemsep}{0.2\baselineskip}
				\setlength{\parsep}{0pt}
				\setlength{\labelsep}{0pt}
				\setlength{\labelwidth}{0pt}
				\setlength{\listparindent}{0pt}
				\setlength{\topsep}{0pt}}
			\renewcommand{\@biblabel}[1]{}} %rimuove le etichette numeriche
		{\end{list}}
	\makeatother
	\setlist[itemize]{label={$\bullet$}} %punto grande nelle liste
\usepackage{tocloft}
	\renewcommand{\cftsecleader}{\cftdotfill{\cftdotsep}} %per visualizzare i puntini nell'indice anche di fianco ai titoli di sezione
	\setcounter{tocdepth}{4}
\usepackage{graphicx} %package to manage images
\usepackage{subfig} %to put images side by side
\usepackage{wrapfig} %used to align image to right or left
\usepackage{mathpartir} % per sequenti
\usepackage{xfrac} %per frazioni oblique
\newcommand{\To}{\Rightarrow} %definisce il comando \To al posto di \Rightarrow
\usepackage{tikz}
	\newcommand{\Dmd}{\mathord{\vcenter{\hbox{\tikz[baseline=-0.6ex]{\draw[line width=0.4pt, scale=1, rotate=45] (0,0) rectangle (0.2,0.2);}}}}} %definisce il comando \Dmd per il diamond
	\usetikzlibrary{arrows.meta, positioning} %per i disegni :3
\usepackage{xcolor}  %definisce colori personalizzati
	\definecolor{purple}{rgb}{0.545, 0, 1}    % Viola (RGB 139, 0, 255)
	\definecolor{pink}{rgb}{1, 0, 0.6}  % Rosa (RGB 255, 0, 155)
\usepackage{hyperref}
	\hypersetup{
		colorlinks=true,
		linkcolor=purple,     
		urlcolor=pink}

\begin{document}

\begin{titlepage}
	\title{\bfseries\scshape (Appunti di) \\
		Metodi logici per la filosofia}
	\author{\bfseries\scshape Docente: Eugenio Orlandelli}
	\date{Anno scolastico 2024-2025}
	\maketitle
	\thispagestyle{empty}

	\vfill

	\begin{center}
		Alma Mater Studiorum \\
		Università degli studi di Bologna \\
		Febbraio-Maggio 2025 \\
	\end{center}
\end{titlepage}

\tableofcontents

\newpage

\section{Logica proposizonale}
\noindent La logica proposizionale è la logica che tratta di proposizioni, ovvero affermazioni che possono essere vere o false (in altre parole, a cui può essere assegnato un valore di verità).

\subsection{Sintassi}
\noindent La logica proposizionale è formata da:
\subsubsubs{Alfabeto}
\begin{itemize}
	\hypertarget{varenprop}{}
	\item \emph{Variabili enunciative:} $$\Phi := \{p_0, p_1, p_2, \ldots\}$$ è un insieme infinito numerabile i cui elementi, detti anche \emph{atomi}, rappresentano le proposizioni semplici, o \emph{atomiche}.
	\item \emph{Simboli logici:} $$\bot, \land, \lor, \to,$$ detti anche \emph{connettivi}, sono \emph{vero-funzionali}: il valore di verità delle proposizioni da essi costituite dipendono esclusivamente da quelli delle variabili enunciative che vi compaiono. $\bot$ (\emph{bottom}) è detta \emph{0-aria}, in quanto non si applica a nessuna variabile: significa \enquote{falso}. Gli altri sono invece connettivi \emph{binari} (o \emph{2-ari}), in quanto si applicano a due variabili alla volta.
	\item \emph{Simboli ausiliari:} $$(, )$$ sono utilizzati per trattare proposizioni complesse senza ambiguità. \\
\end{itemize}

\subsubsubs{Definizione di formule ben formate}
\begin{defin}
	[\emph{Formule ben formate}]
	Le formule ben formate \footnote{A ben vedere, una formula \emph{non} ben formata non è una formula.} sono tutte e sole le formule ammesse nella logica proposizionale. L'insieme delle formule ben formate $\latinmath{fm}^{\Phi}$ è così definito \footnote{La definizione è qui data, come la successiva, per \href{https://www.treccani.it/enciclopedia/induzione/}{induzione}.}:
	\begin{description}
		\item [Base:] \phantom{ciao}
		      \begin{itemize}
			      \item Se $p_i \in \Phi$, allora $p_i \in \latinmath{fm}^{\Phi}$.
			      \item $\bot \in \latinmath{fm}^{\Phi}$.
		      \end{itemize}
		\item [Passo:] \phantom{ciao}
		      \begin{itemize}
			      \item Se $A, B \in \latinmath{fm}^{\Phi}$, allora $(A \land B), (A \lor B), (A \to B) \in \latinmath{fm}^{\Phi}$.
		      \end{itemize}
	\end{description}

	Nient'altro appartiene a $\latinmath{fm}^{\Phi}$. \\
\end{defin}

Utilizziamo $p, q, r, \ldots$ come metavariabili per le variabili enunciative e $A, B, \ldots$ come metavariabili per le formule. Ci si avvale, inoltre, di una serie di convenzioni:
\begin{enumerate}
	\item Le parentesi più esterne si omettono;
	\item La priorità dei simboli logici è: $\{\land, \lor\} \; > \; \{\to\}$;
	\item $\land$ e $\lor$ associano a sinistra \footnote{Ad esempio, $A \land B \land C$ dev'essere letta come $(A \land B) \land C$.}, ma non tra loro (in questo caso c'è bisogno delle parentesi).
\end{enumerate}

Aggiungiamo anche le seguenti definizioni ausiliarie (convenzionali):
\begin{itemize}
	\item $\top \equiv \bot \to \bot$
	\item $\lnot A \equiv A \to \bot$
	\item $A \leftrightarrow B \equiv (A \to B) \land (B \to A)$ \\
\end{itemize}

\subsubsubs{Definizione di lunghezza di una formula ben formata}
\hypertarget{lgfbf}{}
\begin{defin}
	[\emph{Lunghezza di una formula ben formata}]
	La lunghezza \latinmath{lg} (ovvero il numero di connettivi logici che ne fanno parte) di una formula $A \in \latinmath{fm}^{\Phi}$ è così definita:
	\begin{description}
		\item [Base:] \phantom{ciao}
		      \begin{itemize}
			      \item $\latinmath{lg}(p_i) = \latinmath{lg}(\bot) = 0$
		      \end{itemize}
		\item [Passo:] \phantom{ciao}
		      \begin{itemize}
			      \item $\latinmath{lg}(A \land B) = \latinmath{lg}(A \lor B) = \latinmath{lg}(A \to B) = \latinmath{lg}(A) + \latinmath{lg}(B) + 1$
		      \end{itemize}
	\end{description}
\end{defin}

\vspace{4pt}
\hypertarget{ind}{}
\subsubsection{Principio di induzione}
\noindent Avere definito la lunghezza di una formula ben formata permette di associare le formule a numeri naturali e usare quindi con esse il \href{https://www.treccani.it/enciclopedia/induzione-matematica-principio-di_(Enciclopedia-della-Matematica)/}{principio di induzione matematica}.

Il principio di induzione si utilizza per dimostrare che una certa proprietà $P$ è valida per ogni elemento di un dato insieme. Le definizioni per induzione, così come le dimostrazioni, si svolgono analizzando prima i casi \textbf{Base}, poi tutti gli altri nel \textbf{Passo} induttivo. Quest'ultimo viene svolto considerando una o più \emph{ipotesi di induzione} (\latinmath{IH}). \\

\subsubsubs{Principio di induzione completa sulla lunghezza di $A$}
\hypertarget{indlg}{}
\begin{defin}
	[\emph{Principio di induzione completa sulla lunghezza di $A$}] Data una certa proprietà $P$:
	\begin{description}
		\item [Base:] \phantom{ciao}
		      \begin{itemize}
			      \item $P$ vale per ogni formula $A$ con $\latinmath{lg} = 0$.
		      \end{itemize}
		\item [Passo:] \phantom{ciao}
		      \begin{itemize}
			      \item Se $P$ vale per ogni $A \in \latinmath{fm}^{\Phi} : \latinmath{lg} \leq n$, allora $P$ vale per ogni $A \in \latinmath{fm}^{\Phi} : \latinmath{lg} = n+1$. \\
		      \end{itemize}
	\end{description}
\end{defin}

\subsubsubs{Principio di induzione strutturale su $A$}
\begin{defin}
	[\emph{Principio di induzione strutturale su $A$}] Data una certa proprietà $P$:
	\begin{description}
		\item [Base:] \phantom{ciao}
		      \begin{itemize}
			      \item $P(p_i)$ e $P(\bot)$.
		      \end{itemize}
		\item [Passo:] \phantom{ciao}
		      \begin{itemize}
			      \item Se vale $P(A)$ e $P(B)$, allora vale $P(A \circ B)$, dove $\circ \in \{\land, \lor, \to\}$. \\
		      \end{itemize}
	\end{description}
\end{defin}

\subsubsubs{Esempi di dimostrazioni per induzione strutturale su $A$}
\begin{exm}
	[\emph{Dimostrazione dell'uguaglianza del numero di parentesi}] In una formula $A$, il numero ($\#$) di parentesi aperte è uguale al numero di parentesi chiuse:
	$$\#_((A) = \#_)(A)$$

	\begin{dimo}
		[per induzione strutturale su $A$] \phantom{ciao}
		\begin{description}
			\item [Base:] \phantom{ciao}
			      \begin{itemize}
				      \item $\#_((p_i) = \#_)(p_i) \qquad \text{cioè} \; 0=0$
				      \item $\#_((\bot) = \#_)(\bot) \qquad \text{cioè} \; 0=0$.
			      \end{itemize}
			\item [Passo:] va dimostrato $\#_((B \circ C) = \#_)(B \circ C)$, dove $\circ \in \{\land, \lor, \to\}$. \\
			      \latinmath{IH1}: $\#_((B) = \#_)(B)$ \\
			      \latinmath{IH2}: $\#_((C) = \#_)(C)$ \\
			      Per la \hyperlink{lgfbf}{definizione di lunghezza di una formula ben formata}, so che:
			      $$\begin{aligned}
					      \#_((B \circ C) & = \#_((B) + \#_((C) + 1 \\
					      \#_)(B \circ C) & = \#_)(B) + \#_)(C) + 1 \\
				      \end{aligned} $$
			      Dal momento che:
			      \begin{itemize}
				      \item $1=1$;
				      \item $\#_((B) = \#_)(B) \qquad$ per \latinmath{IH1};
				      \item $\#_((C) = \#_)(C) \qquad$ per \latinmath{IH2};
			      \end{itemize}
			      posso concludere che: $$\#_((B) + \#_((C) + 1 = \#_)(B) + \#_)(C) + 1$$
		\end{description}
	\end{dimo}
\end{exm}

\begin{exm}
	[\emph{Dimostrazione della maggior quantità di variabili rispetto ai connettivi}] In una formula $A$, la somma del numero di variabili è sempre maggiore della somma del numero di connettivi:
	$$\#_{p_i, \, \bot}(A) = \#_{\circ}(A)$$
	dove $\circ \in \{\land, \lor, \to\}$.
	\begin{dimo}
		[per induzione strutturale su $A$] \phantom{ciao}
		\begin{description}
			\item [Base:] \phantom{ciao}
			      \begin{itemize}
				      \item $\#_{p_i, \bot}(p_i) > \#_{\circ}(p_i) \qquad \text{cioè} \; 1>0$
				      \item $\#_{p_i, \bot}(\bot) > \#_{\circ}(\bot) \qquad \text{cioè} \; 1>0$.
			      \end{itemize}
			\item [Passo:] va dimostrato $\#_{p_i, \bot}(B \circ C) > \#_{\circ}(B \circ C) \; \equiv \; \#_{p_i, \bot}(B \circ C) \geq \#_{\circ}(B \circ C) +1$. \\
			      \latinmath{IH1}: $\#_{p_i, \bot}(B) > \#_{\circ}(B)$ \\
			      \latinmath{IH2}: $\#_{p_i, \bot}(C) > \#_{\circ}(C)$ \\
			      Per \latinmath{IH1} so che $\#_{p_i, \bot}(B) \geq \#_{\circ}(B) +1$, mentre per \latinmath{IH2} so che $\#_{p_i, \bot}(C) \geq \#_{\circ}(C) +1$. Ne segue che:
			      $$\begin{aligned}
					      \#_{p_i, \bot}(B) + \#_{p_i, \bot}(C) & \geq \#_{\circ}(B) +1 + \#_{\circ}(C) +1 \\
					      \#_{p_i, \bot}(B) + \#_{p_i, \bot}(C) & \geq \#_{\circ}(B) + \#_{\circ}(C) +2    \\
					      \#_{p_i, \bot}(B \circ C)             & \geq \#_{\circ}(B \circ C) +2
				      \end{aligned}$$
		\end{description}
	\end{dimo}
\end{exm}

\newpage
\subsection{Semantica}
\noindent La semantica tratta del significato (o valore di verità) associato agli elementi di un linguaggio.

Per stabilire quali enunciati del linguaggio sono veri e quali falsi, ci si avvale di un modello $M$, caratterizzato da una funzione di interpretazione $I$ \footnote{Nel caso della logica proposizionale, si può affermare che $M = I$.}. \\

\subsubsubs{Definizioni di funzione di interpretazione}
\begin{defin}
	[\emph{Funzione di interpretazione}] La funzione di interpretazione $I$ è definibile alternativamente come:
	\begin{itemize}
		\item L'insieme di tutte e sole le proposizioni atomiche che essa rende vere; tutte le altre, dunque, saranno false. L'insieme in questione sarà un sottoinsieme di \hyperlink{varenprop}{$\Phi$}:
		      $$I \subseteq \Phi$$
		\item La \emph{funzione caratteristica} del suddetto sottoinsieme, che associa a ogni proposizione atomica che si vuole vera il valore 1 e a ogni proposizione atomica che si vuole falsa il valore 0:
		      $$I : \Phi \to \{0, 1\}$$
	\end{itemize}
\end{defin}

\vspace{8pt}
\noindent Per dire che una formula $A$ è valida rispetto a una certa interpretazione $I$ di un modello $M$ (o che l'interpretazione $I$ rende vera $A$ nel modello $M$, che è lo stesso), si scrive $$I \vDash^{M} A.$$

\subsubsubs{Definizione di verità in un modello}
\hypertarget{verprop}{}
\begin{defin}
	[\emph{Verità in un modello}] \phantom{ciao}
	\begin{description}
		\item [Base:] \phantom{ciao}
		      \begin{itemize}
			      \item $\vDash^{M} p_i \quad\! \text{sse} \;\protect\footnote{\enquote{Se e solo se}.} \quad\! p_i \in I \;\; \text{oppure} \;\; I(p_i)=1$
			      \item $\nvDash^{M} \bot$
		      \end{itemize}
		\item [Passo:] \phantom{ciao}
		      \begin{itemize}
			      \item $\vDash^{M} A \land B \quad\; \text{sse} \quad \vDash^{M} A \;\; \text{e} \;\; \vDash^{M} B$
			      \item $\vDash^{M} A \lor B \quad\; \text{sse} \quad \vDash^{M} A \;\; \text{o} \;\; \vDash^{M} B$
			      \item $\vDash^{M} A \to B \quad\! \text{sse} \quad \nvDash^{M} A \;\; \text{o} \;\; \vDash^{M} B$ \\
		      \end{itemize}
	\end{description}
\end{defin}

Uno dei metodi possibili per verificare quali interpretazioni rendano vera una data formula $A$ è quello delle cosiddette \enquote{tavole di verità}. Per mostrarne l’applicabilità, però, abbiamo prima bisogno di dimostrare il seguente \textbf{Lemma}. Questo ci assicura che, per capire se una formula sia resa vera da una data interpretazione, è sufficiente sapere come questa interpretazione definisca le proposizioni atomiche che effettivamente compaiono nella formula  \footnote{In altre parole, se un atomo non compare nella formula, non ci interessa se sia vero o falso in una data interpretazione per capire se questa renda vera la formula.}. \\

\subsubsubs{Lemma: verità di una formula in modelli diversi}
\begin{lem}
	[Verità di una formula in modelli diversi] Per una qualsiasi formula $A$, se tutte le variabili enunciative che la compongono hanno lo stesso valore di verità sia in $M$ che in $M'$, allora $A$ ha lo stesso valore di verità sia in $M$ che in $M'$:
	$$\text{Se} \ I^{M}(p)=I^{M'}(p) \text{, allora} \; \vDash^{M} \! A \; \text{sse} \; \vDash^{M'} \! A \qquad \quad \forall p \in A$$
\end{lem}
\begin{dimo}
	[per induzione strutturale su $A$] \phantom{ciao}
	\begin{description}
		\item [Base:] \phantom{ciao}
		      \begin{itemize}
			      \item $\vDash^{M} p_i \; \text{sse} \; \vDash^{M'} p_i$ \qquad vero per definizione (l'antecedente del \textbf{Lemma}).
			      \item $\nvDash^{M} \bot \; \text{sse} \; \nvDash^{M'} \bot$ \qquad $ \!\bot$ è falso in qualsiasi modello.
		      \end{itemize}
		\item [Passo:] ho un caso per ogni connettivo. \\
		      \latinmath{IH1}: $\vDash^{M} B \; \text{sse} \; \vDash^{M'} B$ \\
		      \latinmath{IH2}: $\vDash^{M} C \; \text{sse} \; \vDash^{M'} C$
		      \begin{itemize}
			      \item $A \equiv B \land C$ (va dimostrato \emph{$\vDash^{M} B \land C \; \text{sse} \; \vDash^{M'} B \land C$}): \\
			            Per \hyperlink{verprop}{definizione di verità in un modello} so che:
			            $$\vDash^{M} B \land C \quad \; \text{sse} \quad \; \vDash^{M} B \; \; \text{e} \; \; \vDash^{M} C$$
			            Per \latinmath{IH1} e \latinmath{IH2}, ho che:
			            $$\vDash^{M} B \; \; \text{e} \; \; \vDash^{M} C \quad \; \text{sse} \quad \; \vDash^{M'} B \; \; \text{e} \; \; \vDash^{M'} C$$
			            Per \hyperlink{verprop}{definizione di verità in un modello}, posso concludere:
			            $$\vDash^{M} B \land C \quad \; \text{sse} \quad \; \vDash^{M'} B \land C$$
			      \item $A \equiv B \lor C$ (va dimostrato \emph{$\vDash^{M} B \lor C \; \text{sse} \; \vDash^{M'} B \lor C$}): \\
			            Per \hyperlink{verprop}{definizione di verità in un modello} so che:
			            $$\vDash^{M} B \lor C \quad \; \text{sse} \quad \; \vDash^{M} B \; \; \text{o} \; \; \vDash^{M} C$$
			            Per \latinmath{IH1} e \latinmath{IH2}, ho che:
			            $$\vDash^{M} B \; \; \text{o} \; \; \vDash^{M} C \quad \; \text{sse} \quad \; \vDash^{M'} B \; \; \text{o} \; \; \vDash^{M'} C$$
			            Per \hyperlink{verprop}{definizione di verità in un modello}, posso concludere:
			            $$\vDash^{M} B \lor C \quad \; \text{sse} \quad \; \vDash^{M'} B \lor C$$
			      \item $A \equiv B \to C$ (va dimostrato \emph{$\vDash^{M} B \to C \; \text{sse} \; \vDash^{M'} B \to C$}): \\
			            Per \hyperlink{verprop}{definizione di verità in un modello} so che:
			            $$\vDash^{M} B \to C \quad \; \text{sse} \quad \; \nvDash^{M} B \; \; \text{o} \; \; \vDash^{M} C$$
			            Per \latinmath{IH1} e \latinmath{IH2}, ho che:
			            $$\nvDash^{M} B \; \; \text{o} \; \; \vDash^{M} C \quad \; \text{sse} \quad \; \nvDash^{M'} B \; \; \text{o} \; \; \vDash^{M'} C$$
			            Per \hyperlink{verprop}{definizione di verità in un modello}, posso concludere:
			            $$\vDash^{M} B \to C \quad \; \text{sse} \quad \; \vDash^{M'} B \to C$$
		      \end{itemize}
	\end{description}
\end{dimo}

\subsubsubs{Lemma: negazioni pari di una formula}
\begin{lem}
	[Negazioni pari di una formula] Una formula $A$ è valida in un modello $M$ se e solo se, applicate un numero di negazioni pari ad $A$, la formula così ottenuta risulta valida:
	$$\vDash^{M} A \quad \text{sse} \quad \vDash^{M} \lnot^{2n} A$$
\end{lem}
\begin{dimo}
	[per induzione strutturale su $n$] \phantom{ciao}
	\begin{description}
		\item [Base:] $k = 0$
		      \begin{itemize}
			      \item $\vDash^{M} A \; \text{sse} \; \vDash^{M} \lnot^{0} A \quad \equiv \quad \vDash^{M} A \; \text{sse} \; \vDash^{M} A$
		      \end{itemize}
		\item [Passo:] $n = k + 1$ (va dimostrato \emph{$\vDash^{M} A \; \text{sse} \; \vDash^{M} \lnot^{2(k+1)} A$}). \\
		      \latinmath{IH}: $\vDash^{M} A \; \text{sse} \; \vDash^{M} \lnot^{2k} A$
		      $$\begin{aligned}
				      \vDash^{M} A               & \;\; \text{sse} \; \vDash^{M} \lnot^{2(k+1)} A \;       &  & \equiv                             \\
				      \equiv \qquad \vDash^{M} A & \;\; \text{sse} \; \vDash^{M} \lnot^{2k+2} A \;         &  & \equiv                             \\
				      \equiv \qquad \vDash^{M} A & \;\; \text{sse} \; \vDash^{M} \lnot\lnot\lnot^{2k} A \; &  & \equiv                             \\
				      \equiv \qquad \vDash^{M} A & \;\; \text{sse} \; \nvDash^{M} \lnot\lnot^{2k} A \;     &  & \equiv                             \\
				      \equiv \qquad \vDash^{M} A & \;\; \text{sse} \; \vDash^{M} \lnot^{2k} A \qquad       &  & \; \text{vera per \latinmath{IH}.}
			      \end{aligned}$$
	\end{description}
\end{dimo}

\subsubsubs{Definizione di formula soddisfatta e soddifacibile, verità/validità logica, conseguenza logica}
\begin{defin}
	[\emph{Formula soddisfatta e soddifacibile, verità/validità logica, conseguenza logica}] In logica proposizionale, una formula $A \in \latinmath{fm}^{\Phi}$:
	\begin{itemize}
		\item è \emph{soddisfatta} da un modello $M$ se e solo se $\vDash^{M} A$;
		\item è \emph{soddisfacibile} se e solo se $\exists M : \; \vDash^{M} A$;
		\item è una \emph{verità logica}/\emph{formula valida} se se solo se $\forall M (\vDash^{M} A)$;
		\item è \emph{conseguenza logica} di un insieme di formule $\Gamma \subseteq \latinmath{fm}^{\Phi}$ (scriviamo $\Gamma \vDash A$) se e solo se $\forall M (\text{se} \ \forall B \in \Gamma, \;\vDash^{M} B \text{, allora}\;\vDash^{M} A)$. \\
	\end{itemize}
\end{defin}

\subsubsubs{Teorema di deduzione semantica}
\begin{theo}
	[Deduzione semantica] Una formula $B$ è conseguenza logica di una formula $A$ se e solo se è vero che $A \to B$:
	$$A \vDash B \quad \text{sse} \quad \vDash A \to B$$
\end{theo}
\noindent Questo teorema permette di stabilire che, in virtù della semantica che abbiamo associato ad esso, all’interno del nostro linguaggio il connettivo \enquote{$\to$} equivale alla nozione di conseguenza logica.
\begin{dimo}
	\phantom{ciao} \\
	Assumo $A \vDash B$; per la definizione di conseguenza logica, so che $A \vDash B \; \text{sse} \; \forall M (\text{se} \;\vDash^{M} A, \text{, allora}\;\vDash^{M} B)$:
	$$\begin{aligned}
			A \vDash B              & \;\; \text{sse} \;\; \forall M (\text{se} \;\vDash^{M} A \text{, allora}\;\vDash^{M} B) &  & \; \equiv \\
			\equiv \quad A \vDash B & \;\; \text{sse} \;\; \forall M (\nvDash^{M} A \;\text{o}\;\vDash^{M} B)                 &  & \; \equiv \\
			\equiv \quad A \vDash B & \;\; \text{sse} \;\; \forall M (\vDash^{M} A \to B)                                     &  & \; \equiv \\
			\equiv \quad A \vDash B & \;\; \text{sse} \; \vDash A \to B
		\end{aligned}$$
	\vspace*{-16pt} \raggedleft \qedhere
\end{dimo}

\newpage
\section{Logica modale}
\noindent Una logica modale \footnote{Il termine \enquote{modale} risale ad una terminologia medievale secondo la quale gli enunciati sono proferiti secondo varie \enquote{modalità}: \emph{dichiarativa}, \emph{negativa}, \emph{della possibilità}, \emph{dell’impossibilità}, \emph{della necessità}.} è una qualsiasi estensione della logica classica che contiene uno o più connettivi \emph{non} vero-funzionali. Qui verranno presi in considerazione i due connettivi $\Box$ e $\Dmd$; la loro interpretazione (nel linguaggio naturale) più comune è quella \emph{aletica}, nella quale $\Box$ significa \enquote{è necessario che} e $\Dmd$ significa \enquote{è possibile che} \footnote{Alcune delle altre intepretazioni sono quella \emph{epistemica}, dove $\Box$ significa \enquote{si sa che} e $\Dmd$ non ha interpretazione; quella \emph{doxastica}, dove $\Box$ significa \enquote{si crede che} e $\Dmd$ non ha interpretazione; quella \emph{deontica}, dove $\Box$ significa \enquote{è obbligatorio che} e $\Dmd$ significa \enquote{è possibile che}.}.

\subsection{Sintassi}
\noindent La logica modale che verrà trattata qui è formata da:
\subsubsubs{Alfabeto}
\begin{itemize}
	\item \emph{Variabili enunciative:} $$\Phi := \{p_0, p_1, p_2, \ldots\}$$ è un insieme infinito numerabile i cui elementi, detti anche \emph{atomi}, rappresentano le proposizioni semplici, o \emph{atomiche}.
	\item \emph{Simboli logici:} $$\bot, \Box, \Dmd, \land, \lor, \to,$$ detti anche \emph{connettivi}. $\bot$ (\emph{bottom}) è detta \emph{0-aria}, in quanto non si applica a nessuna variabile: significa \enquote{falso}. $\Box$ e $\Dmd$ sono connettivi \emph{1-ari}, in quanto si applicano a una sola variabile alla volta. Gli altri sono invece connettivi \emph{binari} (o \emph{2-ari}), in quanto si applicano a due variabili alla volta.
	\item \emph{Simboli ausiliari:} $$(, )$$ sono utilizzati per trattare proposizioni complesse senza ambiguità. \\
\end{itemize}

\subsubsubs{Definizione di formule ben formate}
\hypertarget{fbfmd}{}
\begin{defin}
	[\emph{Formule ben formate}]
	Le formule ben formate sono tutte e sole le formule ammesse nella logica modale. L'insieme delle formule ben formate $\latinmath{fm}^{\Phi}$ è così definito:
	\begin{description}
		\item [Base:] \phantom{ciao}
		      \begin{itemize}
			      \item Se $p_i \in \Phi$, allora $p_i \in \latinmath{fm}^{\Phi}$.
			      \item $\bot \in \latinmath{fm}^{\Phi}$.
		      \end{itemize}
		\item [Passo:] \phantom{ciao}
		      \begin{itemize}
			      \item Se $A \in \latinmath{fm}^{\Phi}$, allora $\Box A, \Dmd A \in \latinmath{fm}^{\Phi}$.
			      \item Se $A, B \in \latinmath{fm}^{\Phi}$, allora $(A \land B), (A \lor B), (A \to B) \in \latinmath{fm}^{\Phi}$.
		      \end{itemize}
	\end{description}

	Nient'altro appartiene a $\latinmath{fm}^{\Phi}$. \\
\end{defin}

Utilizziamo $p, q, r, \ldots$ come metavariabili per le variabili enunciative e $A, B, \ldots$ come metavariabili per le formule. Ci si avvale, inoltre, di una serie di convenzioni:
\begin{enumerate}
	\item Le parentesi più esterne si omettono;
	\item La priorità dei simboli logici è: $\{\Box, \Dmd\} \; > \; \{\land, \lor\} \; > \; \{\to\}$;
	\item $\land$ e $\lor$ associano a sinistra, ma non tra loro (in questo caso c'è bisogno delle parentesi).
\end{enumerate}

Aggiungiamo anche le seguenti definizioni ausiliarie (convenzionali):
\begin{itemize}
	\item $\top \equiv \bot \to \bot$
	\item $\lnot A \equiv A \to \bot$
	\item $A \leftrightarrow B \equiv (A \to B) \land (B \to A)$
	\item $\Box A \leftrightarrow \lnot \Dmd \lnot A$
	\item $\Dmd A \leftrightarrow \lnot \Box \lnot A$\\
\end{itemize}

\subsubsubs{Definizione di lunghezza di una formula ben formata}
\hypertarget{lgfbf}{}
\begin{defin}
	[\emph{Lunghezza di una formula ben formata}]
	La lunghezza \latinmath{lg} (ovvero il numero di connettivi logici che ne fanno parte) di una formula $A \in \latinmath{fm}^{\Phi}$ è così definita:
	\begin{description}
		\item [Base:] \phantom{ciao}
		      \begin{itemize}
			      \item $\latinmath{lg}(p_i) = \latinmath{lg}(\bot) = 0$
		      \end{itemize}
		\item [Passo:] \phantom{ciao}
		      \begin{itemize}
			      \item $\latinmath{lg}(\Box A) = \latinmath{lg}(\Dmd A) = \latinmath{lg}(A) + 1$
			      \item $\latinmath{lg}(A \land B) = \latinmath{lg}(A \lor B) = \latinmath{lg}(A \to B) = \latinmath{lg}(A) + \latinmath{lg}(B) + 1$ \\
		      \end{itemize}
	\end{description}
\end{defin}

Avendo ridefinito la lunghezza di una formula ben formata, rimane invariato il \hyperlink{indlg}{principio di induzione completa sulla lunghezza di $A$} così come fornito per la logica proposizionale. Cambia, invece, il principio di induzione strutturale: \\

\subsubsubs{Principio di induzione strutturale su $A$}
\begin{defin}
	[\emph{Principio di induzione strutturale su $A$}] Data una certa proprietà $P$:
	\begin{description}
		\item [Base:] \phantom{ciao}
		      \begin{itemize}
			      \item $P(p_i)$ e $P(\bot)$.
		      \end{itemize}
		\item [Passo:] \phantom{ciao}
		      \begin{itemize}
			      \item Se vale $P(A)$ e $P(B)$, allora vale $P(A \circ B)$, dove $\circ \in \{\land, \lor, \to\}$.
			      \item Se vale $P(A)$, allora valgono $P(\Box A)$ e $P(\Dmd A)$. \\
		      \end{itemize}
	\end{description}
\end{defin}

\newpage
\subsection{Semantica}
\noindent Dal punto di vista della semantica, la logica modale è un'estensione di quella classica e mantiene il principio di \emph{bivalenza} (vero/falso), ma non quello di \emph{vero-funzionalità}: la verità/falsità delle proposizioni composte non dipende solo da quella delle proposizioni atomiche che le compongono. Tentare di esprimere i concetti della logica modale tramite tavole di verità analoghe a quelle che si utilizzano per la logica proposizionale non sarebbe fattibile: ci sarebbero alcune caselle che non sapremmo come riempire.

Nel suo articolo del 1963 \href{https://files.commons.gc.cuny.edu/wp-content/blogs.dir/1358/files/2019/03/Semantical-Considerations-on-Modal-Logic-PUBLIC.pdf}{\emph{Semantical considerations on modal logic}}, \href{https://it.wikipedia.org/wiki/Saul_Kripke}{Saul Kripke} introdusse una semantica per la logica modale basata sul concetto leibniziano di \emph{mondi possibili} e sulla \emph{relazione di accessibilità} tra essi. Questa semantica, anche detta \emph{semantica dei mondi possibili}, quantifica gli operatori modali $\Box$ e $\Dmd$ su un certo sottoinsieme dei mondi possibili, asserendo che la proposizione modale sia vera sulla base dello stato di cose nei suddetti mondi. A seconda di quale interpretazione scegliamo, opteremo per una diversa relazione di accessibilità che, dato un mondo \enquote{attuale} (di partenza), ci permetta di risalire ai mondi da esso \enquote{visibili}, ossia quelli possibili.

La semantica di Kripke si basa sulle definizioni preliminari di \emph{struttura relazionale} (o \emph{frame}) e di \emph{modello}. \\

\subsubsubs{Definizione di struttura relazionale}
\begin{defin}
	[\emph{Struttura relazionale}] Una struttura relazionale (o \emph{frame}) $\mathcal{F}$ è una coppia:
	$$\mathcal{F} := \langle \mathcal{W}, \mathcal{R} \rangle$$
	dove:
	\begin{itemize}
		\item $\mathcal{W}$ è un insieme non vuoto di mondi (o \enquote{punti}) possibili, i quali vengono indicati con $w$, $v$, $u$:
		      $$\mathcal{W} := \{w, v, u, \ldots\}$$
		\item $\mathcal{R}$ è la \emph{relazione di accessibilità}, un predicato binario definito come un sottoinsieme del prodotto cartesiano tra due insiemi $\mathcal{W}$:
		      $$\mathcal{R} \subseteq \mathcal{W} \times \mathcal{W}$$
	\end{itemize}
	Scriviamo \enquote{$w\mathcal{R}v$} per dire che \enquote{$w$ vede $v$} o \enquote{$v$ è accessibile da $w$}.
\end{defin}

Di seguito, alcuni esempi grafici di strutture: \\
\vspace*{-12pt}
\begin{minipage}{0.3\textwidth}
	\centering
	\begin{tikzpicture}
		[world/.style={circle, draw, fill=black, inner sep=0pt, minimum size=4pt},
			relation/.style={->, >=Stealth, shorten >=1pt, shorten <=1pt}]
		\node[world, label={above:{$w$}}] (w) at (0,0) {};
	\end{tikzpicture}
\end{minipage}
\hfill
\begin{minipage}{0.3\textwidth}
	\centering
	\begin{tikzpicture}
		[world/.style={circle, draw, fill=black, inner sep=0pt, minimum size=4pt},
			relation/.style={->, >=Stealth, shorten >=1pt, shorten <=1pt}]
		\node[world, label={above right:{$w$}}] (w) at (0,0) {};
		\draw[relation] (w) to [out=135, in=-135, looseness=12, min distance=1cm] (w); % auto-relazione
	\end{tikzpicture}
\end{minipage}
\hfill
\begin{minipage}{0.3\textwidth}
	\centering
	\begin{tikzpicture}
		[world/.style={circle, draw, fill=black, inner sep=0pt, minimum size=4pt},
			relation/.style={->, >=Stealth, shorten >=1pt, shorten <=1pt}]
		\node[world, label={above left:{$w$}}] (w) at (0,0) {};
		\node[world, label={above right:{$v$}}] (v) at (1,1.5) {};
		\node[world, label={above left:{$u$}}] (u) at (-1,1.5) {};
		\draw[relation] (w) -- (v);
		\draw[relation] (v) -- (u);
	\end{tikzpicture}
\end{minipage}
\begin{minipage}{0.3\textwidth}
	\centering
	$$\mathcal{F} = \langle \{w\}, \emptyset \rangle$$
\end{minipage}
\hfill
\begin{minipage}{0.3\textwidth}
	\centering
	$$\mathcal{F} = \langle \{w\}, \{\langle w, w \rangle\} \rangle$$
\end{minipage}
\hfill
\begin{minipage}{0.3\textwidth}
	\centering
	$$\mathcal{F} = \langle \{w, v, u\}, \{\langle w, v \rangle, \langle v, u \rangle\} \rangle$$
\end{minipage} \\

\vspace{12pt}
\subsubsubs{Definizione di modello}
\begin{defin}
	[Modello] Un modello $\mathcal{M}$ è una tripla:
	$$\mathcal{M} := \langle \mathcal{W}, \mathcal{R}, I \rangle$$
	dove:
	\begin{itemize}
		\item $\mathcal{W} := \{w, v, u, \ldots\}$
		\item $\mathcal{R} \subseteq \mathcal{W} \times \mathcal{W} \,$ \footnote{Sia $\mathcal{W}$ che $\mathcal{R}$ sono definiti come nella definizione di struttura relazionale.}
		\item $I$ è la \emph{funzione di interpretazione}, che associa ad ogni variabile enunciativa un sottoinsime di $\mathcal{W}$:
		      $$I : \Phi \to \mathscr{P}(\mathcal{W})$$
		      dove $\mathscr{P}$ è l'\emph{operatore potenza}, che associa ad ogni insieme l'insieme dei suoi sottoinsiemi. $I$, dunque, è tale che, per ogni $p_i \in \Phi$:
		      $$I(p_i) \subseteq \mathcal{W}.$$
	\end{itemize}
	Un modello basato su una struttura $\mathcal{F}$ (ovvero, che condivide $\mathcal{W}$ e $\mathcal{R}$ con quella struttura) è detto $\mathcal{F}$-modello.
\end{defin}

\vspace{0pt}
\subsubsection{Verità e validità logica}
\noindent Si parla di \emph{verità} relativamente a punti di modelli e modelli e di \emph{validità} riguardo a strutture e classi di strutture (la discriminante è dove si preserva la \hyperlink{su}{regola di sostituzione uniforme}). \\

\subsubsubs{Definizione di verità in un punto}
\hypertarget{defverp}{}
\begin{defin}
	[\emph{Verità di una formula in un punto}] La verità di una formula $A$ in un punto $w$ di un modello $\mathcal{M}$ (scriviamo \enquote{$\vDash_w^{\mathcal{M}} A$}) è così definita:
	\begin{description}
		\item [Base:] \phantom{ciao}
		      \begin{itemize}
			      \item $\vDash_w^{\mathcal{M}} p_i \;\; \text{sse} \;\; w \in I(p_i)$
			      \item $\nvDash_w^{\mathcal{M}} \bot$
		      \end{itemize}
		\item [Passo:] \phantom{ciao}
		      \begin{itemize}
			      \item $\vDash_w^{\mathcal{M}} A \land B \quad\; \text{sse} \quad\; \vDash_w^{\mathcal{M}} A \;\; \text{e} \;\; \vDash_w^{\mathcal{M}} B$
			      \item $\vDash_w^{\mathcal{M}} A \lor B \quad\; \text{sse} \quad\; \vDash_w^{\mathcal{M}} A \;\; \text{o} \;\; \vDash_w^{\mathcal{M}} B$
			      \item $\vDash_w^{\mathcal{M}} A \to B \;\;\; \text{sse} \quad\; \nvDash_w^{\mathcal{M}} A \;\; \text{o} \;\; \vDash_w^{\mathcal{M}} B$
			      \item $\vDash_w^{\mathcal{M}} \Box A \qquad\, \; \text{sse} \;\;\quad \forall v \in \mathcal{W} (\text{se} \; w\mathcal{R}v  \text{, allora} \; \vDash_v^{\mathcal{M}} A)$ \footnote{In linguaggio naturale: \enquote{$A$ è necessaria per un mondo $w$ se e solo se $A$ è vera per tutti i mondi accessibili da $w$}.} \thinspace \footnote{Il modo in cui è definita la verità dell'operatore $\Box$ implica che, se il mondo $w$ non vede nessun altro mondo, $\Box A$ sarà sempre vera in $w$, poiché l'antecedente dell'implicazione è falso, e dunque l'implicazione risulta essere sempre vera. In altre parole, possiamo sempre affermare che $\vDash_w^{\mathcal{M}} \Box A$ in un mondo cieco.}
			      \item $\vDash_w^{\mathcal{M}} \Dmd A \qquad\ \; \text{sse} \;\;\quad \exists v \in \mathcal{W} (w\mathcal{R}v \;\, \text{e} \; \vDash_v^{\mathcal{M}} A)$ \footnote{In linguaggio naturale: \enquote{$A$ è possibile per un mondo $w$ se e solo se $A$ è vera per almeno un mondo accessibile da $w$}.} \thinspace \footnote{Il modo in cui è definita la verità dell'operatore $\Dmd$ implica che, se il mondo $w$ non vede nessun altro mondo, $\Dmd A$ sarà sempre falsa in $w$, poiché almeno uno dei due congiunti è falso, e dunque la congiunzione risulta essere sempre falsa. In altre parole, non possiamo mai affermare che $\vDash_w^{\mathcal{M}} \Dmd A$ in un mondo cieco.} \\
		      \end{itemize}
	\end{description}
\end{defin}

\subsubsubs{Definizione di verità in un modello}
\hypertarget{vermod}{}
\begin{defin}
	[\emph{Verità di una formula in un modello}] Una formula $A$ è vera in un modello $\mathcal{M}$ se e solo se è vera in ogni punto di $\mathcal{M}$:
	$$\vDash^{\mathcal{M}} A \quad\; \text{sse} \quad\; \forall w \in \mathcal{W}, \; \vDash_w^{\mathcal{M}} A$$
\end{defin}
\vspace{12pt}

\subsubsubs{Definizione di validità in una struttura}
\hypertarget{valstrut}{}
\begin{defin}
	[\emph{Validità di una formula in una struttura relazionale}] Una formula $A$ è valida in una struttura relazionale $\mathcal{F}$ se e solo se è vera in ogni $\mathcal{F}$-modello $\mathcal{M}$:
	$$\mathcal{F} \vDash A \quad \; \text{sse} \; \quad \forall \mathcal{F}\text{-modello} \: \mathcal{M}, \; \vDash^{\mathcal{M}} A$$
\end{defin}
\vspace{12pt}

\subsubsubs{Definizione di validità di una formula}
\begin{defin}
	[\emph{Validità di una formula}] Una formula $A$ è valida se e solo se è valida in ogni struttura $\mathcal{F}$:
	$$\vDash A \quad \; \text{sse} \; \quad \forall \mathcal{F}(\mathcal{F} \vDash A)$$
\end{defin}
\begin{itemize}
	\item Le tautologie classiche sono valide su ogni struttura.
	\item Lo schema $\Box (A \to B) \to (\Box A \to \Box B)$ ($K$) è valido su ogni struttura.
	      \begin{dimo}
		      \phantom{text} \\
		      Sia $\mathcal{M}$ un qualsiasi $\mathcal{F}$-modello e $w$ un qualsiasi punto $\in \mathcal{W}$. \\
		      Assumo $\vDash_w^{\mathcal{M}} \Box (A \to B) \:$ e $\;\vDash_w^{\mathcal{M}} \Box A$; va quindi mostrato che $\vDash_w^{\mathcal{M}} \Box B$. \\
		      Per \hyperlink{defverp}{definizione di verità di una formula in un punto}, ho che:
		      $$\begin{aligned}
				      \vDash_w^{\mathcal{M}} \Box (A \to B) & \quad \; \text{sse} \; \quad \forall v \in \mathcal{W} (\text{se} \; w\mathcal{R}v \text{, allora} \; \vDash_v^{\mathcal{M}} (A \to B)) \\
				      \vDash_w^{\mathcal{M}} \Box A         & \quad \; \text{sse} \;\quad \forall v \in \mathcal{W} (\text{se} \; w\mathcal{R}v \text{, allora} \; \vDash_v^{\mathcal{M}} A)          \\
			      \end{aligned}$$
		      Posso quindi concludere:
		      $$\vDash_w^{\mathcal{M}} \Box B \quad \; \text{sse} \; \quad \forall v \in \mathcal{W} (\text{se} \; w\mathcal{R}v \text{, allora} \; \vDash_v^{\mathcal{M}} B)$$
	      \end{dimo}
\end{itemize}

\subsubsubs{Definizione di validità su una classe di strutture}
\begin{defin}
	[\emph{Validità di una formula su una classe di strutture}] Una formula $A$ è valida su una classe di strutture $\mathcal{C}$ se e solo se è valida in ogni struttura $\mathcal{F}$ che ne fa parte:
	$$\mathcal{C} \vDash A \quad \; \text{sse} \; \quad \forall \mathcal{F} \in \mathcal{C}(\mathcal{F} \vDash A)$$
\end{defin}
\vspace{12pt}

\subsubsubs{Definizione di conseguenza logica di una formula su una classe}
\begin{defin}
	[\emph{Conseguenza logica di una formula su una classe}] Una formula $A$ è conseguenza logica (locale) di un insieme di formule $\Gamma$ su un classe di strutture $\mathcal{C}$ (scriviamo \enquote{$\Gamma \vDash_{\mathcal{C}} A$}) se e solo se  $$\forall w \in \mathcal{F}\text{-modello} \;\mathcal{M}, \ \text{se} \; \forall B \in \Gamma \ (\vDash_w^{\mathcal{M}} B) \text{, allora}\;\vDash_w^{\mathcal{M}} A$$
\end{defin}
\vspace{12pt}

\subsubsection{Regole che preservano la validità}
\noindent Le seguenti regole preservano la verità/validità di una formula $A$ in un punto di un modello, in un modello o in una struttura relazionale.

La proprietà di preservare la verità/validità di una formula si trasmette in ordine di grandezza \emph{crescente}: qualunque regola preservi la verità in un punto di un modello, la preserva anche in quel modello e ne preserva la validità nella struttura corrispondente; qualunque regola preservi la verità in un modello, preserva anche la sua validità nella struttura corrispondente, ma non è detto che ne preservi la verità in ogni punto del modello; infine, una regola che preserva la validità in una struttura non è detto che preservi anche la verità in un modello e tantomeno in ogni punto del modello. \\

\begin{description}
	\subsubsubs{\emph{Modus ponens}}
	\hypertarget{mp}{}
	\item[\emph{Modus ponens}:]
	      \begin{mathpar}
		      \inferrule*[Right=$M \! P$]{\vDash A \\ \vDash A \to B}{\vDash B}
		      \vspace*{-14pt}
	      \end{mathpar}
	      Il \emph{modus ponens} preserva la verità in un punto di un modello:
	      \vspace*{-4pt}
	      \begin{mathpar}
		      \inferrule*[Right=$M \! P$]{\vDash_w^{\mathcal{M}} A \\ \vDash_w^{\mathcal{M}} A \to B}{\vDash_w^{\mathcal{M}} B}
		      \vspace*{-14pt}
	      \end{mathpar}
	      Quindi, in ogni modello:
	      \vspace*{-4pt}
	      \begin{mathpar}
		      \inferrule*[Right=$M \! P$]{\vDash^{\mathcal{M}} A \\ \vDash^{\mathcal{M}} A \to B}{\vDash^{\mathcal{M}} B}
		      \vspace*{-14pt}
	      \end{mathpar}
	      Di conseguenza, preserva la validità in ogni struttura:
	      \begin{mathpar}
		      \inferrule*[Right=$M \! P$]{\mathcal{F} \vDash A \\ \mathcal{F} \vDash A \to B}{\mathcal{F} \vDash B} \subsubsubs{Necessitazione} \hypertarget{n}{}
	      \end{mathpar}
	\item[Necessitazione:]
	      \vspace*{-4pt}
	      \begin{mathpar}
		      \inferrule*[Right=$N$]{\vDash A}{\vDash \Box A}
		      \vspace*{-14pt}
	      \end{mathpar}
	      La necessitazione non preserva la verità in un punto di un modello: si può infatti avere una struttura $\mathcal{F} = \langle \{w, v\}, \{\langle w, v \rangle\} \rangle$ con $I(p) = \{w\}$; avremo quindi contemporaneamente $\vDash_w^{\mathcal{M}} A$ e $\nvDash_w^{\mathcal{M}} \Box A$ (con $A \equiv p$):
	      \begin{center}
		      \begin{tikzpicture}
			      [world/.style={circle, draw, fill=black, inner sep=0pt, minimum size=4pt},
				      relation/.style={->, >=Stealth, shorten >=1pt, shorten <=1pt}]
			      \node[world, label={[font=\Large]left:{$w \atop p$}}] (w) at (0,0) {};
			      \node[world, label={[font=\Large]right:{$v \atop \lnot p$}}] (v) at (2,0) {};
			      \draw[relation] (w) -- (v);
		      \end{tikzpicture}
	      \end{center}
	      Preserva, però, la verità in un modello:
	      \vspace*{-4pt}
	      \begin{mathpar}
		      \inferrule*[Right=$N$]{\vDash^{\mathcal{M}} A}{\vDash^{\mathcal{M}} \Box A}
		      \vspace*{-14pt}
	      \end{mathpar}
	      In quanto, se esistesse un punto $w$ tale che $\nvDash_w^{\mathcal{M}} \Box A$, esisterebbe un punto $v$ accessibile da $w$ tale che $\nvDash_v^{\mathcal{M}} A$; ma ciò non è possibile, data la premessa $\vDash^{\mathcal{M}} A$. \\
	      Di conseguenza, preserva la validità in ogni struttura:
	      \vspace*{-4pt}
	      \begin{mathpar}
		      \inferrule*[Right=$N$]{\mathcal{F} \vDash A}{\mathcal{F} \vDash \Box A} \subsubsubs{Sostituzione uniforme} \hypertarget{su}{}
	      \end{mathpar}
	\item[Sostituzione uniforme:]
	      \begin{mathpar}
		      \inferrule*[Right=$SU$]{\vDash A}{\vDash A \left[\sfrac{B}{p}\right]} \subsubsubs{Definizione di $A \left[\sfrac{B}{p}\right]$}
		      \vspace*{-24pt}
	      \end{mathpar}
	      \vspace*{-12pt} \hypertarget{defsu}{}
	      \begin{defin}
		      [{$A \left[\sfrac{B}{p}\right]$}] \phantom{ciao}
		      \begin{itemize}
			      \item \emph{Diretta:} $A {\left[\sfrac{B}{p}\right]}$ è la formula ottenuta rimpiazzando in $A$ ciascuna occorrenza di $p$ con un'occorrenza di $B$.
			      \item \emph{Per induzione sulla costruzione di $A$:}
			            \begin{description}
				            \item[Base:] \phantom{ciao}
				                  \begin{itemize}
					                  \item $q \left[\sfrac{B}{p}\right] \;\ \equiv \; \left\{
						                        \begin{aligned}
							                         & \text{se} \; q \equiv p\text{, allora} \: B     \\
							                         & \text{se} \; q \not\equiv p\text{, allora} \: q
						                        \end{aligned}
						                        \right.$
					                  \item $\bot \left[\sfrac{B}{p}\right] \; \equiv \; \bot$
				                  \end{itemize}
				            \item[Passo:] \phantom{ciao}
				                  \begin{itemize}
					                  \item $(C \circ D)\left[\sfrac{B}{p}\right] \; \equiv \; C\left[\sfrac{B}{p}\right] \circ D \left[\sfrac{B}{p}\right]$ \quad con $\circ \in \{\land, \lor, \to\}$
					                  \item $(\bigcirc C)\left[\sfrac{B}{p}\right] \quad \equiv \; \bigcirc(C\left[\sfrac{B}{p}\right])$ \qquad \quad con $\bigcirc \in \{\Box, \Dmd\}$ \\
				                  \end{itemize}
			            \end{description}
		      \end{itemize}
	      \end{defin}
	      La sostituzione uniforme non preserva la verità né in un punto di un modello, né in un modello. Infatti, ponendo $B = \bot$, si può avere che $\vDash^{\mathcal{M}} p \:$ e $\;\nvDash^{\mathcal{M}} p\left[\sfrac{B}{p}\right]$.

	      Preserva, però, la validità in una struttura:
	      \vspace*{-4pt}
	      \begin{mathpar}
		      \inferrule*[Right=$SU$]{\mathcal{F} \vDash A}{\mathcal{F} \vDash A \left[\sfrac{B}{p}\right]}
	      \end{mathpar}
	      \vspace*{-14pt} \subsubsubs{Teorema: la sostituzione uniforme preserva la validità in una struttura}
	      \begin{theo}
		      La sostituzione uniforme $SU$ preserva la validità in una struttura relazionale $\mathcal{F}$:
		      $$\text{se} \ \mathcal{F} \vDash A \text{, allora} \ \mathcal{F} \vDash A \left[\sfrac{B}{p}\right]$$
	      \end{theo}
	      \begin{dimo}
		      [per contrapposizione] \phantom{ciao} \\
		      La dimostrazione di questo teorema si avvale di una \textbf{Definizione} e di un \textbf{Lemma}. \\

		      \hypertarget{mod*}{}
		      \begin{defin}
			      [\emph{Modello $\mathcal{M}^{\ast}$}] Siano $\mathcal{M} := \langle \mathcal{W}, \mathcal{R}, I \rangle$, $B \in \latinmath{fm}^{\Phi}$ e $p \in \Phi$; il modello $\mathcal{M}^{\ast}$ è una tripla
			      $$\mathcal{M}^{\ast} := \langle \mathcal{W}, \mathcal{R}, I^{\ast} \rangle $$
			      dove
			      $$I^{\ast}(q) = \left\{
				      \begin{aligned}
					       & I \;   &  & \text{se} \; q \not\equiv p \\
					       & H_B \; &  & \text{se} \; q \equiv p
				      \end{aligned}
				      \right.$$
			      $H_B$ è l'insieme dei punti di $\mathcal{W}$ tali che, nel modello $\mathcal{M}$, $B$ è vera: $$H_B = \{w \in \mathcal{W} : \; \vDash_w^{\mathcal{M}} B\}$$
			      $\mathcal{M}^{\ast}$, quindi, è un modello in cui $p$ è vera in tutti i punti in cui è vera $B$; in altre parole, $\mathcal{M}^{\ast}$ si comporta su una formula $A$ come $\mathcal{M}$ si comporta su $\vDash A \left[\sfrac{B}{p}\right]$. \\
		      \end{defin}
		      \hypertarget{lemdif}{}
		      \begin{lem}
			      [Equivalenza di formule in un punto in modelli diversi] Per una qualsiasi formula $C \in \latinmath{fm}^{\Phi}$ e $\forall x \in \mathcal{W}$, in un punto $x$ di un modello $\mathcal{M}$ si ha che $C$ è vera sostituendo $B$ a $p$ se e solo se $C$ è vera nello stesso punto del modello $\mathcal{M}^{\ast}$:
			      $$\vDash_x^{\mathcal{M}} C \left[\sfrac{B}{p}\right] \quad\; \text{sse} \quad\; \vDash_x^{\mathcal{M^{\ast}}} C$$
		      \end{lem}
		      \begin{dimo}
			      [per induzione strutturale su $C$] \phantom{ciao}
			      \begin{description}
				      \item[Base:] \phantom{ciao}
				            \begin{itemize}
					            \item $C \equiv q \not\equiv p$: \\
					                  $\vDash_x^{\mathcal{M}} q \left[\sfrac{B}{p}\right] \;\; \text{sse} \;\; \vDash_x^{\mathcal{M}} q \;\; \; \text{sse}\;\; \vDash_x^{\mathcal{M^{\ast}}} q$ \quad per definizione di $I^{\ast}$.
					            \item $C \equiv q \equiv p$: \\
					                  $\vDash_x^{\mathcal{M}} p \left[\sfrac{B}{p}\right] \;\; \text{sse} \;\; \vDash_x^{\mathcal{M}} B \;\; \text{sse} \;\; \vDash_x^{\mathcal{M^{\ast}}} p$ \quad per definizione di $I^{\ast}$ e $H_B$.
					            \item $C \equiv \bot$: \\
					                  $\nvDash_x^{\mathcal{M}} \bot \left[\sfrac{B}{p}\right] \; \text{sse} \;\; \nvDash_x^{\mathcal{M^{\ast}}} \bot$ \qquad \qquad \quad $\; \; \, \bot$ è sempre falso.
				            \end{itemize}
				      \item[Passo:] ho un caso per ogni connettivo.\\
				            \latinmath{IH1}: \emph{$\forall x \in \mathcal{W} (\vDash_x^{\mathcal{M}} D\left[\sfrac{B}{p}\right] \; \text{sse} \; \vDash_x^{\mathcal{M^{\ast}}} D)$} \\
				            \latinmath{IH2}: \emph{$\forall x \in \mathcal{W} (\vDash_x^{\mathcal{M}} E\left[\sfrac{B}{p}\right] \; \text{sse} \; \vDash_x^{\mathcal{M^{\ast}}} E)$}
				            \begin{itemize}
					            \item $C \equiv D \land E$ (va dimostrato \emph{$\forall x \in \mathcal{W} (\vDash_x^{\mathcal{M}} (D \land E)\left[\sfrac{B}{p}\right] \; \text{sse} \; \vDash_x^{\mathcal{M^{\ast}}} D \land E)$}). \\
					                  Scompongo (per \hyperlink{defsu}{definizione di $A \left[\sfrac{B}{p}\right]$}):
					                  $$\vDash_x^{\mathcal{M}} (D \land E)\left[\sfrac{B}{p}\right] \quad\; \text{sse} \quad\; \vDash_x^{\mathcal{M}} D\left[\sfrac{B}{p}\right] \land \vDash_x^{\mathcal{M}} E\left[\sfrac{B}{p}\right]$$
					                  Per \hyperlink{defverp}{definizione di verità di una formula in un punto}:
					                  $$\vDash_x^{\mathcal{M}} D\left[\sfrac{B}{p}\right] \land \vDash_x^{\mathcal{M}} E\left[\sfrac{B}{p}\right] \quad\; \text{sse} \quad\; \vDash_x^{\mathcal{M}} D\left[\sfrac{B}{p}\right] \; \text{e} \; \vDash_x^{\mathcal{M}} E\left[\sfrac{B}{p}\right]$$
					                  Per \latinmath{IH1} e \latinmath{IH2}, $\forall x \in \mathcal{W}$:
					                  $$\vDash_x^{\mathcal{M}} D\left[\sfrac{B}{p}\right] \; \text{e} \; \vDash_x^{\mathcal{M}} E\left[\sfrac{B}{p}\right] \quad\; \text{sse} \quad\; \vDash_x^{\mathcal{M^{\ast}}} D \; \text{e} \; \vDash_x^{\mathcal{M^{\ast}}} E$$
					                  Posso quindi concludere, per \hyperlink{defverp}{definizione di verità di una formula in un punto}::
					                  $$\vDash_x^{\mathcal{M^{\ast}}} D \; \text{e} \; \vDash_x^{\mathcal{M^{\ast}}} E \quad\; \text{sse} \quad\; \vDash_x^{\mathcal{M^{\ast}}} D \land E$$
					            \item $C \equiv D \lor E$ (analogo a $C \equiv D \land E$; va dimostrato \emph{$\forall x \in \mathcal{W} (\vDash_x^{\mathcal{M}} (D \lor E)\left[\sfrac{B}{p}\right] \; \text{sse} \; \vDash_x^{\mathcal{M^{\ast}}} D \lor E)$}). \\
					                  Scompongo (per \hyperlink{defsu}{definizione di $A \left[\sfrac{B}{p}\right]$}):
					                  $$\vDash_x^{\mathcal{M}} (D \lor E)\left[\sfrac{B}{p}\right] \quad \; \text{sse} \quad\; \vDash_x^{\mathcal{M}} D\left[\sfrac{B}{p}\right] \lor \vDash_x^{\mathcal{M}} E\left[\sfrac{B}{p}\right] $$
					                  Per \hyperlink{defverp}{definizione di verità di una formula in un punto}:
					                  $$\vDash_x^{\mathcal{M}} D\left[\sfrac{B}{p}\right] \lor \vDash_x^{\mathcal{M}} E\left[\sfrac{B}{p}\right] \quad\; \text{sse} \quad\; \vDash_x^{\mathcal{M}} D\left[\sfrac{B}{p}\right] \; \text{o} \; \vDash_x^{\mathcal{M}} E\left[\sfrac{B}{p}\right]$$
					                  Per \latinmath{IH1} e \latinmath{IH2}, $\forall x \in \mathcal{W}$:
					                  $$\vDash_x^{\mathcal{M}} D\left[\sfrac{B}{p}\right] \; \text{o} \; \vDash_x^{\mathcal{M}} E\left[\sfrac{B}{p}\right] \quad\; \text{sse} \quad\; \vDash_x^{\mathcal{M^{\ast}}} D \; \text{o} \; \vDash_x^{\mathcal{M^{\ast}}} E$$
					                  Posso quindi concludere, per \hyperlink{defverp}{definizione di verità di una formula in un punto}::
					                  $$\vDash_x^{\mathcal{M^{\ast}}} D \; \text{o} \; \vDash_x^{\mathcal{M^{\ast}}} E \quad\; \text{sse} \quad\; \vDash_x^{\mathcal{M^{\ast}}} D \lor E$$
					            \item $C \equiv D \to E$ (analogo a $C \equiv D \land E$; va dimostrato \emph{$\forall x \in \mathcal{W} (\vDash_x^{\mathcal{M}} (D \to E)\left[\sfrac{B}{p}\right] \; \text{sse} \; \vDash_x^{\mathcal{M^{\ast}}} D \to E)$}). \\
					                  Scompongo (per \hyperlink{defsu}{definizione di $A \left[\sfrac{B}{p}\right]$}):
					                  $$\vDash_x^{\mathcal{M}} (D \to E)\left[\sfrac{B}{p}\right] \quad \; \text{sse} \quad\; \vDash_x^{\mathcal{M}} D\left[\sfrac{B}{p}\right] \to \; \vDash_x^{\mathcal{M}} E\left[\sfrac{B}{p}\right] $$
					                  Per \hyperlink{defverp}{definizione di verità di una formula in un punto}:
					                  $$\vDash_x^{\mathcal{M}} D\left[\sfrac{B}{p}\right] \to \; \vDash_x^{\mathcal{M}} E\left[\sfrac{B}{p}\right] \quad\; \text{sse} \quad\; \nvDash_x^{\mathcal{M}} D\left[\sfrac{B}{p}\right] \; \text{o} \; \vDash_x^{\mathcal{M}} E\left[\sfrac{B}{p}\right]$$
					                  Per \latinmath{IH1} e \latinmath{IH2}, $\forall x \in \mathcal{W}$:
					                  $$\nvDash_x^{\mathcal{M}} D\left[\sfrac{B}{p}\right] \; \text{o} \; \vDash_x^{\mathcal{M}} E\left[\sfrac{B}{p}\right] \quad\; \text{sse} \quad\; \nvDash_x^{\mathcal{M^{\ast}}} D \; \text{o} \; \vDash_x^{\mathcal{M^{\ast}}} E$$
					                  Posso quindi concludere, per \hyperlink{defverp}{definizione di verità di una formula in un punto}::
					                  $$\nvDash_x^{\mathcal{M^{\ast}}} D \; \text{o} \; \vDash_x^{\mathcal{M^{\ast}}} E \quad\; \text{sse} \quad\; \vDash_x^{\mathcal{M^{\ast}}} D \to E$$
					            \item $C \equiv \Box D$ (va dimostrato \emph{$\forall x \in \mathcal{W} (\vDash_x^{\mathcal{M}} \Box D \left[\sfrac{B}{p}\right] \; \text{sse} \; \vDash_x^{\mathcal{M^{\ast}}} \Box D)$}). \\
					                  Per \hyperlink{defverp}{definizione di verità di una formula in un punto}:
					                  $$\vDash_x^{\mathcal{M}} \Box D \left[\sfrac{B}{p}\right] \quad\; \text{sse} \quad\; \forall y \in \mathcal{W} (\text{se} \; x\mathcal{R}y \text{, allora} \; \vDash_y^{\mathcal{M}} D \left[\sfrac{B}{p}\right])$$
					                  Per \latinmath{IH1}, $\forall x \in \mathcal{W}$:
					                  $$\forall y \in \mathcal{W} (\text{se} \; x\mathcal{R}y \text{, allora} \; \vDash_y^{\mathcal{M}} D \left[\sfrac{B}{p}\right]) \quad\; \text{sse} \quad\; \forall y \in \mathcal{W} (\text{se} \; x\mathcal{R}y \text{, allora} \; \vDash_y^{\mathcal{M^{\ast}}} D)$$
					                  Posso quindi concludere, per \hyperlink{defverp}{definizione di verità di una formula in un punto}:
					                  $$\forall y \in \mathcal{W} (\text{se} \; x\mathcal{R}y \text{, allora} \; \vDash_y^{\mathcal{M^{\ast}}} D) \quad\; \text{sse} \quad\; \vDash_x^{\mathcal{M^{\ast}}} \Box D$$
					            \item $C \equiv \Dmd D$ (analogo a $C \equiv \Box D$; va dimostrato \emph{$\forall x \in \mathcal{W} (\vDash_x^{\mathcal{M}} \Dmd D \left[\sfrac{B}{p}\right] \; \text{sse} \; \vDash_x^{\mathcal{M^{\ast}}} \Dmd D)$}). \\
					                  Per \hyperlink{defverp}{definizione di verità di una formula in un punto}:
					                  $$\vDash_x^{\mathcal{M}} \Dmd D \left[\sfrac{B}{p}\right] \quad\; \text{sse} \quad\; \exists y \in \mathcal{W} (x\mathcal{R}y \;\, \text{e} \; \vDash_y^{\mathcal{M}} D \left[\sfrac{B}{p}\right])$$
					                  Per \latinmath{IH1}, $\forall x \in \mathcal{W}$:
					                  $$\exists y \in \mathcal{W} (x\mathcal{R}y \;\, \text{e} \; \vDash_y^{\mathcal{M}} D \left[\sfrac{B}{p}\right]) \quad\; \text{sse} \quad\; \exists y \in \mathcal{W} (x\mathcal{R}y \;\, \text{e} \; \vDash_y^{\mathcal{M^{\ast}}} D)$$
					                  Posso quindi concludere, per \hyperlink{defverp}{definizione di verità di una formula in un punto}:
					                  $$\exists y \in \mathcal{W} (x\mathcal{R}y \;\, \text{e} \; \vDash_y^{\mathcal{M^{\ast}}} D) \quad\; \text{sse} \quad\; \vDash_x^{\mathcal{M^{\ast}}} \Dmd D$$
				            \end{itemize}
			      \end{description}
			      \vspace*{-16pt} \raggedleft \qedhere
		      \end{dimo}
		      Procediamo ora con la dimostrazione per contrapposizione. \\
		      Va dimostrato che \emph{$\text{se} \; \mathcal{F} \nvDash A \left[\sfrac{B}{p}\right] \text{, allora} \; \mathcal{F} \nvDash A$}. Assumo $\mathcal{F} \nvDash A \left[\sfrac{B}{p}\right]$; va quindi mostrato che $\mathcal{F} \nvDash A$. \\
		      So che esiste un $\mathcal{F}$-modello $\mathcal{M}$ tale che:
		      $$\exists w \in \mathcal{W} : \; \nvDash_w^{\mathcal{M}} A \left[\sfrac{B}{p}\right]$$
		      Sia $\mathcal{M^{\ast}}$ un modello \hyperlink{mod*}{definito come sopra}; per il precedente \hyperlink{lemdif}{\textbf{Lemma}}, so che:
		      $$\nvDash_w^{\mathcal{M}} A \left[\sfrac{B}{p}\right] \quad\; \text{sse} \quad \; \nvDash_w^{\mathcal{M^{\ast}}} A $$
		      Quindi, $\exists \mathcal{M^{\ast}} \: (\nvDash_w^{\mathcal{M^{\ast}}} A)$. Posso concludere: $$\mathcal{F} \nvDash A$$
	      \end{dimo}
\end{description}

\newpage
\section{Logiche modali normali}
\noindent Possiamo ora dare una prima definizione per le logiche modali normali. \\

\subsubsubs{Definizione di logica modale normale}
\begin{defin}
	[\emph{Logica modale normale}] Una logica modale normale \footnote{Nelle logiche modali \emph{non} normali valgono meno principi.} è un insieme \latinmath{L} di \hyperlink{fbfmd}{formule} ($\latinmath{L} \subseteq \latinmath{fm}^{\Phi}$) tale che:
	\begin{enumerate}
		\item Se $A$ è una tautologia ($A \in \latinmath{TAUT}$), allora $A \in \latinmath{L}$;
		\item Se $A$ è un'istanza dello schema $K$ ($A \equiv \Box (B \to C) \to (\Box B \to \Box C)$), allora $A \in \latinmath{L}$;
		\item Se $A$ è un'istanza dello schema $\Box B \leftrightarrow \lnot \Dmd \lnot B$, allora $A \in \latinmath{L}$ (questo rende $\Box$ il duale di $\Dmd$);
		\item \latinmath{L} è chiuso sotto \hyperlink{mp}{\emph{Modus ponens}} ($M \! P$) \footnote{Il significato di questo e dei punti successivi è che, se le premesse della regola in questione appartengono a \latinmath{L}, allora anche le sue conclusioni apparterranno a \latinmath{L}.};
		\item \latinmath{L} è chiuso sotto \hyperlink{n}{Necessitazione} ($N$);
		\item \latinmath{L} è chiuso sotto \hyperlink{su}{Sostituzione uniforme} ($SU$). \\
	\end{enumerate}
\end{defin}
Quella qui definita è la \emph{più piccola} logica modale normale, \latinmath{K} \footnote{Prende il nome dallo schema $K$, chiamato così in onore di Saul Kripke.}, costituita dall'insieme delle formule valide su \emph{tutte} le strutture relazionali $\mathcal{F}$.

Una logica che estende \latinmath{K} aggiunge una serie di formule $S_1, \ldots, S_n$ (dove $S_1, \ldots, S_n \in \latinmath{fm}^{\Phi}$) ai suoi assiomi. $S_1, \ldots, S_n$ provengono da \hyperlink{tabschemi}{schemi} che hanno validità in una certa classe di strutture (come mostrato nella sezione successiva). Con le logiche ottenute in questo modo, è possibile costruire un cubo: \\

\subsubsubs{Cubo delle logiche modali}
\begin{center}
	\begin{tikzpicture} [world/.style={circle, draw, fill=black, inner sep=0pt, minimum size=5pt},  relation/.style={->, >=Stealth, shorten >=0pt}, plainline/.style={-, line width=1.3pt, shorten >=0pt, shorten <=0pt}, dashedline/.style={-, dashed, shorten >=1pt, shorten <=1pt}]
		% logiche
		\node[world, label={left:{$\latinmath{K}$}}] (K) at (0,0) {};
		\node[world, label={right:{$\latinmath{B}$}}] (B) at (8,0) {};
		\node[world, label={left:{$\latinmath{D}$}}] (D) at (0,4) {};
		\node[world, label={left:{$\latinmath{T}$}}] (T) at (0,8) {};
		\node[world, label={right:{$\latinmath{DB}$}}] (DB) at (8,4) {};
		\node[world, label={right:{$\latinmath{KTB}$}}] (KTB) at (8,8) {};
		\node[world, label={left:{$\latinmath{K4}$}}] (K4) at (2,2) {};
		\node[world, label={right:{$\latinmath{KB5}$}}] (KB5) at (10,2) {};
		\node[world, label={left:{$\latinmath{S4}$}}] (S4) at (2,10) {};
		\node[world, label={right:{$\latinmath{S5}$}}] (S5) at (10,10) {};
		\node[world, label={left:{$\latinmath{D4}$}}] (D4) at (2,6) {};
		\node[world, label={below right:{$\latinmath{D45}$}}] (D45) at (4,6) {};
		\node[world, label={below right:{$\latinmath{K45}$}}] (K45) at (4,2) {};
		\node[world, label={below right:{$\latinmath{D5}$}}] (D5) at (2.2,5) {};
		\node[world, label={below right:{$\latinmath{K5}$}}] (K5) at (2.2,1) {};
		% linee
		\draw[plainline] (K) -- (B);
		\draw[plainline] (K) -- (D);
		\draw[dashedline] (K) -- (K4);
		\draw[plainline] (D) -- (T);
		\draw[plainline] (D) -- (DB);
		\draw[plainline] (T) -- (KTB);
		\draw[plainline] (B) -- (KB5);
		\draw[plainline] (B) -- (DB);
		\draw[plainline] (DB) -- (KTB);
		\draw[plainline] (KB5) -- (S5);
		\draw[plainline] (T) -- (S4);
		\draw[plainline] (S4) -- (S5);
		\draw[plainline] (S5) -- (KTB);
		\draw[dashedline] (K4) -- (K45);
		\draw[dashedline] (K45) -- (KB5);
		\draw[dashedline] (K4) -- (D4);
		\draw[dashedline] (K) -- (K5);
		\draw[dashedline] (K5) -- (K45);
		\draw[dashedline] (K5) -- (D5);
		\draw[dashedline] (K45) -- (D45);
		\draw[dashedline] (D4) -- (D);
		\draw[dashedline] (D4) -- (D45);
		\draw[dashedline] (D4) -- (S4);
		\draw[dashedline] (D45) -- (S5);
		\draw[dashedline] (D45) -- (D5);
		\draw[dashedline] (D5) -- (D);
	\end{tikzpicture}
\end{center}

Il cubo esplicita le relazioni di inclusione esistenti tra le logiche: quelle più distanti da \latinmath{K} includono quelle più vicine. Avremo quindi, ad esempio, che:
\begin{itemize}
	\item \latinmath{S5} = \latinmath{KT4B} = \latinmath{KT5}
	\item \latinmath{S4} = \latinmath{KT4}
	\item \latinmath{T} = \latinmath{KT}
	\item \latinmath{D} = \latinmath{KD}
	\item \latinmath{DB} = \latinmath{KDB}
	\item $\ldots$
\end{itemize}
Le logiche più importanti sono \latinmath{K}, \latinmath{D}, \latinmath{T}, \latinmath{K4}, \latinmath{S4}, \latinmath{S5}. In particolare, \latinmath{S5} è la logica più forte, in cui vale la relazione di equivalenza (ovvero una relazione riflessiva, transitiva e simmetrica) \footnote{\emph{Fun fact:} \latinmath{S5} è l'unica logica modale in cui il problema di soddisfacibilità di una formula è $N \! P$-completo, il che la rende la logica modale più facilmente decidibile, al pari della logica proposizionale. Questo importante risultato è stato mostrato da \href{https://scispace.com/pdf/the-computational-complexity-of-provability-in-systems-of-6h8tx92q7c.pdf}{R. E. Ladder} nel 1977.}.

\newpage
\subsection{Corrispondenza e non esprimibilità}
\noindent Altre logiche modali normali possono essere ottenute estendendo \latinmath{K} con formule aggiuntive. Queste vengono ricavate dal mostrare che una particolare formula permette di isolare una certa classe di strutture rispetto a una certa proprietà $P$ \footnote{Ci si limiterà a proprietà esprimibili mediante il linguaggio del primo ordine.} di $\mathcal{R}$, nel senso che quella classe di strutture è esattamente quella in cui è valida quella particolare formula modale (ovvero, tutte le strutture di quella classe soddisfano la proprietà $P$ di $\mathcal{R}$).

Diremo che una struttura $\mathcal{F} = \langle \mathcal{W}, \mathcal{R} \rangle$ \enquote{gode della proprietà $P$} se la relazione $\mathcal{R}$ ne gode e, in questo caso, scriveremo \enquote{$\mathcal{F} \rhd P$}. \\

\subsubsubs{Definizione di corrispondenza}
\begin{defin}
	[\emph{Corrispondenza}] Una formula $A \in \latinmath{fm}^{\Phi}$ corrisponde a una proprietà $P$ di $\mathcal{R}$ se vale che:
	$$\forall \mathcal{F}, \; \mathcal{F} \vDash A \quad\; \text{sse} \;\quad \mathcal{F} \rhd P$$
\end{defin}

\vspace{12pt}
Nelle seguenti tabelle sono mostrate alcune proprietà di $\mathcal{R}$ esprimibili al primo ordine e gli schemi modali corrispondenti. \\

\begin{center}
	\vspace{12pt}
	\subsubsubs{Proprietà esprimibili al prim'ordine}
	\begin{tabular}{|l|l|}
		\hline
		\textbf{Nome}         & \textbf{Definizione}                                                                                                          \\
		\hline
		Riflessività          & $\forall x (x\mathcal{R}x)$                                                                                                   \\
		\hline
		Transitività          & $\forall x \forall y \forall z (x\mathcal{R}y \land y\mathcal{R}z \to x\mathcal{R}z)$                                         \\
		\hline
		Serialità             & $\forall x \exists y (x\mathcal{R}y)$                                                                                         \\
		\hline
		Densità debole        & $\forall x \forall y (x\mathcal{R}y \to \exists z(x\mathcal{R}z \land z\mathcal{R}y))$                                        \\
		\hline
		Simmetria             & $\forall x \forall y (x\mathcal{R}y \to y\mathcal{R}x)$                                                                       \\
		\hline
		Euclidea              & $\forall x \forall y \forall z (x\mathcal{R}y \land x\mathcal{R}z \to y\mathcal{R}z)$                                         \\
		\hline
		Convergenza debole    & $\forall x \forall y \forall z (x\mathcal{R}y \land x\mathcal{R}z \to \exists w(y\mathcal{R}w \land z\mathcal{R}w))$          \\
		\hline
		Convergenza           & $\forall x \forall y \exists z (x\mathcal{R}z \land y\mathcal{R}z)$                                                           \\
		\hline
		Connessione debole    & $\forall x \forall y \forall z (x\mathcal{R}y \land x\mathcal{R}z \to y\mathcal{R}z \lor z\mathcal{R}y \lor y=z)$             \\
		\hline
		Connessione           & $\forall x \forall y (x\mathcal{R}y \lor y\mathcal{R}x \lor x=y)$                                                             \\
		\hline
		Funzionalità parziale & $\forall x \forall y \forall z (x\mathcal{R}y \land x\mathcal{R}z \to y=z)$                                                   \\
		\hline
		Funzionalità          & $\forall x \exists ! y (x\mathcal{R}y)$                                                                                       \\
		\hline
		Isolamento            & $\forall x \forall y (x\mathcal{R}y \leftrightarrow x=y)$                                                                     \\
		\hline
		Cecità                & $\forall x \forall y \lnot (x\mathcal{R}y)$                                                                                   \\
		\hline
		\emph{mnkj}-Lemmon    & $\forall x \forall y \forall z (x\mathcal{R}^my \land x\mathcal{R}^kz \to \exists w (y\mathcal{R}^nw \land z\mathcal{R}^jw))$ \\
		\hline
	\end{tabular}

	\newpage
	\subsubsubs{Schemi e proprietà corrispondenti}
	\hypertarget{tabschemi}{}
	\begin{tabular}{|l|l|l|}
		\hline
		\textbf{Nome dello schema} & \textbf{Schema}                                               & \textbf{Nome della proprietà} \\
		\hline
		$T$                        & $\Box A \to A$                                                & Riflessività                  \\
		\hline
		$4$                        & $\Box A \to \Box \Box A$                                      & Transitività                  \\
		\hline
		$D$                        & $\Box A \to \Dmd A$                                           & Serialità                     \\
		\hline
		$X$                        & $\Box \Box A \to \Box A$                                      & Densità debole                \\
		\hline
		$B$                        & $A \to \Box \Dmd A$                                           & Simmetria                     \\
		\hline
		$5$                        & $\Dmd A \to \Box \Dmd A$                                      & Euclidea                      \\
		\hline
		$2$                        & $\Dmd \Box A \to \Box \Dmd A$                                 & Convergenza debole            \\
		\hline
		$3$                        & $\Box(A \land \Box A \to B) \lor \Box (\Box B \land B \to A)$ & Connessione debole            \\
		\hline
		$P \! F$                   & $\Dmd A \to \Box A$                                           & Funzionalità parziale         \\
		\hline
		$F$                        & $\Dmd A \leftrightarrow \Box A$                               & Funzionalità                  \\
		\hline
		$\latinmath{Triv}$         & $A \leftrightarrow \Box A$                                    & Isolamento                    \\
		\hline
		$\latinmath{Ver}$          & $\Box A$                                                      & Cecità                        \\
		\hline
		$\latinmath{Lemmon}$       & $\Dmd^m \Box^n A \to \Box^k \Dmd^j A$                         & \emph{mnkj}-Lemmon            \\
		\hline
	\end{tabular}
\end{center}

\vspace{12pt}
\subsubsection{Risultati di corrispondenza}
\noindent Di seguito vengono dati teoremi e dimostrazioni delle corrispondenze elencate nell'ultima tabella. Il procedimento per le dimostrazioni (esclusa la dimostrazione per lo schema $3$, che è leggermente differente) è il seguente:
\begin{itemize}
	\item Si svolge prima l'implicazione verso destra (1) e poi quella verso sinistra (2):
	      \begin{enumerate}
		      \item Si assume che, nella struttura in cui ci si trova, lo schema che si sta considerando sia valido per $A \equiv p$ \footnote{$A$ è una qualsiasi formula $\in \latinmath{fm}^{\Phi}$; consideriamo $A \equiv p$ per semplicità. L'importante è che, per qualsiasi $A$ si scelga, si costruisca $I$ in modo da renderla vera nei mondi desiderati.}. Si considera un generico $\mathcal{F}$-modello $\mathcal{M}$ e almeno un suo generico mondo $x$, per poi definire per $\mathcal{M}$ un'interpretazione minimale (ovvero che tratta tutte e sole le variabili necessarie alla dimostrazione) che renda vero l'antecedente dello schema in $x$ per una variabile enunciativa; si ottiene così che in $x$ è vero il conseguente e, per come si è costruita l'interpretazione minimale, si ricava la proprietà che si voleva dimostrare.
		      \item Si assume che la struttura in cui ci si trova goda della proprietà che si considerando \footnote{Si noti che questo non implica automaticamente che la struttura sarà formata da tutti e soli i mondi corrispondenti alle variabili della proprietà, ma che, se quei mondi esistono, godranno di quella proprietà.}. Si considera un generico $\mathcal{F}$-modello $\mathcal{M}$ e un suo generico mondo $x$, tale che in quel mondo sia vera la premessa dello schema, poi si usa la proprietà di $R$ presa per ipotesi per dimostrare che in $x$ è vera la conclusione dello schema.
	      \end{enumerate}
	      Si adottano procedimenti di questo tipo in quanto, per le definizioni di \hyperlink{valstrut}{validità di una formula in una struttura} e di \hyperlink{vermod}{verità di una formula in un modello}, si ha che $\mathcal{F} \vDash A$ sse $\forall \mathcal{F}$-modello $\mathcal{M}, \forall x \in \mathcal{W} (\vDash_{x}^{\mathcal{M}} A)$. Per brevità, nelle conclusioni raggiunte si eviterà di specificare che esse sono vere $\forall \mathcal{F}$-modello $\mathcal{M}$ e $\forall x \in \mathcal{W}$. \\
\end{itemize}

\subsubsubs{Teorema: $T$ corrisponde alla riflessività}
\begin{theo}
	Lo schema $T$ corrisponde alla riflessività.
	$$\mathcal{F} \vDash \Box A \to A \quad\; \text{sse} \;\quad \mathcal{F} \rhd \forall x (x\mathcal{R}x)$$
\end{theo}
\begin{dimo}
	\phantom{ciao}
	\begin{enumerate}
		\item ($\to$): va dimostrato che \emph{se $\mathcal{F} \vDash \Box A \to A$, allora $\mathcal{F} \rhd \forall x (x\mathcal{R}x)$}. \\
		      Sia $p \in \Phi$; assumo $\mathcal{F} \vDash \Box p \to p$. \\
		      Sia $x \in \mathcal{W}$; definisco un $\mathcal{F}$-modello $\mathcal{M}$ tale che: \\

		      \begin{minipage}{0.48\textwidth}
			      $$I(p) = \{w \in \mathcal{W} : x\mathcal{R}w\}$$
		      \end{minipage}
		      \begin{minipage}{0.48\textwidth}
			      \begin{center}
				      \begin{tikzpicture}
					      [world/.style={circle, draw, fill=black, inner sep=0pt, minimum size=4pt},
						      views/.style={->, >=Stealth, shorten >=1pt, shorten <=1pt},
						      generic/.style={->, >=Stealth, shorten >=1pt, shorten <=1pt, dashed}]
					      \node[world, label={above left:{$x$}}] (x) at (0,0) {};
					      \node[world, label={[font=\Large]right:{$w \atop p$}}] (w) at (2,0) {};
					      \draw[generic] (x) -- (w);
				      \end{tikzpicture}
			      \end{center}
		      \end{minipage}
		      \vspace{8pt}

		      Ho così definito che, in un generico \footnote{Quindi o se stesso, o un altro.} mondo $w$ accessibile da $x$, $p$ è vera. \\
		      Poiché $p$ è vera in tutti i mondi visti da $x$ (cioè ho che $\forall w \in \mathcal{W} (\text{se} \; x\mathcal{R}w  \text{, allora} \; \vDash_w^{\mathcal{M}} A)$), per \hyperlink{defverp}{definizione di verità di una formula in un punto} ho che: \\

		      \begin{minipage}{0.48\textwidth}
			      $$\vDash_x^{\mathcal{M}}\Box p$$
		      \end{minipage}
		      \begin{minipage}{0.48\textwidth}
			      \begin{center}
				      \begin{tikzpicture}
					      [world/.style={circle, draw, fill=black, inner sep=0pt, minimum size=4pt},
						      views/.style={->, >=Stealth, shorten >=1pt, shorten <=1pt},
						      generic/.style={->, >=Stealth, shorten >=1pt, shorten <=1pt, dashed}]
					      \node[world, label={[font=\Large]left:{$x \atop \Box p$}}] (x) at (0,0) {};
					      \node[world, label={[font=\Large]right:{$w \atop p$}}] (w) at (2,0) {};
					      \draw[generic] (x) -- (w);
				      \end{tikzpicture}
			      \end{center}
		      \end{minipage}
		      \vspace{8pt}

		      Per assunzione, ho che $\vDash_x^{\mathcal{M}} \Box p \to p$; quindi, per $M \! P$:

		      \begin{minipage}{0.48\textwidth}
			      \begin{mathpar}
				      \inferrule*[Right=$M \! P$]{\vDash_x^{\mathcal{M}}\Box p \\ \vDash_x^{\mathcal{M}} \Box p \to p}{\vDash_x^{\mathcal{M}} p}
			      \end{mathpar}
		      \end{minipage}
		      \begin{minipage}{0.48\textwidth}
			      \begin{center}
				      \begin{tikzpicture}
					      [world/.style={circle, draw, fill=black, inner sep=0pt, minimum size=4pt},
						      views/.style={->, >=Stealth, shorten >=1pt, shorten <=1pt},
						      generic/.style={->, >=Stealth, shorten >=1pt, shorten <=1pt, dashed}]
					      \node[world, label={[font=\Large]left:{$x \atop p, \: \Box p$}}] (x) at (0,0) {};
					      \node[world, label={[font=\Large]right:{$w \atop p$}}] (w) at (2,0) {};
					      \draw[generic] (x) -- (w);
				      \end{tikzpicture}
			      \end{center}
		      \end{minipage}
		      \vspace{8pt}

		      Dunque, $x \in I(p)$ (per \hyperlink{defverp}{definizione di verità di una formula in un punto}), quindi posso concludere:

		      \begin{minipage}{0.48\textwidth}
			      $$x\mathcal{R}x$$
		      \end{minipage}
		      \begin{minipage}{0.48\textwidth}
			      \begin{center}
				      \begin{tikzpicture}
					      [world/.style={circle, draw, fill=black, inner sep=0pt, minimum size=4pt},
						      relation/.style={->, >=Stealth, shorten >=1pt, shorten <=1pt}]
					      \node[world, label={above right:{$x$}}] (x) at (0,0) {};
					      \draw[relation] (x) to [out=135, in=-135, looseness=12, min distance=1cm] (x); % auto-relazione
				      \end{tikzpicture}
			      \end{center}
		      \end{minipage}
		      \vspace{8pt}

		\item ($\leftarrow$): va dimostrato che \emph{se $\mathcal{F} \rhd \forall x (x\mathcal{R}x)$, allora $ \mathcal{F} \vDash \Box A \to A$}. \\
		      Assumo $\mathcal{F} \rhd \forall x (x\mathcal{R}x)$. \\
		      Sia $x \in \mathcal{W}$; definisco un $\mathcal{F}$-modello $\mathcal{M}$ tale che: \\

		      \begin{minipage}{0.48\textwidth}
			      $$\vDash_x^{\mathcal{M}}\Box A$$
		      \end{minipage}
		      \begin{minipage}{0.48\textwidth}
			      \begin{center}
				      \begin{tikzpicture}
					      [world/.style={circle, draw, fill=black, inner sep=0pt, minimum size=4pt},
						      relation/.style={->, >=Stealth, shorten >=1pt, shorten <=1pt}]
					      \node[world, label={[font=\Large]right:{$x \atop \Box A$}}] (x) at (0,0) {};
				      \end{tikzpicture}
			      \end{center}
		      \end{minipage}
		      \vspace{8pt}

		      Ho così definito che, in un generico mondo $x$, $\Box A$ è vera. Quindi, per \hyperlink{defverp}{definizione di verità di una formula in un punto}, ho che $\vDash_x^{\mathcal{M}} \Box A \; \text{sse} \; \forall w \in \mathcal{W} (\text{se} \; x\mathcal{R}w  \text{, allora} \; \vDash_w^{\mathcal{M}} A)$. \\
		      Per assunzione, so che un qualsiasi mondo di questa struttura sarà accessibile da se stesso (cioè $\forall x (x\mathcal{R}x)$), da cui:
		      \vspace{0pt}

		      \begin{minipage}{0.48\textwidth}
			      $$\vDash_x^{\mathcal{M}} A$$
		      \end{minipage}
		      \begin{minipage}{0.48\textwidth}
			      \begin{center}
				      \begin{tikzpicture}
					      [world/.style={circle, draw, fill=black, inner sep=0pt, minimum size=4pt},
						      relation/.style={->, >=Stealth, shorten >=1pt, shorten <=1pt}]
					      \node[world, label={[font=\Large]right:{$x \atop A, \: \Box A$}}] (x) at (0,0) {};
					      \draw[relation] (x) to [out=135, in=-135, looseness=12, min distance=1cm] (x);
				      \end{tikzpicture}
			      \end{center}
		      \end{minipage}
		      \vspace{4pt}

		      Quindi, posso concludere:
		      $$\vDash_x^{\mathcal{M}} \Box A \to A$$
		      (sia antecedente che conseguente sono veri in $x$) \footnote{La verità del conseguente è sufficiente per rendera vera l'implicazione, ma comunque questa frase è vera.}.
	\end{enumerate}
\end{dimo}
Questa dimostrazione può essere svolta anche per contrapposizione o per assurdo. \\

\subsubsubs{Teorema: $4$ corrisponde alla transitività}
\begin{theo}
	Lo schema $4$ corrisponde alla transitività.
	$$\mathcal{F} \vDash \Box A \to \Box \Box A \quad\; \text{sse} \;\quad \mathcal{F} \rhd \forall x \forall y \forall z (x\mathcal{R}y \land y\mathcal{R}z \to x\mathcal{R}z)$$
\end{theo}
\begin{dimo}
	\phantom{ciao}
	\begin{enumerate}
		\item ($\to$): va dimostrato che \emph{se $\mathcal{F} \vDash \Box A \to \Box \Box A$, allora $\mathcal{F} \rhd \forall x \forall y \forall z (x\mathcal{R}y \land y\mathcal{R}z \to x\mathcal{R}z)$}. \\
		      Sia $p \in \Phi$: assumo $\mathcal{F} \vDash \Box p \to \Box \Box p$; \\
		      Siano $x, y, z \in \mathcal{W}$ tali che:

		      \begin{minipage}{0.48\textwidth}
			      $$x\mathcal{R}y \land y\mathcal{R}z$$
		      \end{minipage}
		      \begin{minipage}{0.48\textwidth}
			      \begin{center}
				      \begin{tikzpicture}
					      [world/.style={circle, draw, fill=black, inner sep=0pt, minimum size=4pt},
						      views/.style={->, >=Stealth, shorten >=1pt, shorten <=1pt},
						      generic/.style={->, >=Stealth, shorten >=1pt, shorten <=1pt, dashed}]
					      \node[world, label={left:{$x$}}] (x) at (0,0) {};
					      \node[world, label={right:{$y$}}] (y) at (1,1.5) {};
					      \node[world, label={left:{$z$}}] (z) at (-1,1.5) {};
					      \draw[views] (x) -- (y);
					      \draw[views] (y) -- (z);
				      \end{tikzpicture}
			      \end{center}
		      \end{minipage}
		      \vspace{0pt}

		      Definisco un $\mathcal{F}$-modello $\mathcal{M}$ tale che:

		      \begin{minipage}{0.48\textwidth}
			      $$I(p) = \{w \in \mathcal{W} : x\mathcal{R}w\}$$
		      \end{minipage}
		      \begin{minipage}{0.48\textwidth}
			      \begin{center}
				      \begin{tikzpicture}
					      [world/.style={circle, draw, fill=black, inner sep=0pt, minimum size=4pt},
						      views/.style={->, >=Stealth, shorten >=1pt, shorten <=1pt},
						      generic/.style={->, >=Stealth, shorten >=1pt, shorten <=1pt, dashed}]
					      \node[world, label={right:{$x$}}] (x) at (0,0) {};
					      \node[world, label={[font=\Large]right:{$y \atop p$}}] (y) at (1,1.5) {};
					      \node[world, label={left:{$z$}}] (z) at (-1,1.5) {};
					      \node[world, label={[font=\Large]left:{$w \atop p$}}] (w) at (-1.5,0) {};
					      \draw[views] (x) -- (y);
					      \draw[views] (y) -- (z);
					      \draw[generic] (x) -- (w);
				      \end{tikzpicture}
			      \end{center}
		      \end{minipage}
		      \vspace{0pt}

		      Ho così definito che, in un generico mondo $w$ accessibile da $x$ (tra cui $y$), $p$ è vera. \\
		      Poiché $p$ è vera in tutti i mondi visti da $x$ (cioè ho che $\forall w \in \mathcal{W} (\text{se} \; x\mathcal{R}w  \text{, allora} \; \vDash_w^{\mathcal{M}} A)$), per \hyperlink{defverp}{definizione di verità di una formula in un punto} ho che $\vDash_x^{\mathcal{M}}\Box p$:
		      \vspace{8pt}

		      \begin{center}
			      \begin{tikzpicture}
				      [world/.style={circle, draw, fill=black, inner sep=0pt, minimum size=4pt},
					      views/.style={->, >=Stealth, shorten >=1pt, shorten <=1pt},
					      generic/.style={->, >=Stealth, shorten >=1pt, shorten <=1pt, dashed}]
				      \node[world, label={[font=\Large]right:{$x \atop \Box p$}}] (x) at (0,0) {};
				      \node[world, label={[font=\Large]right:{$y \atop p$}}] (y) at (1,1.5) {};
				      \node[world, label={left:{$z$}}] (z) at (-1,1.5) {};
				      \node[world, label={[font=\Large]left:{$w \atop p$}}] (w) at (-1.5,0) {};
				      \draw[views] (x) -- (y);
				      \draw[views] (y) -- (z);
				      \draw[generic] (x) -- (w);
			      \end{tikzpicture}
		      \end{center}

		      Per assunzione, ho che $\vDash_x^{\mathcal{M}}\Box p \to \Box \Box p$ \footnote{Aver assunto che nella struttura è valido lo schema $\Box p \to \Box \Box p$ fa sì che esso sia vero in tutti i mondi di tutti i modelli; dunque possiamo affermare che, in particolare, sia vero in $x$.}; quindi, per $M \! P$:

		      \begin{minipage}{0.48\textwidth}
			      \begin{mathpar}
				      \inferrule*[Right=$M \! P$]{\vDash_x^{\mathcal{M}}\Box p \\ \vDash_x^{\mathcal{M}}\Box p \to \Box \Box p}{\vDash_x^{\mathcal{M}}\Box \Box p}
			      \end{mathpar}
		      \end{minipage}
		      \begin{minipage}{0.48\textwidth}
			      \begin{center}
				      \begin{tikzpicture}
					      [world/.style={circle, draw, fill=black, inner sep=0pt, minimum size=4pt},
						      views/.style={->, >=Stealth, shorten >=1pt, shorten <=1pt},
						      generic/.style={->, >=Stealth, shorten >=1pt, shorten <=1pt, dashed}]
					      \node[world, label={[font=\Large]right:{$x \atop \Box p, \Box \Box p$}}] (x) at (0,0) {};
					      \node[world, label={[font=\Large]right:{$y \atop p$}}] (y) at (1,1.5) {};
					      \node[world, label={left:{$z$}}] (z) at (-1,1.5) {};
					      \node[world, label={[font=\Large]left:{$w \atop p$}}] (w) at (-1.5,0) {};
					      \draw[views] (x) -- (y);
					      \draw[views] (y) -- (z);
					      \draw[generic] (x) -- (w);
				      \end{tikzpicture}
			      \end{center}
		      \end{minipage}
		      \vspace{8pt}

		      Quindi, ho che in tutti i mondi accessibili da $x$, $\Box p$ è vera; in particolare:
		      \vspace{0pt}

		      \begin{minipage}{0.48\textwidth}
			      $$\vDash_y^{\mathcal{M}}\Box p$$
		      \end{minipage}
		      \begin{minipage}{0.48\textwidth}
			      \begin{center}
				      \begin{tikzpicture}
					      [world/.style={circle, draw, fill=black, inner sep=0pt, minimum size=4pt},
						      views/.style={->, >=Stealth, shorten >=1pt, shorten <=1pt},
						      generic/.style={->, >=Stealth, shorten >=1pt, shorten <=1pt, dashed}]
					      \node[world, label={[font=\Large]right:{$x \atop \Box p, \Box \Box p$}}] (x) at (0,0) {};
					      \node[world, label={[font=\Large]right:{$y \atop p, \Box p$}}] (y) at (1,1.5) {};
					      \node[world, label={left:{$z$}}] (z) at (-1,1.5) {};
					      \node[world, label={[font=\Large]left:{$w \atop p, \Box p$}}] (w) at (-1.5,0) {};
					      \draw[views] (x) -- (y);
					      \draw[views] (y) -- (z);
					      \draw[generic] (x) -- (w);
				      \end{tikzpicture}
			      \end{center}
		      \end{minipage}
		      \vspace{0pt}

		      In tutti i mondi accessibili da $y$, dunque, $p$ è vera; in particolare:
		      \vspace{0pt}

		      \begin{minipage}{0.48\textwidth}
			      $$\vDash_z^{\mathcal{M}} p$$
		      \end{minipage}
		      \begin{minipage}{0.48\textwidth}
			      \begin{center}
				      \begin{tikzpicture}
					      [world/.style={circle, draw, fill=black, inner sep=0pt, minimum size=4pt},
						      views/.style={->, >=Stealth, shorten >=1pt, shorten <=1pt},
						      generic/.style={->, >=Stealth, shorten >=1pt, shorten <=1pt, dashed}]
					      \node[world, label={[font=\Large]right:{$x \atop \Box p, \Box \Box p$}}] (x) at (0,0) {};
					      \node[world, label={[font=\Large]right:{$y \atop p, \Box p$}}] (y) at (1,1.5) {};
					      \node[world, label={[font=\Large]left:{$z \atop p$}}] (z) at (-1,1.5) {};
					      \node[world, label={[font=\Large]left:{$w \atop p, \Box p$}}] (w) at (-1.5,0) {};
					      \draw[views] (x) -- (y);
					      \draw[views] (y) -- (z);
					      \draw[generic] (x) -- (w);
				      \end{tikzpicture}
			      \end{center}
		      \end{minipage}
		      \vspace{0pt}

		      Dunque, $z \in I(p)$ (per \hyperlink{defverp}{definizione di verità di una formula in un punto}), da cui $x\mathcal{R}z$. Quindi, posso concludere:

		      \begin{minipage}{0.48\textwidth}
			      $$x\mathcal{R}y \land y\mathcal{R}z \to x\mathcal{R}z$$
		      \end{minipage}
		      \begin{minipage}{0.48\textwidth}
			      \begin{center}
				      \begin{tikzpicture}
					      [world/.style={circle, draw, fill=black, inner sep=0pt, minimum size=4pt},
						      views/.style={->, >=Stealth, shorten >=1pt, shorten <=1pt},
						      generic/.style={->, >=Stealth, shorten >=1pt, shorten <=1pt, dashed}]
					      \node[world, label={[font=\Large]right:{$x \atop \Box p, \Box \Box p$}}] (x) at (0,0) {};
					      \node[world, label={[font=\Large]right:{$y \atop p, \Box p$}}] (y) at (1,1.5) {};
					      \node[world, label={[font=\Large]left:{$z \atop p$}}] (z) at (-1,1.5) {};
					      \draw[views] (x) -- (y);
					      \draw[views] (y) -- (z);
					      \draw[views] (x) -- (z);
				      \end{tikzpicture}
			      \end{center}
		      \end{minipage}
		      \vspace{8pt}

		\item ($\leftarrow$): va dimostrato che \emph{se $\mathcal{F} \rhd \forall x \forall y \forall z (x\mathcal{R}y \land y\mathcal{R}z \to x\mathcal{R}z)$, allora $\mathcal{F} \vDash \Box A \to \Box \Box A$}. \\
		      Assumo $\mathcal{F} \rhd \forall x \forall y \forall z (x\mathcal{R}y \land y\mathcal{R}z \to x\mathcal{R}z)$. \\
		      Sia $x \in \mathcal{W}$; definisco un $\mathcal{F}$-modello $\mathcal{M}$ tale che: \\

		      \begin{minipage}{0.48\textwidth}
			      $$\vDash_x^{\mathcal{M}}\Box A$$
		      \end{minipage}
		      \begin{minipage}{0.48\textwidth}
			      \begin{center}
				      \begin{tikzpicture}
					      [world/.style={circle, draw, fill=black, inner sep=0pt, minimum size=4pt},
						      relation/.style={->, >=Stealth, shorten >=1pt, shorten <=1pt}]
					      \node[world, label={[font=\Large]right:{$x \atop \Box A$}}] (x) at (0,0) {};
				      \end{tikzpicture}
			      \end{center}
		      \end{minipage}
		      \vspace{8pt}

		      Ho così definito che, in un generico mondo $x$, $\Box A$ è vera. \\
		      Devo mostrare che anche $\Box \Box A$ è vera in $x$; per \hyperlink{defverp}{definizione di verità di una formula in un punto}, ciò equivale a dimostrare che $\forall y \in \mathcal{W} (\text{se} \; x\mathcal{R}y  \text{, allora} \: \vDash_y^{\mathcal{M}} \Box A)$, che a sua volta equivale a dimostrare che $\forall z \in \mathcal{W} (\text{se} \; y\mathcal{R}z  \text{, allora} \; \vDash_z^{\mathcal{M}} A)$. Riassumendo:
		      $$\vDash_x^{\mathcal{M}} \Box \Box A \quad \; \text{sse} \; \quad \forall y, z \in \mathcal{W} (\text{se} \; x\mathcal{R}y \; \text{e} \; y\mathcal{R}z  \text{, allora} \; \vDash_z^{\mathcal{M}} A)$$
		      Ho due casi:
		      \begin{itemize}
			      \item $\nexists y, z \in \mathcal{W}(x\mathcal{R}y \; \text{e} \; y\mathcal{R}z)$: l'antecedente dell'implicazione è falso, dunque l'implicazione è vera; ho quindi che $\vDash_x^{\mathcal{M}} \Box \Box A$. Di conseguenza, posso concludere:
			            $$\vDash_x^{\mathcal{M}} \Box A \to \Box \Box A$$
			            (sia antecedente che conseguente sono veri in $x$).
			      \item $\exists y, z \in \mathcal{W}(x\mathcal{R}y \; \text{e} \; y\mathcal{R}z)$:
			            per assunzione, ho che:
			            \vspace{8pt}

			            \begin{minipage}{0.48\textwidth}
				            $$\forall y \forall z (x\mathcal{R}y \land y\mathcal{R}z \to x\mathcal{R}z)$$
			            \end{minipage}
			            \begin{minipage}{0.48\textwidth}
				            \begin{center}
					            \begin{tikzpicture}
						            [world/.style={circle, draw, fill=black, inner sep=0pt, minimum size=4pt},
							            views/.style={->, >=Stealth, shorten >=1pt, shorten <=1pt},
							            generic/.style={->, >=Stealth, shorten >=1pt, shorten <=1pt, dashed}]
						            \node[world, label={[font=\Large]right:{$x \atop \Box A$}}] (x) at (0,0) {};
						            \node[world, label={right:{$y$}}] (y) at (1,1.5) {};
						            \node[world, label={left:{$z$}}] (z) at (-1,1.5) {};
						            \draw[views] (x) -- (y);
						            \draw[views] (y) -- (z);
						            \draw[views] (x) -- (z);
					            \end{tikzpicture}
				            \end{center}
			            \end{minipage}
			            \vspace{0pt}

			            Dunque, ho che $A$ è vera sia in $y$ che in $z$, dal momento che entrambi sono accessibili da $x$: \\

			            \begin{center}
				            \begin{tikzpicture}
					            [world/.style={circle, draw, fill=black, inner sep=0pt, minimum size=4pt},
						            views/.style={->, >=Stealth, shorten >=1pt, shorten <=1pt},
						            generic/.style={->, >=Stealth, shorten >=1pt, shorten <=1pt, dashed}]
					            \node[world, label={[font=\Large]right:{$x \atop \Box A$}}] (x) at (0,0) {};
					            \node[world, label={[font=\Large]right:{$y \atop A$}}] (y) at (1,1.5) {};
					            \node[world, label={[font=\Large]left:{$z \atop A$}}] (z) at (-1,1.5) {};
					            \draw[views] (x) -- (y);
					            \draw[views] (y) -- (z);
					            \draw[views] (x) -- (z);
				            \end{tikzpicture}
			            \end{center}
			            \vspace{8pt}

			            Dato che $\vDash_z^{\mathcal{M}} A$, ho che $\vDash_x^{\mathcal{M}} \Box\Box A$. Dunque, posso concludere:
			            $$\vDash_x^{\mathcal{M}} \Box A \to \Box \Box A$$
			            (sia antecedente che conseguente sono veri in $x$).
		      \end{itemize}
	\end{enumerate}
\end{dimo}
Questa dimostrazione può essere svolta anche per contrapposizione. \\

\subsubsubs{Teorema: $D$ corrisponde alla serialità}
\begin{theo}
	Lo schema $D$ corrisponde alla serialità.
	$$\mathcal{F} \vDash \Box A \to \Dmd A \quad\; \text{sse} \;\quad \mathcal{F} \rhd \forall x \exists y (x\mathcal{R}y)$$
\end{theo}
\begin{dimo}
	\phantom{ciao}
	\begin{enumerate}
		\item ($\to$): va dimostrato che \emph{se $\mathcal{F} \vDash \Box A \to \Dmd A$, allora $\mathcal{F} \rhd \forall x \exists y (x\mathcal{R}y)$}. \\
		      Sia $p \in \Phi$: assumo $\mathcal{F} \vDash \Box p \to \Dmd p $. \\
		      Sia $x \in \mathcal{W}$; definisco un $\mathcal{F}$-modello $\mathcal{M}$ tale che: \\

		      \begin{minipage}{0.48\textwidth}
			      $$I(p) = \{w \in \mathcal{W} : x\mathcal{R}w\}$$
		      \end{minipage}
		      \begin{minipage}{0.48\textwidth}
			      \begin{center}
				      \begin{tikzpicture}
					      [world/.style={circle, draw, fill=black, inner sep=0pt, minimum size=4pt},
						      views/.style={->, >=Stealth, shorten >=1pt, shorten <=1pt},
						      generic/.style={->, >=Stealth, shorten >=1pt, shorten <=1pt, dashed}]
					      \node[world, label={above left:{$x$}}] (x) at (0,0) {};
					      \node[world, label={[font=\Large]right:{$w \atop p$}}] (w) at (2,0) {};
					      \draw[generic] (x) -- (w);
				      \end{tikzpicture}
			      \end{center}
		      \end{minipage}
		      \vspace{8pt}

		      Ho così definito che, in un generico mondo $w$ accessibile da $x$, $p$ è vera. \\
		      Poiché $p$ è vera in tutti i mondi visti da $x$ (cioè ho che $\forall w \in \mathcal{W} (\text{se} \; x\mathcal{R}w  \text{, allora} \; \vDash_w^{\mathcal{M}} A)$), per \hyperlink{defverp}{definizione di verità di una formula in un punto} ho che: \\

		      \begin{minipage}{0.48\textwidth}
			      $$\vDash_x^{\mathcal{M}}\Box p$$
		      \end{minipage}
		      \begin{minipage}{0.48\textwidth}
			      \begin{center}
				      \begin{tikzpicture}
					      [world/.style={circle, draw, fill=black, inner sep=0pt, minimum size=4pt},
						      views/.style={->, >=Stealth, shorten >=1pt, shorten <=1pt},
						      generic/.style={->, >=Stealth, shorten >=1pt, shorten <=1pt, dashed}]
					      \node[world, label={[font=\Large]left:{$x \atop \Box p$}}] (x) at (0,0) {};
					      \node[world, label={[font=\Large]right:{$w \atop p$}}] (w) at (2,0) {};
					      \draw[generic] (x) -- (w);
				      \end{tikzpicture}
			      \end{center}
		      \end{minipage}
		      \vspace{8pt}

		      Per assunzione, ho che $\vDash_x^{\mathcal{M}} \Box p \to \Dmd p$; quindi, per $M \! P$:

		      \begin{minipage}{0.48\textwidth}
			      \begin{mathpar}
				      \inferrule*[Right=$M \! P$]{\vDash_x^{\mathcal{M}}\Box p \\ \vDash_x^{\mathcal{M}} \Box p \to \Dmd p}{\vDash_x^{\mathcal{M}} \Dmd p}
			      \end{mathpar}
		      \end{minipage}
		      \begin{minipage}{0.48\textwidth}
			      \begin{center}
				      \begin{tikzpicture}
					      [world/.style={circle, draw, fill=black, inner sep=0pt, minimum size=4pt},
						      views/.style={->, >=Stealth, shorten >=1pt, shorten <=1pt},
						      generic/.style={->, >=Stealth, shorten >=1pt, shorten <=1pt, dashed}]
					      \node[world, label={[font=\Large]left:{$x \atop p, \: \Box p, \: \Dmd p$}}] (x) at (0,0) {};
					      \node[world, label={[font=\Large]right:{$w \atop p$}}] (w) at (2,0) {};
					      \draw[generic] (x) -- (w);
				      \end{tikzpicture}
			      \end{center}
		      \end{minipage}
		      \vspace{8pt}

		      Per \hyperlink{defverp}{definizione di verità di una formula in un punto} (cioè $\vDash_x^{\mathcal{M}} \Dmd A \; \text{sse} \; \exists y \in \mathcal{W} (x\mathcal{R}y \;\, \text{e} \; \vDash_y^{\mathcal{M}} A)$), ho che $y \in I(p)$; dunque, posso concludere: \\

		      \begin{minipage}{0.48\textwidth}
			      $$\exists y (x\mathcal{R}y)$$
		      \end{minipage}
		      \begin{minipage}{0.48\textwidth}
			      \begin{center}
				      \begin{tikzpicture}
					      [world/.style={circle, draw, fill=black, inner sep=0pt, minimum size=4pt},
						      views/.style={->, >=Stealth, shorten >=1pt, shorten <=1pt},
						      generic/.style={->, >=Stealth, shorten >=1pt, shorten <=1pt, dashed}]
					      \node[world, label={left:{$x$}}] (x) at (0,0) {};
					      \node[world, label={right:{$y$}}] (y) at (2,0) {};
					      \draw[views] (x) -- (y);
				      \end{tikzpicture}
			      \end{center}
		      \end{minipage}
		      \vspace{8pt}

		\item ($\leftarrow$): va dimostrato che \emph{se $\mathcal{F} \rhd \forall x \exists y (x\mathcal{R}y)$, allora $\mathcal{F} \vDash \Box A \to \Dmd A$}. \\
		      Assumo $\mathcal{F} \rhd \forall x \exists y (x\mathcal{R}y)$. \\
		      Sia $x \in \mathcal{W}$; definisco un $\mathcal{F}$-modello $\mathcal{M}$ tale che: \\

		      \begin{minipage}{0.48\textwidth}
			      $$\vDash_x^{\mathcal{M}}\Box A$$
		      \end{minipage}
		      \begin{minipage}{0.48\textwidth}
			      \begin{center}
				      \begin{tikzpicture}
					      [world/.style={circle, draw, fill=black, inner sep=0pt, minimum size=4pt},
						      relation/.style={->, >=Stealth, shorten >=1pt, shorten <=1pt}]
					      \node[world, label={[font=\Large]right:{$x \atop \Box A$}}] (x) at (0,0) {};
				      \end{tikzpicture}
			      \end{center}
		      \end{minipage}
		      \vspace{8pt}

		      Ho così definito che, in un generico mondo $x$, $\Box A$ è vera. \\
		      Per assunzione so che, preso un qualsiasi mondo di questa struttura, ne esisterà almeno un altro accessibile dal primo (cioé $\forall x \exists y (x\mathcal{R}y)$): \\

		      \begin{center}
			      \begin{tikzpicture}
				      [world/.style={circle, draw, fill=black, inner sep=0pt, minimum size=4pt},
					      views/.style={->, >=Stealth, shorten >=1pt, shorten <=1pt},
					      generic/.style={->, >=Stealth, shorten >=1pt, shorten <=1pt, dashed}]
				      \node[world, label={[font=\Large]left:{$x \atop \Box A$}}] (x) at (0,0) {};
				      \node[world, label={right:{$y$}}] (y) at (2,0) {};
				      \draw[views] (x) -- (y);
			      \end{tikzpicture}
		      \end{center}
		      \vspace{8pt}

		      Dal momento che $\Box A$ è vera in $x$ e $x\mathcal{R}y$, per \hyperlink{defverp}{definizione di verità di una formula in un punto} avrò che:
		      \vspace{8pt}

		      \begin{minipage}{0.48\textwidth}
			      $$\vDash_y^{\mathcal{M}} A$$
		      \end{minipage}
		      \begin{minipage}{0.48\textwidth}
			      \begin{center}
				      \begin{tikzpicture}
					      [world/.style={circle, draw, fill=black, inner sep=0pt, minimum size=4pt},
						      views/.style={->, >=Stealth, shorten >=1pt, shorten <=1pt},
						      generic/.style={->, >=Stealth, shorten >=1pt, shorten <=1pt, dashed}]
					      \node[world, label={[font=\Large]left:{$x \atop \Box A$}}] (x) at (0,0) {};
					      \node[world, label={[font=\Large]right:{$y \atop A$}}] (y) at (2,0) {};
					      \draw[views] (x) -- (y);
				      \end{tikzpicture}
			      \end{center}
		      \end{minipage}
		      \vspace{8pt}

		      Per \hyperlink{defverp}{definizione di verità di una formula in un punto} (cioè $\vDash_x^{\mathcal{M}} \Dmd A \; \text{sse} \; \exists y \in \mathcal{W} (x\mathcal{R}y \;\, \text{e} \; \vDash_y^{\mathcal{M}} A)$) ho che:
		      \vspace{8pt}

		      \begin{minipage}{0.48\textwidth}
			      $$\vDash_x^{\mathcal{M}}\Dmd A$$
		      \end{minipage}
		      \begin{minipage}{0.48\textwidth}
			      \begin{center}
				      \begin{tikzpicture}
					      [world/.style={circle, draw, fill=black, inner sep=0pt, minimum size=4pt},
						      views/.style={->, >=Stealth, shorten >=1pt, shorten <=1pt},
						      generic/.style={->, >=Stealth, shorten >=1pt, shorten <=1pt, dashed}]
					      \node[world, label={[font=\Large]left:{$x \atop \Box A, \: \Dmd A$}}] (x) at (0,0) {};
					      \node[world, label={[font=\Large]right:{$y \atop A$}}] (y) at (2,0) {};
					      \draw[views] (x) -- (y);
				      \end{tikzpicture}
			      \end{center}
		      \end{minipage}
		      \vspace{8pt}

		      Posso quindi concludere:
		      $$\vDash_x^{\mathcal{M}}\Box A \to \Dmd A$$
		      (sia antecedente che conseguente sono veri in $x$).
	\end{enumerate}
\end{dimo}

\subsubsubs{Teorema: $X$ corrisponde alla densità debole}
\begin{theo}
	Lo schema $X$ corrisponde alla densità debole.
	$$\mathcal{F} \vDash \Box \Box A \to \Box A \quad\; \text{sse} \;\quad \mathcal{F} \rhd \forall x \forall y (x\mathcal{R}y \to \exists z(x\mathcal{R}z \land z\mathcal{R}y))$$
\end{theo}
\begin{dimo}
	\phantom{ciao}
	\begin{enumerate}
		\item ($\to$): va dimostrato che \emph{se $\mathcal{F} \vDash \Box \Box A \to \Box A$, allora $\mathcal{F} \rhd \forall x \forall y (x\mathcal{R}y \to \exists z(x\mathcal{R}z \land z\mathcal{R}y))$}. \\
		      Sia $p \in \Phi$; assumo $\mathcal{F} \vDash \Box \Box p \to \Box p$. \\
		      Siano $x, y \in \mathcal{W}$ tali che: \\

		      \begin{minipage}{0.48\textwidth}
			      $$x\mathcal{R}y$$
		      \end{minipage}
		      \begin{minipage}{0.48\textwidth}
			      \begin{center}
				      \begin{tikzpicture}
					      [world/.style={circle, draw, fill=black, inner sep=0pt, minimum size=4pt},
						      views/.style={->, >=Stealth, shorten >=1pt, shorten <=1pt},
						      generic/.style={->, >=Stealth, shorten >=1pt, shorten <=1pt, dashed}]
					      \node[world, label={left:{$x$}}] (x) at (0,0) {};
					      \node[world, label={right:{$y$}}] (y) at (2,0) {};
					      \draw[views] (x) -- (y);
				      \end{tikzpicture}
			      \end{center}
		      \end{minipage}
		      \vspace{8pt}

		      Definisco un $\mathcal{F}$-modello $\mathcal{M}$ tale che:
		      \vspace{8pt}

		      \begin{minipage}{0.48\textwidth}
			      $$I(p) = \{w \in \mathcal{W} : w \neq y\}$$
		      \end{minipage}
		      \begin{minipage}{0.48\textwidth}
			      \begin{center}
				      \begin{tikzpicture}
					      [world/.style={circle, draw, fill=black, inner sep=0pt, minimum size=4pt},
						      views/.style={->, >=Stealth, shorten >=1pt, shorten <=1pt},
						      generic/.style={->, >=Stealth, shorten >=1pt, shorten <=1pt, dashed}]
					      \node[world, label={[font=\Large]left:{$x \atop p$}}] (x) at (0,0) {};
					      \node[world, label={right:{$y$}}] (y) at (2,0) {};
					      \node[world, label={[font=\Large]right:{$w \atop p$}}] (w) at (1,1) {};
					      \draw[views] (x) -- (y);
				      \end{tikzpicture}
			      \end{center}
		      \end{minipage}
		      \vspace{8pt}

		      Ho così definito che, in un generico punto $w$ diverso da $y$ (tra cui $x$), è vera $p$. Avrò, quindi, $\nvDash_y^{\mathcal{M}} p$ e, poichè non è rispettata la \hyperlink{defverp}{definizione di verità di una formula in un punto}, $\nvDash_x^{\mathcal{M}} \Box p$. \\
		      Per assunzione, so che $\vDash^{\mathcal{M}}\Box \Box p \to \Box p$, che equivale a $\nvDash^{\mathcal{M}}\Box p \to \Box \Box p$. Quindi, per $M \! P$:
		      \begin{mathpar}
			      \inferrule*[Right=$M \! P$]{\nvDash_x^{\mathcal{M}} \Box p \\ \nvDash_x^{\mathcal{M}}\Box p \to \Box \Box p}{\nvDash_x^{\mathcal{M}} \Box \Box p}
		      \end{mathpar}
		      Dal momento che non è rispettata la \hyperlink{defverp}{definizione di verità di una formula in un punto}, ne segue che esiste almeno un punto accessibile da $x$ in cui $\Box p$ non è vera (cioè $\exists z (x\mathcal{R}z \; \text{e} \; \nvDash_z^{\mathcal{M}} \Box p)$), da cui, per lo stesso motivo, segue che $\exists v(z\mathcal{R}v \; \text{e} \; \nvDash_v^{\mathcal{M}} p)$. \\
		      Dunque, $v \in I(p)$ (per \hyperlink{defverp}{definizione di verità di una formula in un punto}), da cui $v=y$; posso quindi concludere:

		      \begin{minipage}{0.48\textwidth}
			      $$\exists z(x\mathcal{R}z \land z\mathcal{R}y)$$
		      \end{minipage}
		      \begin{minipage}{0.48\textwidth}
			      \begin{center}
				      \begin{tikzpicture}
					      [world/.style={circle, draw, fill=black, inner sep=0pt, minimum size=4pt},
						      views/.style={->, >=Stealth, shorten >=1pt, shorten <=1pt},
						      generic/.style={->, >=Stealth, shorten >=1pt, shorten <=1pt, dashed}]
					      \node[world, label={[font=\Large]left:{$x \atop p$}}] (x) at (0,0) {};
					      \node[world, label={right:{$y$}}] (y) at (2,0) {};
					      \node[world, label={left:{$z$}}] (z) at (1,1.5) {};
					      \draw[views] (x) -- (y);
					      \draw[views] (z) -- (y);
					      \draw[views] (x) -- (z);
				      \end{tikzpicture}
			      \end{center}
		      \end{minipage}
		      \vspace{8pt}

		\item ($\leftarrow$): va dimostrato che \emph{se $\mathcal{F} \rhd \forall x \forall y (x\mathcal{R}y \to \exists z(x\mathcal{R}z \land z\mathcal{R}y))$, allora $\mathcal{F} \vDash \Box \Box A \to \Box A$}. \\
		      Assumo $\mathcal{F} \rhd \forall x \forall y (x\mathcal{R}y \to \exists z(x\mathcal{R}z \land z\mathcal{R}y))$. \\
		      Sia $x \in \mathcal{W}$; definisco un $\mathcal{F}$-modello $\mathcal{M}$ tale che:  \\

		      \begin{minipage}{0.48\textwidth}
			      $$\vDash_x^{\mathcal{M}} \Box \Box A$$
		      \end{minipage}
		      \begin{minipage}{0.48\textwidth}
			      \begin{center}
				      \begin{tikzpicture}
					      [world/.style={circle, draw, fill=black, inner sep=0pt, minimum size=4pt},
						      views/.style={->, >=Stealth, shorten >=1pt, shorten <=1pt},
						      generic/.style={->, >=Stealth, shorten >=1pt, shorten <=1pt, dashed}]
					      \node[world, label={[font=\Large]left:{$x \atop \Box \Box A$}}] (x) at (0,0) {};
				      \end{tikzpicture}
			      \end{center}
		      \end{minipage}
		      \vspace{8pt}

		      Devo mostrare che anche $\Box A$ è vero in $x$; per \hyperlink{defverp}{definizione di verità di una formula in un punto}, ciò equivale a dimostrare che $\forall y \in \mathcal{W} (\text{se} \; x\mathcal{R}y  \text{, allora} \; \vDash_y^{\mathcal{M}} A)$. \\
		      Ho due casi:
		      \begin{itemize}
			      \item $\nexists y \in \mathcal{W} (x\mathcal{R}y)$: l'antecedente dell'implicazione è falso, dunque l'implicazione è vera; ho quindi che $\vDash_x^{\mathcal{M}} \Box A$. Di conseguenza, posso concludere:
			            $$\vDash_x^{\mathcal{M}} \Box \Box A \to \Box A$$
			            (sia antecedente che conseguente sono veri in $x$).
			      \item $\exists y \in \mathcal{W} (x\mathcal{R}y)$: per assunzione, so che: \\
			            \begin{minipage}{0.48\textwidth}
				            $$x\mathcal{R}y \to \exists z(x\mathcal{R}z \land z\mathcal{R}y)$$
			            \end{minipage}
			            \begin{minipage}{0.48\textwidth}
				            \begin{center}
					            \begin{tikzpicture}
						            [world/.style={circle, draw, fill=black, inner sep=0pt, minimum size=4pt},
							            views/.style={->, >=Stealth, shorten >=1pt, shorten <=1pt},
							            generic/.style={->, >=Stealth, shorten >=1pt, shorten <=1pt, dashed}]
						            \node[world, label={[font=\Large]left:{$x \atop \Box \Box A$}}] (x) at (0,0) {};
						            \node[world, label={right:{$y$}}] (y) at (2,0) {};
						            \node[world, label={right:{$z$}}] (z) at (1,1.5) {};
						            \draw[views] (x) -- (y);
						            \draw[views] (x) -- (z);
						            \draw[views] (z) -- (y);
					            \end{tikzpicture}
				            \end{center}
			            \end{minipage}
			            \vspace{0pt}

			            Per \hyperlink{defverp}{definizione di verità di una formula in un punto} di $\vDash_x^{\mathcal{M}} \Box \Box A$, ho che $\vDash_z^{\mathcal{M}} \Box A$, da cui $\vDash_y^{\mathcal{M}}A$:

			            \begin{center}
				            \begin{tikzpicture}
					            [world/.style={circle, draw, fill=black, inner sep=0pt, minimum size=4pt},
						            views/.style={->, >=Stealth, shorten >=1pt, shorten <=1pt},
						            generic/.style={->, >=Stealth, shorten >=1pt, shorten <=1pt, dashed}]
					            \node[world, label={[font=\Large]left:{$x \atop \Box \Box A$}}] (x) at (0,0) {};
					            \node[world, label={[font=\Large]right:{$y \atop A$}}] (y) at (2,0) {};
					            \node[world, label={[font=\Large]right:{$z \atop \Box A$}}] (z) at (1,1.5) {};
					            \draw[views] (x) -- (y);
					            \draw[views] (x) -- (z);
					            \draw[views] (z) -- (y);
				            \end{tikzpicture}
			            \end{center}
			            \vspace{8pt}

			            Dal momento che ho $x\mathcal{R}y$ e $\vDash_y^{\mathcal{M}}A$, ho che: \\
			            \begin{minipage}{0.48\textwidth}
				            $$\vDash_x^{\mathcal{M}}\Box A$$
			            \end{minipage}
			            \begin{minipage}{0.48\textwidth}
				            \begin{center}
					            \begin{tikzpicture}
						            [world/.style={circle, draw, fill=black, inner sep=0pt, minimum size=4pt},
							            views/.style={->, >=Stealth, shorten >=1pt, shorten <=1pt},
							            generic/.style={->, >=Stealth, shorten >=1pt, shorten <=1pt, dashed}]
						            \node[world, label={[font=\Large]left:{$x \atop \Box \Box A, \: \Box A$}}] (x) at (0,0) {};
						            \node[world, label={[font=\Large]right:{$y \atop A$}}] (y) at (2,0) {};
						            \node[world, label={[font=\Large]right:{$z \atop \Box A$}}] (z) at (1,1.5) {};
						            \draw[views] (x) -- (y);
						            \draw[views] (x) -- (z);
						            \draw[views] (z) -- (y);
					            \end{tikzpicture}
				            \end{center}
			            \end{minipage}
			            \vspace{8pt}

			            Posso quindi concludere:
			            $$\vDash_x^{\mathcal{M}}\Box \Box A \to \Box A$$
			            (sia antecedente che conseguente sono veri in $x$).
		      \end{itemize}
	\end{enumerate}
\end{dimo}

\subsubsubs{Teorema: $B$ corrisponde alla simmetria}
\begin{theo}
	Lo schema $B$ corrisponde alla simmetria.
	$$\mathcal{F} \vDash A \to \Box \Dmd A \quad\; \text{sse} \;\quad \mathcal{F} \rhd \forall x \forall y (x\mathcal{R}y \to y\mathcal{R}x)$$
\end{theo}
\begin{dimo}
	\phantom{ciao}
	\begin{enumerate}
		\item ($\to$): va dimostrato che \emph{se $\mathcal{F} \vDash A \to \Box \Dmd A$, allora $\mathcal{F} \rhd \forall x \forall y (x\mathcal{R}y \to y\mathcal{R}x)$}. \\
		      Sia $p \in \Phi$: assumo $\mathcal{F} \vDash p \to \Box \Dmd p$. \\
		      Siano $x, y \in \mathcal{W}$ tale che:\\

		      \begin{minipage}{0.48\textwidth}
			      $$x\mathcal{R}y$$
		      \end{minipage}
		      \begin{minipage}{0.48\textwidth}
			      \begin{center}
				      \begin{tikzpicture}
					      [world/.style={circle, draw, fill=black, inner sep=0pt, minimum size=4pt},
						      views/.style={->, >=Stealth, shorten >=1pt, shorten <=1pt},
						      generic/.style={->, >=Stealth, shorten >=1pt, shorten <=1pt, dashed}]
					      \node[world, label={left:{$x$}}] (x) at (0,0) {};
					      \node[world, label={right:{$y$}}] (y) at (2,0) {};
					      \draw[views] (x) -- (y);
				      \end{tikzpicture}
			      \end{center}
		      \end{minipage}
		      \vspace{8pt}

		      Definisco un $\mathcal{F}$-modello $\mathcal{M}$ tale che: \\

		      \begin{minipage}{0.48\textwidth}
			      $$I(p) = \{x\}$$
		      \end{minipage}
		      \begin{minipage}{0.48\textwidth}
			      \begin{center}
				      \begin{tikzpicture}
					      [world/.style={circle, draw, fill=black, inner sep=0pt, minimum size=4pt},
						      views/.style={->, >=Stealth, shorten >=1pt, shorten <=1pt},
						      generic/.style={->, >=Stealth, shorten >=1pt, shorten <=1pt, dashed}]
					      \node[world, label={[font=\Large]left:{$x \atop p$}}] (x) at (0,0) {};
					      \node[world, label={right:{$y$}}] (y) at (2,0) {};
					      \draw[views] (x) -- (y);
				      \end{tikzpicture}
			      \end{center}
		      \end{minipage}
		      \vspace{8pt}

		      Per assunzione, ho che $\vDash_x^{\mathcal{M}} p \to \Box\Dmd p$; quindi, per $M \! P$:

		      \begin{minipage}{0.48\textwidth}
			      \begin{mathpar}
				      \inferrule*[Right=$M \! P$]{\vDash_x^{\mathcal{M}}p \\ \vDash_x^{\mathcal{M}} p \to \Box \Dmd p}{\vDash_x^{\mathcal{M}} \Box\Dmd p}
			      \end{mathpar}
		      \end{minipage}
		      \begin{minipage}{0.48\textwidth}
			      \begin{center}
				      \begin{tikzpicture}
					      [world/.style={circle, draw, fill=black, inner sep=0pt, minimum size=4pt},
						      views/.style={->, >=Stealth, shorten >=1pt, shorten <=1pt},
						      generic/.style={->, >=Stealth, shorten >=1pt, shorten <=1pt, dashed}]
					      \node[world, label={[font=\Large]left:{$x \atop p, \: \Box \Dmd p$}}] (x) at (0,0) {};
					      \node[world, label={right:{$y$}}] (y) at (2,0) {};
					      \draw[views] (x) -- (y);
				      \end{tikzpicture}
			      \end{center}
		      \end{minipage}
		      \vspace{8pt}

		      Quindi, per \hyperlink{defverp}{definizione di verità di una formula in un punto}, ho che $\Dmd p$ dev'essere vera in tutti i mondi accessibili da $x$, in particolare $y$: \\

		      \begin{center}
			      \begin{tikzpicture}
				      [world/.style={circle, draw, fill=black, inner sep=0pt, minimum size=4pt},
					      views/.style={->, >=Stealth, shorten >=1pt, shorten <=1pt},
					      generic/.style={->, >=Stealth, shorten >=1pt, shorten <=1pt, dashed}]
				      \node[world, label={[font=\Large]left:{$x \atop p, \: \Box \Dmd p$}}] (x) at (0,0) {};
				      \node[world, label={[font=\Large]right:{$y \atop \Dmd p$}}] (y) at (2,0) {};
				      \draw[views] (x) -- (y);
			      \end{tikzpicture}
		      \end{center}
		      \vspace{8pt}

		      Dal momento che ho $\vDash_y^{\mathcal{M}} \Dmd p$ e $\vDash_x^{\mathcal{M}} p$, per \hyperlink{defverp}{definizione di verità di una formula in un punto} posso concludere che: \\

		      \begin{minipage}{0.48\textwidth}
			      $$y\mathcal{R}x$$
		      \end{minipage}
		      \begin{minipage}{0.48\textwidth}
			      \begin{center}
				      \begin{tikzpicture}
					      [world/.style={circle, draw, fill=black, inner sep=0pt, minimum size=4pt},
						      views/.style={->, >=Stealth, shorten >=1pt, shorten <=1pt},
						      generic/.style={->, >=Stealth, shorten >=1pt, shorten <=1pt, dashed}]
					      \node[world, label={[font=\Large]left:{$x \atop p, \: \Box \Dmd p$}}] (x) at (0,0) {};
					      \node[world, label={[font=\Large]right:{$y \atop \Dmd p$}}] (y) at (2,0) {};
					      \draw[views] (x) to [bend left] (y);
					      \draw[views] (y) to [bend left] (x);
				      \end{tikzpicture}
			      \end{center}
		      \end{minipage}
		      \vspace{8pt}

		\item ($\leftarrow$): va dimostrato che \emph{se $\mathcal{F} \rhd \forall x \forall y (x\mathcal{R}y \to y\mathcal{R}x)$, allora $\mathcal{F} \vDash A \to \Box \Dmd A$}. \\
		      Assumo $\mathcal{F} \rhd \forall x \forall y (x\mathcal{R}y \to y\mathcal{R}x)$. \\
		      Sia $x \in \mathcal{W}$; definisco un $\mathcal{F}$-modello $\mathcal{M}$ tale che:\\

		      \begin{minipage}{0.48\textwidth}
			      $$\vDash_x^{\mathcal{M}} A$$
		      \end{minipage}
		      \begin{minipage}{0.48\textwidth}
			      \begin{center}
				      \begin{tikzpicture}
					      [world/.style={circle, draw, fill=black, inner sep=0pt, minimum size=4pt},
						      views/.style={->, >=Stealth, shorten >=1pt, shorten <=1pt},
						      generic/.style={->, >=Stealth, shorten >=1pt, shorten <=1pt, dashed}]
					      \node[world, label={[font=\Large]left:{$x \atop A$}}] (x) at (0,0) {};
				      \end{tikzpicture}
			      \end{center}
		      \end{minipage}
		      \vspace{8pt}

		      Devo mostrare che anche $\Box \Dmd A$ è vero in $x$; per \hyperlink{defverp}{definizione di verità di una formula in un punto}, ciò equivale a dimostrare che $\forall y \in \mathcal{W} (\text{se} \; x\mathcal{R}y  \text{, allora} \; \vDash_y^{\mathcal{M}} \Dmd A)$, che a sua volta equivale a dimostrare che $\exists w \in \mathcal{W} (y\mathcal{R}w \;\, \text{e} \; \vDash_w^{\mathcal{M}} A)$.\\
		      Ho due casi:
		      \begin{itemize}
			      \item $\nexists y \in \mathcal{W} (x\mathcal{R}y)$: l'antecedente dell'implicazione è falso, dunque l'implicazione è vera; ho quindi che $\vDash_x^{\mathcal{M}} \Box \Dmd A$. Di conseguenza, posso concludere:
			            $$\vDash_x^{\mathcal{M}} A \to \Box \Dmd A$$
			            (sia antecedente che conseguente sono veri in $x$).
			      \item $\exists y \in \mathcal{W} (x\mathcal{R}y)$: per assunzione, so che qualunque punto accessibile da $x$ vedrà $x$ a sua volta:

			            \begin{minipage}{0.48\textwidth}
				            $$x\mathcal{R}y \to y\mathcal{R}x$$
			            \end{minipage}
			            \begin{minipage}{0.48\textwidth}
				            \begin{center}
					            \begin{tikzpicture}
						            [world/.style={circle, draw, fill=black, inner sep=0pt, minimum size=4pt},
							            views/.style={->, >=Stealth, shorten >=1pt, shorten <=1pt},
							            generic/.style={->, >=Stealth, shorten >=1pt, shorten <=1pt, dashed}]
						            \node[world, label={left:{$x$}}] (x) at (0,0) {};
						            \node[world, label={right:{$y$}}] (y) at (2,0) {};
						            \draw[views] (x) to [bend left] (y);
						            \draw[views] (y) to [bend left] (x);
					            \end{tikzpicture}
				            \end{center}
			            \end{minipage}
			            \vspace{0pt}

			            Dal momento che ho $y\mathcal{R}x$ e $\vDash_x^{\mathcal{M}} A$, ho che: \\

			            \begin{minipage}{0.48\textwidth}
				            $$\vDash_y^{\mathcal{M}} \Dmd A$$
			            \end{minipage}
			            \begin{minipage}{0.48\textwidth}
				            \begin{center}
					            \begin{tikzpicture}
						            [world/.style={circle, draw, fill=black, inner sep=0pt, minimum size=4pt},
							            views/.style={->, >=Stealth, shorten >=1pt, shorten <=1pt},
							            generic/.style={->, >=Stealth, shorten >=1pt, shorten <=1pt, dashed}]
						            \node[world, label={[font=\Large]left:{$x \atop A$}}] (x) at (0,0) {};
						            \node[world, label={[font=\Large]right:{$y \atop \Dmd A$}}] (y) at (2,0) {};
						            \draw[views] (x) to [bend left] (y);
						            \draw[views] (y) to [bend left] (x);
					            \end{tikzpicture}
				            \end{center}
			            \end{minipage}
			            \vspace{8pt}

			            Poiché in tutti i punti accessibili da $x$, $\Dmd A$ è vera, per \hyperlink{defverp}{definizione di verità di una formula in un punto} ho che:

			            \begin{minipage}{0.48\textwidth}
				            $$\vDash_x^{\mathcal{M}} \Box \Dmd A$$
			            \end{minipage}
			            \begin{minipage}{0.48\textwidth}
				            \begin{center}
					            \begin{tikzpicture}
						            [world/.style={circle, draw, fill=black, inner sep=0pt, minimum size=4pt},
							            views/.style={->, >=Stealth, shorten >=1pt, shorten <=1pt},
							            generic/.style={->, >=Stealth, shorten >=1pt, shorten <=1pt, dashed}]
						            \node[world, label={[font=\Large]left:{$x \atop A, \: \Box \Dmd A$}}] (x) at (0,0) {};
						            \node[world, label={[font=\Large]right:{$y \atop \Dmd A$}}] (y) at (2,0) {};
						            \draw[views] (x) to [bend left] (y);
						            \draw[views] (y) to [bend left] (x);
					            \end{tikzpicture}
				            \end{center}
			            \end{minipage}
			            \vspace{0pt}

			            Quindi, posso concludere:
			            $$\vDash_x^{\mathcal{M}} A \to \Box \Dmd A$$
			            (sia antecedente che conseguente sono veri in $x$).
		      \end{itemize}
	\end{enumerate}
\end{dimo}

\subsubsubs{Teorema: $5$ corrisponde alla proprietà euclidea}
\begin{theo}
	Lo schema $5$ corrisponde alla proprietà euclidea.
	$$\mathcal{F} \vDash \Dmd A \to \Box \Dmd A \quad\; \text{sse} \;\quad \mathcal{F} \rhd \forall x \forall y \forall z (x\mathcal{R}y \land x\mathcal{R}z \to y\mathcal{R}z)$$
\end{theo}
\begin{dimo}
	\phantom{ciao}
	\begin{enumerate}
		\item ($\to$): va dimostrato che \emph{se $\mathcal{F} \vDash \Dmd A \to \Box \Dmd A$, allora $\mathcal{F} \rhd \forall x \forall y \forall z (x\mathcal{R}y \land x\mathcal{R}z \to y\mathcal{R}z)$}. \\
		      Sia $p \in \Phi$; assumo $\mathcal{F} \vDash \Dmd p \to \Box \Dmd p$. \\
		      Siano $x, y, z \in \mathcal{W}$ tali che $x\mathcal{R}y \land x\mathcal{R}z$:
		      \vspace{8pt}

		      \begin{center}
			      \begin{tikzpicture}
				      [world/.style={circle, draw, fill=black, inner sep=0pt, minimum size=4pt},
					      views/.style={->, >=Stealth, shorten >=1pt, shorten <=1pt},
					      generic/.style={->, >=Stealth, shorten >=1pt, shorten <=1pt, dashed}]
				      \node[world, label={right:{$x$}}] (x) at (0,0) {};
				      \node[world, label={right:{$y$}}] (y) at (1,1.5) {};
				      \node[world, label={left:{$z$}}] (z) at (-1,1.5) {};
				      \draw[views] (x) -- (y);
				      \draw[views] (x) -- (z);
			      \end{tikzpicture}
		      \end{center}

		      Definisco un $\mathcal{F}$-modello $\mathcal{M}$ tale che:

		      \begin{minipage}{0.48\textwidth}
			      $$I(p) = \{z\}$$
		      \end{minipage}
		      \begin{minipage}{0.48\textwidth}
			      \begin{center}
				      \begin{tikzpicture}
					      [world/.style={circle, draw, fill=black, inner sep=0pt, minimum size=4pt},
						      views/.style={->, >=Stealth, shorten >=1pt, shorten <=1pt},
						      generic/.style={->, >=Stealth, shorten >=1pt, shorten <=1pt, dashed}]
					      \node[world, label={right:{$x$}}] (x) at (0,0) {};
					      \node[world, label={right:{$y$}}] (y) at (1,1.5) {};
					      \node[world, label={[font=\Large]left:{$z \atop p$}}] (z) at (-1,1.5) {};
					      \draw[views] (x) -- (y);
					      \draw[views] (x) -- (z);
				      \end{tikzpicture}
			      \end{center}
		      \end{minipage}
		      \vspace{0pt}

		      Dal momento che esiste almeno un mondo accessibile da $x$ in cui $p$ è vera, per \hyperlink{defverp}{definizione di verità di una formula in un punto} ho che:

		      \begin{minipage}{0.48\textwidth}
			      $$\vDash_x^{\mathcal{M}} \Dmd p$$
		      \end{minipage}
		      \begin{minipage}{0.48\textwidth}
			      \begin{center}
				      \begin{tikzpicture}
					      [world/.style={circle, draw, fill=black, inner sep=0pt, minimum size=4pt},
						      views/.style={->, >=Stealth, shorten >=1pt, shorten <=1pt},
						      generic/.style={->, >=Stealth, shorten >=1pt, shorten <=1pt, dashed}]
					      \node[world, label={[font=\Large]right:{$x \atop \Dmd p$}}] (x) at (0,0) {};
					      \node[world, label={right:{$y$}}] (y) at (1,1.5) {};
					      \node[world, label={[font=\Large]left:{$z \atop p$}}] (z) at (-1,1.5) {};
					      \draw[views] (x) -- (y);
					      \draw[views] (x) -- (z);
				      \end{tikzpicture}
			      \end{center}
		      \end{minipage}
		      \vspace{0pt}

		      Per assunzione, ho che $\vDash_x^{\mathcal{M}} \Dmd p \to \Box\Dmd p$; quindi, per $M \! P$:

		      \begin{minipage}{0.48\textwidth}
			      \begin{mathpar}
				      \inferrule*[Right=$M \! P$]{\vDash_x^{\mathcal{M}} \Dmd p \\ \vDash_x^{\mathcal{M}} \Dmd p \to \Box \Dmd p}{\vDash_x^{\mathcal{M}} \Box\Dmd p}
			      \end{mathpar}
		      \end{minipage}
		      \begin{minipage}{0.48\textwidth}
			      \begin{center}
				      \begin{tikzpicture}
					      [world/.style={circle, draw, fill=black, inner sep=0pt, minimum size=4pt},
						      views/.style={->, >=Stealth, shorten >=1pt, shorten <=1pt},
						      generic/.style={->, >=Stealth, shorten >=1pt, shorten <=1pt, dashed}]
					      \node[world, label={[font=\Large]right:{$x \atop \Dmd p, \: \Box \Dmd p$}}] (x) at (0,0) {};
					      \node[world, label={right:{$y$}}] (y) at (1,1.5) {};
					      \node[world, label={[font=\Large]left:{$z \atop p$}}] (z) at (-1,1.5) {};
					      \draw[views] (x) -- (y);
					      \draw[views] (x) -- (z);
				      \end{tikzpicture}
			      \end{center}
		      \end{minipage}
		      \vspace{8pt}

		      Quindi $\Dmd p$ è vera in tutti i mondi accessibili da $x$ (per \hyperlink{defverp}{definizione di verità di una formula in un punto}), in particolare in $y$:

		      \begin{minipage}{0.48\textwidth}
			      $$\vDash_y^{\mathcal{M}} \Dmd p$$
		      \end{minipage}
		      \begin{minipage}{0.48\textwidth}
			      \begin{center}
				      \begin{tikzpicture}
					      [world/.style={circle, draw, fill=black, inner sep=0pt, minimum size=4pt},
						      views/.style={->, >=Stealth, shorten >=1pt, shorten <=1pt},
						      generic/.style={->, >=Stealth, shorten >=1pt, shorten <=1pt, dashed}]
					      \node[world, label={[font=\Large]right:{$x \atop \Dmd p, \: \Box \Dmd p$}}] (x) at (0,0) {};
					      \node[world, label={[font=\Large]right:{$y \atop \Dmd p$}}] (y) at (1,1.5) {};
					      \node[world, label={[font=\Large]left:{$z \atop p$}}] (z) at (-1,1.5) {};
					      \draw[views] (x) -- (y);
					      \draw[views] (x) -- (z);
				      \end{tikzpicture}
			      \end{center}
		      \end{minipage}
		      \vspace{0pt}

		      Dal momento che ho $\vDash_y^{\mathcal{M}} \Dmd p$ e $\vDash_z^{\mathcal{M}} p$, per \hyperlink{defverp}{definizione di verità di una formula in un punto} posso concludere:
		      \vspace{8pt}

		      \begin{minipage}{0.48\textwidth}
			      $$y\mathcal{R}z$$
		      \end{minipage}
		      \begin{minipage}{0.48\textwidth}
			      \begin{center}
				      \begin{tikzpicture}
					      [world/.style={circle, draw, fill=black, inner sep=0pt, minimum size=4pt},
						      views/.style={->, >=Stealth, shorten >=1pt, shorten <=1pt},
						      generic/.style={->, >=Stealth, shorten >=1pt, shorten <=1pt, dashed}]
					      \node[world, label={[font=\Large]right:{$x \atop \Dmd p, \: \Box \Dmd p$}}] (x) at (0,0) {};
					      \node[world, label={[font=\Large]right:{$y \atop \Dmd p$}}] (y) at (1,1.5) {};
					      \node[world, label={[font=\Large]left:{$z \atop p$}}] (z) at (-1,1.5) {};
					      \draw[views] (x) -- (y);
					      \draw[views] (x) -- (z);
					      \draw[views] (y) -- (z);
				      \end{tikzpicture}
			      \end{center}
		      \end{minipage}
		      \vspace{8pt}

		\item ($\leftarrow$): va dimostrato che \emph{se $\mathcal{F} \rhd \forall x \forall y \forall z (x\mathcal{R}y \land x\mathcal{R}z \to y\mathcal{R}z)$, allora $\mathcal{F} \vDash \Dmd A \to \Box \Dmd A$}. \\
		      Assumo $\mathcal{F} \rhd \forall x \forall y \forall z (x\mathcal{R}y \land x\mathcal{R}z \to y\mathcal{R}z)$. \\
		      Sia $x \in \mathcal{W}$; definisco un $\mathcal{F}$-modello $\mathcal{M}$ tale che: \\

		      \begin{minipage}{0.48\textwidth}
			      $$\vDash_x^{\mathcal{M}} \Dmd A$$
		      \end{minipage}
		      \begin{minipage}{0.48\textwidth}
			      \begin{center}
				      \begin{tikzpicture}
					      [world/.style={circle, draw, fill=black, inner sep=0pt, minimum size=4pt},
						      relation/.style={->, >=Stealth, shorten >=1pt, shorten <=1pt}]
					      \node[world, label={[font=\Large]right:{$x \atop \Dmd A$}}] (x) at (0,0) {};
				      \end{tikzpicture}
			      \end{center}
		      \end{minipage}
		      \vspace{8pt}

		      Da cui, per \hyperlink{defverp}{definizione di verità di una formula in un punto}, so che esiste almeno un punto accessibile da $x$ in cui $A$ è vera: \\

		      \begin{minipage}{0.48\textwidth}
			      $$\exists z (x\mathcal{R}z \; \text{e} \; \vDash_z^{\mathcal{M}} A)$$
		      \end{minipage}
		      \begin{minipage}{0.48\textwidth}
			      \begin{center}
				      \begin{tikzpicture}
					      [world/.style={circle, draw, fill=black, inner sep=0pt, minimum size=4pt},
						      views/.style={->, >=Stealth, shorten >=1pt, shorten <=1pt},
						      generic/.style={->, >=Stealth, shorten >=1pt, shorten <=1pt, dashed}]
					      \node[world, label={[font=\Large]left:{$x \atop \Dmd A$}}] (x) at (0,0) {};
					      \node[world, label={[font=\Large]right:{$z \atop A$}}] (z) at (2,0) {};
					      \draw[views] (x) -- (z);
				      \end{tikzpicture}
			      \end{center}
			      \vspace{8pt}

		      \end{minipage}
		      \vspace{8pt}

		      Devo mostrare che anche $\Box \Dmd A$ è vera in $x$; per \hyperlink{defverp}{definizione di verità di una formula in un punto}, ciò equivale a dimostrare che $\forall y \in \mathcal{W} (\text{se} \; x\mathcal{R}y  \text{, allora} \; \vDash_y^{\mathcal{M}} \Dmd A)$, che a sua volta equivale a dimostrare che $\exists w \in \mathcal{W} (y\mathcal{R}w \;\, \text{e} \; \vDash_w^{\mathcal{M}} A)$.\\
		      Per assunzione so che, comunque presi due punti accessibili da $x$, il primo vedrà il secondo; un caso particolare di questa proprietà è quello in cui i due punti sono lo stesso punto, in questo caso $z$:

		      \begin{minipage}{0.48\textwidth}
			      $$x\mathcal{R}z \land x\mathcal{R}z \to z\mathcal{R}z$$
		      \end{minipage}
		      \begin{minipage}{0.48\textwidth}
			      \begin{center}
				      \begin{tikzpicture}
					      [world/.style={circle, draw, fill=black, inner sep=0pt, minimum size=4pt},
						      views/.style={->, >=Stealth, shorten >=1pt, shorten <=1pt},
						      generic/.style={->, >=Stealth, shorten >=1pt, shorten <=1pt, dashed}]
					      \node[world, label={[font=\Large]left:{$x \atop \Dmd A$}}] (x) at (0,0) {};
					      \node[world, label={[font=\Large]right:{$\phantom{A.} z \atop \phantom{A.} A$}}] (z) at (2,0) {};
					      \draw[views] (x) -- (z);
					      \draw[views] (z) to [out=45, in=-45, looseness=12, min distance=1cm] (z); % auto-relazione
				      \end{tikzpicture}
			      \end{center}
		      \end{minipage}
		      \vspace{8pt}

		      Dati $z\mathcal{R}z$ e $\vDash_z^{\mathcal{M}} A$, per \hyperlink{defverp}{definizione di verità di una formula in un punto} so che: \\

		      \begin{minipage}{0.48\textwidth}
			      $$\vDash_z^{\mathcal{M}} \Dmd A$$
		      \end{minipage}
		      \begin{minipage}{0.48\textwidth}
			      \begin{center}
				      \begin{tikzpicture}
					      [world/.style={circle, draw, fill=black, inner sep=0pt, minimum size=4pt},
						      views/.style={->, >=Stealth, shorten >=1pt, shorten <=1pt},
						      generic/.style={->, >=Stealth, shorten >=1pt, shorten <=1pt, dashed}]
					      \node[world, label={[font=\Large]left:{$x \atop \Dmd A$}}] (x) at (0,0) {};
					      \node[world, label={[font=\Large]right:{$\phantom{A.} z \atop \phantom{A.} A, \: \Dmd A$}}] (z) at (2,0) {};
					      \draw[views] (x) -- (z);
					      \draw[views] (z) to [out=45, in=-45, looseness=12, min distance=1cm] (z); % auto-relazione
				      \end{tikzpicture}
			      \end{center}
		      \end{minipage}
		      \vspace{8pt}

		      A questo punto, ho due casi:
		      \begin{itemize}
			      \item $\nexists y \in \mathcal{W} (x\mathcal{R}y)$: ho che, in tutti i mondi accessibili da $x$, $\Dmd A$ è vera ($x\mathcal{R}z \; \text{e} \; \vDash_z^{\mathcal{M}} \Dmd A$); da cui $\vDash_x^{\mathcal{M}} \Box \Dmd A$. Quindi, posso concludere:
			            $$\vDash_x^{\mathcal{M}} \Dmd A \to \Box \Dmd A$$
			            (sia antecedente che conseguente sono veri in $x$).
			      \item $\exists y \in \mathcal{W} (x\mathcal{R}y)$: per assunzione so che:

			            \begin{minipage}{0.48\textwidth}
				            $$x\mathcal{R}y \land x\mathcal{R}z \to y\mathcal{R}z$$
			            \end{minipage}
			            \begin{minipage}{0.48\textwidth}
				            \begin{center}
					            \begin{tikzpicture}
						            [world/.style={circle, draw, fill=black, inner sep=0pt, minimum size=4pt},
							            views/.style={->, >=Stealth, shorten >=1pt, shorten <=1pt},
							            generic/.style={->, >=Stealth, shorten >=1pt, shorten <=1pt, dashed}]
						            \node[world, label={[font=\Large]left:{$x \atop \Dmd A$}}] (x) at (0,0) {};
						            \node[world, label={[font=\Large]right:{$\phantom{A.} z \atop \phantom{A.} A, \: \Dmd A$}}] (z) at (2,0) {};
						            \node[world, label={left:{$y$}}] (y) at (1,1.5) {};
						            \draw[views] (x) -- (z);
						            \draw[views] (z) to [out=45, in=-45, looseness=12, min distance=1cm] (z); % auto-relazione
						            \draw[views] (x) -- (y);
						            \draw[views] (y) -- (z);
					            \end{tikzpicture}
				            \end{center}
			            \end{minipage}

			            Dati $y\mathcal{R}z$ e $\vDash_z^{\mathcal{M}} A$, per \hyperlink{defverp}{definizione di verità di una formula in un punto} so che:
			            \vspace{8pt}

			            \begin{minipage}{0.48\textwidth}
				            $$\vDash_y^{\mathcal{M}} \Dmd A$$
			            \end{minipage}
			            \begin{minipage}{0.48\textwidth}
				            \begin{center}
					            \begin{tikzpicture}
						            [world/.style={circle, draw, fill=black, inner sep=0pt, minimum size=4pt},
							            views/.style={->, >=Stealth, shorten >=1pt, shorten <=1pt},
							            generic/.style={->, >=Stealth, shorten >=1pt, shorten <=1pt, dashed}]
						            \node[world, label={[font=\Large]left:{$x \atop \Dmd A$}}] (x) at (0,0) {};
						            \node[world, label={[font=\Large]right:{$\phantom{A.} z \atop \phantom{A.} A, \: \Dmd A$}}] (z) at (2,0) {};
						            \node[world, label={[font=\Large]left:{$y \atop \Dmd A$}}] (y) at (1,1.5) {};
						            \draw[views] (x) -- (z);
						            \draw[views] (z) to [out=45, in=-45, looseness=12, min distance=1cm] (z); % auto-relazione
						            \draw[views] (x) -- (y);
						            \draw[views] (y) -- (z);
					            \end{tikzpicture}
				            \end{center}
			            \end{minipage}

			            Quindi, in tutti i punti accessibili da $x$, è vero $\Dmd A$; per \hyperlink{defverp}{definizione di verità di una formula in un punto}, ho che:

			            \begin{minipage}{0.48\textwidth}
				            $$\vDash_x^{\mathcal{M}} \Box \Dmd A$$
			            \end{minipage}
			            \begin{minipage}{0.48\textwidth}
				            \begin{center}
					            \begin{tikzpicture}
						            [world/.style={circle, draw, fill=black, inner sep=0pt, minimum size=4pt},
							            views/.style={->, >=Stealth, shorten >=1pt, shorten <=1pt},
							            generic/.style={->, >=Stealth, shorten >=1pt, shorten <=1pt, dashed}]
						            \node[world, label={[font=\Large]left:{$x \atop \Dmd A, \: \Box \Dmd A$}}] (x) at (0,0) {};
						            \node[world, label={[font=\Large]right:{$\phantom{A.} z \atop \phantom{A.} A, \: \Dmd A$}}] (z) at (2,0) {};
						            \node[world, label={[font=\Large]left:{$y \atop \Dmd A$}}] (y) at (1,1.5) {};
						            \draw[views] (x) -- (z);
						            \draw[views] (z) to [out=45, in=-45, looseness=12, min distance=1cm] (z); % auto-relazione
						            \draw[views] (x) -- (y);
						            \draw[views] (y) -- (z);
					            \end{tikzpicture}
				            \end{center}
			            \end{minipage}

			            Posso quindi concludere:
			            $$\vDash_x^{\mathcal{M}} \Dmd A \to \Box \Dmd A$$
			            (sia antecedente che conseguente sono veri in $x$).
		      \end{itemize}
	\end{enumerate}
\end{dimo}

\subsubsubs{Teorema: $2$ corrisponde alla convergenza debole}
\begin{theo}
	Lo schema $2$ corrisponde alla convergenza debole.
	$$\mathcal{F} \vDash \Dmd \Box A \to \Box \Dmd A \quad\; \text{sse} \;\quad \mathcal{F} \rhd \forall x \forall y \forall z (x\mathcal{R}y \land x\mathcal{R}z \to \exists w(y\mathcal{R}w \land z\mathcal{R}w))$$
\end{theo}
\begin{dimo}
	\phantom{ciao}
	\begin{enumerate}
		\item ($\to$): va dimostrato che \emph{se $\mathcal{F} \vDash \Dmd \Box A \to \Box \Dmd A$, allora $\mathcal{F} \rhd \forall x \forall y \forall z (x\mathcal{R}y \land x\mathcal{R}z \to \exists w(y\mathcal{R}w \land z\mathcal{R}w))$}. \\
		      Assumo $\mathcal{F} \vDash \Dmd \Box A \to \Box \Dmd A$. \\
		      Siano $p \in \Phi$, $x, y, z \in \mathcal{W}$ tali che $x\mathcal{R}y \land x\mathcal{R}z$:
		      \vspace{8pt}

		      \begin{center}
			      \begin{tikzpicture}
				      [world/.style={circle, draw, fill=black, inner sep=0pt, minimum size=4pt},
					      views/.style={->, >=Stealth, shorten >=1pt, shorten <=1pt},
					      generic/.style={->, >=Stealth, shorten >=1pt, shorten <=1pt, dashed}]
				      \node[world, label={right:{$x$}}] (x) at (0,0) {};
				      \node[world, label={right:{$y$}}] (y) at (1,1.5) {};
				      \node[world, label={left:{$z$}}] (z) at (-1,1.5) {};
				      \draw[views] (x) -- (y);
				      \draw[views] (x) -- (z);
			      \end{tikzpicture}
		      \end{center}

		      Definisco un $\mathcal{F}$-modello $\mathcal{M}$ tale che:

		      \begin{minipage}{0.48\textwidth}
			      $$I(p) = \{w \in \mathcal{W} : y\mathcal{R}w\}$$
		      \end{minipage}
		      \begin{minipage}{0.48\textwidth}
			      \begin{center}
				      \begin{tikzpicture}
					      [world/.style={circle, draw, fill=black, inner sep=0pt, minimum size=4pt},
						      views/.style={->, >=Stealth, shorten >=1pt, shorten <=1pt},
						      generic/.style={->, >=Stealth, shorten >=1pt, shorten <=1pt, dashed}]
					      \node[world, label={right:{$x$}}] (x) at (0,0) {};
					      \node[world, label={right:{$y$}}] (y) at (1,1) {};
					      \node[world, label={left:{$z$}}] (z) at (-1,1) {};
					      \node[world, label={[font=\Large]right:{$w \atop p$}}] (w) at (0,2) {};
					      \draw[views] (x) -- (y);
					      \draw[views] (x) -- (z);
					      \draw[generic] (y) -- (w);
				      \end{tikzpicture}
			      \end{center}
		      \end{minipage}
		      \vspace{0pt}

		      Ho così definito che, in un generico mondo $w$ accessibile da $y$, $p$ è vera. \\
		      Poiché $p$ è vera in tutti i mondi visti da $y$ (cioè ho che $\forall w \in \mathcal{W} (\text{se} \; y\mathcal{R}w  \text{, allora} \; \vDash_y^{\mathcal{M}} A)$), per \hyperlink{defverp}{definizione di verità di una formula in un punto} ho che:

		      \begin{minipage}{0.48\textwidth}
			      $$\vDash_y^{\mathcal{M}} \Box p$$
		      \end{minipage}
		      \begin{minipage}{0.48\textwidth}
			      \begin{center}
				      \begin{tikzpicture}
					      [world/.style={circle, draw, fill=black, inner sep=0pt, minimum size=4pt},
						      views/.style={->, >=Stealth, shorten >=1pt, shorten <=1pt},
						      generic/.style={->, >=Stealth, shorten >=1pt, shorten <=1pt, dashed}]
					      \node[world, label={right:{$x$}}] (x) at (0,0) {};
					      \node[world, label={[font=\Large]right:{$y \atop \Box p$}}] (y) at (1,1) {};
					      \node[world, label={left:{$z$}}] (z) at (-1,1) {};
					      \node[world, label={[font=\Large]right:{$w \atop p$}}] (w) at (0,2) {};
					      \draw[views] (x) -- (y);
					      \draw[views] (x) -- (z);
					      \draw[generic] (y) -- (w);
				      \end{tikzpicture}
			      \end{center}
		      \end{minipage}
		      \vspace{0pt}

		      Quindi, esiste almeno un punto accessibile da $x$ in cui è vera $\Box p$; dunque, per \hyperlink{defverp}{definizione di verità di una formula in un punto}, ho che:

		      \begin{minipage}{0.48\textwidth}
			      $$\vDash_x^{\mathcal{M}} \Dmd \Box p$$
		      \end{minipage}
		      \begin{minipage}{0.48\textwidth}
			      \begin{center}
				      \begin{tikzpicture}
					      [world/.style={circle, draw, fill=black, inner sep=0pt, minimum size=4pt},
						      views/.style={->, >=Stealth, shorten >=1pt, shorten <=1pt},
						      generic/.style={->, >=Stealth, shorten >=1pt, shorten <=1pt, dashed}]
					      \node[world, label={[font=\Large]right:{$x \atop \Dmd \Box p$}}] (x) at (0,0) {};
					      \node[world, label={[font=\Large]right:{$y \atop \Box p$}}] (y) at (1,1) {};
					      \node[world, label={left:{$z$}}] (z) at (-1,1) {};
					      \node[world, label={[font=\Large]right:{$w \atop p$}}] (w) at (0,2) {};
					      \draw[views] (x) -- (y);
					      \draw[views] (x) -- (z);
					      \draw[generic] (y) -- (w);
				      \end{tikzpicture}
			      \end{center}
		      \end{minipage}
		      \vspace{0pt}

		      Per assunzione, ho che $\vDash_x^{\mathcal{M}} \Dmd \Box p \to \Box\Dmd p$; quindi, per $M \! P$:

		      \begin{minipage}{0.48\textwidth}
			      \begin{mathpar}
				      \inferrule*[Right=$M \! P$]{\vDash_x^{\mathcal{M}} \Dmd \Box p \\ \vDash_x^{\mathcal{M}} \Dmd \Box p \to \Box \Dmd p}{\vDash_x^{\mathcal{M}} \Box \Dmd p}
			      \end{mathpar}
		      \end{minipage}
		      \begin{minipage}{0.48\textwidth}
			      \begin{center}
				      \begin{tikzpicture}
					      [world/.style={circle, draw, fill=black, inner sep=0pt, minimum size=4pt},
						      views/.style={->, >=Stealth, shorten >=1pt, shorten <=1pt},
						      generic/.style={->, >=Stealth, shorten >=1pt, shorten <=1pt, dashed}]
					      \node[world, label={[font=\Large]right:{$x \atop \Dmd \Box p, \: \Box \Dmd p$}}] (x) at (0,0) {};
					      \node[world, label={[font=\Large]right:{$y \atop \Box p$}}] (y) at (1,1) {};
					      \node[world, label={left:{$z$}}] (z) at (-1,1) {};
					      \node[world, label={[font=\Large]right:{$w \atop p$}}] (w) at (0,2) {};
					      \draw[views] (x) -- (y);
					      \draw[views] (x) -- (z);
					      \draw[generic] (y) -- (w);
				      \end{tikzpicture}
			      \end{center}
		      \end{minipage}
		      \vspace{8pt}

		      Quindi, in tutti i punti accessibili da $x$, $\Dmd p$ è vera (per \hyperlink{defverp}{definizione di verità di una formula in un punto}); in particolare:

		      \begin{minipage}{0.48\textwidth}
			      $$\vDash_z^{\mathcal{M}} \Dmd p$$
		      \end{minipage}
		      \begin{minipage}{0.48\textwidth}
			      \begin{center}
				      \begin{tikzpicture}
					      [world/.style={circle, draw, fill=black, inner sep=0pt, minimum size=4pt},
						      views/.style={->, >=Stealth, shorten >=1pt, shorten <=1pt},
						      generic/.style={->, >=Stealth, shorten >=1pt, shorten <=1pt, dashed}]
					      \node[world, label={[font=\Large]right:{$x \atop \Dmd \Box p$}}] (x) at (0,0) {};
					      \node[world, label={[font=\Large]right:{$y \atop \Box p$}}] (y) at (1,1) {};
					      \node[world, label={[font=\Large]left:{$z \atop \Dmd p$}}] (z) at (-1,1) {};
					      \node[world, label={[font=\Large]right:{$w \atop p$}}] (w) at (0,2) {};
					      \draw[views] (x) -- (y);
					      \draw[views] (x) -- (z);
					      \draw[generic] (y) -- (w);
				      \end{tikzpicture}
			      \end{center}
		      \end{minipage}
		      \vspace{0pt}

		      Da cui (per \hyperlink{defverp}{definizione di verità di una formula in un punto}) so che deve esistere almeno un punto accessibile da $z$ in cui $p$ è vera. Per come ho definito $I(p)$, so che $p$ è vera solo nei punti accessibili da $y$, dunque ho che $z\mathcal{R}w$. Posso quindi concludere:

		      \begin{minipage}{0.48\textwidth}
			      $$\exists w (y\mathcal{R}w \land z\mathcal{R}w)$$
		      \end{minipage}
		      \begin{minipage}{0.48\textwidth}
			      \begin{center}
				      \begin{tikzpicture}
					      [world/.style={circle, draw, fill=black, inner sep=0pt, minimum size=4pt},
						      views/.style={->, >=Stealth, shorten >=1pt, shorten <=1pt},
						      generic/.style={->, >=Stealth, shorten >=1pt, shorten <=1pt, dashed}]
					      \node[world, label={[font=\Large]right:{$x \atop \Dmd \Box p$}}] (x) at (0,0) {};
					      \node[world, label={[font=\Large]right:{$y \atop \Box p$}}] (y) at (1,1) {};
					      \node[world, label={[font=\Large]left:{$z \atop \Dmd p$}}] (z) at (-1,1) {};
					      \node[world, label={[font=\Large]right:{$w \atop p$}}] (w) at (0,2) {};
					      \draw[views] (x) -- (y);
					      \draw[views] (x) -- (z);
					      \draw[views] (y) -- (w);
					      \draw[views] (z) -- (w);
				      \end{tikzpicture}
			      \end{center}
		      \end{minipage}
		      \vspace{0pt}

		\item ($\leftarrow$): va dimostrato che \emph{se $\mathcal{F} \rhd \forall x \forall y \forall z (x\mathcal{R}y \land x\mathcal{R}z \to \exists w(y\mathcal{R}w \land z\mathcal{R}w))$, allora $\mathcal{F} \vDash \Dmd \Box A \to \Box \Dmd A$}. \\
		      Assumo $\mathcal{F} \rhd \forall x \forall y \forall z (x\mathcal{R}y \land x\mathcal{R}z \to \exists w(y\mathcal{R}w \land z\mathcal{R}w))$. \\
		      Sia $x \in \mathcal{W}$; definisco un $\mathcal{F}$-modello $\mathcal{M}$ tale che: \\

		      \begin{minipage}{0.48\textwidth}
			      $$\vDash_x^{\mathcal{M}} \Dmd \Box A$$
		      \end{minipage}
		      \begin{minipage}{0.48\textwidth}
			      \begin{center}
				      \begin{tikzpicture}
					      [world/.style={circle, draw, fill=black, inner sep=0pt, minimum size=4pt},
						      relation/.style={->, >=Stealth, shorten >=1pt, shorten <=1pt}]
					      \node[world, label={[font=\Large]right:{$x \atop \Dmd \Box A$}}] (x) at (0,0) {};
				      \end{tikzpicture}
			      \end{center}
		      \end{minipage}
		      \vspace{8pt}

		      Da cui, per \hyperlink{defverp}{definizione di verità di una formula in un punto}, so che esiste almeno un punto accessibile da $x$ in cui $\Box A$ è vera: \\

		      \begin{minipage}{0.48\textwidth}
			      $$\exists y (x\mathcal{R}y \; \text{e} \; \vDash_y^{\mathcal{M}}\Box  A)$$
		      \end{minipage}
		      \begin{minipage}{0.48\textwidth}
			      \begin{center}
				      \begin{tikzpicture}
					      [world/.style={circle, draw, fill=black, inner sep=0pt, minimum size=4pt},
						      views/.style={->, >=Stealth, shorten >=1pt, shorten <=1pt},
						      generic/.style={->, >=Stealth, shorten >=1pt, shorten <=1pt, dashed}]
					      \node[world, label={[font=\Large]left:{$x \atop \Dmd \Box A$}}] (x) at (0,0) {};
					      \node[world, label={[font=\Large]right:{$y \atop \Box A$}}] (z) at (2,0) {};
					      \draw[views] (x) -- (z);
				      \end{tikzpicture}
			      \end{center}
			      \vspace{8pt}

		      \end{minipage}
		      \vspace{8pt}

		      Devo mostrare che anche $\Box \Dmd A$ è vera in $x$; per \hyperlink{defverp}{definizione di verità di una formula in un punto}, ciò equivale a dimostrare che $\forall z \in \mathcal{W} (\text{se} \; x\mathcal{R}z  \text{, allora} \: \vDash_z^{\mathcal{M}} \Dmd A)$, che a sua volta equivale a dimostrare che $\exists w \in \mathcal{W} (z\mathcal{R}w \;\, \text{e} \; \vDash_w^{\mathcal{M}} A)$.\\
		      Per assunzione so che, comunque presi due punti accessibili da $x$, esisterà un terzo punto accessibile da entrambi; un caso particolare di questa proprietà è quello in cui i due punti sono lo stesso punto, in questo caso $y$:

		      \begin{minipage}{0.48\textwidth}
			      $$x\mathcal{R}y \land x\mathcal{R}y \to \exists w (y\mathcal{R}w \land y\mathcal{R}w)$$
		      \end{minipage}
		      \begin{minipage}{0.48\textwidth}
			      \begin{center}
				      \begin{tikzpicture}
					      [world/.style={circle, draw, fill=black, inner sep=0pt, minimum size=4pt},
						      views/.style={->, >=Stealth, shorten >=1pt, shorten <=1pt},
						      generic/.style={->, >=Stealth, shorten >=1pt, shorten <=1pt, dashed}]
					      \node[world, label={[font=\Large]left:{$x \atop \Dmd \Box A$}}] (x) at (0,0) {};
					      \node[world, label={[font=\Large]right:{$y \atop \Box A$}}] (y) at (2,0) {};
					      \node[world, label={left:{$w$}}] (w) at (1,1.5) {};
					      \draw[views] (x) -- (y);
					      \draw[views] (y) -- (w);
				      \end{tikzpicture}
			      \end{center}
		      \end{minipage}
		      \vspace{0pt}

		      Per \hyperlink{defverp}{definizione di verità di una formula in un punto} so che:

		      \begin{minipage}{0.48\textwidth}
			      $$\vDash_w^{\mathcal{M}} A$$
		      \end{minipage}
		      \begin{minipage}{0.48\textwidth}
			      \begin{center}
				      \begin{tikzpicture}
					      [world/.style={circle, draw, fill=black, inner sep=0pt, minimum size=4pt},
						      views/.style={->, >=Stealth, shorten >=1pt, shorten <=1pt},
						      generic/.style={->, >=Stealth, shorten >=1pt, shorten <=1pt, dashed}]
					      \node[world, label={[font=\Large]left:{$x \atop \Dmd \Box A$}}] (x) at (0,0) {};
					      \node[world, label={[font=\Large]right:{$y \atop \Box A$}}] (y) at (2,0) {};
					      \node[world, label={[font=\Large]left:{$w \atop A$}}] (w) at (1,1.5) {};
					      \draw[views] (x) -- (y);
					      \draw[views] (y) -- (w);
				      \end{tikzpicture}
			      \end{center}
		      \end{minipage}
		      \vspace{0pt}

		      Quindi, poiché è rispettata la \hyperlink{defverp}{definizione di verità di una formula in un punto} (cioè $\exists w \in \mathcal{W} (y\mathcal{R}w \;\, \text{e} \; \vDash_w^{\mathcal{M}} A)$), ho che:

		      \begin{minipage}{0.48\textwidth}
			      $$\vDash_y^{\mathcal{M}} \Dmd A$$
		      \end{minipage}
		      \begin{minipage}{0.48\textwidth}
			      \begin{center}
				      \begin{tikzpicture}
					      [world/.style={circle, draw, fill=black, inner sep=0pt, minimum size=4pt},
						      views/.style={->, >=Stealth, shorten >=1pt, shorten <=1pt},
						      generic/.style={->, >=Stealth, shorten >=1pt, shorten <=1pt, dashed}]
					      \node[world, label={[font=\Large]left:{$x \atop \Dmd \Box A$}}] (x) at (0,0) {};
					      \node[world, label={[font=\Large]right:{$y \atop \Box A, \: \Dmd A$}}] (y) at (2,0) {};
					      \node[world, label={[font=\Large]left:{$w \atop A$}}] (w) at (1,1.5) {};
					      \draw[views] (x) -- (y);
					      \draw[views] (y) -- (w);
				      \end{tikzpicture}
			      \end{center}
		      \end{minipage}
		      \vspace{0pt}

		      A questo punto, ho due casi:
		      \begin{itemize}
			      \item $\nexists z \in \mathcal{W} (x\mathcal{R}z)$: ho che, in tutti i mondi accessibili da $x$, $\Dmd A$ è vera ($x\mathcal{R}y \; \text{e} \; \vDash_y^{\mathcal{M}} \Dmd A$); da cui $\vDash_x^{\mathcal{M}} \Box \Dmd A$. Quindi, posso concludere:
			            $$\vDash_x^{\mathcal{M}} \Dmd \Box A \to \Box \Dmd A$$
			            (sia antecedente che conseguente sono veri in $x$).
			      \item $\exists z \in \mathcal{W} (x\mathcal{R}z)$: per assunzione so che:

			            \begin{minipage}{0.48\textwidth}
				            $$x\mathcal{R}y \land x\mathcal{R}z \to \exists w (y\mathcal{R}w \land z\mathcal{R}w)$$
			            \end{minipage}
			            \begin{minipage}{0.48\textwidth}
				            \begin{center}
					            \begin{tikzpicture}
						            [world/.style={circle, draw, fill=black, inner sep=0pt, minimum size=4pt},
							            views/.style={->, >=Stealth, shorten >=1pt, shorten <=1pt},
							            generic/.style={->, >=Stealth, shorten >=1pt, shorten <=1pt, dashed}]
						            \node[world, label={[font=\Large]below:{$x \atop \Dmd \Box A$}}] (x) at (0,0) {};
						            \node[world, label={[font=\Large]right:{$y \atop \Box A, \: \Dmd A$}}] (y) at (1,1) {};
						            \node[world, label={[font=\Large]above:{$w \atop A$}}] (w) at (0,2) {};
						            \node[world, label={left:{$z$}}] (z) at (-1,1) {};
						            \draw[views] (x) -- (y);
						            \draw[views] (x) -- (z);
						            \draw[views] (y) -- (w);
						            \draw[views] (z) -- (w);
					            \end{tikzpicture}
				            \end{center}
			            \end{minipage}
			            \vspace{0pt}

		      \end{itemize}

		      Quindi, so che esiste almeno un punto accessibile da $z$ in cui $A$ è vera; da cui, per \hyperlink{defverp}{definizione di verità di una formula in un punto} ho che:

		      \begin{minipage}{0.48\textwidth}
			      $$\vDash_z^{\mathcal{M}} \Dmd A$$
		      \end{minipage}
		      \begin{minipage}{0.48\textwidth}
			      \begin{center}
				      \begin{tikzpicture}
					      [world/.style={circle, draw, fill=black, inner sep=0pt, minimum size=4pt},
						      views/.style={->, >=Stealth, shorten >=1pt, shorten <=1pt},
						      generic/.style={->, >=Stealth, shorten >=1pt, shorten <=1pt, dashed}]
					      \node[world, label={[font=\Large]below:{$x \atop \Dmd \Box A$}}] (x) at (0,0) {};
					      \node[world, label={[font=\Large]right:{$y \atop \Box A, \: \Dmd A$}}] (y) at (1,1) {};
					      \node[world, label={[font=\Large]above:{$w \atop A$}}] (w) at (0,2) {};
					      \node[world, label={[font=\Large]left:{$z \atop \Dmd A$}}] (z) at (-1,1) {};
					      \draw[views] (x) -- (y);
					      \draw[views] (x) -- (z);
					      \draw[views] (y) -- (w);
					      \draw[views] (z) -- (w);
				      \end{tikzpicture}
			      \end{center}
		      \end{minipage}
		      \vspace{0pt}

		      Poiché in tutti i punti accessibili da $x$, $\Dmd A$ è vera, per \hyperlink{defverp}{definizione di verità di una formula in un punto} ho che:

		      \begin{minipage}{0.48\textwidth}
			      $$\vDash_x^{\mathcal{M}} \Box \Dmd A$$
		      \end{minipage}
		      \begin{minipage}{0.48\textwidth}
			      \begin{center}
				      \begin{tikzpicture}
					      [world/.style={circle, draw, fill=black, inner sep=0pt, minimum size=4pt},
						      views/.style={->, >=Stealth, shorten >=1pt, shorten <=1pt},
						      generic/.style={->, >=Stealth, shorten >=1pt, shorten <=1pt, dashed}]
					      \node[world, label={[font=\Large]below:{$x \atop \Dmd \Box A, \: \Box \Dmd A$}}] (x) at (0,0) {};
					      \node[world, label={[font=\Large]right:{$y \atop \Box A, \: \Dmd A$}}] (y) at (1,1) {};
					      \node[world, label={[font=\Large]left:{$z \atop \Dmd A$}}] (z) at (-1,1) {};
					      \node[world, label={[font=\Large]above:{$w \atop A$}}] (w) at (0,2) {};
					      \draw[views] (x) -- (y);
					      \draw[views] (x) -- (z);
					      \draw[views] (y) -- (w);
					      \draw[views] (z) -- (w);
				      \end{tikzpicture}
			      \end{center}
		      \end{minipage}

		      Posso quindi concludere:
		      $$\vDash_x^{\mathcal{M}} \Dmd \Box A \to \Box \Dmd A$$
		      (sia antecedente che conseguente sono veri in $x$).
	\end{enumerate}
\end{dimo}

\subsubsubs{Teorema: $3$ corrisponde alla connessione debole}
\begin{theo}
	Lo schema $3$ corrisponde alla connessione debole.
	$$\mathcal{F} \vDash \Box(A \land \Box A \to B) \lor \Box (\Box B \land B \to A) \quad\; \text{sse} \;\quad \mathcal{F} \rhd \forall x \forall y \forall z (x\mathcal{R}y \land x\mathcal{R}z \to y\mathcal{R}z \lor z\mathcal{R}y \lor y=z)$$
\end{theo}
\begin{dimo}
	\phantom{ciao}
	\begin{enumerate}
		\item ($\to$): va dimostrato che \emph{se $\mathcal{F} \vDash \Box(A \land \Box A \to B) \lor \Box (\Box B \land B \to A)$, allora $\mathcal{F} \rhd \forall x \forall y \forall z (x\mathcal{R}y \land x\mathcal{R}z \to y\mathcal{R}z \lor z\mathcal{R}y \lor y=z)$}. \\
		      Siano $p, q \in \Phi$; assumo $\mathcal{F} \vDash \Box(p \land \Box p \to q) \lor \Box (\Box q \land q \to p)$. \\
		      Siano $x, y, z \in \mathcal{W}$ tali che $x\mathcal{R}y \land x\mathcal{R}z$:
		      \vspace{8pt}

		      \begin{center}
			      \begin{tikzpicture}
				      [world/.style={circle, draw, fill=black, inner sep=0pt, minimum size=4pt},
					      views/.style={->, >=Stealth, shorten >=1pt, shorten <=1pt},
					      generic/.style={->, >=Stealth, shorten >=1pt, shorten <=1pt, dashed}]
				      \node[world, label={right:{$x$}}] (x) at (0,0) {};
				      \node[world, label={right:{$y$}}] (y) at (1,1.5) {};
				      \node[world, label={left:{$z$}}] (z) at (-1,1.5) {};
				      \draw[views] (x) -- (y);
				      \draw[views] (x) -- (z);
			      \end{tikzpicture}
		      \end{center}

		      Definisco un $\mathcal{F}$-modello $\mathcal{M}$ tale che:

		      \begin{minipage}{0.48\textwidth}
			      $$\begin{aligned}
					       & I(p) = \{w : y\mathcal{R}w\} \cup \{y\} \\
					       & I(q) = \{w : z\mathcal{R}w\} \cup \{z\}
				      \end{aligned}$$
		      \end{minipage}
		      \begin{minipage}{0.48\textwidth}
			      \begin{center}
				      \begin{tikzpicture}
					      [world/.style={circle, draw, fill=black, inner sep=0pt, minimum size=4pt},
						      views/.style={->, >=Stealth, shorten >=1pt, shorten <=1pt},
						      generic/.style={->, >=Stealth, shorten >=1pt, shorten <=1pt, dashed}]
					      \node[world, label={right:{$x$}}] (x) at (0,0) {};
					      \node[world, label={[font=\Large]right:{$y \atop p$}}] (y) at (1,1) {};
					      \node[world, label={[font=\Large]left:{$z \atop q$}}] (z) at (-1,1) {};
					      \node[world, label={[font=\Large]right:{$w \atop \phantom{,} p, \: q$}}] (w) at (0,2) {};
					      \draw[views] (x) -- (y);
					      \draw[views] (x) -- (z);
					      \draw[generic] (y) -- (w);
					      \draw[generic] (z) -- (w);
				      \end{tikzpicture}
			      \end{center}
		      \end{minipage}
		      \vspace{0pt}

		      Ho così definito che, in tutti i generici punto $w$ accessibili sia da $z$ che da $y$, sia $p$ che $q$ sono vere. Quindi, per \hyperlink{defverp}{definizione di verità di una formula in un punto} ho che $\Box p$ è vera in $y$ e $\Box q$ è vera in $z$. Inoltre, ho che $p$ è vera in $y$ e che $q$ è vera in $z$. Da cui:

		      \begin{minipage}{0.48\textwidth}
			      $$\vDash_y^{\mathcal{M}} p \land \Box p \quad \text{e} \quad \vDash_z^{\mathcal{M}} \Box q \land q$$
		      \end{minipage}
		      \begin{minipage}{0.48\textwidth}
			      \begin{center}
				      \begin{tikzpicture}
					      [world/.style={circle, draw, fill=black, inner sep=0pt, minimum size=4pt},
						      views/.style={->, >=Stealth, shorten >=1pt, shorten <=1pt},
						      generic/.style={->, >=Stealth, shorten >=1pt, shorten <=1pt, dashed}]
					      \node[world, label={right:{$x$}}] (x) at (0,0) {};
					      \node[world, label={[font=\Large]right:{$y \atop p \land \Box p$}}] (y) at (1,1) {};
					      \node[world, label={[font=\Large]left:{$z \atop \Box q \land q$}}] (z) at (-1,1) {};
					      \node[world, label={[font=\Large]right:{$w \atop \phantom{,} p, \: q$}}] (w) at (0,2) {};
					      \draw[views] (x) -- (y);
					      \draw[views] (x) -- (z);
					      \draw[generic] (y) -- (w);
					      \draw[generic] (z) -- (w);
				      \end{tikzpicture}
			      \end{center}
		      \end{minipage}
		      \vspace{0pt}

		      Per assunzione, so che $\vDash_y^{\mathcal{M}} \Box (p \land \Box p \to q) \; \text{o} \; \vDash_z^{\mathcal{M}} \Box (\Box q \land q \to p)$. Quindi, procedo per casi:
		      \begin{enumerate}
			      \item $\vDash_y^{\mathcal{M}} \Box (p \land \Box p \to q)$: \\
			            Ho che $\vDash_y^{\mathcal{M}} p \land \Box p$, quindi, per $M \! P$:
			            \begin{mathpar}
				            \inferrule*[Right=$M \! P$]{\vDash_y^{\mathcal{M}} p \land \Box p \\ \vDash_y^{\mathcal{M}} p \land \Box p \to q}{\vDash_y^{\mathcal{M}} q}
			            \end{mathpar}
			            Da come ho definito $I(q)$, so che $z\mathcal{R}y$ oppure $z=y$.
			      \item $\vDash_z^{\mathcal{M}} \Box (\Box q \land q \to p)$ \\
			            Ho che $\vDash_z^{\mathcal{M}} \Box q \land q$, quindi, per $M \! P$:
			            \begin{mathpar}
				            \inferrule*[Right=$M \! P$]{\vDash_z^{\mathcal{M}} \Box q \land q \\ \vDash_z^{\mathcal{M}} \Box q \land q \to p}{\vDash_z^{\mathcal{M}} p}
			            \end{mathpar}
			            Da come ho definito $I(p)$, so che $y\mathcal{R}z$ oppure $y=z$.
		      \end{enumerate}
		      Complessivamente, dunque, ho concluso che $y\mathcal{R}z \lor z\mathcal{R}y \lor y=z$.
		\item ($\leftarrow$): va dimostrato che \emph{se $\mathcal{F} \rhd \forall x \forall y \forall z (x\mathcal{R}y \land x\mathcal{R}z \to y\mathcal{R}z \lor z\mathcal{R}y \lor y=z)$, allora $\mathcal{F} \vDash \Box(A \land \Box A \to B) \lor \Box (\Box B \land B \to A)$}. La dimostrazione è per assurdo. \\
		      Assumo, per assurdo, che \emph{se $\mathcal{F} \rhd \forall x \forall y \forall z (x\mathcal{R}y \land x\mathcal{R}z \to y\mathcal{R}z \lor z\mathcal{R}y \lor y=z)$, allora $\mathcal{F} \nvDash \Box(A \land \Box A \to B) \lor \Box (\Box B \land B \to A)$}. \\
		      Assumo $\mathcal{F} \rhd \forall x \forall y \forall z (x\mathcal{R}y \land x\mathcal{R}z \to y\mathcal{R}z \lor z\mathcal{R}y \lor y=z)$. Quindi, per $M \! P$, ho che:
		      $$\mathcal{F} \nvDash \Box(A \land \Box A \to B) \lor \Box (\Box B \land B \to A)$$
		      Questo significa che esiste un $\mathcal{F}$-modello $\mathcal{M}$ in cui:
		      $$\exists w \in \mathcal{W} : \; \nvDash_w^{\mathcal{M}} \Box(A \land \Box A \to B) \lor \Box (\Box B \land B \to A)$$
		      Quindi, per \hyperlink{defverp}{definizione di verità di una formula in un punto}:
		      $$\nvDash_w^{\mathcal{M}} \Box(A \land \Box A \to B) \;\; \text{e} \;\; \nvDash_w^{\mathcal{M}} \Box (\Box B \land B \to A)$$
		      Questo significa che $\exists y, z : x\mathcal{R}y \land x\mathcal{R}z$ e che:
		      \begin{itemize}
			      \item $\vDash_y^{\mathcal{M}} A$;
			      \item $\vDash_y^{\mathcal{M}} \Box A$;
			      \item $\nvDash_y^{\mathcal{M}} B$;
			      \item $\vDash_z^{\mathcal{M}} B$;
			      \item $\vDash_z^{\mathcal{M}} \Box B$;
			      \item $\nvDash_z^{\mathcal{M}} A$;
		      \end{itemize}
		      Queste condizioni escludono $y\mathcal{R}z \lor z\mathcal{R}y \lor y=z$; quindi,  contraddizione.
	\end{enumerate}
\end{dimo}

\subsubsubs{Teorema: \latinmath{Lemmon} corrisponde alla proprietà \emph{mnkj}-Lemmon}
\begin{theo}
	Lo schema \latinmath{Lemmon} corrisponde alla proprietà \emph{mnkj}-Lemmon.
	$$\mathcal{F} \vDash \Dmd^m \Box^n A \to \Box^k \Dmd^j A \quad\; \text{sse} \;\quad \mathcal{F} \rhd \forall x \forall y \forall z (x\mathcal{R}^my \land x\mathcal{R}^kz \to \exists w (y\mathcal{R}^nw \land z\mathcal{R}^jw))$$
\end{theo}
\noindent Tutti gli altri schemi sono istanze di questo.
\begin{dimo}
	\phantom{ciao}
	\begin{enumerate}
		\item ($\to$): va dimostrato che \emph{se $\mathcal{F} \vDash \Dmd^m \Box^n A \to \Box^k \Dmd^j A$, allora $\mathcal{F} \rhd \forall x \forall y \forall z (x\mathcal{R}^my \land x\mathcal{R}^kz \to \exists w (y\mathcal{R}^nw \land z\mathcal{R}^jw))$}. \\
		      Sia $p \in \Phi$; assumo $\mathcal{F} \vDash \Dmd^m \Box^n p \to \Box^k \Dmd^j p$. \\
		      Siano $x, y, z \in \mathcal{W}$ tali che $x\mathcal{R}^my \land x\mathcal{R}^kz$:
		      \vspace{8pt}

		      \begin{center}
			      \begin{tikzpicture}
				      [world/.style={circle, draw, fill=black, inner sep=0pt, minimum size=4pt},
					      views/.style={->, >=Stealth, shorten >=1pt, shorten <=1pt},
					      generic/.style={->, >=Stealth, shorten >=1pt, shorten <=1pt, dashed}]
				      \node[world, label={right:{$x$}}] (x) at (0,0) {};
				      \node[world, label={right:{$y$}}] (y) at (1,1.5) {};
				      \node[world, label={left:{$z$}}] (z) at (-1,1.5) {};
				      \draw[views] (x) -- (y) node[midway, below, sloped] {$\mathcal{R}^m$};
				      \draw[views] (x) -- (z) node[midway, below, sloped] {$\mathcal{R}^k$};
			      \end{tikzpicture}
		      \end{center}

		      Definisco un $\mathcal{F}$-modello $\mathcal{M}$ tale che:

		      \begin{minipage}{0.48\textwidth}
			      $$I(p) = \{w \in \mathcal{W} : y\mathcal{R}^nw\}$$
		      \end{minipage}
		      \begin{minipage}{0.48\textwidth}
			      \begin{center}
				      \begin{tikzpicture}
					      [world/.style={circle, draw, fill=black, inner sep=0pt, minimum size=4pt},
						      views/.style={->, >=Stealth, shorten >=1pt, shorten <=1pt},
						      generic/.style={->, >=Stealth, shorten >=1pt, shorten <=1pt, dashed}]
					      \node[world, label={below:{$x$}}] (x) at (0,0) {};
					      \node[world, label={right:{$y$}}] (y) at (1,1) {};
					      \node[world, label={left:{$z$}}] (z) at (-1,1) {};
					      \node[world, label={[font=\Large]left:{$w \atop p$}}] (w) at (0,2) {};
					      \draw[views] (x) -- (y) node[midway, below, sloped] {$\mathcal{R}^m$};
					      \draw[views] (x) -- (z) node[midway, below, sloped] {$\mathcal{R}^k$};
					      \draw[generic] (y) -- (w) node[midway, above, sloped] {$\mathcal{R}^n$};
				      \end{tikzpicture}
			      \end{center}
		      \end{minipage}
		      \vspace{0pt}

		      Ho così definito che, in un generico mondo $w$ accessibile da $y$ in $\mathcal{R}^n$ \footnote{Sarebbe corretto scrivere \enquote{accessibile da $y$ in $n$ passi}. Usiamo questa notazione per velocizzare la lettura.}, $p$ è vera. \\
		      So, quindi, che $p$ è vera in tutti i mondi accessibili da $y$ in $\mathcal{R}^n$; dunque, per \hyperlink{defverp}{definizione di verità di una formula in un punto}, ho che:

		      \begin{minipage}{0.48\textwidth}
			      $$\vDash_y^{\mathcal{M}} \Box^n p$$
		      \end{minipage}
		      \begin{minipage}{0.48\textwidth}
			      \begin{center}
				      \begin{tikzpicture}
					      [world/.style={circle, draw, fill=black, inner sep=0pt, minimum size=4pt},
						      views/.style={->, >=Stealth, shorten >=1pt, shorten <=1pt},
						      generic/.style={->, >=Stealth, shorten >=1pt, shorten <=1pt, dashed}]
					      \node[world, label={below:{$x$}}] (x) at (0,0) {};
					      \node[world, label={[font=\Large]right:{$y \atop \Box^n p$}}] (y) at (1,1) {};
					      \node[world, label={left:{$z$}}] (z) at (-1,1) {};
					      \node[world, label={[font=\Large]left:{$w \atop p$}}] (w) at (0,2) {};
					      \draw[views] (x) -- (y) node[midway, below, sloped] {$\mathcal{R}^m$};
					      \draw[views] (x) -- (z) node[midway, below, sloped] {$\mathcal{R}^k$};
					      \draw[generic] (y) -- (w) node[midway, above, sloped] {$\mathcal{R}^n$};
				      \end{tikzpicture}
			      \end{center}
		      \end{minipage}
		      \vspace{0pt}

		      Quindi, esiste almeno un mondo accessibile da $x$ in $\mathcal{R}^m$ in cui è vera $\Box^n p$; dunque, per \hyperlink{defverp}{definizione di verità di una formula in un punto}, ho che:

		      \begin{minipage}{0.48\textwidth}
			      $$\vDash_x^{\mathcal{M}} \Dmd^m \Box^n p$$
		      \end{minipage}
		      \begin{minipage}{0.48\textwidth}
			      \begin{center}
				      \begin{tikzpicture}
					      [world/.style={circle, draw, fill=black, inner sep=0pt, minimum size=4pt},
						      views/.style={->, >=Stealth, shorten >=1pt, shorten <=1pt},
						      generic/.style={->, >=Stealth, shorten >=1pt, shorten <=1pt, dashed}]
					      \node[world, label={[font=\Large]below:{$x \atop \Dmd^m \Box^n p$}}] (x) at (0,0) {};
					      \node[world, label={[font=\Large]right:{$y \atop \Box^n p$}}] (y) at (1,1) {};
					      \node[world, label={left:{$z$}}] (z) at (-1,1) {};
					      \node[world, label={[font=\Large]left:{$w \atop p$}}] (w) at (0,2) {};
					      \draw[views] (x) -- (y) node[midway, below, sloped] {$\mathcal{R}^m$};
					      \draw[views] (x) -- (z) node[midway, below, sloped] {$\mathcal{R}^k$};
					      \draw[generic] (y) -- (w) node[midway, above, sloped] {$\mathcal{R}^n$};
				      \end{tikzpicture}
			      \end{center}
		      \end{minipage}
		      \vspace{0pt}

		      Per assunzione, ho che $\vDash_x^{\mathcal{M}} \Dmd^m \Box^n p \to \Box^k \Dmd^j p$; quindi, per $M \! P$:

		      \begin{minipage}{0.48\textwidth}
			      \begin{mathpar}
				      \inferrule*[Right=$M \! P$]{\vDash_x^{\mathcal{M}} \Dmd^m \Box^n p \\ \vDash_x^{\mathcal{M}} \Dmd^m \Box^n p \to \Box^k \Dmd^j p}{\vDash_x^{\mathcal{M}} \Box^k \Dmd^j p}
			      \end{mathpar}
		      \end{minipage}
		      \begin{minipage}{0.48\textwidth}
			      \begin{center}
				      \begin{tikzpicture}
					      [world/.style={circle, draw, fill=black, inner sep=0pt, minimum size=4pt},
						      views/.style={->, >=Stealth, shorten >=1pt, shorten <=1pt},
						      generic/.style={->, >=Stealth, shorten >=1pt, shorten <=1pt, dashed}]
					      \node[world, label={[font=\Large]below:{$x \atop \Dmd^m \Box^n p, \: \Box^k \Dmd^j p$}}] (x) at (0,0) {};
					      \node[world, label={[font=\Large]right:{$y \atop \Box^n p$}}] (y) at (1,1) {};
					      \node[world, label={left:{$z$}}] (z) at (-1,1) {};
					      \node[world, label={[font=\Large]left:{$w \atop p$}}] (w) at (0,2) {};
					      \draw[views] (x) -- (y) node[midway, below, sloped] {$\mathcal{R}^m$};
					      \draw[views] (x) -- (z) node[midway, below, sloped] {$\mathcal{R}^k$};
					      \draw[generic] (y) -- (w) node[midway, above, sloped] {$\mathcal{R}^n$};
				      \end{tikzpicture}
			      \end{center}
		      \end{minipage}
		      \vspace{8pt}

		      Quindi, so che in tutti i mondi accessibili da $x$ in $\mathcal{R}^k$, $\Dmd^j$ è vera; in particolare:

		      \begin{minipage}{0.48\textwidth}
			      $$\vDash_z^{\mathcal{M}} \Dmd^j p$$
		      \end{minipage}
		      \begin{minipage}{0.48\textwidth}
			      \begin{center}
				      \begin{tikzpicture}
					      [world/.style={circle, draw, fill=black, inner sep=0pt, minimum size=4pt},
						      views/.style={->, >=Stealth, shorten >=1pt, shorten <=1pt},
						      generic/.style={->, >=Stealth, shorten >=1pt, shorten <=1pt, dashed}]
					      \node[world, label={[font=\Large]below:{$x \atop \Dmd^m \Box^n p, \: \Box^k \Dmd^j p$}}] (x) at (0,0) {};
					      \node[world, label={[font=\Large]right:{$y \atop \Box^n p$}}] (y) at (1,1) {};
					      \node[world, label={[font=\Large]left:{$z \atop \Dmd^j p$}}] (z) at (-1,1) {};
					      \node[world, label={[font=\Large]left:{$w \atop p$}}] (w) at (0,2) {};
					      \draw[views] (x) -- (y) node[midway, below, sloped] {$\mathcal{R}^m$};
					      \draw[views] (x) -- (z) node[midway, below, sloped] {$\mathcal{R}^k$};
					      \draw[generic] (y) -- (w) node[midway, above, sloped] {$\mathcal{R}^n$};
				      \end{tikzpicture}
			      \end{center}
		      \end{minipage}
		      \vspace{0pt}

		      Da cui (per \hyperlink{defverp}{definizione di verità di una formula in un punto}) so che deve esistere almeno un punto accessibile da $z$ in $\mathcal{R}^j$in cui $p$ è vera. Per come ho definito $I(p)$, so che $p$ è vera solo nei punti accessibili da $y$ in $\mathcal{R}^n$, dunque ho che $z\mathcal{R}^jw$. Quindi, posso concludere:

		      \begin{minipage}{0.48\textwidth}
			      $$\exists w (y\mathcal{R}^nw \land z\mathcal{R}^jw)$$
		      \end{minipage}
		      \begin{minipage}{0.48\textwidth}
			      \begin{center}
				      \begin{tikzpicture}
					      [world/.style={circle, draw, fill=black, inner sep=0pt, minimum size=4pt},
						      views/.style={->, >=Stealth, shorten >=1pt, shorten <=1pt},
						      generic/.style={->, >=Stealth, shorten >=1pt, shorten <=1pt, dashed}]
					      \node[world, label={[font=\Large]below:{$x \atop \Dmd^m \Box^n p, \: \Box^k \Dmd^j p$}}] (x) at (0,0) {};
					      \node[world, label={[font=\Large]right:{$y \atop \Box^n p$}}] (y) at (1,1) {};
					      \node[world, label={[font=\Large]left:{$z \atop \Dmd^j p$}}] (z) at (-1,1) {};
					      \node[world, label={[font=\Large]above:{$w \atop p$}}] (w) at (0,2) {};
					      \draw[views] (x) -- (y) node[midway, below, sloped] {$\mathcal{R}^m$};
					      \draw[views] (x) -- (z) node[midway, below, sloped] {$\mathcal{R}^k$};
					      \draw[views] (y) -- (w) node[midway, above, sloped] {$\mathcal{R}^n$};
					      \draw[views] (z) -- (w) node[midway, above, sloped] {$\mathcal{R}^j$};
				      \end{tikzpicture}
			      \end{center}
		      \end{minipage}
		      \vspace{0pt}
		\item ($\leftarrow$): va dimostrato che \emph{se $\mathcal{F} \rhd \forall x \forall y \forall z (x\mathcal{R}^my \land x\mathcal{R}^kz \to \exists w (y\mathcal{R}^nw \land z\mathcal{R}^jw))$, allora $\mathcal{F} \vDash \Dmd^m \Box^n A \to \Box^k \Dmd^j A$}. \\
		      Assumo $\mathcal{F} \rhd \forall x \forall y \forall z (x\mathcal{R}^my \land x\mathcal{R}^kz \to \exists w (y\mathcal{R}^nw \land z\mathcal{R}^jw))$. \\
		      Sia $x \in \mathcal{W}$; definisco un $\mathcal{F}$-modello $\mathcal{M}$ tale che: \\

		      \begin{minipage}{0.48\textwidth}
			      $$\vDash_x^{\mathcal{M}} \Dmd^m \Box^n A$$
		      \end{minipage}
		      \begin{minipage}{0.48\textwidth}
			      \begin{center}
				      \begin{tikzpicture}
					      [world/.style={circle, draw, fill=black, inner sep=0pt, minimum size=4pt},
						      relation/.style={->, >=Stealth, shorten >=1pt, shorten <=1pt}]
					      \node[world, label={[font=\Large]right:{$x \atop \Dmd^m \Box^n A$}}] (x) at (0,0) {};
				      \end{tikzpicture}
			      \end{center}
		      \end{minipage}
		      \vspace{8pt}

		      Da cui, per \hyperlink{defverp}{definizione di verità di una formula in un punto}, so che esiste almeno un punto accessibile da $x$ in $\mathcal{R}^m$ in cui $\Box^n A$ è vera: \\

		      \begin{minipage}{0.48\textwidth}
			      $$\exists y (x\mathcal{R}^my \; \text{e} \; \vDash_y^{\mathcal{M}}\Box^n  A)$$
		      \end{minipage}
		      \begin{minipage}{0.48\textwidth}
			      \begin{center}
				      \begin{tikzpicture}
					      [world/.style={circle, draw, fill=black, inner sep=0pt, minimum size=4pt},
						      views/.style={->, >=Stealth, shorten >=1pt, shorten <=1pt},
						      generic/.style={->, >=Stealth, shorten >=1pt, shorten <=1pt, dashed}]
					      \node[world, label={[font=\Large]left:{$x \atop \Dmd^m \Box^n A$}}] (x) at (0,0) {};
					      \node[world, label={[font=\Large]right:{$y \atop \Box^n A$}}] (y) at (2,0) {};
					      \draw[views] (x) -- (y) node[midway, above, sloped] {$\mathcal{R}^m$};
				      \end{tikzpicture}
			      \end{center}
			      \vspace{8pt}

		      \end{minipage}
		      \vspace{8pt}

		      Devo mostrare che anche $\Box^k \Dmd^j A$ è vera in $x$; per \hyperlink{defverp}{definizione di verità di una formula in un punto}, ciò equivale a mostrare che $\forall z \in \mathcal{W} (\text{se} \; x\mathcal{R}^kz  \text{, allora} \: \vDash_z^{\mathcal{M}} \Dmd^j A)$, che a sua volta equivale a mostrare che $\exists w \in \mathcal{W} (z\mathcal{R}^jw \;\, \text{e} \; \vDash_w^{\mathcal{M}} A)$.\\
		      Ho due casi:
		      \begin{itemize}
			      \item $\nexists z \in \mathcal{W} (x\mathcal{R}^kz)$: l'antecedente dell'implicazione è falso, dunque l'implicazione è vera; ho quindi che $\vDash_x^{\mathcal{M}} \Box^k \Dmd^j A$. Di conseguenza, posso concludere:
			            $$\vDash_x^{\mathcal{M}} \Dmd^m \Box^n A \to \Box^k \Dmd^j A$$
			            (sia antecedente che conseguente sono veri in $x$).
			      \item $\exists z \in \mathcal{W} (x\mathcal{R}^kz)$: per assunzione so che:

			            \begin{minipage}{0.48\textwidth}
				            $$x\mathcal{R}^my \land x\mathcal{R}^kz \to \exists w (y\mathcal{R}^nw \land z\mathcal{R}^jw)$$
			            \end{minipage}
			            \begin{minipage}{0.48\textwidth}
				            \begin{center}
					            \begin{tikzpicture}
						            [world/.style={circle, draw, fill=black, inner sep=0pt, minimum size=4pt},
							            views/.style={->, >=Stealth, shorten >=1pt, shorten <=1pt},
							            generic/.style={->, >=Stealth, shorten >=1pt, shorten <=1pt, dashed}]
						            \node[world, label={[font=\Large]below:{$x \atop \Dmd^m \Box^n A$}}] (x) at (0,0) {};
						            \node[world, label={[font=\Large]right:{$y \atop \Box^n A$}}] (y) at (1,1) {};
						            \node[world, label={above:{$w$}}] (w) at (0,2) {};
						            \node[world, label={left:{$z$}}] (z) at (-1,1) {};
						            \draw[views] (x) -- (y) node[midway, below, sloped] {$\mathcal{R}^m$};
						            \draw[views] (x) -- (z) node[midway, below, sloped] {$\mathcal{R}^k$};
						            \draw[views] (y) -- (w) node[midway, above, sloped] {$\mathcal{R}^n$};
						            \draw[views] (z) -- (w) node[midway, above, sloped] {$\mathcal{R}^j$};;
					            \end{tikzpicture}
				            \end{center}
			            \end{minipage}
			            \vspace{0pt}

			            Per \hyperlink{defverp}{definizione di verità di una formula in un punto}, ho che in tutti i mondi visti da $y$ in $\mathcal{R}^n$:

			            \begin{minipage}{0.48\textwidth}
				            $$\vDash_w^{\mathcal{M}} A$$
			            \end{minipage}
			            \begin{minipage}{0.48\textwidth}
				            \begin{center}
					            \begin{tikzpicture}
						            [world/.style={circle, draw, fill=black, inner sep=0pt, minimum size=4pt},
							            views/.style={->, >=Stealth, shorten >=1pt, shorten <=1pt},
							            generic/.style={->, >=Stealth, shorten >=1pt, shorten <=1pt, dashed}]
						            \node[world, label={[font=\Large]below:{$x \atop \Dmd^m \Box^n A$}}] (x) at (0,0) {};
						            \node[world, label={[font=\Large]right:{$y \atop \Box^n A$}}] (y) at (1,1) {};
						            \node[world, label={[font=\Large]above:{$w \atop A$}}] (w) at (0,2) {};
						            \node[world, label={left:{$z$}}] (z) at (-1,1) {};
						            \draw[views] (x) -- (y) node[midway, below, sloped] {$\mathcal{R}^m$};
						            \draw[views] (x) -- (z) node[midway, below, sloped] {$\mathcal{R}^k$};
						            \draw[views] (y) -- (w) node[midway, above, sloped] {$\mathcal{R}^n$};
						            \draw[views] (z) -- (w) node[midway, above, sloped] {$\mathcal{R}^j$};;
					            \end{tikzpicture}
				            \end{center}
			            \end{minipage}
			            \vspace{0pt}

			            Quindi, so che esiste almeno un punto accessibile da $z$ in $\mathcal{R}^k$ in cui $A$ è vera; da cui, per \hyperlink{defverp}{definizione di verità di una formula in un punto}, ho che:

			            \begin{minipage}{0.48\textwidth}
				            $$\vDash_z^{\mathcal{M}} \Dmd^j A$$
			            \end{minipage}
			            \begin{minipage}{0.48\textwidth}
				            \begin{center}
					            \begin{tikzpicture}
						            [world/.style={circle, draw, fill=black, inner sep=0pt, minimum size=4pt},
							            views/.style={->, >=Stealth, shorten >=1pt, shorten <=1pt},
							            generic/.style={->, >=Stealth, shorten >=1pt, shorten <=1pt, dashed}]
						            \node[world, label={[font=\Large]below:{$x \atop \Dmd^m \Box^n A$}}] (x) at (0,0) {};
						            \node[world, label={[font=\Large]right:{$y \atop \Box^n A$}}] (y) at (1,1) {};
						            \node[world, label={[font=\Large]above:{$w \atop A$}}] (w) at (0,2) {};
						            \node[world, label={[font=\Large]left:{$z \atop \Dmd^j A$}}] (z) at (-1,1) {};
						            \draw[views] (x) -- (y) node[midway, below, sloped] {$\mathcal{R}^m$};
						            \draw[views] (x) -- (z) node[midway, below, sloped] {$\mathcal{R}^k$};
						            \draw[views] (y) -- (w) node[midway, above, sloped] {$\mathcal{R}^n$};
						            \draw[views] (z) -- (w) node[midway, above, sloped] {$\mathcal{R}^j$};;
					            \end{tikzpicture}
				            \end{center}
			            \end{minipage}
			            \vspace{0pt}

			            Poiché in tutti i punti accessibili da $x$ in $\mathcal{R}^k$, $\Dmd^j A$ è vera, per \hyperlink{defverp}{definizione di verità di una formula in un punto} ho che:

			            \begin{minipage}{0.48\textwidth}
				            $$\vDash_x^{\mathcal{M}} \Box^k \Dmd^j A$$
			            \end{minipage}
			            \begin{minipage}{0.48\textwidth}
				            \begin{center}
					            \begin{tikzpicture}
						            [world/.style={circle, draw, fill=black, inner sep=0pt, minimum size=4pt},
							            views/.style={->, >=Stealth, shorten >=1pt, shorten <=1pt},
							            generic/.style={->, >=Stealth, shorten >=1pt, shorten <=1pt, dashed}]
						            \node[world, label={[font=\Large]below:{$x \atop \Dmd^m \Box^n A, \: \Box^k \Dmd^j A$}}] (x) at (0,0) {};
						            \node[world, label={[font=\Large]right:{$y \atop \Box^n A$}}] (y) at (1,1) {};
						            \node[world, label={[font=\Large]above:{$w \atop A$}}] (w) at (0,2) {};
						            \node[world, label={[font=\Large]left:{$z \atop \Dmd^j A$}}] (z) at (-1,1) {};
						            \draw[views] (x) -- (y) node[midway, below, sloped] {$\mathcal{R}^m$};
						            \draw[views] (x) -- (z) node[midway, below, sloped] {$\mathcal{R}^k$};
						            \draw[views] (y) -- (w) node[midway, above, sloped] {$\mathcal{R}^n$};
						            \draw[views] (z) -- (w) node[midway, above, sloped] {$\mathcal{R}^j$};;
					            \end{tikzpicture}
				            \end{center}
			            \end{minipage}

			            Posso quindi concludere:
			            $$\vDash_x^{\mathcal{M}} \Dmd^m \Box^n A \to \Box^k \Dmd^j A$$
			            (sia antecedente che conseguente sono veri in $x$).
		      \end{itemize}
	\end{enumerate}
\end{dimo}

\subsubsection{Proprietà non esprimibili}
\noindent Vi sono proprietà del primo ordine che non sono esprimibili da formule modali. Per mostrare ciò, si utilizzano tecniche che permettono di trasformare modelli o metterli in relazione con altri \enquote{simili}. \\

\subsubsubs{Definizione di chiusura riflessiva e transitiva}
\begin{defin}
	[\emph{Chiusura riflessiva e transitiva}] Data una struttura relazionale $\mathcal{F} = \langle \mathcal{W}, \mathcal{R} \rangle$, $\mathcal{R}^{\star}$ \footnote{Il simbolo $\star$ è chiamato \enquote{\href{https://it.wikipedia.org/wiki/Star_di_Kleene}{stella di Kleene}}, e indica precisamente una chiusura riflessiva e transitiva.} è detta \enquote{chiusura riflessiva e transitiva} di $\mathcal{R}$ se: $$w\mathcal{R}^{\star}v \quad\; \text{sse} \;\quad \exists n \in \mathbb{N} (w\mathcal{R}^nv)$$
	Dunque, una chiusura riflessiva e transitiva rappresenta l'insieme di tutti i mondi che sono o il mondo di partenza stesso (nel caso in cui $n=0$) o i mondi da esso accessibili in un numero finito di passi (in altre parole, raggiungibili per riflessività o transitività). \\
\end{defin}

\subsubsubs{Definizione di sottomodello generato da un punto}
\hypertarget{modgen}{}
\begin{defin}
	[\emph{Sottomodello generato da un punto}] Dato un modello $\mathcal{M} = \langle \mathcal{W}, \mathcal{R}, I \rangle$, sia $v \in \mathcal{W}$; chiamiamo \enquote{sottomodello generato da $v$} il modello definito nel seguente modo:
	$$\mathcal{M}^{v} = \langle \mathcal{W}^{v}, \mathcal{R}^{v}, I^{v} \rangle $$
	dove:
	\begin{itemize}
		\item $\mathcal{W}^{v}$ è l'insieme dei mondi accessibili da $v$ tramite la chiusura riflessiva e transitiva $\mathcal{R}^{\star}$ (in altre parole, tutti e soli i mondi raggiungibili da $v$ in un numero finito di passi):
		      $$\mathcal{W}^{v} = \{u \in \mathcal{W} (v\mathcal{R}^{\star}u)\}$$
		\item $\mathcal{R}^{v}$ è la parte di relazione di accessibilità del modello di partenza che riguarda i mondi accessibili da $v$ tramite $\mathcal{R}^{\star}$:
		      $$\mathcal{R}^{v} = \mathcal{R} \cap (\mathcal{W}^{v} \times \mathcal{W}^{v})$$
		\item $I^{v}(p)$ è la parte di funzione di interpretazione del modello di partenza che riguarda i mondi accessibili da $v$ tramite $\mathcal{R}^{\star}$:
		      $$I^{v}(p) = I(p) \cap \mathcal{W}^{v}$$
	\end{itemize}
	La struttura $\mathcal{F}^{v} = \langle \mathcal{W}^{v}, \mathcal{R}^{v} \rangle$ è la \emph{sottostruttura} di $\mathcal{F} = \langle \mathcal{W}, \mathcal{R} \rangle$ generata da $v$, anche detta struttura con \emph{radice} $v$. \\
	Intuitivamente, possiamo affermare che un sottomodello generato da un punto contiene tutti e soli i punti raggiungibili dal primo in un numero finito di passi; pertanto, qualsiasi relazione che intercorre nel sottomodello generato si troverà anche in quello di partenza, mentre non è vero il contrario. \\
\end{defin}

\subsubsubs{Lemma: verità in un modello generato}
\hypertarget{lemgen}{}
\begin{lem}
	[Verità in un modello generato] Per ogni modello $\mathcal{M}$, comunque presa una formula $A \in \latinmath{fm}^{\Phi}$, $\forall u \in \mathcal{W}^{v}$:
	$$\vDash_u^{\mathcal{M}} A \quad \; \text{sse} \; \quad \; \vDash_u^{\mathcal{M}^{v}} A$$
\end{lem}
\begin{dimo}
	[per induzione strutturale su $A$] \phantom{ciao}
	\begin{description}
		\item[Base:] \phantom{ciao}
		      \begin{itemize}
			      \item $A \equiv \bot$: \\
			            $\nvDash_u^{\mathcal{M}} \bot \; \text{sse} \; \nvDash_u^{\mathcal{M}^{v}} \bot$ \qquad \quad $\bot$ è falso in qualsiasi modello.
			      \item $A \equiv p$: \\
			            Ho $\vDash_u^{\mathcal{M}} p$. Per \hyperlink{modgen}{definizione di sottomodello generato}, so che per ogni $u \in \mathcal{W}^{v}$:
			            $$u \in I(p) \quad \; \text{sse} \; \quad u \in I^{v}(p)$$
			            Per \hyperlink{defverp}{definizione di verità di una formula in un punto}, posso concludere:
			            $$\vDash_u^{\mathcal{M}} p \quad \; \text{sse} \; \quad \; \vDash_u^{\mathcal{M}^{v}} p$$
		      \end{itemize}
		\item[Passo:] ho un caso per ogni connettivo. \\
		      \latinmath{IH1}: $\forall u \in \mathcal{W}^{v} (\vDash_u^{\mathcal{M}} B \; \text{sse} \; \vDash_u^{\mathcal{M}^{v}} B)$ \\
		      \latinmath{IH2}: $\forall u \in \mathcal{W}^{v} (\vDash_u^{\mathcal{M}} C \; \text{sse} \; \vDash_u^{\mathcal{M}^{v}} C)$
		      \begin{itemize}
			      \item $A \equiv B \land C$ (va dimostrato \emph{$\forall u \in \mathcal{W}^{v} (\vDash_u^{\mathcal{M}} B \land C \; \text{sse} \; \vDash_u^{\mathcal{M}^{v}} B \land C)$}):\\
			            Per \hyperlink{defverp}{definizione di verità di una formula in un punto}, ho $\vDash_u^{\mathcal{M}} B \land C \; \text{sse} \; \vDash_u^{\mathcal{M}} B \; \text{e} \; \vDash_u^{\mathcal{M}} C$. \\
			            Per \latinmath{IH1} e \latinmath{IH2}, $\forall u \in \mathcal{W}^{v}$:
			            $$\vDash_u^{\mathcal{M}} B \;\; \text{e} \;\; \vDash_u^{\mathcal{M}} C \quad \; \text{sse} \quad \; \vDash_u^{\mathcal{M}^{v}} B \;\; \text{e} \;\; \vDash_u^{\mathcal{M}^{v}} C$$
			            Per \hyperlink{defverp}{definizione di verità di una formula in un punto}, posso concludere che, $\forall u \in \mathcal{W}^{v}$:
			            $$\vDash_u^{\mathcal{M}} B \land C \quad \; \text{sse} \quad \; \vDash_u^{\mathcal{M}^{v}} B \land C$$
			      \item $A \equiv B \lor C$ (analogo a $A \equiv B \land C$; va dimostrato \emph{$\forall u \in \mathcal{W}^{v} (\vDash_u^{\mathcal{M}} B \lor C \; \text{sse} \; \vDash_u^{\mathcal{M}^{v}} B \lor C)$}):\\
			            Per \hyperlink{defverp}{definizione di verità di una formula in un punto}, ho $\vDash_u^{\mathcal{M}} B \lor C \; \text{sse} \; \vDash_u^{\mathcal{M}} B \; \text{o} \; \vDash_u^{\mathcal{M}} C$. \\
			            Per \latinmath{IH1} e \latinmath{IH2}, $\forall u \in \mathcal{W}^{v}$:
			            $$\vDash_u^{\mathcal{M}} B \;\; \text{o} \;\; \vDash_u^{\mathcal{M}} C \quad \; \text{sse} \quad \; \vDash_u^{\mathcal{M}^{v}} B \;\; \text{o} \;\; \vDash_u^{\mathcal{M}^{v}} C$$
			            Per \hyperlink{defverp}{definizione di verità di una formula in un punto}, posso concludere che, $\forall u \in \mathcal{W}^{v}$:
			            $$\vDash_u^{\mathcal{M}} B \lor C \quad \; \text{sse} \quad \; \vDash_u^{\mathcal{M}^{v}} B \lor C$$
			      \item $A \equiv B \to C$ (analogo a $A \equiv B \land C$; va dimostrato \emph{$\forall u \in \mathcal{W}^{v} (\vDash_u^{\mathcal{M}} B \to C \; \text{sse} \; \vDash_u^{\mathcal{M}^{v}} B \to C)$}):\\
			            \hyperlink{defverp}{definizione di verità di una formula in un punto}, ho $\vDash_u^{\mathcal{M}} B \to C \; \text{sse} \; \nvDash_u^{\mathcal{M}} B \; \text{o} \; \vDash_u^{\mathcal{M}} C$. \\
			            Per \latinmath{IH1} e \latinmath{IH2}, $\forall u \in \mathcal{W}^{v}$:
			            $$\nvDash_u^{\mathcal{M}} B \;\; \text{o} \;\; \vDash_u^{\mathcal{M}} C \quad \; \text{sse} \quad \; \nvDash_u^{\mathcal{M}^{v}} B \;\; \text{o} \;\; \vDash_u^{\mathcal{M}^{v}} C$$
			            Per \hyperlink{defverp}{definizione di verità di una formula in un punto}, posso concludere che, $\forall u \in \mathcal{W}^{v}$:
			            $$\vDash_u^{\mathcal{M}} B \to C \quad \; \text{sse} \quad \; \vDash_u^{\mathcal{M}^{v}} B \to C$$
			      \item $A \equiv \Box B$ (va dimostrato \emph{$\forall u \in \mathcal{W}^{v} (\vDash_u^{\mathcal{M}} \Box B \; \text{sse} \; \vDash_u^{\mathcal{M}^{v}} \Box B)$}). Mostro prima l'implicazione verso sinistra e poi quella verso destra.
			            \begin{enumerate}
				            \item ($\to$): va dimostrato che \emph{$\forall u \in \mathcal{W}^{v}$ (se $\vDash_u^{\mathcal{M}} \Box B$, allora $\vDash_u^{\mathcal{M}^{v}} \Box B$)}. \\
				                  Assumo che $\forall u \in \mathcal{W}^{v} (\vDash_u^{\mathcal{M}} \Box B)$. \\
				                  Sia $z \in \mathcal{W}^{v} (u\mathcal{R}^{v}z)$; per \hyperlink{modgen}{definizione di sottomodello generato}, ho che $u\mathcal{R}z$; quindi, per \hyperlink{defverp}{definizione di verità di una formula in un punto} (cioè $\forall z \in \mathcal{W} (\text{se} \; u\mathcal{R}z \text{, allora} \; \vDash_z^{\mathcal{M}} B)$), ho che $\vDash_z^{\mathcal{M}} B$. \\
				                  Per \latinmath{IH1}, $\forall z \in \mathcal{W}^{v}$:
				                  $$\vDash_z^{\mathcal{M}} B\; \text{sse} \; \vDash_z^{\mathcal{M}^{v}} B$$
				                  Poiché è rispettata la \hyperlink{defverp}{definizione di verità di una formula in un punto} (cioè $\forall z \in \mathcal{W}^v (\text{se} \; u\mathcal{R}^vz \text{, allora} \; \vDash_z^{\mathcal{M}^v} B)$), posso concludere:
				                  $$\vDash_u^{\mathcal{M}^{v}} \Box B$$
				            \item ($\leftarrow$): va dimostrato che \emph{$\forall u \in \mathcal{W}^{v}$ (se $\vDash_u^{\mathcal{M}^{v}} \Box B$, allora $\vDash_u^{\mathcal{M}} \Box B$)}. \\
				                  Assumo che $\forall u \in \mathcal{W}^{v} (\vDash_u^{\mathcal{M}^{v}} \Box B)$. \\
				                  Sia $z \in \mathcal{W} (u\mathcal{R}z)$. Per assunzione ho che $u \in \mathcal{W}^{v}$, quindi so che $v\mathcal{R}^{\star}z$; dunque, per \hyperlink{modgen}{definizione di sottomodello generato}, ho che $z \in \mathcal{W}^{v}$ e $u\mathcal{R}^{v}z$. \\
				                  Quindi, per \hyperlink{defverp}{definizione di verità di una formula in un punto} (cioè $\forall z \in \mathcal{W}^{v} (\text{se} \; u\mathcal{R}^{v}z \text{, allora} \; \vDash_z^{\mathcal{M}^{v}} B)$), ho che $\vDash_z^{\mathcal{M}^{v}} B$. \\
				                  Per \latinmath{IH1}: $$\vDash_z^{\mathcal{M}^{v}} B \quad \; \text{sse} \quad \; \vDash_z^{\mathcal{M}} B$$
				                  Dunque, poiché è rispettata la \hyperlink{defverp}{definizione di verità di una formula in un punto} (cioè $\forall z \in \mathcal{W} (\text{se} \; u\mathcal{R}z \text{, allora} \; \vDash_z^{\mathcal{M}} B)$), posso concludere:
				                  $$\vDash_u^{\mathcal{M}} \Box B$$
			            \end{enumerate}
			      \item $A \equiv \Dmd B$ (va dimostrato \emph{$\forall u \in \mathcal{W}^{v} (\vDash_u^{\mathcal{M}} \Dmd B \; \text{sse} \; \vDash_u^{\mathcal{M}^{v}} \Dmd B)$}). Mostro prima l'implicazione verso sinistra e poi quella verso destra.
			            \begin{enumerate}
				            \item ($\to$): va dimostrato che \emph{$\forall u \in \mathcal{W}^{v}$ (se $\vDash_u^{\mathcal{M}} \Dmd B$, allora $\vDash_u^{\mathcal{M}^{v}} \Dmd B$)}. \\
				                  Assumo che $\forall u \in \mathcal{W}^{v} (\vDash_u^{\mathcal{M}} \Dmd B)$; dunque, per \hyperlink{defverp}{definizione di verità di una formula in un punto}, so che $\exists x \in \mathcal{W} (u\mathcal{R}x \;\, \text{e} \; \vDash_x^{\mathcal{M}} B)$. \\
				                  Per assunzione ho che $u \in \mathcal{W}^{v}$, quindi so che $v\mathcal{R}^{\star}x$; dunque, per \hyperlink{modgen}{definizione di sottomodello generato}, ho che $u\mathcal{R}^{v}x$; per \latinmath{IH1}, ho che $\forall u \in \mathcal{W}^{v}$:
				                  $$\vDash_x^{\mathcal{M}} B\; \text{sse} \; \vDash_x^{\mathcal{M}^{v}} B$$
				                  Quindi, poiché è rispettata la \hyperlink{defverp}{definizione di verità di una formula in un punto} (cioè $\exists x \in \mathcal{W}^{v}(u\mathcal{R}^{v}x \;\, \text{e} \; \vDash_x^{\mathcal{M}^{v}} B)$), posso concludere:
				                  $$\vDash_u^{\mathcal{M}^{v}} \Dmd B$$
				            \item ($\leftarrow$): va dimostrato che \emph{$\forall u \in \mathcal{W}^{v}$ (se $\vDash_u^{\mathcal{M}^{v}} \Dmd B$, allora $\vDash_u^{\mathcal{M}} \Dmd B$)}. \\
				                  Assumo che $\forall u \in \mathcal{W}^{v} (\vDash_u^{\mathcal{M}^{v}} \Dmd B)$; dunque, per \hyperlink{defverp}{definizione di verità di una formula in un punto}, so che $\exists x \in \mathcal{W}^{v} (u\mathcal{R}^{v}x \;\, \text{e} \; \vDash_x^{\mathcal{M}^{v}} B)$. \\
				                  Per \hyperlink{modgen}{definizione di sottomodello generato}, ho che $u\mathcal{R}x$; per \latinmath{IH1}, ho che $\forall u \in \mathcal{W}^{v}$:
				                  $$\vDash_x^{\mathcal{M}^{v}} B\; \text{sse} \; \vDash_x^{\mathcal{M}} B$$
				                  Quindi, poiché è rispettata la \hyperlink{defverp}{definizione di verità di una formula in un punto} (cioè $\exists x \in \mathcal{W}(u\mathcal{R}x \;\, \text{e} \; \vDash_x^{\mathcal{M}} B)$), posso concludere:
				                  $$\vDash_u^{\mathcal{M}} \Dmd B$$
			            \end{enumerate}
		      \end{itemize}
	\end{description}
\end{dimo}

\subsubsubs{Teorema: la convergenza non è modalmente esprimibile}
\begin{theo}
	La convergenza non è modalmente esprimibile:
	$$\nexists A \in \latinmath{fm}^{\Phi} (\mathcal{F} \vDash A \; \text{sse} \; \mathcal{F} \rhd \forall x \forall y \exists z (x\mathcal{R}z \land y\mathcal{R}z))$$
\end{theo}
\begin{dimo}
	[per assurdo] \phantom{ciao} \\
	Assumo, per assurdo, che $\exists A \in \latinmath{fm}^{\Phi} (\mathcal{F} \vDash A \; \text{sse} \; \mathcal{F} \rhd \forall x \forall y \exists z (x\mathcal{R}z \land y\mathcal{R}z))$. \\
	Sia $\mathcal{F} = \langle  \{w, v\}, \{\langle w, w \rangle, \langle v, v\rangle\} \rangle$, con $w \neq v$. \\
	$\mathcal{F}$, quindi, non è convergente, in quanto non rende vera la proprietà: ho che $\mathcal{F} \nvDash A$. Dunque, preso un qualsiasi $\mathcal{F}$-modello $\mathcal{M}$, ho che $\forall x \in \mathcal{W} (\nvDash_x^{\mathcal{M}} A)$; in particolare, $\nvDash_w^{\mathcal{M}} A \;\, \text{e} \; \nvDash_v^{\mathcal{M}} A$. \\
	Ho due casi:
	\begin{enumerate}
		\item Sia $\mathcal{M}^{w}$ il \hyperlink{modgen}{sottomodello generato} da $w$; la sottostruttura  $\mathcal{F}^{w}$ sarà allora convergente, in quanto rende valida la proprietà (nel caso particolare in cui $x=y=z=w$). Quindi, ho che:
		      $$\vDash_w^{\mathcal{M}^{w}} A$$
		      Per il \hyperlink{lemgen}{\textbf{Lemma} precedente}: $$\vDash_w^{\mathcal{M}^{w}} A \quad \; \text{sse} \quad \; \vDash_w^{\mathcal{M}} A$$
		      Ma questo è assurdo, perché avevo che $\nvDash_w^{\mathcal{M}} A$!
		\item Sia $\mathcal{M}^{v}$ il \hyperlink{modgen}{sottomodello generato} da $v$; la sottostruttura  $\mathcal{F}^{v}$ sarà allora convergente,  in quanto rende valida la proprietà (nel caso particolare in cui $x=y=z=v$). Quindi, ho che:
		      $$\vDash_v^{\mathcal{M}^{v}} A$$
		      Per il \hyperlink{lemgen}{\textbf{Lemma} precedente}: $$\vDash_v^{\mathcal{M}^{v}} A \quad \; \text{sse} \quad \; \vDash_v^{\mathcal{M}} A$$
		      Ma questo è assurdo, perché avevo che $\nvDash_v^{\mathcal{M}} A$!
	\end{enumerate}
	Ho contraddizione in entrambi i casi. \\
\end{dimo}

\subsubsubs{Teorema: la connessione non è modalmente esprimibile}
\begin{theo}
	La connessione non è modalmente esprimibile:
	$$\nexists A \in \latinmath{fm}^{\Phi} (\mathcal{F} \vDash A \; \text{sse} \; \mathcal{F} \rhd \forall x \forall y (x\mathcal{R}y \lor y\mathcal{R}x \lor x=y)$$
\end{theo}
\begin{dimo}
	[per assurdo; analoga alla precedente] \phantom{ciao} \\
	Assumo, per assurdo, che $\exists A \in \latinmath{fm}^{\Phi} (\mathcal{F} \vDash A \; \text{sse} \; \mathcal{F} \rhd \forall x \forall y (x\mathcal{R}y \lor y\mathcal{R}x \lor x=y)$. \\
	Sia $\mathcal{F} = \langle  \{w, v\}, \{\langle w, w \rangle, \langle v, v\rangle\} \rangle$, con $w \neq v$. \\
	$\mathcal{F}$, quindi, non è connessa, in quanto non rende vera la proprietà: ho che $\mathcal{F} \nvDash A$. Dunque, preso un qualsiasi $\mathcal{F}$-modello $\mathcal{M}$, ho che $\forall x \in \mathcal{W} (\nvDash_x^{\mathcal{M}} A)$; in particolare, $\nvDash_w^{\mathcal{M}} A \;\, \text{e} \; \nvDash_v^{\mathcal{M}} A$. \\
	Ho due casi:
	\begin{enumerate}
		\item Sia $\mathcal{M}^{w}$ il \hyperlink{modgen}{sottomodello generato} da $w$; la sottostruttura  $\mathcal{F}^{w}$ sarà allora convergente, in quanto rende valida la proprietà (nel caso particolare in cui $x=y=w$). Quindi, ho che:
		      $$\vDash_w^{\mathcal{M}^{w}} A$$
		      Per il \hyperlink{lemgen}{\textbf{Lemma} precedente}: $$\vDash_w^{\mathcal{M}^{w}} A \quad \; \text{sse} \quad \; \vDash_w^{\mathcal{M}} A$$
		      Ma questo è assurdo, perché avevo che $\nvDash_w^{\mathcal{M}} A$!
		\item Sia $\mathcal{M}^{v}$ il \hyperlink{modgen}{sottomodello generato} da $v$; la sottostruttura  $\mathcal{F}^{v}$ sarà allora convergente,  in quanto rende valida la proprietà (nel caso particolare in cui $x=y=v$). Quindi, ho che:
		      $$\vDash_v^{\mathcal{M}^{v}} A$$
		      Per il \hyperlink{lemgen}{\textbf{Lemma} precedente}: $$\vDash_v^{\mathcal{M}^{v}} A \quad \; \text{sse} \quad \; \vDash_v^{\mathcal{M}} A$$
		      Ma questo è assurdo, perché avevo che $\nvDash_v^{\mathcal{M}} A$!
	\end{enumerate}
	Ho contraddizione in entrambi i casi. \\
\end{dimo}
\vspace{12pt}

\subsubsubs{Definizione di p-morfismo tra strutture}
\hypertarget{pmorfstrut}{}
\begin{defin}
	[\emph{p-morfismo \protect\footnote{\enquote{\emph{P}-morfismo} sta per \enquote{\emph{pseudo}-morfismo}.} tra strutture}] Date una \emph{struttura di partenza} $\mathcal{F}_1 = \langle \mathcal{W}_1, \mathcal{R}_1 \rangle$ e una \emph{struttura di arrivo} $\mathcal{F}_2 = \langle \mathcal{W}_2, \mathcal{R}_2 \rangle$, un \emph{p-morfismo} è una funzione che associa i mondi della struttura di partenza a quelli della struttura di arrivo:
	$$f : \mathcal{W}_1 \to \mathcal{W}_2$$
	Essa deve rispettare le seguenti condizioni:
	\begin{enumerate}
		\hypertarget{forth}{}
		\item \emph{Forth condition:}
		      $$\forall w, v \in \mathcal{W}_1 (\text{se}\; w\mathcal{R}_1v \text{, allora} \; f(w)\mathcal{R}_2f(v))$$
		      Questa condizione assicura che sia preservata l'accessibilità della struttura di partenza.
		      \hypertarget{back}{}
		\item \emph{Back condition:}
		      $$\forall w \in \mathcal{W}_1, \forall v \in \mathcal{W}_2 (\text{se} \; f(w)\mathcal{R}_2v \text{, allora} \; \exists u \in \mathcal{W}_1 (w\mathcal{R}_1u \land f(u)=v))$$
	\end{enumerate}
\end{defin}
\vspace{12pt}

\subsubsubs{Definizione di p-morfismo tra modelli}
\hypertarget{pmorf}{}
\begin{defin}
	[\emph{p-morfismo tra modelli}] Dati due modelli $\mathcal{M}_1 = \langle \mathcal{W}_1, \mathcal{R}_1, I_1 \rangle$ e $\mathcal{M}_2 = \langle \mathcal{W}_2, \mathcal{R}_2, I_2 \rangle$, un \emph{p-morfismo} è una funzione:
	$$f : \mathcal{M}_1 \to \mathcal{M}_2$$
	Essa deve rispettare le seguenti condizioni:
	\begin{enumerate}
		\item \hyperlink{forth}{\emph{Forth condition}}
		\item \hyperlink{back}{\emph{Back condition}} \\
		      Le condizioni 1 e 2 assicurano che $f$ sia un p-morfismo da $\langle \mathcal{W}_1, \mathcal{R}_1 \rangle$ a $\langle \mathcal{W}_2, \mathcal{R}_2 \rangle$.
		\item $$\forall w \in \mathcal{W}_1 (w \in I_1(p) \; \text{sse} \; f(w) \in I_2(p))$$
		      Questa condizione assicura che venga preservata la funzione di interpretazione della struttura di partenza. \\
	\end{enumerate}
\end{defin}

\subsubsubs{Lemma: verità in un p-morfismo tra modelli}
\hypertarget{lempmorfmod}{}
\begin{lem}
	[verità in un p-morfismo tra modelli] Siano $\mathcal{M}_1 = \langle \mathcal{W}_1, \mathcal{R}_1, I_1 \rangle$ e $\mathcal{M}_2 = \langle \mathcal{W}_2, \mathcal{R}_2, I_2 \rangle$ e sia $f : \mathcal{M}_1 \to \mathcal{M}_2$ il p-morfismo tra essi; sarà allora vero che, $\forall x \in \mathcal{W}_1$:
	$$\vDash_x^{\mathcal{M}_1} A \quad \; \text{sse} \; \quad \vDash_{f(x)}^{\mathcal{M}_2} A$$
\end{lem}
\begin{dimo}
	[per induzione strutturale su $A$] \phantom{ciao}
	\begin{description}
		\item [Base:] \phantom{ciao}
		      \begin{itemize}
			      \item $A \equiv \bot$: \\
			            $\nvDash_x^{\mathcal{M}_1} \bot \; \text{sse} \; \nvDash_{f(x)}^{\mathcal{M}_2} \bot$ \qquad \quad $\bot$ è falso in qualsiasi modello.
			      \item $A \equiv p$: \\
			            Ho $\vDash_x^{\mathcal{M}_1} p$. Per \hyperlink{pmorf}{definizione di p-morfismo tra modelli}, so che:
			            $$x \in I_1(p) \quad \; \text{sse} \; \quad f(x) \in I_2(p)$$
			            Per \hyperlink{defverp}{definizione di verità di una formula in un punto}, posso concludere:
			            $$\vDash_x^{\mathcal{M}_1} p \quad \; \text{sse} \; \quad \; \nvDash_{f(x)}^{\mathcal{M}_2} p$$
		      \end{itemize}
		\item [Passo:] ho un caso per ogni connettivo. \\
		      \latinmath{IH1}: $\forall x \in \mathcal{W}_1 (\vDash_x^{\mathcal{M}_1} B \; \text{sse} \; \vDash_{f(x)}^{\mathcal{M}_2} B$) \\
		      \latinmath{IH2}: $\forall x \in \mathcal{W}_1 (\vDash_x^{\mathcal{M}_1} C \; \text{sse} \; \vDash_{f(x)}^{\mathcal{M}_2} C)$
		      \begin{itemize}
			      \item $A \equiv B \land C$ (va dimostrato \emph{$\forall x \in \mathcal{W}_1 (\vDash_x^{\mathcal{M}_1} B \land C \; \text{sse} \; \vDash_{f(x)}^{\mathcal{M}_2} B \land C)$}): \\
			            Per \hyperlink{defverp}{definizione di verità di una formula in un punto}, ho $\vDash_x^{\mathcal{M}_1} B \land C \; \text{sse} \; \vDash_x^{\mathcal{M}_1} B \; \text{e} \; \vDash_x^{\mathcal{M}_1} C$. \\
			            Per \latinmath{IH1} e \latinmath{IH2}, $\forall x \in \mathcal{W}_1$:
			            $$\vDash_x^{\mathcal{M}_1} B \; \text{e} \; \vDash_x^{\mathcal{M}_1} C \quad \; \text{sse} \quad \; \vDash_{f(x)}^{\mathcal{M}_2} B \; \text{e} \; \vDash_{f(x)}^{\mathcal{M}_2} C$$
			            Per \hyperlink{defverp}{definizione di verità di una formula in un punto}, posso concludere, $\forall x \in \mathcal{W}_1$:
			            $$\vDash_x^{\mathcal{M}_1} B \land C \quad \; \text{sse} \quad \; \vDash_{f(x)}^{\mathcal{M}_2} B \land C$$
			      \item $A \equiv B \lor C$ (analogo a $A \equiv B \land C$; va dimostrato \emph{$\forall x \in \mathcal{W}_1 (\vDash_x^{\mathcal{M}_1} B \lor C \; \text{sse} \; \vDash_{f(x)}^{\mathcal{M}_2} B \lor C)$}): \\
			            Per \hyperlink{defverp}{definizione di verità di una formula in un punto}, ho $\vDash_x^{\mathcal{M}_1} B \lor C \; \text{sse} \; \vDash_x^{\mathcal{M}_1} B \; \text{o} \; \vDash_x^{\mathcal{M}_1} C$. \\
			            Per \latinmath{IH1} e \latinmath{IH2}, $\forall x \in \mathcal{W}_1$:
			            $$\vDash_x^{\mathcal{M}_1} B \; \text{o} \; \vDash_x^{\mathcal{M}_1} C \quad \; \text{sse} \quad \; \vDash_{f(x)}^{\mathcal{M}_2} B \; \text{o} \; \vDash_{f(x)}^{\mathcal{M}_2} C$$
			            Per \hyperlink{defverp}{definizione di verità di una formula in un punto}, posso concludere, $\forall x \in \mathcal{W}_1$:
			            $$\vDash_x^{\mathcal{M}_1} B \lor C \quad \; \text{sse} \quad \; \vDash_{f(x)}^{\mathcal{M}_2} B \lor C$$
			      \item $A \equiv B \to C$ (analogo a $A \equiv B \land C$; va dimostrato \emph{$\forall x \in \mathcal{W}_1 (\vDash_x^{\mathcal{M}_1} B \to C \; \text{sse} \; \vDash_{f(x)}^{\mathcal{M}_2} B \to C)$}): \\
			            Per \hyperlink{defverp}{definizione di verità di una formula in un punto}, ho $\vDash_x^{\mathcal{M}_1} B \to C \; \text{sse} \; \nvDash_x^{\mathcal{M}_1} B \; \text{o} \; \vDash_x^{\mathcal{M}_1} C$. \\
			            Per \latinmath{IH1} e \latinmath{IH2}, $\forall x \in \mathcal{W}_1$:
			            $$\nvDash_x^{\mathcal{M}_1} B \; \text{o} \; \vDash_x^{\mathcal{M}_1} C \quad \; \text{sse} \quad \; \nvDash_{f(x)}^{\mathcal{M}_2} B \; \text{o} \; \vDash_{f(x)}^{\mathcal{M}_2} C$$
			            Per \hyperlink{defverp}{definizione di verità di una formula in un punto}, posso concludere, $\forall x \in \mathcal{W}_1$:
			            $$\vDash_x^{\mathcal{M}_1} B \to C \quad \; \text{sse} \quad \; \vDash_{f(x)}^{\mathcal{M}_2} B \to C$$
			      \item $A \equiv \Box B$ (va dimostrato \emph{$\forall x \in \mathcal{W}_1 (\vDash_x^{\mathcal{M}_1} \Box B \; \text{sse} \; \vDash_{f(x)}^{\mathcal{M}_2} \Box B)$}). Mostro prima l'implicazione verso sinistra e poi quella verso destra.
			            \begin{enumerate}
				            \item ($\to$): va dimostrato che \emph{$\forall x \in \mathcal{W}_1$ (se $\vDash_x^{\mathcal{M}_1} \Box B$, allora $\vDash_{f(x)}^{\mathcal{M}_2} \Box B$)}. \\
				                  Assumo che $\forall x \in \mathcal{W}_1 (\vDash_x^{\mathcal{M}_1} \Box B)$; devo mostrare che $\vDash_{f(x)}^{\mathcal{M}_2} \Box B$. \\
				                  Sia $z \in \mathcal{W}_2 (f(x)\mathcal{R}_2z)$; per \hyperlink{back}{\emph{back condition}}, so che:
				                  $$\exists v \in \mathcal{W}_1 (x\mathcal{R}_1v \; \text{e} \; f(v)=z)$$
				                  Dunque, per \hyperlink{defverp}{definizione di verità di una formula in un punto}, ho che $\vDash_v^{\mathcal{M}_1} B$. Quindi, per \latinmath{IH1}, $\forall v \in \mathcal{W}_1$:
				                  $$\vDash_v^{\mathcal{M}_1} B \quad \; \text{sse} \quad \; \vDash_{z}^{\mathcal{M}_2} B $$
				                  Poiché è rispettata la \hyperlink{defverp}{definizione di verità di una formula in un punto} (cioé $\forall z \in \mathcal{M}_2 (\text{se} \; f(x)\mathcal{R}_2z \text{, allora} \; \vDash_z^{\mathcal{M}_2} B)$), posso concludere:
				                  $$\vDash_{f(x)}^{\mathcal{M}_2} \Box B$$
				            \item ($\leftarrow$): va dimostrato che \emph{$\forall x \in \mathcal{W}_1$ (se $\vDash_{f(x)}^{\mathcal{M}_2} \Box B$, allora $\vDash_x^{\mathcal{M}_1} \Box B$)}. \\
				                  Assumo che $\forall x \in \mathcal{W}_1 (\vDash_{f(x)}^{\mathcal{M}_2} \Box B)$; devo mostrare che $\vDash_x^{\mathcal{M}_1} \Box B$. \\
				                  Sia $w \in \mathcal{W}_1(x\mathcal{R}_1w)$; per \hyperlink{forth}{\emph{forth condition}}, ho che:
				                  $$f(x)\mathcal{R}_2f(w)$$
				                  Dunque, per \hyperlink{defverp}{definizione di verità di una formula in un punto}, ho che $\vDash_{f(w)}^{\mathcal{M}_2} B$. Quindi, per \latinmath{IH1}, $\forall w \in \mathcal{W}_1$:
				                  $$\vDash_{f(w)}^{\mathcal{M}_2} B \quad \; \text{sse} \quad \; \vDash_{w}^{\mathcal{M}_1} B$$
				                  Poiché è rispettata la \hyperlink{defverp}{definizione di verità di una formula in un punto} (cioè $\forall w \in \mathcal{M}_1 (\text{se} \; x\mathcal{R}_1w \text{, allora} \; \vDash_w^{\mathcal{M}_1} B)$), posso concludere:
				                  $$\vDash_x^{\mathcal{M}_1} \Box B$$
			            \end{enumerate}
			      \item $A \equiv \Dmd B$ (va dimostrato che \emph{$\forall x \in \mathcal{W}_1 (\vDash_x^{\mathcal{M}_1} \Dmd B \; \text{sse} \; \vDash_{f(x)}^{\mathcal{M}_2} \Dmd B)$}). Mostro prima l'implicazione verso sinistra e poi quella verso destra.
			            \begin{enumerate}
				            \item ($\to$): va dimostrato che \emph{$\forall x \in \mathcal{W}_1$ (se $\vDash_x^{\mathcal{M}_1} \Dmd B$, allora  $\vDash_{f(x)}^{\mathcal{M}_2} \Dmd B$)}. \\
				                  Assumo che $\forall x \in \mathcal{W}_1 (\vDash_x^{\mathcal{M}_1} \Dmd B)$; devo mostrare che $\vDash_{f(x)}^{\mathcal{M}_2} \Dmd B$. \\
				                  Sia $w \in \mathcal{W}_1 (x\mathcal{R}_1w)$; per assunzione, ho che $\vDash_{w}^{\mathcal{M}_1} B$. Per \hyperlink{forth}{\emph{forth condition}}, ho che:
				                  $$f(x)\mathcal{R}_2f(w)$$
				                  Per \latinmath{IH1} so che, $\forall w \in \mathcal{W}_1$:
				                  $$\vDash_{w}^{\mathcal{M}_1} B \; \text{sse} \; \vDash_{f(w)}^{\mathcal{M}_2} B$$
				                  Poiché è rispettata la \hyperlink{defverp}{definizione di verità di una formula in un punto} (cioè $\exists w \in \mathcal{M}_2 (f(x)\mathcal{R}_2f(w) \; \text{e} \; \vDash_{f(w)}^{\mathcal{M}_2} B)$), posso concludere:
				                  $$\vDash_{f(x)}^{\mathcal{M}_2} \Dmd B$$
				            \item ($\leftarrow$): va dimostrato che \emph{$\forall x \in \mathcal{W}_1$ (se $\vDash_{f(x)}^{\mathcal{M}_2} \Dmd B$, allora $\vDash_x^{\mathcal{M}_1} \Dmd B$)}. \\
				                  Assumo che $\forall x \in \mathcal{W}_1 (\vDash_{f(x)}^{\mathcal{M}_2} \Dmd B)$; devo mostrare che $\vDash_x^{\mathcal{M}_1} \Dmd B$. \\
				                  Sia $z \in \mathcal{W}_2 (f(x)\mathcal{R}_2z)$; per assunzione, ho che $\vDash_{z}^{\mathcal{M}_2} B$. Per \hyperlink{back}{\emph{back condition}}, ho che:
				                  $$\exists v \in \mathcal{W}_1 (x\mathcal{R}_1v \; \text{e} \; f(v)=z)$$
				                  Quindi, per \latinmath{IH1}, $\forall v \in \mathcal{W}_1$:
				                  $$\vDash_{z}^{\mathcal{M}_2} B \quad \; \text{sse} \quad \; \vDash_v^{\mathcal{M}_1} B$$
				                  Poiché è rispettata la \hyperlink{defverp}{definizione di verità di una formula in un punto} (cioé $\exists v \in \mathcal{M}_2 (x\mathcal{R}_1v \; \text{e} \; \vDash_v^{\mathcal{M}_1} B)$), posso concludere:
				                  $$\vDash_x^{\mathcal{M}_1} \Dmd B$$
			            \end{enumerate}
		      \end{itemize}
	\end{description}
\end{dimo}

\subsubsubs{Definizione di p-morfismo suriettivo}
\begin{defin}
	[\emph{p-morfismo suriettivo \protect\footnote{Una funzione si dice \emph{suriettiva} quando ogni elemento del codominio (insieme di arrivo) è immagine (sottoinsieme ottenuto applicando la funzione al dominio, è quindi un sottoinsieme del codominio) di almeno un elemento del dominio (insieme di partenza).}}]
	Un p-morfismo $f : \mathcal{W}_1 \to \mathcal{W}_2$ si dice \emph{suriettivo} se e solo se:
	$$\forall x \in \mathcal{W}_2, \exists y \in \mathcal{W}_1 (x=f(y))$$
	Rispetto a un p-morfismo generico, dunque, un p-morfismo suriettivo garantisce che ogni mondo nel dominio di arrivo abbia un corrispettivo nel dominio di partenza.
\end{defin}
\vspace{12pt}

\subsubsubs{Lemma: p-morfismo suriettivo tra modelli}
\hypertarget{lempmorfsur}{}
\begin{lem}
	[p-morfismo suriettivo tra modelli] Se $f$ è un p-morfismo suriettivo da $\mathcal{M}_1$ a $\mathcal{M}_2$, allora è vero che:
	$$\vDash^{\mathcal{M}_1} A \quad \; \text{sse} \quad \; \vDash^{\mathcal{M}_2} A$$
\end{lem}
\noindent Questo non era vero nel \hyperlink{lempmorfmod}{generico p-morfismo tra modelli}, che conservava la verità solo in specifici punti.

\begin{dimo}
	\phantom{ciao} \\
	Per \hyperlink{vermod}{definizione di verità di un modello}, so che \emph{$\vDash^{\mathcal{M}_1} A \; \text{sse} \; \forall w \in \mathcal{W}_1 (\vDash_w^{\mathcal{M}_1} A)$}. Per il \hyperlink{lempmorfmod}{\textbf{Lemma} del generico p-morfismo tra modelli}, so che:
	$$\forall w \in \mathcal{W}_1 (\vDash_w^{\mathcal{M}_1} A) \; \quad \text{sse} \quad \; \forall f(w) \in \mathcal{W}_2 (\vDash_{f(w)}^{\mathcal{M}_2} A)$$
	dove $f(\mathcal{M}_1) = \{y \in \mathcal{M}_2 : \exists w \in \mathcal{M}_1 (f(w)=y)\}$. Dato che $f$ è suriettiva, questo equivale a:
	$$\forall w \in \mathcal{W}_1 (\vDash_{w}^{\mathcal{M}_1} A) \quad \; \text{sse} \; \quad \forall s \in \mathcal{W}_2 (\vDash_{s}^{\mathcal{M}_2} A)$$
	Quindi, per \hyperlink{vermod}{definizione di verità di un modello}, posso concludere:
	$$\vDash^{\mathcal{M}_1} A \quad \; \text{sse} \quad \; \vDash^{\mathcal{M}_2} A$$
\end{dimo}
\vspace{12pt}

\subsubsubs{Lemma: modello generato da un p-morfismo}
\hypertarget{pmorfgen}{}
\begin{lem}
	[modello generato da un p-morfismo] Se $f$ è un p-morfismo da $\mathcal{F}_1$ a $\mathcal{F}_2$, allora per ogni $\mathcal{F}_2$-modello $\mathcal{M}_2$ esiste un $\mathcal{F}_1$-modello $\mathcal{M}_1$ tale $f$ è un \hyperlink{pmorf}{p-morfismo tra modelli} tra $\mathcal{M}_1$ e $\mathcal{M}_2$ \footnote{In altre parole, questo Lemma afferma che, comunque prese due strutture e il p-morfismo tra esse, per ogni modello della struttura di arrivo esiste un modello della struttura di partenza che rende il p-morfismo tra le strutture un p-morfismo tra modelli.}.
\end{lem}

\begin{dimo}
	\phantom{ciao} \\
	Esiste una funzione di interpretazione che rispetta la \hyperlink{pmorf}{definizione di p-morfismo tra modelli}:
	$$I_1(p): = \{x \in \mathcal{W}_1 : f(x) = I_2(p)\}$$
\end{dimo}

\subsubsubs{Lemma: p-morfismo suriettivo tra strutture}
\hypertarget{lempmorfsurstrut}{}
\begin{lem}
	[p-morfismo suriettivo tra strutture] Se $f$ è un p-morfismo suriettivo da $\mathcal{F}_1$ a $\mathcal{F}_2$, allora è vero che:
	$$\text{Se} \; \mathcal{F}_1 \vDash A \text{, allora} \; \mathcal{F}_2 \vDash A$$
\end{lem}
\begin{dimo}
	[per contrapposizione] Va dimostrato che \emph{se $\mathcal{F}_2 \nvDash A$, allora $\mathcal{F}_1 \nvDash A$}. \\
	Assumo che $\mathcal{F}_2 \nvDash A$. Di conseguenza, poiché non è rispettata la \hyperlink{valstrut}{definizione di validità in una struttura}, so che esiste un $\mathcal{F}_2$-modello $\mathcal{M}_2$ tale che: $$\nvDash^{\mathcal{M}_2} A$$
	Quindi, per il \hyperlink{pmorfgen}{\textbf{Lemma} del modello generato da un p-morfismo} so che esiste un $\mathcal{F}_1$-modello $\mathcal{M}_1$ tale che:
	$$f : \mathcal{M}_1 \to \mathcal{M}_2$$
	So che $f$ è suriettiva; quindi, per il \hyperlink{lempmorfsur}{\textbf{Lemma} della verità in un p-morfismo suriettivo tra modelli}, da $\nvDash^{\mathcal{M}_2} A$ segue che $\nvDash^{\mathcal{M}_1} A$. Posso quindi concludere:
	$$\mathcal{F}_1 \nvDash A$$
\end{dimo}

\subsubsubs{Teorema: l'irriflessività non è modalmente esprimibile}
\begin{theo}
	L'irriflessività non è modalmente esprimibile:
	$$\nexists A \in \latinmath{fm}^{\Phi} (\mathcal{F} \vDash A \; \text{sse} \; \mathcal{F} \rhd \forall x \in \mathcal{W} \lnot(x\mathcal{R}x))$$
\end{theo}
\begin{dimo}
	[per assurdo] \phantom{ciao} \\
	Assumo, per assurdo, che $\exists A \in \latinmath{fm}^{\Phi} (\mathcal{F} \vDash A \; \text{sse} \; \mathcal{F} \rhd \forall x \in \mathcal{W} \lnot(x\mathcal{R}x))$. \\
	Costruisco un p-morfismo che mappi tutti i numeri naturali su 0, dunque tra due strutture $\mathcal{F}_1 = \langle \mathbb{N}, < \rangle$ e $\mathcal{F}_2 = \langle \{0\}, \{\langle 0, 0\rangle\} \rangle$:
	$$f : \mathbb{N} \to \{0\}$$
	Ho quindi che $\forall n \in \mathbb{N} (f(n) = 0)$. \\
	Verifico che $f$ soddisfi le condizioni dei \hyperlink{pmorfstrut}{p-morfismi tra strutture}:
	\begin{enumerate}
		\item \hyperlink{forth}{\emph{Forth condition}}\emph{:} \emph{$\forall n, m \in \mathbb{N} (\text{se} \; n<m \text{, allora} \; f(n)\mathcal{R}_2f(m))$} è vera, in quanto $f(n) = f(m) = 0$.
		\item \hyperlink{back}{\emph{Back condition}}\emph{:} $\forall n \in \mathbb{N} (f(n)= 0)$ per definizione di $f$. Sia $0\mathcal{R}_2y$, allora $y=0$. Poiché $n<n+1 \land f(n+1)=0$, la condizione è soddisfatta.
	\end{enumerate}
	So che $f$ è un \hyperlink{lempmorfsurstrut}{p-morfismo suriettivo tra strutture} da $\mathcal{F}_1$ a $\mathcal{F}_2$ (perchè non c'è nessun mondo di $\mathcal{F}_2$ che non abbia un corrispettivo in $\mathcal{F}_1$), dunque:
	\begin{itemize}
		\item Poichè $\mathcal{F}_1$ è irriflessiva, per il \hyperlink{lempmorfsurstrut}{\textbf{Lemma} precedente} anche $\mathcal{F}_2 \vDash A$ è irriflessiva;
		\item Per sua definizione, $\mathcal{F}_2$ è riflessiva: $\mathcal{F}_2 \nvDash A$.
	\end{itemize}
	Ho sia $\mathcal{F}_2 \vDash A$ che $\mathcal{F}_2 \nvDash A$, il che è assurdo. \\
\end{dimo}

\subsubsubs{Teorema: l'antisimmetria non è modalmente esprimibile}
\begin{theo}
	L'antisimmetria non è modalmente esprimibile:
	$$\nexists A \in \latinmath{fm}^{\Phi} (\mathcal{F} \vDash A \; \text{sse} \; \mathcal{F} \rhd \forall x, y (x\mathcal{R}y \land y\mathcal{R}x \to y=x)$$
\end{theo}
\begin{dimo}
	[per assurdo] \phantom{ciao} \\
	Assumo, per assurdo, che $\exists A \in \latinmath{fm}^{\Phi} (\mathcal{F} \vDash A \; \text{sse} \; \mathcal{F} \rhd \forall x, y (x\mathcal{R}y \land y\mathcal{R}x \to y=x)$. \\
	Costruisco un p-morfismo tra due strutture $\mathcal{F}_1 = \langle \mathbb{N}, \leq \rangle$ e $\mathcal{F}_2 = \langle \{p, d\}, \{\langle p, p \rangle, \langle p, d \rangle, \langle d, p \rangle, \langle d,d \rangle\} \rangle$:
	$$f : \mathbb{N} \to \{p, d\}$$
	Ho quindi che:
	$$\forall n \in \mathbb{N}, f(n) = \left\{
		\begin{aligned}
			p & \quad \text{se $n$ è pari}    \\
			d & \quad \text{se $n$ è dispari}
		\end{aligned}
		\right.$$
	Verifico che $f$ soddisfi le condizioni dei \hyperlink{pmorfstrut}{p-morfismi tra strutture}:
	\begin{enumerate}
		\item \hyperlink{forth}{\emph{Forth condition}}\emph{:} \emph{\text{se }$n \leq m \text{, allora} \; f(n)\mathcal{R}_2f(m)$} è vera, poiché $p$ e $d$ sono sempre in relazione tra loro.
		\item \hyperlink{back}{\emph{Back condition}}\emph{:} dato che $n \in \mathbb{N}$, ci sono due immagini possibili:
		      \begin{enumerate}
			      \item $f(n) = d$: \\
			            per $d\mathcal{R}_2p$, esiste $y : f(y) = p$ (basta prendere $y = n+1$); \\
			            per $d\mathcal{R}_2d$, basta prendere $n$, dato che $n \leq n$.
			      \item $f(n) = p$ (analogo): \\
			            per $p\mathcal{R}_2d$, esiste $y : f(y) = d$ (basta prendere $y = n+1$); \\
			            per $p\mathcal{R}_2p$, basta prendere $n$, dato che $n \leq n$.
		      \end{enumerate}
	\end{enumerate}
	So che $f$ è un \hyperlink{lempmorfsurstrut}{p-morfismo suriettivo} da $\mathcal{F}_1$ a $\mathcal{F}_2$ (perchè non c'è nessun mondo di $\mathcal{F}_2$ che non abbia un corrispettivo in $\mathcal{F}_1$), dunque:
	\begin{itemize}
		\item Poichè $\mathcal{F}_1$ è antisimmetrica, per il \hyperlink{lempmorfsurstrut}{\textbf{Lemma} precedente} anche $\mathcal{F}_2 \vDash A$ è antisimmetrica;
		\item Per sua definizione, $\mathcal{F}_2$ è simmetrica: $\mathcal{F}_2 \nvDash A$.
	\end{itemize}
	Ho sia $\mathcal{F}_2 \vDash A$ che $\mathcal{F}_2 \nvDash A$, il che è assurdo. \\
\end{dimo}

\newpage
\subsection{Calcoli assiomatici}
\noindent A differenza della deduzione naturale o dei \hyperlink{seq}{sequenti}, i calcoli assiomatici utilizzano un numero di assiomi relativamente ampio e pochissime regole. Nei calcoli assiomatici è molto difficile derivare dimostrazioni; il loro vantaggio è che sono invece molto compatti per introdurre logiche. La logica K, ad esempio, può essere definita come segue: \\

\subsubsubs{Semantizzazione di \latinmath{K}}
\hypertarget{ksem}{}
\noindent \textbf{\emph{Semantizzazione di \latinmath{K}}}:
\begin{description}
	\item[Assiomi:] \phantom{ciao}
	      \begin{enumerate}
		      \item \latinmath{TAUT}: tutte le tautologie.
		      \item $K$: $\Box (A \to B) \to (\Box A \to \Box B)$
		      \item $\Box A \leftrightarrow \lnot \Dmd \lnot A$
	      \end{enumerate}
	\item[Regole:] \phantom{ciao}
	      \begin{enumerate}
		      \item $M \! P$:
		            \vspace*{-4pt}
		            \begin{mathpar}
			            \inferrule{\Gamma \vdash A \to B \\ \Gamma \vdash A}{\Gamma \vdash B}
			            \vspace*{-14pt}
		            \end{mathpar}
		      \item $N$:
		            \vspace*{-4pt}
		            \begin{mathpar}
			            \inferrule{\emptyset \vdash A}{\emptyset \vdash \Box A}
			            \vspace*{-14pt}
		            \end{mathpar}
	      \end{enumerate}
\end{description}

\noindent Dove \enquote{$\vdash$} significa \enquote{essere derivabile} ed è definito come segue: \\

\subsubsubs{Definizione di $\vdash$}
\hypertarget{defvdash}{}
\begin{defin}
	[$\vdash$] Sia \latinmath{L} una logica modale normale; diremo che una derivazione di una formula $A$ a partire da un insieme di formule $\Gamma$ è una sequenza finita di formule $\alpha_1, \alpha_2, \ldots, \alpha_n$ tali che:
	\begin{enumerate}
		\item $\alpha_n \equiv A$;
		\item $\alpha_{i(\leq n)} = \left\{\begin{aligned}
				       & \text{un assioma di \latinmath{L};}                                                                \\
				       & \in \Gamma\text{;}                                                                                 \\
				       & \text{è ottenuto da formule che lo precedono (nella sequenza) usando una regola di \latinmath{L}.}
			      \end{aligned}\right.$ \\
	\end{enumerate}
	Scriveremo \enquote{$\Gamma \vdash_{\latinmath{L}} A$} se esiste una derivazione di $A$ da $\Gamma$ di \latinmath{L}.
\end{defin}

\noindent Le derivazioni possono svolgersi linearmente o ad albero; quelle ad albero sono più compatte, dunque da preferire. \\

\subsubsubs{Definizione di teorema in L}
\begin{defin}
	[\emph{Teorema in \latinmath{L}}] Una formula $A$ è un teorema di \latinmath{L} (ovvero $\vdash_{\latinmath{L}} A$) se e solo se:
	$$\emptyset \vdash_{\latinmath{L}} A$$
\end{defin}
\vspace{12pt}

\subsubsubs{Esempi di derivazioni in \latinmath{K}}
\begin{exm}
	[\emph{Derivazione di teorema di \latinmath{K}}] $\vdash_{\latinmath{K}} \Box \top$:
	\begin{mathpar}
		\inferrule*[Right=$N$]{\emptyset \vdash_{\latinmath{K}} \bot \to \bot}{\emptyset \vdash_{\latinmath{K}} \Box (\bot \to \bot)}
	\end{mathpar}
	Questo è un assioma, in quanto $\in \latinmath{TAUT}$. \\
\end{exm}

\begin{exm}
	[\emph{Derivazione di teorema di \latinmath{K}}] $\vdash_{\latinmath{K}} \Box (A \land B) \to (\Box A \land \Box B)$:
	\vspace*{-12pt}
	\begin{mathpar}
		\inferrule*[Right=$M \! P$]{\inferrule*[Right=\phantom{R}]{\inferrule*[Right=$M \! P$]{\vdash_{\latinmath{K}} \text{Schema} \, K \\ \inferrule*[Right=$N$]{\inferrule*[Right=$Ax$]{\phantom{\vdash_{\latinmath{K}} A \land B \to A}}{\vdash_{\latinmath{K}} A \land B \to A}}{\vdash_{\latinmath{K}} \Box (A \land B \to A)}}{\vdash_{\latinmath{K}} \Box (A \land B) \to \Box A} \\ \inferrule*[Right=$M \! P$]{\vdash_{\latinmath{K}} \text{Schema} \, K \\ \inferrule*[Right=$N$]{\inferrule*[Right=$Ax$]{\phantom{\vdash_{\latinmath{K}} A \land B \to B}}{\vdash_{\latinmath{K}} A \land B \to B}}{\vdash_{\latinmath{K}} \Box (A \land B \to B)}}{\vdash_{\latinmath{K}} \Box (A \land B) \to \Box B}}{\vdash_{\latinmath{K}} (\Box (A \land B) \to \Box A) \land (\Box (A \land B) \to \Box B)} \\ S}{\vdash_{\latinmath{K}} \Box (A \land B) \to (\Box A \land \Box B)}
	\end{mathpar}
	$S = (\alpha\to \beta \land \alpha \to \gamma) \to (\alpha \to \beta \land \gamma)$.
\end{exm}
\vspace{12pt}

\begin{exm}
	[\emph{Derivazione di teorema di \latinmath{K}}] $\vdash_{\latinmath{K}} (\Box A \land \Box B) \to (\Box A \land \Box B)$:
	\vspace*{-12pt}
	\begin{mathpar}
		\inferrule*[Right=$M \! P$]{S \\ \inferrule*[Right=\latinmath{Trans}]{\vdash_{\latinmath{K}} \Box (B \to A \land B) \to (\Box B \to \Box (A \land B)) \\ \inferrule*[Right=$M \! P$]{ \inferrule*[Right=$N$]{\inferrule*[Right=$\latinmath{TAUT}$]{\phantom{\vdash_{\latinmath{K}} A \to (B \to A \land B)}}{\vdash_{\latinmath{K}} A \to (B \to A \land B)}}{\vdash_{\latinmath{K}} \Box (A \to (B \to A \land B))} \\ U}{\vdash_{\latinmath{K}} \Box A \to \Box (B \to A \land B)}}{\vdash_{\latinmath{K}} \Box A \to (\Box B \to \Box (A \land B))}}{\vdash_{\latinmath{K}} (\Box A \land \Box B) \to (\Box A \land \Box B)}
	\end{mathpar}
	$S = \alpha \to (\beta \to \gamma) \to (\alpha \land \beta \to \gamma)$; \\
	$U = \Box (\alpha \to \beta) \to (\Box \alpha \to \Box \beta)$. \\
\end{exm}

\noindent In queste derivazioni si sono utilizzate delle \emph{regole ammissibili}, che sono così definite: \\

\subsubsubs{Definizione di regola ammissibile in \latinmath{L}}
\begin{defin}
	[\emph{Regola ammissibile in \latinmath{L}}]
	Una regola
	\vspace*{-4pt}
	\begin{mathpar}
		\inferrule*[Right=$R$]{\alpha_1, \ldots, \alpha_n}{\beta}
		\vspace*{-14pt}
	\end{mathpar}
	è ammissibile in \latinmath{L} se e solo se esiste una derivazione
	$$\Gamma \vdash_L \beta$$
	assumendo che esistano derivazioni $\Gamma \vdash_L \alpha_n \; \forall \alpha_n \in \latinmath{L}$. \\
\end{defin}

\subsubsubs{Esempio di regola ammissibile in \latinmath{L}}
\begin{exm}
	[Regola ammissibile in \latinmath{L}] La regola $RM$ è ammissibile in \latinmath{L}:
	\begin{mathpar}
		\inferrule*[Right=$RM$]{\vdash_{\latinmath{L}} A \to B}{\vdash_{\latinmath{L}} \Box A \to \Box B}
	\end{mathpar}
	Il procedimento per la derivazione è:
	\begin{enumerate}
		\item assumo che $A \to B$ sia un teorema ($\vdash_{\latinmath{L}} A \to B$);
		\item applico $N$;
		\item prendo l'istanza di $K$ appropriata e applico $M \! P$. \\
	\end{enumerate}
	\begin{mathpar}
		\inferrule*[Right=$M \! P$]{\inferrule*[Right=$N$]{\vdash_{\latinmath{L}} A \to B}{\vdash_{\latinmath{L}} \Box (A \to B)} \\ \vdash_{\latinmath{L}} \Box (A \to B) \to (\Box A \to \Box B)}{\vdash_{\latinmath{L}} \Box A \to \Box B}
	\end{mathpar}
	$RM$ permette di introdurre direttamente la distribuzione di $\Box$. \\
\end{exm}

\subsubsubs{Teoremi di \latinmath{L}}
\begin{theo}
	Per i teoremi di \latinmath{L} valgono le seguenti proprietà:
	\begin{enumerate}
		\item Se $A \in \Gamma$, allora $\Gamma \vdash_{\latinmath{L}} A$.
		\item Se $\Gamma \subseteq \Delta$ e $\Gamma \vdash_{\latinmath{L}} A$, allora $\Delta \vdash_{\latinmath{L}} A$.
		\item Se $\emptyset \vdash_{\latinmath{L}} A$, allora $\Gamma \vdash_{\latinmath{L}} A$.
		\item Se $\emptyset \vdash_{\latinmath{L}} A$, allora $\Gamma \vdash_{\latinmath{L}} \Box A$.
	\end{enumerate}
\end{theo}
\begin{dimo}
	\phantom{ciao}
	\begin{enumerate}
		\item Segue per \hyperlink{defvdash}{definizione di $\vdash$}.
		\item Segue da 1.
		\item Segue da 2, o meglio ne è un caso particolare: $\forall \Gamma, \emptyset \subseteq \Gamma$.
		\item Segue da 3 via necessitazione.
	\end{enumerate}
\end{dimo}

\subsubsubs{Teorema di deduzione assiomatica}
\begin{theo}
	[\emph{Deduzione assiomatica}]
	$$\Gamma, B \vdash_{\latinmath{L}} A \quad \; \text{sse} \; \quad \Gamma \vdash_{\latinmath{L}} B \to A$$
\end{theo}
\begin{dimo}
	Intuitiva. (?)
	$$\begin{array}{rcl}
			            & \alpha_1 &      \\
			\phantom{B} & \vdots   & \! B
		\end{array}$$
	\vspace*{-12pt}
	\begin{mathpar}
		\inferrule{\inferrule{\alpha_n}{A}}{\Gamma \vdash_{\latinmath{L}} B \to A}
	\end{mathpar}
\end{dimo}

La validità del teorema di deduzione dipende dal modo in cui si definisce la necessitazione nella \hyperlink{ksem}{semantizzazione} \footnote{La validità di questo teorema è \enquote{discussa}; Orlandelli ne è un sostenitore, ma altre persone che studiano logiche modali, tra cui Giovanna Corsi, non la condividono: è questo il motivo per cui non compare nel testo in bibliografia.}. Senza l'applicabilità ristretta a $\emptyset$, si avrebbe infatti:
\vspace{-4pt}
\begin{mathpar}
	\inferrule*[Right=$\latinmath{DED}$]{\inferrule*[Right=$N$]{A \vdash A}{A \vdash \Box A}}{\vdash A \to \Box A}
	\vspace*{-14pt}
\end{mathpar}
che è una conclusione chiaramente falsa. Dunque, definendo la necessitazione in modo diverso il teorema di deduzione non sarebbe valido. \\

\subsubsubs{Teorema di validità}
\begin{theo}
	[\emph{Validità/Correttezza/Soundness}] Se una formula $A$ è un teorema di \latinmath{K}, allora è valida:
	$$\text{Se} \; \Gamma \vdash_{\latinmath{K}} A \text{, allora} \; \Gamma \vDash A$$
\end{theo}
\begin{dimo}
	[per induzione sulla lunghezza della derivazione di $\Gamma \vdash_{\latinmath{K}} A$] \phantom{ciao}
	\begin{description}
		\item [Base:] $n = 1$
		      \begin{itemize}
			      \item Ho due casi:
			            \begin{enumerate}
				            \item $A$ è un assioma $Ax$: \\
				                  \vspace*{-12pt}

				                  $A \left\{\begin{aligned}
						                   & \in \latinmath{TAUT} \\
						                   & \equiv K
					                  \end{aligned}\right.$.
				                  \vspace{4pt}
				            \item $A \in \Gamma$: \\
				                  $\Gamma = \Gamma', A$ \\
				                  $\Gamma, A \vDash A$.
			            \end{enumerate}
		      \end{itemize}
		\item [Passo:] $n=k+1$
		      \begin{itemize}
			      \item Ho due casi:
			            \begin{enumerate}
				            \item $M \! P$ (preserva la verità in ogni punto di ogni modello):
				                  \vspace*{-8pt}
				                  $$\begin{array}{cl}
						                  B \to A & \quad \Gamma \vDash B \to A \\
						                  \vdots  &                             \\
						                  B       & \quad \Gamma \vDash B       \\
						                  \vdots  &                             \\
						                  A       &                             \\
					                  \end{array}$$
				                  \vspace*{-16pt}
				            \item $N$ (è vera in ogni punto di ogni modello):
				                  \vspace*{-8pt}
				                  $$\vdots$$
				                  \vspace*{-30pt}
				                  \begin{mathpar}
					                  \inferrule*[Right=$N \qquad \emptyset \vDash \Box B$]{\emptyset \vdash B}{\Box B}
				                  \end{mathpar}
			            \end{enumerate}
		      \end{itemize}
	\end{description}
\end{dimo}


\newpage
\section{Sequenti}
\noindent Il calcolo dei sequenti permette di esprimere legami tra proposizioni complesse. Le sue prime formalizzazioni sono dovute alle ricerche del matematico e logico \href{https://it.wikipedia.org/wiki/Gerhard_Gentzen}{Gerhard Gentzen}.

Le regole dei calcoli dei sequenti sono strutturate su due piani: ciò che si trova nella parte inferiore è chiamato \emph{conclusione}, mentre ciò che si trova nella parte superiore è chiamato \emph{premessa} (o \emph{premesse}, a seconda che siano una o più). \\

\subsubsubs{Definizione di multiinsieme}
\hypertarget{multi}{}
\begin{defin}
	[\emph{Multiinsieme}]
	Un multiinsieme è un insieme i cui elementi possono essere ripetuti (ovvero in cui sono ammesse ripetizioni) \footnote{Questo significa che le ripetizioni contano! Per eliminare formule ripetute abbiamo bisogno di regole apposite.}. Non conta, invece, l'ordine degli elementi. \\

	La classificazione è:
	\begin{itemize}
		\item \emph{Insiemi:} $\{A, B\} = \{A, A, B\}$
		\item \emph{Multiinsiemi:} $\{A, B\} = \{B, A\} \neq \{A, A, B\}$
		\item \emph{Sequenze:} $\{A, B\} \neq \{B, A\}$ \\
	\end{itemize}
\end{defin}
\subsubsubs{Definizione di sequente}
\hypertarget{seq}{}
\begin{defin}
	[\emph{Sequente}]
	Un sequente è una coppia ordinata $$\Gamma \To \Delta$$ in cui $\Gamma$ e $\Delta$ sono multiinsiemi finiti \footnote{Questo è vero nella visione della logica classica. In logica intuizionista, $\Delta$ è formato da \emph{una sola} proposizione.}.
\end{defin}

\newpage
\hypertarget{LK}{}
\subsection{Calcolo \latinmath{LK}}
\noindent Il calcolo \latinmath{LK}, o calcolo di Gentzen, è un calcolo dei \hyperlink{seq}{sequenti} per la logica classica. \`{E} definito da:
\begin{itemize}
	\subsubsubs{Sequenti iniziali}
	\item \emph{Sequenti iniziali:}
	      $$\begin{aligned}
			       & A \To A \qquad \text{e} \qquad
			       & \bot \To A
			      \subsubsubs{Regole logiche}\end{aligned}$$
	\item \emph{Regole logiche:}

	      \vspace*{-24pt}
	      \begin{minipage}[t]{0.48\textwidth}
		      \centering
		      \begin{mathpar}
			      \inferrule*[right=$L\land_i \; (\star)$]{A_i, \Gamma \To \Delta}{A_1 \land A_2, \Gamma \To \Delta} \\
			      \inferrule*[right=$L\lor$]{A, \Gamma \To \Delta \\ B, \Gamma \To \Delta}{A \lor B, \Gamma \To \Delta} \\
			      \inferrule*[right=$L\to$]{\Gamma \To \Delta, A \\ B, \Pi \To \Sigma}{A \to B, \Gamma, \Pi \To \Delta, \Sigma}
		      \end{mathpar}
	      \end{minipage}
	      \hspace*{-2cm}
	      \begin{minipage}[t]{0.48\textwidth}
		      \centering
		      \begin{mathpar}
			      \inferrule*[right=$R\land$]{\Gamma \To \Delta, A \\ \Gamma \To \Delta, B}{\Gamma \To \Delta, A \land B} \\
			      \inferrule*[right=$R\lor_i \; (\star)$]{\Gamma \To \Delta, A_i}{\Gamma \To \Delta, A_1 \lor A_2} \\
			      \inferrule*[right=$R\to$]{A, \Gamma \To \Delta, B}{\Gamma \To \Delta, A \to B} \\
		      \end{mathpar}
	      \end{minipage} \\
	      ($\star$) $i \in \{1, 2\}$
	      \subsubsubs{Regole strutturali}
	      \hypertarget{cut}{} \hypertarget{cont}{} \hypertarget{weak}{}
	\item \emph{Regole strutturali:}

	      \vspace*{-24pt}
	      \begin{minipage}[t]{0.48\textwidth}
		      \centering
		      \begin{mathpar}
			      \inferrule*[right=$LW$]{\Gamma \To \Delta}{A, \Gamma \To \Delta} \\
			      \inferrule*[right=$LC$]{A, A \Gamma \To \Delta}{A, \Gamma \To \Delta} \\
		      \end{mathpar}
	      \end{minipage}
	      \hspace*{-2cm}
	      \begin{minipage}[t]{0.48\textwidth}
		      \centering
		      \begin{mathpar}
			      \inferrule*[right=$RW$]{\Gamma \To \Delta}{\Gamma \To \Delta, A} \\
			      \inferrule*[right=$RC$]{\Gamma \To \Delta, A, A}{\Gamma \To \Delta, A}
		      \end{mathpar}
	      \end{minipage} \\
	      \centering
	      \begin{mathpar}
		      \inferrule*[right=\latinmath{cut}]{\Gamma \To \Delta, A \\ A, \Pi \To \Sigma}{\Gamma, \Pi \To \Delta, \Sigma}
	      \end{mathpar} \raggedright

	      $C$ = Contrazione (\emph{contraction}), $W$ = Indebolimento (\emph{weakening}).
	      \subsubsubs{Teorema: eliminabilità della regola di taglio}
	\item \textbf{Teorema}:
	      $$\vdash_{\latinmath{LK}} \Gamma \To A \quad \; \text{\emph{sse}} \; \quad \vdash_{\latinmath{LK}-\,\latinmath{cut}} \Gamma \To A$$
	      in cui si afferma l'eliminabilità della regola di taglio (\latinmath{cut}). Intuitivamente, ciò che fa il \latinmath{cut} è eliminare un passaggio intermedio in una derivazione.
\end{itemize}
A differenza del \latinmath{cut}, le regole di contrazione e indebolimento non sono eliminabili. La presenza della regola di contrazione, tuttavia, rende questo calcolo indecidibile, poiché può essere applicata infinite volte, in quanto la formula $A$ di partenza viene conservata \footnote{In maniera molto simile a quanto accade nel calcolo per la logica proposizionale con la regola $L\forall$.}. Anche il \latinmath{cut} presenterebbe questo problema, ma è eliminabile in questo calcolo (in altre parole, qualsiasi derivazione che lo utilizzi può essere riscritta senza utilizzarlo).

Nessuna regola di \latinmath{LK} è invertibile. \\

\subsubsubs{Necessarietà dell'indebolimento}
\begin{theo}
	[Necessarietà dell'indebolimento in \latinmath{LK}] Le regole di indebolimento non sono eliminabili in \latinmath{LK}.
\end{theo}
\begin{dimo}
	Per derivare \enquote{$\To A \to (B \to A)$} (\emph{a fortiori}) è necessario utilizzare la regola $LW$:
	\vspace*{-12pt}
	\begin{mathpar}
		\inferrule*[Right=$R\to$]{\inferrule*[Right=$R\to$]{\inferrule*[Right=$LW$]{\inferrule*[Right=$Ax$]{\phantom{A \To A}}{A \To A}}{A, B \To A}}{A \To B \to A)}}{\To A \to (B \to A)}
	\end{mathpar}
\end{dimo}

\subsubsubs{Necessarietà della contrazione}
\begin{theo}
	[Necessarietà della contrazione in \latinmath{LK}] Le regole di contrazione non sono eliminabili in \latinmath{LK}.
\end{theo}

\begin{dimo}
	Per derivare \enquote{$\To A \lor \lnot A$} è necessario utilizzare la regola $RC$ (in particolare, qui $A \equiv p$):
	\vspace*{-12pt}
	\begin{mathpar}
		\inferrule*[Right=$RC$]{\inferrule*[Right=$R\lor$]{\inferrule*[Right=$R\lor$]{\inferrule*[Right=$R\to$]{\inferrule*[Right=$RW$]{\inferrule*[Right=$Ax$]{\phantom{p \lor \lnot p}}{p \To p}}{p \To p, \bot}}{\To p, p \to \bot}}{\To p, p \lor \lnot p}}{\To p \lor \lnot p, p \lor \lnot p}}{\To p \lor \lnot p}
	\end{mathpar}
\end{dimo}

\newpage
\subsection{Calcolo \latinmath{G3cp}}
\noindent Il calcolo \latinmath{G3cp} è un calcolo dei \hyperlink{seq}{sequenti} per la logica proposizionale classica, introdotto da \href{https://en.wikipedia.org/wiki/Sara_Negri}{Sara Negri} e \href{https://de.wikipedia.org/wiki/Jan_von_Plato}{Jan von Plato} nel loro testo \href{https://www.cambridge.org/core/books/structural-proof-theory/487F9F5F1E6174867B458B819043C36B}{\emph{Structural Proof Theory}} del 2001.

\`{E} una variante di  \hyperlink{LK}{\latinmath{LK}}, ma differisce da quest'ultimo in quanto tutte le sue regole sono invertibili. Ciò permette una lettura semantica del calcolo, dal basso verso l'alto, in cui tutte le formule a sinistra (di $\To$) sono vere e tutte quelle a destra sono false \footnote{Questo spiega la mancanza di una regola $R\bot$: non c'è ragione di aggiungere $\bot$ a destra.}.

\`{E} definito da:
\begin{itemize}
	\subsubsubs{Sequenti iniziali}
	\item \emph{Sequenti iniziali:}
	      \subsubsubs{Regole logiche} \hypertarget{Lbotcp}{} $$p, \Gamma \Rightarrow \Delta, p$$
	\item \emph{Regole logiche:}

	      \vspace*{-24pt}
	      \begin{minipage}[t]{0.48\textwidth}
		      \centering
		      \begin{mathpar}
			      \inferrule*[right=$L\bot$]{ }{\bot, \Gamma \To \Delta} \\
			      \inferrule*[right=$L\land$]{A, B, \Gamma \To \Delta}{A \land B, \Gamma \To \Delta} \\
			      \inferrule*[right=$L\lor$]{A, \Gamma \To \Delta \\ B, \Gamma \To \Delta}{A \lor B, \Gamma \To \Delta} \\
			      \inferrule*[right=$L\to$]{\Gamma \To \Delta, A \\ B, \Gamma \To \Delta}{A \to B, \Gamma, \To \Delta}
		      \end{mathpar}
	      \end{minipage}
	      \hspace*{-2cm}
	      \begin{minipage}[t]{0.48\textwidth}
		      \centering
		      \begin{mathpar}
			      \phantom{\inferrule*[right=$L\bot$]{ }{\bot \To}} \\
			      \inferrule*[right=$R\land$]{\Gamma \To \Delta, A \\ \Gamma \To \Delta, B}{\Gamma \To \Delta, A \land B} \\
			      \inferrule*[right=$R\lor$]{\Gamma \To \Delta, A, B}{\Gamma \To \Delta, A \lor B} \\
			      \inferrule*[right=$R\to$]{A, \Gamma \To \Delta, B}{\Gamma \To \Delta, A \to B} \\
		      \end{mathpar}
	      \end{minipage}

	      \subsubsubs{Regole di abbreviazione}
	\item \emph{Regole di abbreviazione:}

	      \vspace*{-24pt}
	      \begin{minipage}[t]{0.48\textwidth}
		      \centering
		      \begin{mathpar}
			      \inferrule*[right=$L\lnot$]{\Gamma \To \Delta, p}{\lnot p, \Gamma \To \Delta} \\
		      \end{mathpar}
	      \end{minipage}
	      \hspace*{-2cm}
	      \begin{minipage}[t]{0.48\textwidth}
		      \centering
		      \begin{mathpar}
			      \inferrule*[right=$R\lnot$]{p, \Gamma \To \Delta}{\Gamma \To \Delta, \lnot p} \\
		      \end{mathpar}
	      \end{minipage} \\ \raggedright
\end{itemize}

\subsubsection{Metateoria di \latinmath{G3cp}}
\subsubsubs{Definizione di contesti, formule principali, formule attive}
\begin{defin}
	[\emph{Contesti, formule principali, formule attive}]
	Nelle regole di \latinmath{G3cp}:
	\begin{itemize}
		\item i \hyperlink{multi}{multiinsiemi} $\Gamma$ e $\Delta$ sono chiamati \emph{contesti};
		\item le formule che occorrono nella conclusione e non fanno parte di $\Gamma$ o $\Delta$ sono chiamate \emph{formule principali};
		\item le formule che occorrono nelle premesse, non occorrono nella conclusione e non fanno parte di $\Gamma$ o $\Delta$ sono chiamate \emph{formule attive}. \\
	\end{itemize}
\end{defin}

\subsubsubs{Definizione di \latinmath{G}-derivazione}
\begin{defin}
	[\emph{\latinmath{G}$\,$-$\,$derivazione} \protect\footnote{Viene usata la dicitura \latinmath{G} in quanto questa definizione si applica a tutti i calcoli basati su \latinmath{G3}.}]
	Una \latinmath{G}-derivazione \enquote{$\latinmath{G} \vdash \Gamma \To \Delta$} di $\Gamma \To \Delta$ è un albero \footnote{Un albero è una struttura con una radice e nodi che si sviluppano verso l'alto. Un nodo che si trova sopra un altro è chiamato \emph{figlio} del nodo sottostante. Un nodo che non ha nessun nodo soprastante è chiamato \emph{foglia}. Il concetto di \enquote{albero di sequenti} presenta diverse analogie con quello di \enquote{albero genealogico}.} finito di sequenti tale che:
	\begin{enumerate}
		\item La radice è $\Gamma \To \Delta$;
		\item Le foglie sono sequenti iniziali o conclusioni di \hyperlink{Lbotcp}{$L\bot$};
		\item Ogni nodo cresce secondo le regole di \latinmath{G}, ovvero è ottenuto dai suoi figli tramite una regola di \latinmath{G}. \\
	\end{enumerate}
\end{defin}

\subsubsubs{Definizione di profondità di una derivazione}
\begin{defin}
	[\emph{Profondità di una derivazione}]
	La profondità di una derivazione $D$, che si indica con $p(D)$, è il numero di nodi di un ramo massimale (cioè quello di lunghezza maggiore) in $D$, a cui va sottratto 1. \\
\end{defin}

\subsubsubs{Definizione di regola ammissibile}
\begin{defin}
	[\emph{Regola ammissibile}]
	Una regola
	\vspace*{-4pt}
	\begin{mathpar}
		\inferrule*[Right=$R$]{S_1, \ldots, S_n}{S}
		\vspace*{-14pt}
	\end{mathpar}
	è ammissibile in \latinmath{G} se, qualora le sue premesse siano derivabili, anche la sua conclusione lo è:
	$$\text{Se} \; \latinmath{G} \vdash S_1 \ldots \latinmath{G} \vdash S_n \text{, allora} \; \latinmath{G} \vdash S$$
\end{defin}
\vspace{12pt}

\subsubsubs{Definizione di regola \emph{pp}-ammissibile}
\begin{defin}
	[\emph{Regola \emph{pp}-ammissibile}]
	Una regola
	\vspace*{-4pt}
	\begin{mathpar}
		\inferrule*[Right=$R$]{S_1, \ldots, S_n}{S}
		\vspace*{-14pt}
	\end{mathpar}
	è \emph{pp} \footnote{\emph{P}reservando la \emph{p}rofondità.}-ammissibile in \latinmath{G} se, qualora le sue premesse siano derivabili con profondità al più $k$, anche la sua conclusione è derivabile con profondità al più $k$:
	$$\text{Se} \, \latinmath{G} \vdash^k S_1 \ldots \latinmath{G} \vdash^k S_n \text{, allora} \; \latinmath{G} \vdash^k S$$
	Le regole di questo tipo non sono \enquote{utili}, nel senso che non fanno nulla per accelerare le derivazioni. \\
\end{defin}

\subsubsubs{Definizione di regola \emph{pp}-invertibile}
\begin{defin}
	[\emph{Regola \emph{pp}-invertibile}]
	Una regola
	\vspace*{-4pt}
	\begin{mathpar}
		\inferrule*[Right=$R$]{S_1, \ldots, S_n}{S}
		\vspace*{-14pt}
	\end{mathpar}
	è \emph{pp}-invertibile in \latinmath{G} se, qualora la sua conclusione sia derivabile con profondità al più $k$, anche le sue premesse sono derivabili con profondità al più $k$:
	$$\text{Se} \; \latinmath{G} \vdash^k S \text{, allora} \; \latinmath{G} \vdash^k S_i \quad \forall i \leq n$$
\end{defin}

\vspace{12pt}
\subsubsection{Proprietà strutturali di \latinmath{G3cp}}
\latinmath{G3cp} presenta una buona serie di proprietà strutturali:
\begin{enumerate}
	\item I sequenti del tipo $A, \Gamma \To \Delta, A$ sono derivabili in \latinmath{G3cp}.
	\item Le regole di \hyperlink{weak}{indebolimento} ($W$) sono \emph{pp}-ammissibili in \latinmath{G3cp} (non sono primitive, non permettono di derivare nulla, non accelerano le derivazioni).
	\item Le regole di \hyperlink{cont}{contrazione} ($C$) sono \emph{pp}-ammissibili in \latinmath{G3cp} (non sono primitive, non permettono di derivare nulla, non accelerano le derivazioni).
	\item Tutte le regole di \latinmath{G3cp} sono \emph{pp}-invertibili.
	\item La regola di \hyperlink{cut}{taglio} è ammissibile in \latinmath{G3cp}.
	\item \latinmath{G3cp} è corretto e completo rispetto alla conseguenza logica classica:
	      $$\latinmath{G3cp} \vdash \Gamma \To \Delta \quad \; \text{sse} \; \quad \land \Gamma \vDash \lor \Delta$$
\end{enumerate}

\subsubsection{Estensioni di \latinmath{G3cp}}
\noindent Le seguenti regole estendono il calcolo \latinmath{G3cp}:
\begin{description}
	\item [\latinmath{G3K}:]
	      \begin{mathpar}
		      \inferrule*[Right=$LR \Box$]{\Gamma \To \Delta}{\Box \Gamma, \Pi \To \Delta, \Box A}
	      \end{mathpar}
	      con $\Box \Gamma = \{\Box A : A \in \Gamma\}$. Questa regola non è invertibile.
	\item [\latinmath{G3T}:]
	      \begin{mathpar}
		      \inferrule*[Right=$T$]{\Gamma, A \To \Delta}{\Gamma, \Box A \To \Delta}
	      \end{mathpar}
	\item [\latinmath{G345}: ]
	      \begin{mathpar}
		      \inferrule*[Right=$45$]{\Gamma^{\Box} \To A, \Delta^{\Box}}{\Gamma \To \Box A, \Delta}
	      \end{mathpar}
	      con $\Gamma^{\Box} = \{A : A \in \Gamma \: \text{e} \: A \equiv \Box B\}$.
\end{description}
Le estensioni che hanno come assiomi lo schema $5$ o lo schema $B$ non possono essere \hyperlink{cut}{\latinmath{cut}}-\emph{free}.

\newpage
\subsection{Calcoli etichettati}
\noindent I calcoli etichettati differiscono dai calcoli di sequenti per le logiche modali ordinari in quanto permettono di utilizzare nel calcolo la semantica relazionale; consentono dunque di trattare in modo unifome una vasta classe di logiche modali. Per fare ciò, si avvalgono di estensioni del linguaggio chiamate \emph{etichette}, tramite le quali è possibile definire le \emph{formule etichettate}. \\

\subsubsubs{Definizione di insieme di etichette}
\hypertarget{et}{}
\begin{defin}
	[\emph{Insieme di etichette}]
	Le etichette sono oggetti \emph{sintattici} utilizzati per indicare i mondi (i quali sono invece oggetti \emph{semantici} \footnote{Allo stesso modo, per indicare la relazione di accessibilità tra etichette useremo $R$, che è un oggetto sintattico ed è differente da $\mathcal{R}$, oggetto semantico che abbiamo usato fino ad ora per indicare la relazione semantica tra mondi.}). $$\varepsilon = \{w_1, w_2, w_3, \ldots\}$$ è un insieme infinito numerabile di etichette. Utilizziamo $w$, $u$, $v$ come metavariabili. \\
\end{defin}

\subsubsubs{Definizione di formule etichettate}
\hypertarget{fmet}{}
\begin{defin}
	[\emph{Formule etichettate}]
	L'insieme delle formule etichettate $\latinmath{fm}^{\varepsilon}$ è così definito:
	\begin{description}
		\item [Base:] \phantom{ciao}
		      \begin{itemize}
			      \item Siano $w, v \in \varepsilon$; $wRv$ e $w=v \; \in \latinmath{fm}^{\varepsilon}$.
		      \end{itemize}
		      Queste sono dette \emph{formule atomiche relazionali}.
		\item [Passo:] \phantom{ciao}

		      \begin{itemize}
			      \item Siano $w \in \varepsilon, A \in \latinmath{fm}^{\Phi}$; $w:A$ (che significa \enquote{$A$ è vera nel mondo $w$}, o $\, \vDash_w A$) $\in \latinmath{fm}^{\varepsilon}$.
		      \end{itemize}
	\end{description}

	Nient'altro appartiene a $\latinmath{fm}^{\varepsilon}$. \\
\end{defin}

\subsubsubs{Definizione di sequente etichettato}
\hypertarget{seqet}{}
\begin{defin}
	[\emph{Sequente etichettato}]
	Un sequente etichettato è una coppia $$\Gamma \To \Delta$$ dove:
	\begin{itemize}
		\item $\Gamma$ è un \hyperlink{multi}{multiinsieme} di formule etichettate e atomi relazionali;
		\item $\Delta$ è un insieme di formule etichettate.
	\end{itemize}
\end{defin}

\newpage
\hypertarget{G3K}{}
\subsubsection{Calcolo \latinmath{G3K}}
\noindent Il calcolo \latinmath{G3K} è un calcolo dei \hyperlink{seqet}{sequenti etichettati}. \`{E} definito da:
\begin{itemize}
	\subsubsubs{Sequenti iniziali}
	\hypertarget{seqetin}{}
	\item \emph{Sequenti iniziali:}
	      \subsubsubs{Regole logiche} \hypertarget{Lbotet}{}	$$w:p, \Gamma \To \Delta, w:p$$
	\item \emph{Regole logiche:}

	      \vspace*{-24pt}
	      \begin{minipage}[t]{0.48\textwidth}
		      \centering
		      \begin{mathpar}
			      \inferrule*[right=$L\bot$]{ }{w:\bot, \Gamma \To \Delta} \\
			      \inferrule*[right=$L\land$]{w:A, w:B, \Gamma \To \Delta}{w:A \land B, \Gamma \To \Delta} \\
			      \inferrule*[right=$L\lor$]{w:A, \Gamma \To \Delta \\ w:B, \Gamma \To \Delta}{w:A \lor B, \Gamma \To \Delta} \\
			      \inferrule*[right=$L\to$]{\Gamma \To \Delta, w:A \\ w:B, \Gamma \To \Delta}{w:A \to B, \Gamma, \To \Delta} \\
			      \inferrule*[right=$L \Box$]{v:A, wRv, w:\Box A, \Gamma \To \Delta}{wRv, w:\Box A, \Gamma, \To \Delta} \\
			      \inferrule*[right=$L\Dmd \; (\star)$]{wRv, v:A, \Gamma \To \Delta}{w:\Dmd A, \Gamma, \To \Delta}
		      \end{mathpar}
	      \end{minipage}
	      \hspace*{-2cm}
	      \begin{minipage}[t]{0.48\textwidth}
		      \centering
		      \begin{mathpar}
			      \phantom{\inferrule*[right=$L\bot$]{ }{w:\bot \To \Delta}} \\
			      \inferrule*[right=$R\land$]{\Gamma \To \Delta, w:A \\ \Gamma \To \Delta, w:B}{\Gamma \To \Delta, w:A \land B} \\
			      \inferrule*[right=$R\lor$]{\Gamma \To \Delta, w:A, w:B}{\Gamma \To \Delta, w:A \lor B} \\
			      \inferrule*[right=$R\to$]{w:A, \Gamma \To \Delta, w:B}{\Gamma \To \Delta, w:A \to B} \\
			      \inferrule*[right=$R \Box \; (\star)$]{wRv, \Gamma \To \Delta, v:A}{\Gamma, \To \Delta, w:\Box A} \\
			      \inferrule*[right=$R\Dmd$]{wRv, \Gamma \To \Delta, w:\Dmd A, v:A}{wRv, \Gamma, \To \Delta, w:\Dmd A}
		      \end{mathpar}
	      \end{minipage} \\

	      ($\star$) $v \notin \{\Gamma, \Delta, w\}$ (non deve comparire nella conclusione della regola)
\end{itemize}
Per le regole $L$, la lettura sintattica e quella semantica coincidono; si leggono solo dal basso verso l'alto. Per le regole $R$, la lettura sintattica è preferibile; si leggono dall'alto verso il basso. \\

\subsubsubs{Esempi di derivazioni}
\begin{exm}
	[\emph{Derivazione in \latinmath{G3K}}] Derivazione di \enquote{$\To w:\Box(A \to B) \to (\Box A \to \Box B)$} (schema $K$) nel calcolo \latinmath{G3K}.
	\begin{mathpar}
		\inferrule*[Right=$R\to$]{\inferrule*[Right=$R\to$]{\inferrule*[Right=$R \Box$]{\inferrule*[Right=$L \Box$]{\inferrule*[Right=$L \Box$]{\inferrule*[Right=$L\to$]{v \! : \! A, wRv, w \!:\!\Box(A \! \to\! B), w:\Box A \To v \!: \! A, v \!: \! B \! \qquad \! v \!: \! B, v \!: \! A, wRv, w \!: \!\Box(A \! \to\! B), w \!: \!\Box A \To v \! : \! B}{v:A \to B, v:A, wRv, w:\Box(A \to B), w:\Box A \To v:B}}{v:A, wRv, w:\Box(A \to B), w:\Box A \To v:B}}{wRv, w:\Box(A \to B), w:\Box A \To v:B}}{w:\Box(A \to B), w:\Box A \To w:\Box B}}{w:\Box(A \to B) \To w:\Box A \to \Box B}}{\To w:\Box(A \to B) \to (\Box A \to \Box B)}
	\end{mathpar}
	Questo schema, dunque, è derivabile solo se si definisce il \hyperlink{lemGax}{\textbf{Lemma}}:
	$$\latinmath{G3K} \vdash w:A, \Gamma \To \Delta, w:A$$
	che afferma la derivabilità dei sequenti del tipo descritto, per una qualsiasi formula $A$, in \latinmath{G3K}. \\
\end{exm}

\begin{exm}
	[\emph{Derivazione in \latinmath{G3K}}] Derivazione di \enquote{$\To w:\Box(A \land B) \to w:(\Box A \land \Box B)$} nel calcolo \latinmath{G3K}.
	\begin{mathpar}
		\inferrule*[Right=$R\to$]{\inferrule*[Right=$R\land$]{\inferrule*[Right=$R\Box$]{\inferrule*[Right=$L\Box$]{\inferrule*[Right=$L\land$]{v:A, v:B, wRv, w:\Box (A \land B) \To v:A}{v:A \land B, wRv, w:\Box (A \land B) \To v:A}}{wRv, w:\Box (A \land B) \To v:A}}{w:\Box (A \land B) \To w:\Box A} \\ \inferrule*[Right=$R\Box$]{\inferrule*[Right=$L\Box$]{\inferrule*[Right=$L\land$]{v:A, v:B, wRv, w:\Box (A \land B) \To v:B}{v:A \land B, wRv, w:\Box (A \land B) \To v:B}}{wRv, w:\Box (A \land B) \To v:B}}{w:\Box (A \land B) \To w:\Box B}}{w:\Box(A \land B) \To w:(\Box A \land \Box B)}}{\To w:\Box(A \land B) \to w:(\Box A \land \Box B)}
	\end{mathpar}
\end{exm}

\hypertarget{estG3K}{}
\subsubsection{Estendere \latinmath{G3K}}
\noindent Sia \latinmath{L} un'estensione di \latinmath{K} (\latinmath{L} $\supseteq$ \latinmath{K}); come definire \latinmath{G3L}?

Una prima opzione potrebbe essere quella di aggiungere assiomi propri \footnote{Cioè singole proposizioni, come \enquote{$\To w:\Box A \to  A$} o \enquote{$\To wRw$}, propri di una struttura riflessiva $T$.} al calcolo a seconda della struttura che sto considerando (e dello schema che le corrisponde), ma si trova che ci sono proposizioni \footnote{Gli esempi riportati in classe sono stati tentativi di derivare lo schema $D$, \enquote{$\To w:\Box A \to \Dmd A$}, che deve essere reso vero da ogni struttura \latinmath{T}.} che, pur dovendo essere derivabili per definizione di quella struttura, lo sono solo utilizzando la regola di taglio, che generalmente non è ammissibile.

La soluzione, dunque, è aggiungere regole: si prende un assioma proprio della logica del prim'ordine e lo si trasforma in una regola per il calcolo dei \hyperlink{seqet}{sequenti etichettati} con introduzione di variabili atomiche a sinistra:
$$\begin{aligned}
		 & \forall \vec{x} (\land P \to \lor \exists \vec{y}(\land Q))                                                                                 \\
		 & \forall \vec{x} (P_1 \land \ldots \land P_n \to \exists \vec{y_1}(Q_1 \land \ldots Q_n) \lor \exists \vec{y_2}(R_1 \land \ldots \land R_n))
	\end{aligned}$$
La regola generale sarà dunque:
\begin{mathpar}
	\inferrule*{Q_1\left[\sfrac{z}{y_1}\right], \ldots, Q_n\left[\sfrac{z}{y_1}\right], P_1, \ldots, P_n, \Gamma \To \Delta \\ R_1\left[\sfrac{z}{y_2}\right], \ldots, R_n\left[\sfrac{z}{y_2}\right], P_1, \ldots, P_n, \Gamma \To \Delta}{P_1, \ldots, P_n, \Gamma \To \Delta}
\end{mathpar}
con $z \notin \{P_1, \ldots, P_n, \Gamma, \Delta \}$.

Applicando questa metodologia, si possono introdurre regole a sinistra che trattano con atomi relazionali per definire calcoli per nuove logiche a partire dai teoremi di corrispondenza.

\newpage
\subsection{Calcolo \latinmath{G3L}}
\noindent Un calcolo \latinmath{G3L} è una qualsiasi estensione del calcolo \hyperlink{G3K}{\latinmath{G3K}} ottenuta come descritto nella \hyperlink{estG3K}{sezione precedente}. Viene presentato in questo punto in quanto, da qui in poi, le definizioni e i teoremi saranno enunciati in maniera tale da applicarsi ad un qualunque calcolo \latinmath{G3L}.

Di seguito vengono presentati alcune regole che, se aggiunte a quelle di \hyperlink{G3K}{\latinmath{G3K}}, permettono di ottenere un nuovo calcolo per i \hyperlink{seqet}{sequenti etichettati} in una particolare struttura. In particolare, queste regole permettono di derivare le formule modali che corrispondono alle proprietà di $\mathcal{R}$ che esprimono; inoltre, permettono di derivare tutti gli assiomi delle logiche incluse in quella su cui si sta basando il proprio calcolo. \\

\begin{description}
	\item[Riflessività:] (\latinmath{G3T})
	      \vspace*{-4pt}
	      \begin{mathpar}
		      \inferrule*[Right=\latinmath{\emph{Rif}}]{wRw, \Gamma \To \Delta}{\Gamma \To \Delta}
		      \vspace*{-14pt}
	      \end{mathpar}
	\item[Serialità:] (\latinmath{G3D})
	      \vspace*{-4pt}
	      \begin{mathpar}
		      \inferrule*[Right=\latinmath{\emph{Ser}}]{wRv, \Gamma \To \Delta}{\Gamma \To \Delta}
		      \vspace*{-14pt}
	      \end{mathpar}
	      con $v \notin \{\Gamma, \Delta, w\}$.
	\item[Transitività:] (\latinmath{G3K4})
	      \vspace*{-4pt}
	      \begin{mathpar}
		      \inferrule*[Right=\latinmath{\emph{Trans}}]{wRu, wRv, vRu, \Gamma \To \Delta}{wRv, vRu, \Gamma \To \Delta}
		      \vspace*{-14pt}
	      \end{mathpar}
	\item[Simmetria:] (\latinmath{G3B})
	      \vspace*{-4pt}
	      \begin{mathpar}
		      \inferrule*[Right=\latinmath{\emph{Sim}}]{vRw, wRv, \Gamma \To \Delta}{wRv, \Gamma \To \Delta}
		      \vspace*{-14pt}
	      \end{mathpar}
	\item[Convergenza debole:] (\latinmath{G3K2})  \footnote{La seconda regola (\emph{convergenza debole contratta}) serve per evitare di avere copie della stessa relazione nei casi in cui $v = u$. La sua necessarietà è dovuta al fatto di non avere la contrazione tra le regole primitive del calcolo.}
	      \vspace*{-4pt}
	      \begin{mathpar}
		      \inferrule*[Right=\latinmath{\emph{ConvDeb}}]{vRt, uRt, wRv, wRu, \Gamma \To \Delta}{wRv, wRu, \Gamma \To \Delta}
		      \vspace*{-14pt}
	      \end{mathpar}
	      con $t \notin \{\Gamma, \Delta, w, v, u\}$.
	      \vspace*{-4pt}
	      \begin{mathpar}
		      \inferrule*[Right=$\latinmath{\emph{ConvDeb}}^C$]{vRt, wRv, \Gamma \To \Delta}{wRv, \Gamma \To \Delta}
		      \vspace*{-14pt}
	      \end{mathpar}
	      con $t \notin \{\Gamma, \Delta, w, v, u\}$.
	\item[Euclidea:] (\latinmath{G3K5})  \footnote{Come nel caso della convergenza debole, la seconda regola (\emph{euclidea contratta}) serve per evitare di avere copie della stessa relazione nei casi in cui $v = u$. La sua necessarietà è dovuta al fatto di non avere la contrazione tra le regole primitive del calcolo.}
	      \vspace*{-4pt}
	      \begin{mathpar}
		      \inferrule*[Right=\latinmath{\emph{Euclid}}]{vRu, wRv, wRu, \Gamma \To \Delta}{wRv, wRu, \Gamma \To \Delta}
	      \end{mathpar}
	      \begin{mathpar}
		      \inferrule*[Right=$\latinmath{\emph{Euclid}}^{\, C}$]{vRv, wRv, \Gamma \To \Delta}{wRv, \Gamma \To \Delta}
		      \vspace*{-14pt}
	      \end{mathpar}
\end{description}
\vspace{12pt}


\subsubsection{Metateoria di \latinmath{G3L}}
\subsubsubs{Definizione di lunghezza di un sequente etichettato}
\begin{defin}
	[\emph{Lunghezza di un sequente etichettati}]
	La lunghezza \latinmath{lg} di un sequente etichettato è così definita:
	\begin{description}
		\item [Base:] \phantom{ciao}
		      \begin{itemize}
			      \item $\latinmath{lg} (wRv) = 0$
		      \end{itemize}
		\item [Passo:] \phantom{ciao}
		      \begin{itemize}
			      \item $\latinmath{lg} (w:A) = \latinmath{lg} (A)$ \\
		      \end{itemize}
	\end{description}

	\subsubsubs{Definizione di profondità di una derivazione di un sequente etichettato}
	\begin{defin}
		[\emph{Profondità di una derivazione di un sequente etichettato}]
		La profondità di una derivazione $\Gamma \To \Delta$, che si indica con $p(D)$, è il numero di nodi di un ramo massimale (cioè quello di lunghezza maggiore) in $\Gamma \To \Delta$, a cui va sottratto 1 \footnote{Sottrarre 1 fa sì che un sequente iniziale abbia profondità uguale a 0, così che possiamo trattare le dimostrazioni per induzione coerentemente con quanto fatto fino ad ora; si tratta però di una convenzione, che potrebbe essere definita altrimenti.}. \\
	\end{defin}

\end{defin}
\subsubsubs{Definizione di regola ammissibile in \latinmath{G3L}}
\hypertarget{ammisset}{}
\begin{defin}
	[\emph{Regola ammissibile in \latinmath{G3L}}]
	Una regola
	\vspace*{-4pt}
	\begin{mathpar}
		\inferrule*[Right=$R$]{S_1, \ldots, S_n}{S}
		\vspace*{-14pt}
	\end{mathpar}
	è ammissibile in \latinmath{G3L} se, qualora le sue premesse siano derivabili, anche la sua conclusione lo è:
	$$\text{Se} \; \latinmath{G3L} \vdash S_1 \ldots \latinmath{G3L} \vdash S_n \text{, allora} \; \latinmath{G3L} \vdash S$$
\end{defin}

\vspace{12pt}
\subsubsubs{Definizione di regola \emph{pp}-ammissibile in \latinmath{G3L}}
\hypertarget{ppamet}{}
\begin{defin}
	[\emph{Regola \emph{pp}-ammissibile in \latinmath{G3L}}]
	Una regola
	\vspace*{-4pt}
	\begin{mathpar}
		\inferrule*[Right=$R$]{S_1, \ldots, S_n}{S}
		\vspace*{-14pt}
	\end{mathpar}
	è \emph{pp}-ammissibile in \latinmath{G3L} se, qualora le sue premesse siano derivabili con profondità al più $k$, anche la sua conclusione è derivabile con profondità al più $k$:
	$$\text{Se} \; \latinmath{G3L} \vdash^k S_1 \ldots \latinmath{G3L} \vdash^k S_n \text{, allora} \; \latinmath{G3L} \vdash^k S$$
\end{defin}

\subsubsubs{Definizione di sostituzione di etichetta}
\hypertarget{defsostet}{}
\begin{defin}
	[\emph{Sostituzione di etichetta}]
	Una sostituzione di \hyperlink{et}{etichetta} avviene quando tutte le occorrenze di un'etichetta $u$ vengono sostituite con un'etichetta $v$ e si indica con $\left[\sfrac{v}{u}\right]$:
	\vspace{4pt}
	\begin{itemize}
		\item $w\left[\sfrac{v}{u}\right]=\left\{
			      \begin{aligned}
				       & w &  & \text{se} \; w \not\equiv  u \\
				       & v &  & \text{se} \; w \equiv  u
			      \end{aligned}
			      \right. $ \\

		\item $w:A \left[\sfrac{v}{u}\right] = w\left[\sfrac{v}{u}\right] : A$ \\

		\item $w_1Rw_2 \left[\sfrac{v}{u}\right] = w_1\left[\sfrac{v}{u}\right] R w_2\left[\sfrac{v}{u}\right]$ \\
	\end{itemize}
\end{defin}

\subsubsubs{Definizione di regola \emph{pp}-invertibile in \latinmath{G3L}}
\hypertarget{ppinv}{}
\begin{defin}
	[\emph{Regola \emph{pp}-invertibile in \latinmath{G3L}}]
	Una regola
	\vspace*{-4pt}
	\begin{mathpar}
		\inferrule*[Right=$R$]{S_1, \ldots, S_n}{S}
		\vspace*{-14pt}
	\end{mathpar}
	è \emph{pp}-invertibile in \latinmath{G3L} se, qualora la sua conclusione sia derivabile con profondità al più $k$, anche le sue premesse sono derivabili con profondità al più $k$:
	$$\text{Se} \; \latinmath{G3L} \vdash^k S \text{, allora} \; \latinmath{G3L} \vdash^k S_i \quad \forall i \leq n$$
\end{defin}

\vspace{4pt}
\subsubsection{Proprietà strutturali di \latinmath{G3L}}
\noindent Le seguenti proprietà strutturali sono dimostrabili in qualsiasi calcolo basato su \latinmath{G3L}. \\
\subsubsubs{Lemma \latinmath{G}$Ax$}
\hypertarget{lemGax}{}
\begin{lem}
	[\latinmath{G}$Ax$]
	Ogni sequente etichettato della forma $w:A, \Gamma \To \Delta, w:A$, $\forall A \in \latinmath{fm}^{\varepsilon}$, è derivabile in \latinmath{G3L}:
	$$\latinmath{G3L} \vdash w:A, \Gamma \To \Delta, w:A $$
\end{lem}

\begin{dimo}
	[per induzione strutturale su $w:A$] \phantom{ciao}
	\begin{description}
		\item [Base:] \phantom{ciao}
		      \begin{itemize}
			      \item $A \equiv p$: \\
			            $w:p, \Gamma \To \Delta, w:p \;\;$ \qquad \qquad è un \hyperlink{seqetin}{sequente iniziale}.
			      \item $A \equiv \bot$: \\
			            $w:\bot, \Gamma \To \Delta, w:\bot$ \qquad \qquad è un \hyperlink{seqetin}{sequente iniziale}.
		      \end{itemize}
		\item [Passo:] ho un caso per ogni connettivo. \\
		      \latinmath{IH}: $\forall A \in \latinmath{fm}^{\varepsilon} (\latinmath{G3L} \vdash w:A, \Gamma \To \Delta, w:A)$ \\
		      \begin{itemize}
			      \item $A \equiv B \land C$ (va derivato $w: B \land C, \Gamma \To \Delta, w: B \land C$):
			            \vspace*{-12pt}
			            \begin{mathpar}
				            \inferrule*[Right=$L\land$]{\inferrule*[Right=$R\land$]{\inferrule*[Right=$\latinmath{\emph{IH}}$]{\phantom{x}}{w:B, w:C, \Gamma \To \Delta, w:B} \\ \inferrule*[Right=$\latinmath{\emph{IH}}$]{\phantom{x}}{w:B, w:C, \Gamma \To \Delta, w:C}}{w:B, w:C, \Gamma \To \Delta, w:B \land C}}{w: B \land C, \Gamma \To \Delta, w: B \land C}
			            \end{mathpar}
			      \item $A \equiv B \lor C$ (analogo a $A \equiv B \land C$; va derivato $w: B \lor C, \Gamma \To \Delta, w: B \lor C$):
			            \vspace*{-12pt}
			            \begin{mathpar}
				            \inferrule*[Right=$R\lor$]{\inferrule*[Right=$L\lor$]{\inferrule*[Right=$\latinmath{\emph{IH}}$]{\phantom{x}}{w:B, \Gamma \To \Delta, w:B, w:C} \\ \inferrule*[Right=$\latinmath{\emph{IH}}$]{\phantom{x}}{w:C, \Gamma \To \Delta, w:B, w:C}}{w:B \lor C, \Gamma \To \Delta, w:B, w:C}}{w: B \lor C, \Gamma \To \Delta, w: B \lor C}
			            \end{mathpar}
			      \item $A \equiv B \to C$ (analogo a $A \equiv B \land C$; va derivato $w: B \to C, \Gamma \To \Delta, w: B \to C$):
			            \vspace*{-12pt}
			            \begin{mathpar}
				            \inferrule*[Right=$R\to$]{\inferrule*[Right=$L\to$]{\inferrule*[Right=$\latinmath{\emph{IH}}$]{\phantom{x}}{w:B, \Gamma \To \Delta, w:B, w:C} \\ \inferrule*[Right=$\latinmath{\emph{IH}}$]{\phantom{x}}{w:B, w:C, \Gamma \To \Delta, w:C}}{w:B \to C, w:B, \Gamma \To \Delta, w:C}}{w: B \to C, \Gamma \To \Delta, w: B \to C}
			            \end{mathpar}
			      \item $A \equiv \Box B$ (va derivato $w: \Box B, \Gamma \To \Delta, w: \Box B$):
			            \vspace*{-12pt}
			            \begin{mathpar}
				            \inferrule*[Right=$R\Box$]{\inferrule*[Right=$L\Box$]{\inferrule*[Right=$\latinmath{\emph{IH}}$]{\phantom{x}}{v:B, wRv, w: \Box B, \Gamma \To \Delta, v:B}}{wRv, w: \Box B, \Gamma \To \Delta, v:B}}{w: \Box B, \Gamma \To \Delta, w: \Box B}
			            \end{mathpar}
			      \item $A \equiv \Dmd B$ (analogo a $A \equiv \Box B$; va derivato $w: \Dmd B, \Gamma \To \Delta, w: \Dmd B$):
			            \vspace*{-12pt}
			            \begin{mathpar}
				            \inferrule*[Right=$L\Dmd$]{\inferrule*[Right=$R\Dmd$]{\inferrule*[Right=$\latinmath{\emph{IH}}$]{\phantom{x}}{v:B, wRv, \Gamma \To \Delta, w: \Dmd B, v:B}}{v:B, wRv, \Gamma \To \Delta, w: \Dmd B}}{w: \Dmd B, \Gamma \To \Delta, w: \Dmd B}
			            \end{mathpar}
		      \end{itemize}
	\end{description}
\end{dimo}

\subsubsubs{Teorema: \emph{pp}-ammissibilità della regola di sostituzione uniforme}
\hypertarget{thsoset}{}
\begin{theo}
	La seguente regola di sostituzione uniforme è \hyperlink{ppamet}{\emph{pp}-ammissibile} in \latinmath{G3L}:
	\begin{mathpar}
		\inferrule*[Right={$\left[\sfrac{v}{u}\right]$}]{\Gamma \To \Delta}{\Gamma \left[\sfrac{v}{u}\right] \To \Delta \left[\sfrac{v}{u}\right]}
	\end{mathpar}
\end{theo}

\begin{dimo}
	[per induzione sulla profondità della derivazione $D$ della premessa $\Gamma \To \Delta$] \phantom{ciao} \\
	\begin{description}
		\item [Base:] $p(D)=0$
		      \begin{itemize}
			      \item $w:p, \Gamma' \; \protect\footnote{Si scrive \enquote{$\Gamma'$} per indicare che dal multiinsieme $\Gamma$ sono escluse le formule che si trovano nello stesso lato del sequente.} \To \Delta', w:p$: \\
			            Per \hyperlink{defsostet}{definizione di sostituzione di etichette}, ho due casi:
			            \begin{enumerate}
				            \item $w \not\equiv u$: $\;\; v:p, \Gamma' \left[\sfrac{v}{u}\right] \To \Delta'\left[\sfrac{v}{u}\right], v:p \;\,\;$ \qquad è un \hyperlink{seqetin}{sequente iniziale}.
				            \item $w \equiv u$: $\;\; w:p, \Gamma'\left[\sfrac{v}{u}\right] \To \Delta'\left[\sfrac{v}{u}\right], w:p \;$ \qquad è un \hyperlink{seqetin}{sequente iniziale}.
			            \end{enumerate}
			      \item $w:\bot, \Gamma' \To \Delta$: \\
			            Per \hyperlink{defsostet}{definizione di sostituzione di etichette}, ho due casi:
			            \begin{enumerate}
				            \item $w \not\equiv u$: $\;\; v:\bot, \Gamma'\left[\sfrac{v}{u}\right] \To \Delta\left[\sfrac{v}{u}\right] \:$ \qquad \qquad \quad è derivabile con \hyperlink{Lbotet}{$L\bot$} \footnote{Si può anche dire \enquote{è un'istanza di \hyperlink{Lbotet}{$L\bot$}}.}.
				            \item $w \equiv u$: $\;\; w:\bot, \Gamma'\left[\sfrac{v}{u}\right] \To \Delta\left[\sfrac{v}{u}\right]$ \qquad \qquad \quad è derivabile con \hyperlink{Lbotet}{$L\bot$}.
			            \end{enumerate}
		      \end{itemize}
		\item [Passo:] $p(D)=n+1$. \`{E} stata applicata almeno una regola; si esamina l'ultima applicata. Di ogni regola so che, se la sua conclusione è derivabile in $n+1$ passi, le sue premesse saranno derivabili in $n$ passi. \\
		      \latinmath{IH}: \emph{se $\latinmath{G3L} \vdash^n \Gamma \To \Delta$, allora $\latinmath{G3L} \vdash^n \Gamma \left[\sfrac{v}{u}\right] \To \Delta \left[\sfrac{v}{u}\right]$}
		      \begin{itemize}
			      \item $L\land$:
			            \begin{mathpar}
				            \inferrule*[Right=$L\land$]{\vdash^n \quad w:A, w:B, \Gamma' \To \Delta}{\vdash^{n+1} \quad w: A \land B, \Gamma' \To \Delta}
			            \end{mathpar}
			            Per \latinmath{IH}, applico la sostituzione sulla premessa, poi riapplico la regola:
			            \begin{mathpar}
				            \inferrule*[Right=$L\land$]{\inferrule*[Right=$\latinmath{IH}$]{\vdash^n \quad w:A, w:B, \Gamma' \To \Delta}{\vdash^n \quad w:A\left[\sfrac{v}{u}\right], w:B\left[\sfrac{v}{u}\right], \Gamma'\left[\sfrac{v}{u}\right] \To \Delta\left[\sfrac{v}{u}\right]}}{\vdash^{n+1} \quad w: A \land B\left[\sfrac{v}{u}\right], \Gamma'\left[\sfrac{v}{u}\right] \To \Delta\left[\sfrac{v}{u}\right]}
			            \end{mathpar}
			      \item $R\land$:
			            \begin{mathpar}
				            \inferrule*[Right=$R\land$]{\vdash^n \; \Gamma \To \Delta', w:A \\ \vdash^n \; \Gamma \To \Delta', w:B}{\vdash^{n+1} \quad \Gamma \To \Delta', w: A \land B}
			            \end{mathpar}
			            Per \latinmath{IH}, applico la sostituzione sulle premesse, poi riapplico la regola:
			            \begin{mathpar}
				            \inferrule*[Right=$R\land$]{\inferrule*[Right=$\latinmath{IH}$]{\vdash^n \quad \Gamma \To \Delta', w:A}{\vdash^n \; \Gamma\left[\sfrac{v}{u}\right] \To \Delta'\left[\sfrac{v}{u}\right], w:A\left[\sfrac{v}{u}\right]} \\ \inferrule*[Right=$\latinmath{IH}$]{\vdash^n \quad \Gamma \To \Delta', w:B}{\vdash^n \; \Gamma\left[\sfrac{v}{u}\right] \To \Delta'\left[\sfrac{v}{u}\right], w:B\left[\sfrac{v}{u}\right]}}{\vdash^{n+1} \quad \Gamma\left[\sfrac{v}{u}\right] \To \Delta'\left[\sfrac{v}{u}\right], w: A \land B\left[\sfrac{v}{u}\right]}
			            \end{mathpar}
			      \item $L\lor$ (analogo a $R\land$):
			            \begin{mathpar}
				            \inferrule*[Right=$L\lor$]{\vdash^n \; w:A, \Gamma' \To \Delta \\ \vdash^n \; w:B, \Gamma' \To \Delta}{\vdash^{n+1} \quad w:A \lor B, \Gamma' \To \Delta}
			            \end{mathpar}
			            Per \latinmath{IH}, applico la sostituzione sulle premesse, poi riapplico la regola:
			            \begin{mathpar}
				            \inferrule*[Right=$L\lor$]{\inferrule*[Right=$\latinmath{IH}$]{\vdash^n \quad w:A, \Gamma' \To \Delta}{\vdash^n w:A\left[\sfrac{v}{u}\right], \Gamma'\left[\sfrac{v}{u}\right] \To \Delta\left[\sfrac{v}{u}\right]} \\ \inferrule*[Right=$\latinmath{IH}$]{\vdash^n \quad w:B, \Gamma' \To \Delta}{\vdash^n w:B\left[\sfrac{v}{u}\right], \Gamma'\left[\sfrac{v}{u}\right] \To \Delta\left[\sfrac{v}{u}\right]}}{\vdash^{n+1} \quad w: A \lor B\left[\sfrac{v}{u}\right], \Gamma'\left[\sfrac{v}{u}\right] \To \Delta\left[\sfrac{v}{u}\right]}
			            \end{mathpar}
			      \item $R\lor$ (analogo a $L\land$):
			            \begin{mathpar}
				            \inferrule*[Right=$R\lor$]{\vdash^n \quad \Gamma \To \Delta', w:A, w:B}{\vdash^{n+1} \quad \Gamma \To \Delta', w: A \lor B}
			            \end{mathpar}
			            Per \latinmath{IH}, applico la sostituzione sulla premessa, poi riapplico la regola:
			            \begin{mathpar}
				            \inferrule*[Right=$R\lor$]{\inferrule*[Right=$\latinmath{IH}$]{\vdash^n \quad \Gamma \To \Delta', w:A, w:B}{\vdash^n \Gamma\left[\sfrac{v}{u}\right] \To \Delta'\left[\sfrac{v}{u}\right], w:A\left[\sfrac{v}{u}\right], w:B\left[\sfrac{v}{u}\right]}}{\vdash^{n+1} \quad \Gamma\left[\sfrac{v}{u}\right] \To \Delta'\left[\sfrac{v}{u}\right], w: A \lor B\left[\sfrac{v}{u}\right]}
			            \end{mathpar}
			      \item $L\to$ (analogo a $R\land$):
			            \begin{mathpar}
				            \inferrule*[Right=$L\to$]{\vdash^n \; \Gamma' \To \Delta, w:A \\ \vdash^n \; w:B, \Gamma' \To \Delta}{\vdash^{n+1} \quad w:A \to B, \Gamma' \To \Delta}
			            \end{mathpar}
			            Per \latinmath{IH}, applico la sostituzione sulle premesse, poi riapplico la regola:
			            \begin{mathpar}
				            \inferrule*[Right=$L\to$]{\inferrule*[Right=$\latinmath{IH}$]{\vdash^n \quad \Gamma' \To \Delta, w:A}{\vdash^n \Gamma'\left[\sfrac{v}{u}\right] \To \Delta\left[\sfrac{v}{u}\right], w:A\left[\sfrac{v}{u}\right]} \\ \inferrule*[Right=$\latinmath{IH}$]{\vdash^n \quad w:B, \Gamma' \To \Delta}{\vdash^n w:B\left[\sfrac{v}{u}\right], \Gamma'\left[\sfrac{v}{u}\right] \To \Delta\left[\sfrac{v}{u}\right]}}{\vdash^{n+1} \quad w: A \to B\left[\sfrac{v}{u}\right], \Gamma'\left[\sfrac{v}{u}\right] \To \Delta\left[\sfrac{v}{u}\right]}
			            \end{mathpar}
			      \item $R\to$ (analogo a $L\land$):
			            \begin{mathpar}
				            \inferrule*[Right=$R\to$]{\vdash^n \quad w:A, \Gamma \To \Delta', w:B}{\vdash^{n+1} \; \Gamma' \To \Delta', w: A \to B}
			            \end{mathpar}
			            Per \latinmath{IH}, applico la sostituzione sulla premessa, poi riapplico la regola:
			            \begin{mathpar}
				            \inferrule*[Right=$R\to$]{\inferrule*[Right=$\latinmath{IH}$]{\vdash^n \quad w:A, \Gamma \To \Delta', w:B}{\vdash^n w:A\left[\sfrac{v}{u}\right], \Gamma \left[\sfrac{v}{u}\right] \To \Delta'\left[\sfrac{v}{u}\right], w:B\left[\sfrac{v}{u}\right]}}{\vdash^{n+1} \quad \Gamma \left[\sfrac{v}{u}\right] \To \Delta'\left[\sfrac{v}{u}\right], w: A \to B\left[\sfrac{v}{u}\right]}
			            \end{mathpar}
			      \item $L \Box$:
			            \begin{mathpar}
				            \inferrule*[Right=$L \Box$]{\vdash^n \quad w_2:A, w_1Rw_2, w_1:\Box A, \Gamma' \To \Delta}{\vdash^{n+1} \quad w_1Rw_2, w_1:\Box A, \Gamma' \To \Delta}
			            \end{mathpar}
			            Per \latinmath{IH}, applico la sostituzione sulla premessa, poi riapplico la regola:
			            \begin{mathpar}
				            \inferrule*[Right=$L \Box$]{\inferrule*[Right=$\latinmath{IH}$]{\vdash^n \quad w_2:A, w_1Rw_2, w_1: \Box A, \Gamma' \To \Delta}{\vdash^n \quad w_2:A\left[\sfrac{v}{u}\right], w_1\left[\sfrac{v}{u}\right]R \, w_2\left[\sfrac{v}{u}\right], w_1: \Box A\left[\sfrac{v}{u}\right], \Gamma'\left[\sfrac{v}{u}\right] \To \Delta\left[\sfrac{v}{u}\right]}}{\vdash^{n+1} \quad w_1\left[\sfrac{v}{u}\right]R \, w_2\left[\sfrac{v}{u}\right], w_1: \Box A\left[\sfrac{v}{u}\right], \Gamma'\left[\sfrac{v}{u}\right] \To \Delta\left[\sfrac{v}{u}\right]}
			            \end{mathpar}
			      \item $R\Dmd$ (analogo a $L \Box$):
			            \begin{mathpar}
				            \inferrule*[Right=$R\Dmd$]{\vdash^n \quad w_1Rw_2, \Gamma' \To \Delta', w_1:\Dmd A, w_2:A}{\vdash^{n+1} \quad w_1Rw_2, \Gamma' \To \Delta', w_1:\Dmd A}
			            \end{mathpar}
			            Per \latinmath{IH}, applico la sostituzione sulla premessa, poi riapplico la regola:
			            \begin{mathpar}
				            \inferrule*[Right=$R\Dmd$]{\inferrule*[Right=$\latinmath{IH}$]{\vdash^n \quad  w_1Rw_2, \Gamma' \To \Delta', w_1: \Dmd A, w_2:A}{\vdash^n \quad w_1\left[\sfrac{v}{u}\right]R \, w_2\left[\sfrac{v}{u}\right], \Gamma'\left[\sfrac{v}{u}\right] \To \Delta'\left[\sfrac{v}{u}\right], w_1:\Dmd A\left[\sfrac{v}{u}\right], w_2:A\left[\sfrac{v}{u}\right]}}{\vdash^{n+1} \quad w_1\left[\sfrac{v}{u}\right]R \, w_2\left[\sfrac{v}{u}\right], \Gamma'\left[\sfrac{v}{u}\right] \To \Delta'\left[\sfrac{v}{u}\right], w_1:\Dmd A\left[\sfrac{v}{u}\right]}
			            \end{mathpar}
			      \item \latinmath{\emph{Trans}} (analogo a $L \Box$):
			            \begin{mathpar}
				            \inferrule*[Right=$\latinmath{\emph{Trans}}$]{\vdash^n \quad w_1Rw_3, w_1Rw_2, w_2Rw_3, \Gamma' \To \Delta}{\vdash^{n+1} \quad w_1Rw_2, w_2Rw_3, \Gamma' \To \Delta}
			            \end{mathpar}
			            Per \latinmath{IH}, applico la sostituzione sulla premessa, poi riapplico la regola:
			            \begin{mathpar}
				            \inferrule*[Right=$\latinmath{\emph{Trans}}$]{\inferrule*[Right=$\latinmath{IH}$]{\vdash^n \quad w_1Rw_3, w_1Rw_2, w_2Rw_3, \Gamma' \To \Delta}{\vdash^n \quad w_1\left[\sfrac{v}{u}\right]R \, w_3\left[\sfrac{v}{u}\right], w_1\left[\sfrac{v}{u}\right]R \, w_2\left[\sfrac{v}{u}\right], w_2\left[\sfrac{v}{u}\right]R \, w_3\left[\sfrac{v}{u}\right], \Gamma'\left[\sfrac{v}{u}\right] \To \Delta\left[\sfrac{v}{u}\right]}}{\vdash^{n+1} \quad w_1\left[\sfrac{v}{u}\right]R \, w_2\left[\sfrac{v}{u}\right], w_2\left[\sfrac{v}{u}\right]R \, w_3\left[\sfrac{v}{u}\right], \Gamma'\left[\sfrac{v}{u}\right] \To \Delta\left[\sfrac{v}{u}\right]}
			            \end{mathpar}
			            Tutti gli altri casi di regole non logiche sono analoghi.
		      \end{itemize}
		      Le uniche eccezioni sono rappresentate dalle regole che hanno restrizioni sulle variabili:
		      \begin{itemize}
			      \item $R \Box$:
			            \begin{mathpar}
				            \inferrule*[Right=$R \Box$]{\vdash^n \quad w_1Rw_2, \Gamma \To \Delta', w_2:A}{\vdash^{n+1} \quad \Gamma \To \Delta', w_1:\Box A}
			            \end{mathpar}
			            con $w_2 \notin \{\Gamma, \Delta', w_1\}$. \\
			            Si sceglie un'etichetta che non si trovi nella derivazione e la si sostituisce a $w_2$ (per \latinmath{IH}) nella premessa:
			            $$\left[\sfrac{w_2}{w_3}\right] \qquad w_1Rw_3, \Gamma \To \Delta', w_3:A$$
			            con $w_3 \notin \{\Gamma, \Delta, w_1, v, u\}$. \\
			            Non serve segnare la sostituzione nel calcolo perché, date le restrizioni su $w_2$, si conosce esattamente ogni sua occorrenza. \\
			            A questo punto si conclude come per le altre regole; per \latinmath{IH}, applico la sostituzione sulla premessa e riapplico la regola:
			            \begin{mathpar}
				            \inferrule*[Right=$R\Box$]{\inferrule*[Right=$\latinmath{IH}$]{\vdash^n \quad w_1Rw_3, \Gamma \To \Delta', w_3:A}{\vdash^n \quad w_1\left[\sfrac{v}{u}\right]R \, w_3, \Gamma\left[\sfrac{v}{u}\right] \To \Delta'\left[\sfrac{v}{u}\right], w_3:A}}{\vdash^{n+1} \quad \Gamma\left[\sfrac{v}{u}\right] \To \Delta'\left[\sfrac{v}{u}\right], w_1:\Box A\left[\sfrac{v}{u}\right]}
			            \end{mathpar}
			            con $w_3 \notin \{\Gamma, \Delta, w_1, v, u\}$.
			      \item $L\Dmd$ (analogo a $R \Box$):
			            \begin{mathpar}
				            \inferrule*[Right=$L\Dmd$]{\vdash^n \quad w_1Rw_2, w_2:A, \Gamma' \To \Delta}{\vdash^{n+1} \quad w_1:\Dmd A,\Gamma' \To \Delta}
			            \end{mathpar}
			            con $ w_2 \notin \{\Gamma, \Delta, w_1\}$. Per \latinmath{IH}, applico una prima sostituzione:
			            $$\left[\sfrac{w_2}{w_3}\right] \qquad w_1Rw_3, w_3:A, \Gamma' \To \Delta$$
			            con $w_3 \notin \{\Gamma, \Delta, w_1, v, u\}$. \\
			            Per \latinmath{IH}, applico la sostituzione sulla premessa, poi riapplico la regola:
			            \begin{mathpar}
				            \inferrule*[Right=$L\Dmd$]{\inferrule*[Right=$\latinmath{IH}$]{\vdash^n \quad w_1Rw_3, w_3:A, \Gamma' \To \Delta}{\vdash^n \quad w_1\left[\sfrac{v}{u}\right]R \, w_3, w_3:A, \Gamma'\left[\sfrac{v}{u}\right] \To \Delta\left[\sfrac{v}{u}\right]}}{\vdash^{n+1} \quad w_1:\Dmd A\left[\sfrac{v}{u}\right], \Gamma'\left[\sfrac{v}{u}\right] \To \Delta\left[\sfrac{v}{u}\right]}
			            \end{mathpar}
			            con $w_3 \notin \{\Gamma, \Delta, w_1, v, u\}$.
			      \item \latinmath{\emph{Ser}} (analogo a $R \Box$):
			            \begin{mathpar}
				            \inferrule*[Right=$\latinmath{\emph{Ser}}$]{\vdash^n \quad w_1Rw_2, \Gamma \To \Delta}{\vdash^{n+1} \quad \Gamma \To \Delta}
			            \end{mathpar}
			            con $ w_2 \notin \{\Gamma, \Delta, w_1\}$. Per \latinmath{IH}, applico una prima sostituzione:
			            $$\left[\sfrac{w_2}{w_3}\right] \qquad w_1Rw_3, \Gamma \To \Delta$$
			            con $w_3 \notin \{\Gamma, \Delta, w_1, v, u\}$. \\
			            Per \latinmath{IH}, applico la sostituzione sulla premessa, poi riapplico la regola:
			            \begin{mathpar}
				            \inferrule*[Right=$\latinmath{\emph{Ser}}$]{\inferrule*[Right=$\latinmath{IH}$]{\vdash^n \quad w_1Rw_3, \Gamma \To \Delta}{\vdash^n \quad w_1\left[\sfrac{v}{u}\right]R \, w_3, \Gamma\left[\sfrac{v}{u}\right] \To \Delta\left[\sfrac{v}{u}\right]}}{\vdash^{n+1} \quad \Gamma\left[\sfrac{v}{u}\right] \To \Delta\left[\sfrac{v}{u}\right]}
			            \end{mathpar}
			            con $w_3 \notin \{\Gamma, \Delta, w_1, v, u\}$. \\
		      \end{itemize}
	\end{description}
\end{dimo}

\subsubsubs{Teorema: \emph{pp}-ammissibilità della regola di indebolimento}
\hypertarget{weakg3l}{}
\begin{theo}
	La seguente regola di indebolimento è \hyperlink{ppamet}{\emph{pp}-ammissibile} in \latinmath{G3L}:
	\begin{mathpar}
		\inferrule*[Right=$W$]{\Gamma \To \Delta}{\Gamma, \Pi \To \Delta, \Sigma}
		\vspace*{-16pt}
	\end{mathpar}
	dove:
	\begin{itemize}
		\item $\Pi$ è un \hyperlink{multi}{multiinsieme} di formule etichettate e atomi relazionali;
		\item $\Sigma$ è un \hyperlink{multi}{multiinsieme} di sole formule etichettate. \\
	\end{itemize}
\end{theo}

\begin{dimo}
	[per induzione sulla profondità della derivazione $D$ della premessa $\Gamma \To \Delta$] \phantom{ciao}
	\begin{description}
		\item [Base:] $p(D)=0$
		      \begin{itemize}
			      \item $w:p, \Gamma' \To \Delta', w:p$: \\
			            $w:p, \Gamma', \Pi \To \Delta', \Sigma, w:p$ \qquad \qquad è un \hyperlink{seqetin}{sequente iniziale}.
			      \item $w:\bot, \Gamma' \To \Delta$: \\
			            $w:\bot, \Gamma', \Pi \To \Delta, \Sigma$ \qquad \qquad \qquad \, è derivabile con \hyperlink{Lbotet}{$L\bot$}.
		      \end{itemize}
		\item [Passo:] $p(D)=n+1$. \`{E} stata applicata almeno una regola; si esamina l'ultima applicata. Di ogni regola so che, se la sua conclusione è derivabile in $n+1$ passi, le sue premesse saranno derivabili in $n$ passi. \\
		      \latinmath{IH}: \emph{se $\latinmath{G3L} \vdash^n \Gamma \To \Delta$, allora $\latinmath{G3L} \vdash^n \Gamma, \Pi \To \Delta, \Sigma$}
		      \begin{itemize}
			      \item $L\land$:
			            \begin{mathpar}
				            \inferrule*[Right=$L\land$]{\vdash^n \quad w:A, w:B, \Gamma' \To \Delta}{\vdash^{n+1} \quad w: A \land B, \Gamma' \To \Delta}
			            \end{mathpar}
			            Per \latinmath{IH}, applico $W$ sulla premessa, poi riapplico la regola:
			            \begin{mathpar}
				            \inferrule*[Right=$L\land$]{\inferrule*[Right=$\latinmath{IH}$]{\vdash^n \quad w:A, w:B, \Gamma' \To \Delta}{\vdash^n \quad w:A, w:B, \Gamma', \Pi \To \Delta, \Sigma}}{\vdash^{n+1} \quad w: A \land B, \Gamma', \Pi \To \Delta, \Sigma}
			            \end{mathpar}
			      \item $R\land$:
			            \begin{mathpar}
				            \inferrule*[Right=$R\land$]{\vdash^n \; \Gamma \To \Delta', w:A \\ \vdash^n \; \Gamma \To \Delta', w:B}{\vdash^{n+1} \quad \Gamma \To \Delta', w: A \land B}
			            \end{mathpar}
			            Per \latinmath{IH}, applico $W$ sulle premesse, poi riapplico la regola:
			            \begin{mathpar}
				            \inferrule*[Right=$R\land$]{\inferrule*[Right=$\latinmath{IH}$]{\vdash^n \; \Gamma \To \Delta', w:A}{\vdash^n \; \Gamma, \Pi \To \Delta', \Sigma, w:A} \\ \inferrule*[Right=$\latinmath{IH}$]{\vdash^n \; \Gamma \To \Delta', w:B}{\vdash^n \; \Gamma, \Pi \To \Delta', \Sigma, w:B}}{\vdash^{n+1} \quad \Gamma, \Pi \To \Delta', \Sigma, w: A \land B}
			            \end{mathpar}
			      \item $L\lor$ (analogo a $R\land$):
			            \begin{mathpar}
				            \inferrule*[Right=$L\lor$]{\vdash^n \; w:A, \Gamma' \To \Delta \\ \vdash^n \; w:B, \Gamma' \To \Delta}{\vdash^{n+1} \quad w:A \lor B, \Gamma' \To \Delta}
			            \end{mathpar}
			            Per \latinmath{IH}, applico $W$ sulle premesse, poi riapplico la regola:
			            \begin{mathpar}
				            \inferrule*[Right=$L\lor$]{\inferrule*[Right=$\latinmath{IH}$]{\vdash^n \; w:A, \Gamma' \To \Delta}{\vdash^n \; w:A, \Gamma', \Pi \To \Delta, \Sigma} \\ \inferrule*[Right=$\latinmath{IH}$]{\vdash^n \; w:B, \Gamma' \To \Delta}{\vdash^n \; w:B, \Gamma', \Pi \To \Delta, \Sigma}}{\vdash^{n+1} \quad w:A \lor B, \Gamma', \Pi \To \Delta, \Sigma}
			            \end{mathpar}
			      \item $R\lor$ (analogo a $L\land$):
			            \begin{mathpar}
				            \inferrule*[Right=$R\lor$]{\vdash^n \quad \Gamma \To \Delta', w:A, w:B}{\vdash^{n+1} \quad \Gamma \To \Delta', w: A \lor B}
			            \end{mathpar}
			            Per \latinmath{IH}, applico $W$ sulla premessa, poi riapplico la regola:
			            \begin{mathpar}
				            \inferrule*[Right=$R\lor$]{\inferrule*[Right=$\latinmath{IH}$]{\vdash^n \quad \Gamma \To \Delta', w:A, w:B}{\vdash^n \quad \Gamma, \Pi \To \Delta', \Sigma, w:A, w:B}}{\vdash^{n+1} \quad \Gamma, \Pi \To \Delta', \Sigma, w: A \lor B}
			            \end{mathpar}
			      \item $L\to$ (analogo a $R\land$):
			            \begin{mathpar}
				            \inferrule*[Right=$L\to$]{\vdash^n \; \Gamma' \To \Delta, w:A \\ \vdash^n \; w:B, \Gamma' \To \Delta}{\vdash^{n+1} \quad w:A \to B, \Gamma' \To \Delta}
			            \end{mathpar}
			            Per \latinmath{IH}, applico $W$ sulle premesse, poi riapplico la regola:
			            \begin{mathpar}
				            \inferrule*[Right=$L\to$]{\inferrule*[Right=$\latinmath{IH}$]{\vdash^n \; \Gamma' \To \Delta, w:A}{\vdash^n \; \Gamma', \Pi \To \Delta, \Sigma, w:A} \\ \inferrule*[Right=$\latinmath{IH}$]{\vdash^n \; w:B, \Gamma' \To \Delta}{\vdash^n \; w:B, \Gamma', \Pi \To \Delta, \Sigma}}{\vdash^{n+1} \quad w:A \to B, \Gamma', \Pi \To \Delta, \Sigma}
			            \end{mathpar}
			      \item $R\to$ (analogo a $L\land$):
			            \begin{mathpar}
				            \inferrule*[Right=$R\to$]{\vdash^n \quad w:A, \Gamma \To \Delta', w:B}{\vdash^{n+1} \; \Gamma \To \Delta', w: A \to B}
			            \end{mathpar}
			            Per \latinmath{IH}, applico $W$ sulla premessa, poi riapplico la regola:
			            \begin{mathpar}
				            \inferrule*[Right=$R\to$]{\inferrule*[Right=$\latinmath{IH}$]{\vdash^n \quad w:A, \Gamma \To \Delta', w:B}{\vdash^n \quad w:A, \Gamma, \Pi \To \Delta', \Sigma, w:B}}{\vdash^{n+1} \; \Gamma, \Pi \To \Delta', \Sigma, w: A \to B}
			            \end{mathpar}
			      \item $L\Box$ (analogo a $L\land$):
			            \begin{mathpar}
				            \inferrule*[Right=$L\Box$]{\vdash^n \quad v:A, wRv, w:\Box A, \Gamma' \To \Delta}{\vdash^{n+1} \quad wRv, w:\Box A, \Gamma' \To \Delta}
			            \end{mathpar}
			            Per \latinmath{IH}, applico $W$ sulla premessa, poi riapplico la regola:
			            \begin{mathpar}
				            \inferrule*[Right=$L\Box$]{\inferrule*[Right=$\latinmath{IH}$]{\vdash^n \quad v:A, wRv, w:\Box A, \Gamma' \To \Delta}{\vdash^n \quad v:A, wRv, w:\Box A, \Gamma', \Pi \To \Delta, \Sigma}}{\vdash^{n+1} \quad wRv, w:\Box A, \Gamma', \Pi \To \Delta, \Sigma}
			            \end{mathpar}
			      \item $R\Dmd$ (analogo a $L\land$):
			            \begin{mathpar}
				            \inferrule*[Right=$R\Dmd$]{\vdash^n \quad wRv, \Gamma' \To \Delta', w:\Dmd A, v:A}{\vdash^{n+1} \quad wRv, \Gamma' \To \Delta', w:\Dmd A}
			            \end{mathpar}
			            Per \latinmath{IH}, applico $W$ sulla premessa, poi riapplico la regola:
			            \begin{mathpar}
				            \inferrule*[Right=$R\Dmd$]{\inferrule*[Right=$\latinmath{IH}$]{\vdash^n \quad wRv, \Gamma' \To \Delta', w:\Dmd A, v:A}{\vdash^n \quad wRv, \Gamma', \Pi \To \Delta', \Sigma, w:\Dmd A, v:A}}{\vdash^{n+1} \quad wRv, \Gamma', \Pi \To \Delta', \Sigma, w:\Dmd A}
			            \end{mathpar}
			      \item \latinmath{\emph{Trans}} (analogo a $L\land$):
			            \begin{mathpar}
				            \inferrule*[Right=$\latinmath{\emph{Trans}}$]{\vdash^n \quad wRu, wRv, vRu, \Gamma' \To \Delta}{\vdash^{n+1} \quad wRv, vRu, \Gamma' \To \Delta}
			            \end{mathpar}
			            Per \latinmath{IH}, applico $W$ sulla premessa, poi riapplico la regola:
			            \begin{mathpar}
				            \inferrule*[Right=$\latinmath{\emph{Trans}}$]{\inferrule*[Right=$\latinmath{IH}$]{\vdash^n \quad wRu, wRv, vRu, \Gamma' \To \Delta}{\vdash^n \quad wRu, wRv, vRu, \Gamma', \Pi \To \Delta, \Sigma}}{\vdash^{n+1} \quad wRv, vRu, \Gamma', \Pi \To \Delta, \Sigma}
			            \end{mathpar}
			            Tutti gli altri casi di regole non logiche sono analoghi.
		      \end{itemize}
		      Le uniche eccezioni sono rappresentate dalle regole che hanno restrizioni sulle variabili:
		      \begin{itemize}
			      \item $R \Box$:
			            \begin{mathpar}
				            \inferrule*[Right=$R \Box$]{\vdash^n \quad wRv, \Gamma \To \Delta', v:A}{\vdash^{n+1} \quad \Gamma \To \Delta', w:\Box A}
			            \end{mathpar}
			            con $v \notin \{\Gamma, \Delta, w\}$. \\
			            Si applica una \hyperlink{thsoset}{sostituzione} \sfrac{u}{v}, con $u \notin \{\Gamma, \Delta, \Pi, \Sigma, w\}$, sulla premessa:
			            \begin{mathpar}
				            \inferrule*[Right={$\left[\sfrac{u}{v}\right]$}]{\vdash^n \quad wRv, \Gamma \To \Delta', v:A}{\vdash^n \quad wRu, \Gamma \To \Delta', u:A}
			            \end{mathpar}
			            A questo punto, si conclude come per le altre regole; per \latinmath{IH}, applico $W$ sulla premessa e riapplico la regola:
			            \begin{mathpar}
				            \inferrule*[Right=$R \Box$]{\inferrule*[Right=$\latinmath{IH}$]{\vdash^n \quad wRu, \Gamma \To \Delta', u:A}{\vdash^n \quad wRu, \Gamma, \Pi \To \Delta', \Sigma, u:A}}{\vdash^{n+1} \quad \Gamma, \Pi \To \Delta', \Sigma, w:\Box A}
			            \end{mathpar}
			            con $u \notin \{\Gamma, \Delta, \Pi, \Sigma, w\}$.
			      \item $L\Dmd$ (analogo a $R \Box$):
			            \begin{mathpar}
				            \inferrule*[Right=$L\Dmd$]{\vdash^n \quad wRv, v:A, \Gamma' \To \Delta}{\vdash^{n+1} \quad w:\Dmd A,\Gamma' \To \Delta}
			            \end{mathpar}
			            con $v \notin \{\Gamma, \Delta, w\}$. Applico una \hyperlink{thsoset}{sostituzione} \sfrac{u}{v} sulla premessa:
			            \begin{mathpar}
				            \inferrule*[Right={$\left[\sfrac{u}{v}\right]$}]{\vdash^n \quad wRv, v:A, \Gamma' \To \Delta}{\vdash^n \quad wRu, u:A, \Gamma' \To \Delta}
			            \end{mathpar}
			            con $u \notin \{\Gamma, \Delta, \Pi, \Sigma, w\}$. Per \latinmath{IH}, applico $W$ sulla premessa e riapplico la regola:
			            \begin{mathpar}
				            \inferrule*[Right=$L\Dmd$]{\inferrule*[Right=$\latinmath{IH}$]{\vdash^n \quad wRu, u:A, \Gamma' \To \Delta}{\vdash^n \quad wRu, u:A, \Gamma', \Pi \To \Delta, \Sigma}}{\vdash^{n+1} \quad w:\Dmd A,\Gamma', \Pi \To \Delta, \Sigma}
			            \end{mathpar}
			            con $u \notin \{\Gamma, \Delta, \Pi, \Sigma, w\}$.
			      \item \latinmath{\emph{Ser}} (analogo a $R \Box$):
			            \begin{mathpar}
				            \inferrule*[Right=$\latinmath{\emph{Ser}}$]{\vdash^n \quad wRv, \Gamma \To \Delta}{\vdash^{n+1} \quad \Gamma \To \Delta}
			            \end{mathpar}
			            con $v \notin \{\Gamma, \Delta, w\}$. Applico una \hyperlink{thsoset}{sostituzione} \sfrac{u}{v} sulla premessa:
			            \begin{mathpar}
				            \inferrule*[Right={$\left[\sfrac{u}{v}\right]$}]{\vdash^n \quad wRv, \Gamma \To \Delta}{\vdash^n \quad wRu, \Gamma \To \Delta}
			            \end{mathpar}
			            con $u \notin \{\Gamma, \Delta, \Pi, \Sigma, w\}$. Per \latinmath{IH}, applico $W$ sulla premessa e riapplico la regola:
			            \begin{mathpar}
				            \inferrule*[Right=$\latinmath{\emph{Ser}}$]{\inferrule*[Right=$\latinmath{IH}$]{\vdash^n \quad wRu, \Gamma \To \Delta}{\vdash^n \quad wRu, \Gamma, \Pi \To \Delta, \Sigma}}{\vdash^{n+1} \quad \Gamma, \Pi \To \Delta, \Sigma}
			            \end{mathpar}
			            con $u \notin \{\Gamma, \Delta, \Pi, \Sigma, w\}$. \\
		      \end{itemize}
	\end{description}
\end{dimo}

\subsubsubs{Corollario: ammissibilità della regola di necessitazione}
\hypertarget{namg3l}{}
\noindent \textbf{Corollario.} \emph{La seguente regola di necessitazione $N$ è \hyperlink{ammisset}{ammissibile} in \latinmath{G3L}:}
\begin{mathpar}
	\inferrule*[Right=$N$]{\vdash^{n} \quad \To w:A}{\vdash^{n+1} \quad \To w:\Box A}
\end{mathpar}

\begin{dimo}
	\begin{mathpar}
		\inferrule*[Right=$R\Box \quad (u \neq w)$]{\inferrule*[Right=$W$]{\inferrule*[Right={$\left[\sfrac{u}{w}\right]$}]{\vdash^{n} \quad \To w:A}{\vdash^{n} \quad \To u:A}}{\vdash^{n} \quad wRu \To u:A}}{\vdash^{n+1} \quad \To w:\Box A}
	\end{mathpar}
\end{dimo}

\subsubsubs{Teorema: \emph{pp}-invertibilità delle regole}
\hypertarget{invg3l}{}
\begin{theo}
	Tutte le regole di \latinmath{G3L} sono \hyperlink{ppinv}{\emph{pp}-invertibili}; ovvero,
	se la conclusione di una regola è derivabile con profondità al più $k$, allora anche le sue premesse saranno derivabili con profondità al più $k$. \\
\end{theo}
\begin{dimo}
	Ho un caso per ogni regola. Per ogni caso, la dimostrazione avviene per induzione sulla profondità della derivazione $D$ della premessa $\Gamma \To \Delta$. Fanno eccezione le regole che ripetono le formule principali nelle premesse.
	\begin{itemize}
		\item $R \equiv L\land$ (va dimostrato che \emph{se $\latinmath{G3L} \vdash^k w:A \land B, \Gamma \To \Delta$, allora $\latinmath{G3L} \vdash^k w:A, w:B, \Gamma \To \Delta$}):
		      \begin{description}
			      \item[Base:] $p(D)=0$
			            \begin{itemize}
				            \item $w:A \land B, v:p, \Gamma' \To \Delta', v:p$: \\
				                  $w:A, w:B, v:p, \Gamma' \To \Delta', v:p$ \qquad \qquad è un \hyperlink{seqetin}{sequente iniziale}.
				            \item $w:A \land B, w:\bot, \Gamma' \To \Delta$: \\
				                  $w:A, w:B, w:\bot, \Gamma' \To \Delta$ \qquad \qquad \qquad è derivabile con \hyperlink{Lbotet}{$L\bot$}.
			            \end{itemize}
			      \item[Passo:] $p(D)=n+1$. \`{E} stata applicata almeno una regola; si esamina l'ultima applicata. Di ogni regola so che, se la sua conclusione è derivabile in $n+1$ passi, le sue premesse saranno derivabili in $n$ passi. \\
			            \latinmath{IH}: \emph{se $\latinmath{G3L} \vdash^n w:A \land B, \Pi' \To \Sigma$, allora $\latinmath{G3L} \vdash^n w:A, w:B, \Pi' \To \Sigma$} \\
			            Ho due casi:
			            \begin{enumerate}
				            \item \emph{$w:A \land B$ è principale nell'ultima regola applicata}: l'ultima regola applicata sarà un'istanza di $L\land$.
				                  \begin{mathpar}
					                  \inferrule*[Right=$L\land$]{\vdash^{n} \quad w:A, w:B, \Gamma \To \Delta}{\vdash^{n+1} \quad w:A \land B, \Gamma \To \Delta}
				                  \end{mathpar}
				                  Dal momento che $w:A, w:B, \Gamma \To \Delta$ è derivabile con profondità $n$, lo sarà anche con profondità $n+1$.
				            \item \emph{$w:A \land B$ \emph{non} è principale nell'ultima regola applicata}:
				                  \begin{mathpar}
					                  \inferrule*[Right=$\latinmath{\emph{Reg}}$]{\vdash^n \quad w:A \land B, \Pi' \To \Sigma}{\vdash^{n+1} \quad w:A \land B, \Gamma' \To \Delta}
				                  \end{mathpar}
				                  Per \latinmath{IH}, ho che:
				                  $$\vdash^{n} \quad w:A, w:B, \Pi' \To \Sigma$$
				                  Riapplico $\latinmath{\emph{Reg}}$:
				                  \begin{mathpar}
					                  \inferrule*[Right=$\latinmath{\emph{Reg}}$]{\vdash^n \quad w:A, w:B, \Pi' \To \Sigma}{\vdash^{n+1} \quad w:A, w:B, \Gamma' \To \Delta}
				                  \end{mathpar}
			            \end{enumerate}
		      \end{description}
		\item $R \equiv R\land$ (va dimostrato che \emph{se $\latinmath{G3L} \vdash^{k} \Gamma \To \Delta, w:A \land B$, allora $\latinmath{G3L} \vdash^{k} \Gamma \To \Delta, w:A \; \text{e} \; \latinmath{G3L} \vdash^{k} \Gamma \To \Delta, w:B$}):
		      \begin{description}
			      \item[Base:] $p(D)=0$
			            \begin{itemize}
				            \item $v:p, \Gamma' \To \Delta', v:p, w:A \land B$: \\
				                  $v:p, \Gamma' \To \Delta', v:p, w:A$ \qquad \qquad è un \hyperlink{seqetin}{sequente iniziale}. \\
				                  $v:p, \Gamma' \To \Delta', v:p, w:B$ \qquad \qquad è un \hyperlink{seqetin}{sequente iniziale}.
				            \item $w:\bot, \Gamma' \To \Delta, w:A \land B$: \\
				                  $w:\bot, \Gamma' \To \Delta, w:A$ \qquad \qquad \qquad è derivabile con \hyperlink{Lbotet}{$L\bot$}. \\
				                  $w:\bot, \Gamma' \To \Delta, w:B$ \qquad \qquad \qquad è derivabile con \hyperlink{Lbotet}{$L\bot$}.
			            \end{itemize}
			      \item[Passo:] $p(D)=n+1$. \`{E} stata applicata almeno una regola; si esamina l'ultima applicata. Di ogni regola so che, se la sua conclusione è derivabile in $n+1$ passi, le sue premesse saranno derivabili in $n$ passi. \\
			            \latinmath{IH}: \emph{se $\latinmath{G3L} \vdash^{n} \Pi \To \Sigma', w:A \land B$, allora $\latinmath{G3L} \vdash^{n} \Pi \To \Sigma', w:A \; \text{e} \; \latinmath{G3L} \vdash^{n} \Pi \To \Sigma', w:B$} \\
			            Ho due casi:
			            \begin{enumerate}
				            \item \emph{$w:A \land B$ è principale nell'ultima regola applicata}: l'ultima regola applicata sarà un'istanza di $R\land$.
				                  \begin{mathpar}
					                  \inferrule*[Right=$R\land$]{\vdash^{n} \; \Gamma \To \Delta, w:A \\ \vdash^{n} \; \Gamma \To \Delta, w:B}{\vdash^{n+1} \quad \Gamma \To \Delta, w:A \land B}
				                  \end{mathpar}
				                  Dal momento che $\Gamma \To \Delta, w:A \; \text{e} \; \Gamma \To \Delta, w:B$ sono derivabili con profondità $n$, lo saranno anche con profondità $n+1$.
				            \item \emph{$w:A \land B$ \emph{non} è principale nell'ultima regola applicata}:
				                  \begin{mathpar}
					                  \inferrule*[Right=$\latinmath{\emph{Reg}}$]{\vdash^n \quad \Pi \To \Sigma', w:A \land B}{\vdash^{n+1} \quad \Gamma \To \Delta', w:A \land B}
				                  \end{mathpar}
				                  Per \latinmath{IH}, ho che:
				                  $$\vdash^{n} \; \Pi \To \Sigma', w:A \quad \text{e} \quad \vdash^{n} \; \Pi \To \Sigma', w:B$$
				                  Riapplico $\latinmath{\emph{Reg}}$ su entrambe le premesse:
				                  \begin{mathpar}
					                  \inferrule*[Right=$\latinmath{\emph{Reg}}$]{\vdash^n \quad \Pi \To \Sigma', w:A}{\vdash^{n+1} \quad \Gamma \To \Delta', w:A} \qquad \qquad
					                  \inferrule*[Right=$\latinmath{\emph{Reg}}$]{\vdash^n \quad \Pi \To \Sigma', w:B}{\vdash^{n+1} \quad \Gamma \To \Delta', w:B}
				                  \end{mathpar}
			            \end{enumerate}
		      \end{description}
		\item $R \equiv L\lor$ (analogo a $R\land$; va dimostrato che \emph{se $\latinmath{G3L} \vdash^{k} w:A \lor B, \Gamma \To \Delta$, allora $\latinmath{G3L} \vdash^{k} w:A, \Gamma \To \Delta \; \text{e} \; \latinmath{G3L} \vdash^{k} w:B, \Gamma \To \Delta$}):
		      \begin{description}
			      \item[Base:] $p(D)=0$
			            \begin{itemize}
				            \item $w:A \lor B, v:p, \Gamma' \To \Delta', v:p$: \\
				                  $w:A, v:p, \Gamma' \To \Delta', v:p$ \qquad \qquad è un \hyperlink{seqetin}{sequente iniziale}. \\
				                  $w:B, v:p, \Gamma' \To \Delta', v:p$ \qquad \qquad è un \hyperlink{seqetin}{sequente iniziale}.
				            \item $w:A \lor B, w:\bot, \Gamma' \To \Delta$: \\
				                  $w:A, w:\bot, \Gamma' \To \Delta$ \qquad \qquad \qquad è derivabile con \hyperlink{Lbotet}{$L\bot$}. \\
				                  $w:B, w:\bot, \Gamma' \To \Delta$ \qquad \qquad \qquad è derivabile con \hyperlink{Lbotet}{$L\bot$}.
			            \end{itemize}
			      \item[Passo:] $p(D)=n+1$. \`{E} stata applicata almeno una regola; si esamina l'ultima applicata. Di ogni regola so che, se la sua conclusione è derivabile in $n+1$ passi, le sue premesse saranno derivabili in $n$ passi. \\
			            \latinmath{IH}: \emph{se $\latinmath{G3L} \vdash^{n} w:A \lor B, \Pi' \To \Sigma$, allora $\latinmath{G3L} \vdash^{n} w:A, \Pi' \To \Sigma \: \text{e} \; \latinmath{G3L} \vdash^{n} w:B, \Pi' \To \Sigma$} \\
			            Ho due casi:
			            \begin{enumerate}
				            \item \emph{$w:A \lor B$ è principale nell'ultima regola applicata}: l'ultima regola applicata sarà un'istanza di $L\lor$.
				                  \begin{mathpar}
					                  \inferrule*[Right=$L\lor$]{\vdash^{n} \; w:A, \Gamma \To \Delta \\ \vdash^{n} \; w:B, \Gamma \To \Delta}{\vdash^{n+1} \quad w:A \lor B, \Gamma \To \Delta}
				                  \end{mathpar}
				                  Dal momento che $w:A, \Gamma \To \Delta \; \text{e} \; w:B, \Gamma \To \Delta$ sono derivabili con profondità $n$, lo saranno anche con profondità $n+1$.
				            \item \emph{$w:A \lor B$ \emph{non} è principale nell'ultima regola applicata}:
				                  \begin{mathpar}
					                  \inferrule*[Right=$\latinmath{\emph{Reg}}$]{\vdash^n \quad w:A \lor B, \Pi' \To \Sigma}{\vdash^{n+1} \quad w:A \lor B, \Gamma' \To \Delta}
				                  \end{mathpar}
				                  Per \latinmath{IH}, ho che:
				                  $$\vdash^{n} \; w:A, \Pi' \To \Sigma_1 \quad \text{e} \quad \vdash^{n} \; w:B, \Pi' \To \Sigma$$
				                  Riapplico $\latinmath{\emph{Reg}}$ su entrambe le premesse:
				                  \begin{mathpar}
					                  \inferrule*[Right=$\latinmath{\emph{Reg}}$]{\vdash^n \quad w:A, \Pi' \To \Sigma}{\vdash^{n+1} \quad w:A, \Gamma' \To \Delta} \qquad \qquad
					                  \inferrule*[Right=$\latinmath{\emph{Reg}}$]{\vdash^n \quad w:B, \Pi' \To \Sigma}{\vdash^{n+1} \quad w:B, \Gamma' \To \Delta}
				                  \end{mathpar}
			            \end{enumerate}
		      \end{description}
		\item $R \equiv R\lor$ (analogo a $L\land$; va dimostrato che \emph{se $\latinmath{G3L} \vdash^k \Gamma \To \Delta, w:A \lor B$, allora $\latinmath{G3L} \vdash^k \Gamma \To \Delta, w:A, w:B$}):
		      \begin{description}
			      \item[Base:] $p(D)=0$
			            \begin{itemize}
				            \item $v:p, \Gamma' \To \Delta', v:p, w:A \lor B$: \\
				                  $v:p, \Gamma' \To \Delta', v:p, w:A, w:B \,$ \qquad \qquad è un \hyperlink{seqetin}{sequente iniziale}.
				            \item $w:\bot, \Gamma' \To \Delta, w:A \lor B$: \\
				                  $w:\bot, \Gamma' \To \Delta, w:A, w:B$ \qquad \qquad \qquad \thinspace è derivabile con \hyperlink{Lbotet}{$L\bot$}.
			            \end{itemize}
			      \item[Passo:] $p(D)=n+1$. \`{E} stata applicata almeno una regola; si esamina l'ultima applicata. Di ogni regola so che, se la sua conclusione è derivabile in $n+1$ passi, le sue premesse saranno derivabili in $n$ passi. \\
			            \latinmath{IH}: \emph{se $\latinmath{G3L} \vdash^n \Pi \To \Sigma', w:A \lor B$, allora $\latinmath{G3L} \vdash^n \Pi \To \Sigma', w:A, w:B$} \\
			            Ho due casi:
			            \begin{enumerate}
				            \item \emph{$w:A \lor B$ è principale nell'ultima regola applicata}: l'ultima regola applicata sarà un'istanza di $R\lor$.
				                  \begin{mathpar}
					                  \inferrule*[Right=$R\lor$]{\vdash^{n} \quad \Gamma \To \Delta, w:A, w:B}{\vdash^{n+1} \quad \Gamma \To \Delta, w:A \lor B}
				                  \end{mathpar}
				                  Dal momento che $\Gamma \To \Delta, w:A, w:B$ è derivabile con profondità $n$, lo sarà anche con profondità $n+1$.
				            \item \emph{$w:A \lor B$ \emph{non} è principale nell'ultima regola applicata}:
				                  \begin{mathpar}
					                  \inferrule*[Right=$\latinmath{\emph{Reg}}$]{\vdash^n \quad \Pi \To \Sigma', w:A \lor B}{\vdash^{n+1} \quad \Gamma \To \Delta', w:A \lor B}
				                  \end{mathpar}
				                  Per \latinmath{IH}, ho che:
				                  $$\vdash^{n} \quad \Pi \To \Sigma', w:A, w:B$$
				                  Riapplico $\latinmath{\emph{Reg}}$:
				                  \begin{mathpar}
					                  \inferrule*[Right=$\latinmath{\emph{Reg}}$]{\vdash^n \quad \Pi \To \Sigma', w:A, w:B}{\vdash^{n+1} \quad \Gamma \To \Delta', w:A, w:B}
				                  \end{mathpar}
			            \end{enumerate}
		      \end{description}
		\item $R \equiv L\to$ (analogo a $R\land$; va dimostrato che \emph{se $\latinmath{G3L} \vdash^{k} w:A \to B, \Gamma \To \Delta$, allora $\latinmath{G3L} \vdash^{k} \Gamma \To \Delta, w:A \; \text{e} \; \latinmath{G3L} \vdash^{k} w:B, \Gamma \To \Delta$}):
		      \begin{description}
			      \item[Base:] $p(D)=0$
			            \begin{itemize}
				            \item $w:A \to B, v:p, \Gamma' \To \Delta', v:p$: \\
				                  $v:p, \Gamma' \To \Delta', v:p, w:A$ \qquad \qquad è un \hyperlink{seqetin}{sequente iniziale}. \\
				                  $w:B, v:p, \Gamma' \To \Delta', v:p$ \qquad \qquad è un \hyperlink{seqetin}{sequente iniziale}.
				            \item $w:A \to B, w:\bot, \Gamma' \To \Delta$: \\
				                  $w:\bot, \Gamma' \To \Delta, w:A$ \qquad \qquad \qquad è derivabile con \hyperlink{Lbotet}{$L\bot$}. \\
				                  $w:B, w:\bot, \Gamma' \To \Delta'$ \qquad \qquad \qquad è derivabile con \hyperlink{Lbotet}{$L\bot$}.
			            \end{itemize}
			      \item[Passo:] $p(D)=n+1$. \`{E} stata applicata almeno una regola; si esamina l'ultima applicata. Di ogni regola so che, se la sua conclusione è derivabile in $n+1$ passi, le sue premesse saranno derivabili in $n$ passi. \\
			            \latinmath{IH}: \emph{se $\latinmath{G3L} \vdash^{n} w:A \to B, \Pi' \To \Sigma$, allora $\latinmath{G3L} \vdash^{n} \Pi' \To \Sigma, w:A \; \text{e} \; w:B, \Pi' \To \Sigma$} \\
			            Ho due casi:
			            \begin{enumerate}
				            \item \emph{$w:A \to B$ è principale nell'ultima regola applicata}: l'ultima regola applicata sarà un'istanza di $L\to$.
				                  \begin{mathpar}
					                  \inferrule*[Right=$L\to$]{\vdash^{n} \; \Pi_1 \To \Sigma_1\Gamma \To \Delta, w:A \\ \vdash^{n} \; w:B, \Gamma \To \Delta}{\vdash^{n+1} \quad w:A \to B, \Gamma \To \Delta}
				                  \end{mathpar}
				                  Dal momento che $\Gamma \To \Delta, w:A \; \text{e} \; w:B, \Gamma \To \Delta$ sono derivabili con profondità $n$, lo saranno anche con profondità $n+1$.
				            \item \emph{$w:A \to B$ \emph{non} è principale nell'ultima regola applicata}:
				                  \begin{mathpar}
					                  \inferrule*[Right=$\latinmath{\emph{Reg}}$]{\vdash^n \quad w:A \to B, \Pi' \To \Sigma}{\vdash^{n+1} \quad w:A \to B, \Gamma' \To \Delta}
				                  \end{mathpar}
				                  Per \latinmath{IH}, ho che:
				                  $$\vdash^{n} \; \Pi' \To \Sigma, w:A \quad \text{e} \quad \vdash^{n} \; w:B, \Pi' \To \Sigma$$
				                  Riapplico $\latinmath{\emph{Reg}}$ su entrambe le premesse:
				                  \begin{mathpar}
					                  \inferrule*[Right=$\latinmath{\emph{Reg}}$]{\vdash^n \quad \Pi' \To \Sigma, w:A}{\vdash^{n+1} \quad \Gamma' \To \Delta, w:A} \qquad \qquad
					                  \inferrule*[Right=$\latinmath{\emph{Reg}}$]{\vdash^n \quad w:B, \Pi' \To \Sigma}{\vdash^{n+1} \quad w:B, \Gamma' \To \Delta}
				                  \end{mathpar}
			            \end{enumerate}
		      \end{description}
		\item $R \equiv R\to$ (analogo a $L\land$; va dimostrato che \emph{se $\latinmath{G3L} \vdash^k \Gamma \To \Delta, w:A \to B$, allora $\latinmath{G3L} \vdash^k w:A, \Gamma \To \Delta, w:B$}):
		      \begin{description}
			      \item[Base:] $p(D)=0$
			            \begin{itemize}
				            \item $v:p, \Gamma' \To \Delta', v:p, w:A \to B$: \\
				                  $w:A, v:p, \Gamma' \To \Delta', v:p, w:B \,$ \qquad \qquad è un \hyperlink{seqetin}{sequente iniziale}.
				            \item $w:\bot, \Gamma' \To \Delta, w:A \to B$: \\
				                  $w:A, w:\bot, \Gamma' \To \Delta, w:B$ \qquad \qquad \qquad \thinspace è derivabile con \hyperlink{Lbotet}{$L\bot$}.
			            \end{itemize}
			      \item[Passo:] $p(D)=n+1$. \`{E} stata applicata almeno una regola; si esamina l'ultima applicata. Di ogni regola so che, se la sua conclusione è derivabile in $n+1$ passi, le sue premesse saranno derivabili in $n$ passi. \\
			            \latinmath{IH}: \emph{se $\latinmath{G3L} \vdash^n \Pi \To \Sigma', w:A \to B$, allora $\latinmath{G3L} \vdash^n w:A, \Pi \To \Sigma', w:B$} \\
			            Ho due casi:
			            \begin{enumerate}
				            \item \emph{$w:A \to B$ è principale nell'ultima regola applicata}: l'ultima regola applicata sarà un'istanza di $R\to$.
				                  \begin{mathpar}
					                  \inferrule*[Right=$R\to$]{\vdash^{n} \quad w:A, \Gamma \To \Delta, w:B}{\vdash^{n+1} \quad \Gamma \To \Delta, w:A \to B}
				                  \end{mathpar}
				                  Dal momento che $w:A, \Gamma \To \Delta, w:B$ è derivabile con profondità $n$, lo sarà anche con profondità $n+1$.
				            \item \emph{$w:A \to B$ \emph{non} è principale nell'ultima regola applicata}:
				                  \begin{mathpar}
					                  \inferrule*[Right=$\latinmath{\emph{Reg}}$]{\vdash^n \quad \Pi \To \Sigma', w:A \to B}{\vdash^{n+1} \quad \Gamma \To \Delta', w:A \to B}
				                  \end{mathpar}
				                  Per \latinmath{IH}, ho che:
				                  $$\vdash^{n} \quad w:A, \Pi \To \Sigma', w:B$$
				                  Riapplico $\latinmath{\emph{Reg}}$:
				                  \begin{mathpar}
					                  \inferrule*[Right=$\latinmath{\emph{Reg}}$]{\vdash^n \quad w:A, \Pi \To \Sigma', w:B}{\vdash^{n+1} \quad w:A, \Gamma \To \Delta', w:B}
				                  \end{mathpar}
			            \end{enumerate}
		      \end{description}
		      Le uniche eccezioni a questo procedimento sono rappresentate dalle regole con restrizioni sulle variabili.
		\item $R \equiv R\Box$ (va dimostrato che \emph{se $\latinmath{G3L} \vdash^{k} \Gamma \To \Delta, w:\Box A$, allora $\latinmath{G3L} \vdash^{k} wRv, \Gamma \To \Delta, v:A$}):
		      \begin{description}
			      \item[Base:] $p(D)=0$
			            \begin{itemize}
				            \item $u:p, \Gamma' \To \Delta', u:p, w:\Box A$: \\
				                  $wRv, u:p, \Gamma' \To \Delta', u:p, v:A$, con $v \notin \{\Gamma', \Delta', w\}$ \; è un \hyperlink{seqetin}{sequente iniziale}.
				            \item $w:\bot, \Gamma' \To \Delta, w:\Box A$: \\
				                  $wRv, w:\bot, \Gamma' \To \Delta, v:A$, con $v \notin \{\Gamma', \Delta', w\}$ \: \qquad è derivabile con \hyperlink{Lbotet}{$L\bot$}.
			            \end{itemize}
			      \item[Passo:] $p(D)=n+1$. \`{E} stata applicata almeno una regola; si esamina l'ultima applicata. Di ogni regola so che, se la sua conclusione è derivabile in $n+1$ passi, le sue premesse saranno derivabili in $n$ passi. \\
			            \latinmath{IH}: \emph{se \latinmath{G3L} $\vdash^n \Pi \To \Sigma', w:\Box A$, allora \latinmath{G3L} $\vdash^n wRv, \Pi \To \Sigma', v:A$, con $v \notin \{\Pi, \Sigma', w\}$} \\
			            Ho due casi:
			            \begin{enumerate}
				            \item \emph{$w:\Box A$ è principale nell'ultima regola applicata}: l'ultima regola applicata sarà un'istanza di $R\Box$.
				                  \begin{mathpar}
					                  \inferrule*[Right=$R\Box$]{\vdash^{n} \quad wRv, \Gamma \To \Delta, v:A}{\vdash^{n+1} \quad \Gamma \To \Delta, w: \Box A}
				                  \end{mathpar}
				                  con $v \notin \{\Gamma, \Delta, w\}$. Dal momento che $wRv, \Gamma \To \Delta, v:A$ è derivabile con profondità $n$, lo sarà anche con profondità $n+1$.
				            \item \emph{$w:\Box A$ \emph{non} è principale nell'ultima regola applicata}:
				                  \begin{mathpar}
					                  \inferrule*[Right=$\latinmath{\emph{Reg}}$]{\vdash^n \quad \Pi \To \Sigma', w: \Box A}{\vdash^{n+1} \quad \Gamma \To \Delta', w: \Box A}
				                  \end{mathpar}
				                  Per \latinmath{IH}, ho che:
				                  $$\vdash^n \quad wRv, \Pi \To \Sigma', v:A$$
				                  con $v \notin \{\Pi, \Sigma', w\}$.	Applico una \hyperlink{thsoset}{sostituzione} $\sfrac{v}{u}$:
				                  \begin{mathpar}
					                  \inferrule*[Right={$\left[\sfrac{v}{u}\right]$}]{\vdash^{n} \quad wRv, \Pi \To \Sigma', v:A}{\vdash^{n} \quad wRu, \Pi \To \Sigma', u:A}
				                  \end{mathpar}
				                  con $u \notin \{\Gamma, \Delta', \Pi, \Sigma', w, v\}$. Riapplico $\latinmath{\emph{Reg}}$:
				                  \begin{mathpar}
					                  \inferrule*[Right=$\latinmath{\emph{Reg}}$]{\vdash^n \quad wRu, \Pi \To \Sigma', u:A}{\vdash^{n+1} \quad wRu, \Gamma \To \Delta', u:A}
				                  \end{mathpar}
			            \end{enumerate}
		      \end{description}
		\item $R \equiv L\Dmd$ (analogo a $R\Box$; va dimostrato che \emph{se $\latinmath{G3L} \vdash^{k} w:\Dmd A, \Gamma \To \Delta$, allora $\latinmath{G3L} \vdash^{k} wRv, v:A, \Gamma \To \Delta$}):
		      \begin{description}
			      \item[Base:] $p(D)=0$
			            \begin{itemize}
				            \item $w:\Dmd A, u:p, \Gamma' \To \Delta', u:p$: \\
				                  $wRv, v:A, u:p, \Gamma' \To \Delta', u:p$, con $v \notin \{\Gamma', \Delta', w\}$ \; è un \hyperlink{seqetin}{sequente iniziale}.
				            \item $w:\Dmd A, w:\bot, \Gamma' \To \Delta$: \\
				                  $wRv, v:A, w:\bot, \Gamma' \To \Delta$, con $v \notin \{\Gamma', \Delta', w\}$ \: \qquad è derivabile con \hyperlink{Lbotet}{$L\bot$}.
			            \end{itemize}
			      \item[Passo:] $p(D)=n+1$. \`{E} stata applicata almeno una regola; si esamina l'ultima applicata. Di ogni regola so che, se la sua conclusione è derivabile in $n+1$ passi, le sue premesse saranno derivabili in $n$ passi. \\
			            \latinmath{IH}: \emph{se \latinmath{G3L} $\vdash^n w:\Dmd A, \Pi' \To \Sigma$, allora \latinmath{G3L} $\vdash^n wRv, v:A, \Pi' \To \Sigma$, con $v \notin \{\Pi', \Sigma, w\}$} \\
			            Ho due casi:
			            \begin{enumerate}
				            \item \emph{$w:\Dmd A$ è principale nell'ultima regola applicata}: l'ultima regola applicata sarà un'istanza di $L\Dmd$.
				                  \begin{mathpar}
					                  \inferrule*[Right=$L\Dmd$]{\vdash^{n} \quad wRv, v:A, \Gamma \To \Delta}{\vdash^{n+1} \quad w: \Dmd A, \Gamma \To \Delta}
				                  \end{mathpar}
				                  con $v \notin \{\Gamma, \Delta, w\}$. Dal momento che $wRv, v:A, \Gamma \To \Delta$ è derivabile con profondità $n$, lo sarà anche con profondità $n+1$.
				            \item \emph{$w:\Dmd A$ \emph{non} è principale nell'ultima regola applicata}:
				                  \begin{mathpar}
					                  \inferrule*[Right=$\latinmath{\emph{Reg}}$]{\vdash^n \quad w: \Dmd A, \Pi' \To \Sigma}{\vdash^{n+1} \quad w: \Dmd A, \Gamma' \To \Delta}
				                  \end{mathpar}
				                  Per \latinmath{IH}, ho che:
				                  $$\vdash^n \quad wRv, v:A, \Pi' \To \Sigma$$
				                  con $v \notin \{\Pi', \Sigma, w\}$. Applico una \hyperlink{thsoset}{sostituzione} $\sfrac{v}{u}$:
				                  \begin{mathpar}
					                  \inferrule*[Right={$\left[\sfrac{v}{u}\right]$}]{\vdash^{n} \quad wRv, v:A, \Pi' \To \Sigma}{\vdash^{n} \quad wRu, u:A, \Pi' \To \Sigma}
				                  \end{mathpar}
				                  con $u \notin \{\Gamma', \Delta, \Pi', \Sigma, w, v\}$. Riapplico $\latinmath{\emph{Reg}}$:
				                  \begin{mathpar}
					                  \inferrule*[Right=$\latinmath{\emph{Reg}}$]{\vdash^n \quad wRu, u:A, \Pi' \To \Sigma}{\vdash^{n+1} \quad wRu, u:A, \Gamma' \To \Delta}
				                  \end{mathpar}
			            \end{enumerate}
		      \end{description}
		      Per quanto riguarda le regole che ripetono le formule principali nelle premesse, il procedimento è invece il seguente:
		\item $R \equiv L\Box$ (va dimostrato che \emph{se $\latinmath{G3L} \vdash^k wRv, w: \Box A, \Gamma \To \Delta$, allora $\latinmath{G3L} \vdash^k v:A, wRv, w: \Box A, \Gamma \To \Delta$}):
		      \begin{mathpar}
			      \inferrule*[Right=$W$]{\vdash^n \quad wRv, w: \Box A, \Gamma \To \Delta}{\vdash^n \quad v:A, wRv, w: \Box A, \Gamma \To \Delta}
		      \end{mathpar}
		      Segue immediatamente dalla \hyperlink{weakg3l}{\emph{pp}-ammissibilità della regola di indebolimento}.
		\item $R \equiv R\Dmd$ (analogo a $L\Box$; va dimostrato che \emph{se $\latinmath{G3L} \vdash^k wRv, \Gamma \To \Delta, w:\Dmd A, v:A$, allora $\latinmath{G3L} \vdash^k wRv, \Gamma \To \Delta, w:\Dmd A$}):
		      \begin{mathpar}
			      \inferrule*[Right=$W$]{\vdash^n \quad wRv, \Gamma \To \Delta, w:\Dmd A}{\vdash^n \quad wRv, \Gamma \To \Delta, w:\Dmd A, v:A}
		      \end{mathpar}
		      Segue immediatamente dalla \hyperlink{weakg3l}{\emph{pp}-ammissibilità della regola di indebolimento}.
		\item \emph{Regole non logiche}: tutte le regole non logiche ripetono nelle premesse le formule principali, dunque il procedimento è analogo ai due casi precedenti. \\
	\end{itemize}
\end{dimo}

\subsubsubs{Teorema: \emph{pp}-ammissibilità delle regole di contrazione}
\hypertarget{amcont}{}
\begin{theo}
	Le seguenti regole di contrazione sono \hyperlink{ppamet}{\emph{pp}-ammissibili} in \latinmath{G3L}:
	\begin{mathpar}
		\inferrule*[Right=$LCR$]{wRv, wRv, \Gamma \To \Delta}{wRv, \Gamma \To \Delta}
		\hspace{1.5cm}
		\inferrule*[Right=$LC$]{w:A, w:A, \Gamma \To \Delta}{w:A, \Gamma \To \Delta}
		\hspace{1.5cm}
		\inferrule*[Right=$RC$]{\Gamma \To \Delta, w:A, w:A}{\Gamma \To \Delta, w:A}
	\end{mathpar}
\end{theo}

\begin{dimo}
	[per induzione simultanea \protect\footnote{Con \enquote{simultanea} si intende che vi è una congiunzione che unisce le diverse ipotesi induttive.} per le regole di contrazione sulla profondità della derivazione $D$ della premessa] \phantom{ciao}
	\begin{description}
		\item[Base:] $p(D) = 0$
		      \begin{itemize}
			      \item $w:p, \Gamma \To \Delta, w:p$: \\
			            Supponendo che sia premessa di $LC$, ho due casi:
			            \begin{enumerate}
				            \item $\Gamma \equiv w:p, \Gamma'$ \qquad \qquad in $\Gamma$ occorre un'altra istanza della formula principale.
				            \item $\Gamma \equiv w:A, w:A, \Gamma'$ \quad la formula da contrarre non è principale.
			            \end{enumerate}
			            Se la premessa di una regola di contrazione è un \hyperlink{seqetin}{sequente iniziale}, lo sarà anche la sua conclusione.
		      \end{itemize}
		\item[Passo:] $p(D) = n+1$. \`{E} stata applicata almeno una regola; si esamina l'ultima applicata. Di ogni regola so che, se la sua conclusione è derivabile in $n+1$ passi, le sue premesse saranno derivabili in $n$ passi. \\
		      \latinmath{IH}: \emph{tutte le derivazioni di profondità al più $n$ in \latinmath{G3L} sono derivazioni la cui contrazione è \hyperlink{ppamet}{\emph{pp}-ammissibile}.} \\
		      Va dimostrato che \emph{se $\latinmath{G3L} \vdash^{n+1} \Gamma \To \Delta, w:A, w:A$, allora $\latinmath{G3L} \vdash^{n+1} \Gamma \To \Delta, w:A$}. Distinguo tre casi di una generica regola:
		      \vspace*{-4pt}
		      \begin{mathpar}
			      \inferrule*[Right=$R$]{\Gamma' \To \Delta' \\ (\Gamma'' \To \Delta'')}{\Gamma \To \Delta, w:A, w:A}
			      \vspace*{-12pt}
		      \end{mathpar}
		      \begin{enumerate}
			      \item \emph{Nessuna istanza della contrazione è principale in $R$:} l'ultima regola applicata è una qualsiasi regola ad una premessa.
			            \begin{mathpar}
				            \inferrule*[Right=$R$]{\vdash^{n} \quad \Gamma \To \Delta', w:A, w:A}{\vdash^{n+1} \quad \Gamma \To \Delta', w:A, w:A}
			            \end{mathpar}
			            Per \latinmath{IH}, ho che:
			            $$\vdash^{n} \quad \Gamma \To \Delta', w:A$$
			            Riapplico $R$:
			            \begin{mathpar}
				            \inferrule*[Right=$R$]{\vdash^{n} \quad \Gamma \To \Delta', w:A}{\vdash^{n+1} \quad \Gamma \To \Delta', w:A}
			            \end{mathpar}
			      \item \emph{Una istanza della contrazione è principale in $R$:} ho un caso per ogni regola, a seconda dell'ultima applicata (solo alcuni casi).
			            \begin{itemize}
				            \item $R\land$:
				                  \begin{mathpar}
					                  \inferrule*[Right=$R\land$]{\vdash^{n} \; \Gamma \To \Delta, w:B \land C, w:B \\ \vdash^{n} \; \Gamma \To \Delta, w:B \land C, w:C}{\vdash^{n+1} \quad \Gamma \To \Delta, w:B \land C, w:B \land C}
				                  \end{mathpar}
				                  Applico l'\hyperlink{invg3l}{invertibilità} sulle premesse:
				                  $$\vdash^{n} \; \Gamma \To \Delta, w:B, w:B \quad \; \text{e} \quad \; \vdash^{n} \; \Gamma \To \Delta, w:C, w:C$$
				                  Per \latinmath{IH} ($RC$):
				                  $$\vdash^{n} \; \Gamma \To \Delta, w:B \quad \; \text{e} \quad \; \vdash^{n} \; \Gamma \To \Delta, w:C$$
				                  Riapplico $R\land$:
				                  \begin{mathpar}
					                  \inferrule*[Right=$R\land$]{\vdash^{n} \; \Gamma \To \Delta, w:B \\ \vdash^{n} \; \Gamma \To \Delta, w:C}{\vdash^{n+1} \quad \Gamma \To \Delta, w:B \land C}
				                  \end{mathpar}
				            \item $L\to$ (analogo a $R\land$):
				                  \begin{mathpar}
					                  \inferrule*[Right=$L\to$]{\vdash^{n} \quad w:B \to C, \Gamma \To \Delta, w:B  \\ \vdash^{n} \quad w:C, w:B \to C, \Gamma \To \Delta}{\vdash^{n+1} \quad w:B \to C, w:B \to C, \Gamma \To \Delta}
				                  \end{mathpar}
				                  Applico l'\hyperlink{invg3l}{invertibilità} sulle premesse:
				                  $$\vdash^{n} \; \Gamma \To \Delta, w:B, w:B \quad \; \text{e} \quad \; \vdash^{n} \; w:C, w:C, \Gamma \To \Delta$$
				                  Per \latinmath{IH} ($LC$ e $RC$):
				                  $$\vdash^{n} \; \Gamma \To \Delta, w:B \quad \; \text{e} \quad \; \vdash^{n} \; w:C, \Gamma \To \Delta$$
				                  Riapplico $L\to$:
				                  \begin{mathpar}
					                  \inferrule*[Right=$L\to$]{\vdash^{n} \; \Gamma \To \Delta, w:B \\ \vdash^{n} \; w:C, \Gamma \To \Delta}{\vdash^{n+1} \quad \Gamma \To \Delta, w:B \land C}
				                  \end{mathpar}
				            \item $R\to$:
				                  \begin{mathpar}
					                  \inferrule*[Right=$R\to$]{\vdash^{n} \quad w:B, \Gamma \To \Delta, w:C, w:B \to C}{\vdash^{n+1} \quad \Gamma \To \Delta, w:B \to C, w:B \to C}
				                  \end{mathpar}
				                  Applico l'\hyperlink{invg3l}{invertibilità} sulla premessa:
				                  $$\vdash^{n} \quad w:B, w:B, \Gamma \To \Delta, w:C, w:C$$
				                  Per \latinmath{IH} ($LC$ e $RC$):
				                  $$\vdash^{n} \quad w:B, \Gamma \To \Delta, w:C$$
				                  Riapplico $R\to$:
				                  \begin{mathpar}
					                  \inferrule*[Right=$R\to$]{\vdash^{n} \quad w:B, \Gamma \To \Delta, w:C}{\vdash^{n+1} \quad \Gamma \To \Delta, w:B \to C}
				                  \end{mathpar}
				            \item $R\Box$:
				                  \begin{mathpar}
					                  \inferrule*[Right=$R\Box$]{\vdash^{n} \quad wRv, \Gamma \To \Delta, w:\Box B, v:B}{\vdash^{n+1} \quad \Gamma \To \Delta, w:\Box B, w:\Box B}
				                  \end{mathpar}
				                  con $v \notin \{\Gamma, \Delta, w\}$. Applico l'\hyperlink{invg3l}{invertibilità} sulla premessa:
				                  $$\vdash^{n} \quad wRu, wRv, \Gamma \To \Delta, u:B, v:B$$
				                  con $u \notin \{\Gamma, \Delta, w\}$. Applico una \hyperlink{thsoset}{sostituzione} $\sfrac{v}{u}$:
				                  \begin{mathpar}
					                  \inferrule*[Right={$\left[\sfrac{v}{u}\right]$}]{\vdash^{n} \quad wRu, wRv, \Gamma \To \Delta, u:B, v:B}{\vdash^{n} \quad wRv, wRv, \Gamma \To \Delta, v:B, v:B}
				                  \end{mathpar}
				                  con $v \notin \{\Gamma, \Delta, w\}$. Per \latinmath{IH} ($LCR$ e $RC$):
				                  $$\vdash^{n} \quad wRv, \Gamma \To \Delta, v:B$$
				                  Riapplico $R\Box$:
				                  \begin{mathpar}
					                  \inferrule*[Right=$R\Box$]{\vdash^{n} \quad wRv, \Gamma \To \Delta, v:B}{\vdash^{n+1} \quad \Gamma \To \Delta, w:\Box B}
				                  \end{mathpar}
				                  con $v \notin \{\Gamma, \Delta, w\}$.
			            \end{itemize}
			      \item \emph{Entrambe le istanze della contrazione sono principali in $R$:} può darsi solo con le regole non logiche con almeno due formule principali; di fatto, solo con \latinmath{Trans} e \latinmath{Euclid}.
			            \begin{itemize}
				            \item \latinmath{Euclid}:
				                  \begin{mathpar}
					                  \inferrule*[Right=$\latinmath{Euclid}$]{vRu, wRv, wRu, \Gamma \To \Delta}{wRv, wRu, \Gamma \To \Delta}
				                  \end{mathpar}
				                  Se \latinmath{Euclid} è l'ultima regola ed entrambe le istanze della contrazione sono principali in essa, avrò che $v$ e $u$ sono la stessa etichetta:
				                  \begin{mathpar}
					                  \inferrule*[Right=$\latinmath{Euclid}$]{\vdash^{n} \quad vRv, wRv, wRv, \Gamma \To \Delta}{\vdash^{n+1} \quad wRv, wRv, \Gamma \To \Delta}
				                  \end{mathpar}
				                  Per \latinmath{IH}, la premessa diventa:
				                  $$\vdash^{n} \quad vRv, wRv, \Gamma \To \Delta$$
				                  A questo punto \latinmath{Euclid} non è più applicabile, ma lo è la sua istanza contratta, che ho come primitiva nel calcolo:
				                  \begin{mathpar}
					                  \inferrule*[Right=$\latinmath{Euclid}^C$]{\vdash^{n} \quad vRv, wRv, \Gamma \To \Delta}{\vdash^{n+1} \quad wRv, \Gamma \To \Delta}
				                  \end{mathpar}
				                  Ho reso quindi la contrazione ammissibile implicitandola in un'altra regola. Senza di essa, non si avrebbe un calcolo completo (ad esempio, $\Box (\Box A \to A)$ non sarebbe derivabile in \latinmath{G35}).
				            \item \latinmath{Trans}:
				                  \begin{mathpar}
					                  \inferrule*[Right=$\latinmath{Trans}$]{wRu, wRv, vRu, \Gamma \To \Delta}{wRv, vRu, \Gamma \To \Delta}
				                  \end{mathpar}
				                  Anche questa regola può avere un'istanza contratta, ma la si ottiene solo se le due formule principali (quelle da contrarre) sono la stessa formula, ovvero se $w$, $v$ e $u$ sono la stessa etichetta:
				                  \begin{mathpar}
					                  \inferrule*[Right=$\latinmath{Trans}$]{\vdash^n \quad wRw, wRw, wRw, \Gamma \To \Delta}{\vdash^{n+1} \quad wRw, wRw, \Gamma \To \Delta}
				                  \end{mathpar}
				                  Per \latinmath{IH}, la premessa diventa:
				                  $$\vdash^n \quad wRw, wRw, \Gamma \To \Delta$$
				                  A questo punto \latinmath{Trans} non è più applicabile, ma posso usare ancora \latinmath{IH}:
				                  $$\vdash^n \quad wRw, \Gamma \To \Delta$$
				                  La regola \latinmath{Trans}, quindi, è \hyperlink{ppamet}{\emph{pp}-ammissibile} (e, dunque, $\latinmath{Trans}^C$ non è necessaria).
			            \end{itemize}
		      \end{enumerate}
	\end{description}
\end{dimo}

\subsubsubs{Teorema: ammissibilità della regola di taglio}
\begin{theo}
	La seguente regola di taglio è \hyperlink{ammisset}{ammissibile} in \latinmath{G3L}:
	\begin{mathpar}
		\inferrule*[Right=\latinmath{cut}]{\Gamma \To \Delta, w:A \\ w:A, \Pi \To \Sigma}{\Gamma, \Pi \To \Delta, \Sigma}
	\end{mathpar}
\end{theo}

\begin{dimo}
	[per induzione sul numero di applicazioni di \latinmath{cut}] \phantom{ciao} \\
	Per dimostrare questo teorema, bisogna fornire un \textbf{Lemma}. \\
	\hypertarget{lemcut}{}
	\begin{lem}
		Un \latinmath{cut} \emph{uppermost} (che non ha altri \latinmath{cut} sopra di sé) è \hyperlink{ammisset}{ammissibile} in \latinmath{G3L}.
	\end{lem}
	\begin{dimo}
		La dimostrazione si svolge per induzione su:
		\begin{enumerate}
			\item $\latinmath{lg}(w:A)$ \qquad \qquad \quad \thinspace (principale: sulla lunghezza della formula di \latinmath{cut}) \\
			      %\latinmath{IH1}: \emph{per tutte le formule di lunghezza minore di $A$, il \latinmath{cut} \emph{uppermost} è ammissibile}. \\
			      \emph{$\forall B \in \latinmath{fm}^{\varepsilon}$, se $\latinmath{lg}(B)<\latinmath{lg}A$, ogni \latinmath{cut} \emph{uppermost} con formula $B$ è ammissibile.}
			\item $p(D_{sx}) + p(D_{dx})$ \qquad (secondaria: sulla profondità delle derivazioni delle premesse) \\
			      %\latinmath{IH2}: \emph{per la stessa formula $A$, il \latinmath{cut} \emph{uppermost} è ammissibile se la somma delle profondità delle derivazioni delle premesse è minore}. \\
			      \emph{Se $p(D'_{sx}) + p(D'_{dx})<p(D_{sx}) + p(D_{dx})$, allora il \latinmath{cut} \emph{uppermost} su $A$ è ammissibile}.
		\end{enumerate}
		Una dimostrazione a doppia induzione utilizza una coppia di valori ordinati in modo lessicografico \footnote{Come in un dizionario.}: a parità del primo parametro, si prende in considerazione il secondo (ovvero, a parità di lunghezza della formula si considera la profondità delle derivazioni delle premesse).

		Di fatto, le \latinmath{IH} funzioneranno come dei \latinmath{cut}, che però sappiamo essere ammissibili; ovvero, ciò che deriviamo con essi lo possiamo derivare anche senza.

		Si noti che la formula di \latinmath{cut} non può mai essere un atomo relazionale, poiché essi non compaiono mai nella parte destra del sequente.

		In tutti i casi, devo sempre ipotizzare etichette diverse da $w$ (che è quella della formula del \latinmath{cut}), poiché se non lo facessi starei dando per scontato che sia la stessa ovunque. \\
		Ho 3 casi possibili:
		\begin{enumerate}
			\item \footnote{Questo punto funge da caso base della dimostrazione.}\emph{Almeno una premessa è una foglia \footnote{Ovvero, non ha altri sequenti sopra di sé.}:} supponendo che $D_{dx}$ sia una foglia, ho due casi:
			      \begin{enumerate}
				      \item un \hyperlink{seqetin}{sequente iniziale}; ho due casi:
				            \begin{itemize}
					            \item $v:p$ \emph{non} è la formula di \latinmath{cut}:
					                  \begin{mathpar}
						                  \inferrule*[Right=\latinmath{cut}]{\Gamma \To \Delta, w:A \\ w:A, v:p, \Pi' \To \Sigma', v:p}{v:p, \Gamma, \Pi' \To \Delta, \Sigma', v:p}
					                  \end{mathpar}
					                  La conclusione è derivabile senza \latinmath{cut}, trattandosi di un sequente iniziale.
					            \item $v:p$ è la formula di \latinmath{cut}:
					                  \begin{mathpar}
						                  \inferrule*[Right=\latinmath{cut}]{\Gamma \To \Delta, w:p \\ w:p, \Pi \To \Sigma', w:p}{\Gamma, \Pi \To \Delta, \Sigma', w:p}
					                  \end{mathpar}
					                  Si può ottenere la stessa conclusione applicando la \hyperlink{weakg3l}{regola di indebolimento} su $D_{sx}$:
					                  \begin{mathpar}
						                  \inferrule*[Right=$W$]{\Gamma \To \Delta, w:p}{\Gamma, \Pi \To \Delta, \Sigma', w:p}
					                  \end{mathpar}
					                  Si può concludere che, in questi casi, il \latinmath{cut} è addirittura eliminabile.
				            \end{itemize}
				      \item ottenuta via \hyperlink{Lbotet}{$L\bot$}; ho due casi:
				            \begin{itemize}
					            \item $v:\bot$ \emph{non} è la formula di \latinmath{cut}:
					                  \begin{mathpar}
						                  \inferrule*[Right=\latinmath{cut}]{\Gamma \To \Delta, w:A \\ w:A, v:\bot, \Pi' \To \Sigma}{v:\bot, \Gamma, \Pi' \To \Delta, \Sigma}
					                  \end{mathpar}
					                  La conclusione è derivabile senza \latinmath{cut}, trattandosi di un'istanza di $L\bot$.
					            \item $v:\bot$ è la formula di \latinmath{cut}:
					                  \begin{mathpar}
						                  \inferrule*[Right=\latinmath{cut}]{\Gamma \To \Delta, w:\bot \\ w:\bot, \Pi \To \Sigma}{\Gamma, \Pi \To \Delta, \Sigma}
					                  \end{mathpar}
					                  Per proseguire, devo dimostrare un \textbf{Lemma}:
					                  \begin{lem}
						                  La regola $R\bot$ è \hyperlink{ppamet}{\emph{pp}-ammissibile} in \latinmath{G3L}:
						                  \begin{mathpar}
							                  \inferrule*[Right=$R\bot$]{\Gamma \To \Delta, w:\bot}{\Gamma \To \Delta}
						                  \end{mathpar}
					                  \end{lem}
					                  Questo \textbf{Lemma} afferma che un'occorrenza del falso a destra si può eliminare.
					                  \begin{dimo}
						                  [per induzione sulla profondità della derivazione $D$ della premessa $\Gamma \To \Delta, w:\bot$] \phantom{ciao}
						                  \begin{description}
							                  \item[Base:] $p(D)=0$
							                        \begin{itemize}
								                        \item $v:p, \Gamma' \To \Delta', v:p, w:\bot$: \\
								                              $v:p, \Gamma' \To \Delta', v:p$ \qquad \quad è un \hyperlink{seqetin}{sequente iniziale}.
								                        \item $v:\bot, \Gamma' \To \Delta, w:\bot$: \\
								                              $v:\bot, \Gamma' \To \Delta$ \qquad \qquad \quad \thinspace è derivabile con \hyperlink{Lbotet}{$L\bot$}.
							                        \end{itemize}
							                  \item[Passo:] $p(D)=n+1$. \`{E} stata applicata almeno una regola; si esamina l'ultima applicata. Di ogni regola so che, se la sua conclusione è derivabile in $n+1$ passi, le sue premesse saranno derivabili in $n$ passi (solo alcuni casi). \\
							                        \latinmath{IH}: \emph{se $\latinmath{G3L} \vdash^{n} \Gamma \To \Delta, w:\bot$, allora $\latinmath{G3L} \vdash^{n} \Gamma \To \Delta$}
							                        \begin{itemize}
								                        \item $R\lor$:
								                              \begin{mathpar}
									                              \inferrule*[Right=$R\lor$]{\vdash^{n} \quad \Gamma \To \Delta', v:A, v:B, w:\bot}{\vdash^{n+1} \quad \Gamma \To \Delta', v:A \land B, w:\bot}
								                              \end{mathpar}
								                              Per \latinmath{IH}, la premessa diventa:
								                              $$\vdash^{n} \quad \Gamma \To \Delta', v:A, v:B$$
								                              Riapplico $R\lor$:
								                              \begin{mathpar}
									                              \inferrule*[Right=$R\lor$]{\vdash^{n} \quad \Gamma \To \Delta', v:A, v:B}{\vdash^{n+1} \quad \Gamma \To \Delta', v:A \land B}
								                              \end{mathpar}
								                              Il risultato ottenuto è lo stesso in questo modo o applicando $R\bot$.
								                        \item $L\to$:
								                              \begin{mathpar}
									                              \inferrule*[Right=$L\to$]{\vdash^{n} \; \Gamma' \To \Delta, v:A, w:\bot \\ \vdash^{n} \; v:B, \Gamma' \To \Delta, w:\bot}{\vdash^{n+1} \quad v:A \to B, \Gamma' \To \Delta, w:\bot}
								                              \end{mathpar}
								                              Per \latinmath{IH}, le premesse diventano:
								                              $$\vdash^{n} \; \Gamma' \To \Delta, v:A \; \quad \text{e} \; \quad \vdash^{n} \; v:B, \Gamma' \To \Delta$$
								                              Riapplico $L\to$:
								                              \begin{mathpar}
									                              \inferrule*[Right=$L\to$]{\vdash^{n} \; \Gamma' \To \Delta, v:A \\ \vdash^{n} \; v:B, \Gamma' \To \Delta, w:\bot}{\vdash^{n+1} \quad v:A \to B, \Gamma' \To \Delta}
								                              \end{mathpar}
							                        \end{itemize}
							                        Tutti gli altri casi sono analoghi.
						                  \end{description}
					                  \end{dimo}
				            \end{itemize}
				            Possiamo proseguire con la dimostrazione del caso in cui $D_{dx}$ sia una foglia ottenuta via $L\bot$ che contiene $v:\bot$ come formula di \latinmath{cut}. \\
				            Applico il \textbf{Lemma} precedente su $D_{sx}$:
				            \begin{mathpar}
					            \inferrule*[Right=$R\bot$]{\Gamma \To \Delta, w:\bot}{\Gamma \To \Delta}
				            \end{mathpar}
				            Applico la \hyperlink{weakg3l}{regola di indebolimento} sulla conclusione ottenuta:
				            \begin{mathpar}
					            \inferrule*[Right=$W$]{\Gamma \To \Delta}{\Gamma, \Pi \To \Delta, \Sigma}
				            \end{mathpar}
				            La conclusione, dunque, è derivabile senza \latinmath{cut}.
			      \end{enumerate}
			      Nel caso in cui è $D_{sx}$ ad essere una foglia, procedo allo stesso modo, ma non ho da dimostrare quest'ultimo caso (non sarebbe stato possibile avere $w:\bot$ come formula di \latinmath{cut} \footnote{Se $w:\bot$ è a sinistra in $D_{sx}$, non si può applicare \latinmath{cut}; se è a destra, $D_{sx}$ non è una foglia.}).
			\item \emph{La formula di taglio non è principale in almeno una premessa:} supponendo non lo sia in $D_{sx}$, allora significa è stata applicata una regola che ha come conclusione $\Gamma \To \Delta, w:A$; quindi, ho un caso per ogni regola, eccetto $L\bot$ (solo alcuni casi).

			      In tutti questi casi la lunghezza della formula di \latinmath{cut} è sempre la stessa (abbiamo sempre a che fare con $A$), dunque dovrò applicare \latinmath{IH2}. Si noti inoltre che, nel trasformare le derivazioni, si utilizzano sempre le stesse premesse, cioè si lascia invariato il ramo che ha portato ad esse (volendo, si può scrivere $D_x$ sopra ad ogni premessa per sistemarle più facilmente).
			      \begin{itemize}
				      \item $L\land$:
				            \begin{mathpar}
					            \inferrule*[Right=\latinmath{cut}]{\inferrule*[Right=$L\land$]{v:B, v:C, \Gamma' \To \Delta, w:A}{v:B \land C, \Gamma' \To \Delta, w:A} \\ w:A, \Pi \To \Sigma}{v:B \land C, \Gamma', \Pi \To \Delta, \Sigma}
				            \end{mathpar}
				            Posso trasformare la derivazione in modo che le istanze di \latinmath{cut} si trovino ad una profondità minore, e siano dunque ammissibili per \latinmath{IH2}:
				            \begin{mathpar}
					            \inferrule*[Right=$L\land$]{\inferrule*[Right=$\latinmath{IH2}$]{v:B, v:C, \Gamma' \To \Delta, w:A \\ w:A, \Pi \To \Sigma}{v:B, v:C, \Gamma', \Pi \To \Delta, \Sigma}}{v:B \land C, \Gamma', \Pi \To \Delta, \Sigma}
				            \end{mathpar}
				      \item $R\land$:
				            \begin{mathpar}
					            \inferrule*[Right=\latinmath{cut}]{\inferrule*[Right=$R\land$]{\Gamma \To \Delta', v:B, w:A \\ \Gamma \To \Delta', v:C, w:A}{\Gamma \To \Delta', v:B \land C, w:A} \\ w:A, \Pi \To \Sigma}{\Gamma, \Pi \To \Delta', \Sigma, v:B \land C}
				            \end{mathpar}
				            Posso trasformare la derivazione in modo che le istanze di \latinmath{cut} si trovino ad una profondità minore, e siano dunque ammissibili per \latinmath{IH2}; inoltre, trattandosi di una regola a due premesse, devo applicare una o più \hyperlink{amcont}{regole di contrazione} (scrivo $C$, ma sto applicando $LC$ e $RC$ contemporaneamente e due volte ciascuna):
				            \begin{mathpar}
					            \inferrule*[Right=$C$]{\inferrule*[Right=$R\land$]{\inferrule*[Right=$\latinmath{IH2}$]{\Gamma \To \Delta', v:B, w:A \! \\ \! w:A, \Pi \To \Sigma}{\Gamma, \Pi \To \Delta', \Sigma, v:B}\! \\ \! \inferrule*[Right=$\latinmath{IH2}$]{\Gamma \To \Delta', v:C, w:A \! \\ \! w:A, \Pi \To \Sigma}{\Gamma, \Pi \To \Delta', \Sigma, v:C}}{\Gamma, \Gamma, \Pi, \Pi \To \Delta', \Delta', \Sigma, \Sigma, v:B \land C}}{\Gamma, \Pi \To \Delta', \Sigma, v:B \land C}
				            \end{mathpar}
				      \item $L\to$:
				            \begin{mathpar}
					            \inferrule*[Right=$\latinmath{cut}$]{\inferrule*[Right=$L\to$]{\Gamma' \To \Delta, v:B, w:A \\ v:C, \Gamma' \To \Delta, w:A}{v:B \to C, \Gamma' \To \Delta, w:A} \\ w:A, \Pi \To \Sigma}{v:B \to C, \Gamma', \Pi \To \Delta, \Sigma}
				            \end{mathpar}
				            Posso trasformare la derivazione in modo che le istanze di \latinmath{cut} si trovino ad una profondità minore, e siano dunque ammissibili per \latinmath{IH2}; inoltre, trattandosi di una regola a due premesse, devo applicare una o più \hyperlink{amcont}{regole di contrazione}:
				            \begin{mathpar}
					            \inferrule*[Right=$C$]{\inferrule*[Right=$L\to$]{\inferrule*[Right=$\latinmath{IH2}$]{\Gamma' \To \Delta, v:B, w:A \! \\ \! w:A, \Pi \To \Sigma}{\Gamma', \Pi \To \Delta, \Sigma, v:B}\! \\ \! \inferrule*[Right=$\latinmath{IH2}$]{v:C, \Gamma' \To \Delta', w:A \! \\ \! w:A, \Pi \To \Sigma}{v:C, \Gamma', \Pi \To \Delta, \Sigma}}{v:B \to C, \Gamma', \Gamma', \Pi, \Pi \To \Delta, \Delta, \Sigma, \Sigma}}{v:B \to C, \Gamma', \Pi \To \Delta, \Sigma}
				            \end{mathpar}
				      \item $L\Dmd$:
				            \begin{mathpar}
					            \inferrule*[Right=\latinmath{cut}]{\inferrule*[Right=$L\Dmd$]{vRu, u:B, \Gamma' \To \Delta, w:A}{v: \Dmd B, \Gamma' \To \Delta, w:A} \\ w:A, \Pi \To \Sigma}{v: \Dmd B, \Gamma', \Pi \To \Delta, \Sigma}
				            \end{mathpar}
				            con $u \notin \{\Gamma', \Delta, v, w\}$. Non posso trasformare la derivazione direttamente, poiché non so con certezza che $u$ non occorra in $\Pi$ o $\Sigma$; dunque, devo prima applicare una \hyperlink{thsoset}{sostituzione} \sfrac{t}{u}. Poi, per \latinmath{IH2}:
				            \begin{mathpar}
					            \inferrule*[Right=$L\Dmd$]{\inferrule*[Right=$\latinmath{IH2}$]{\inferrule*[Right={$\left[\sfrac{t}{u}\right]$}]{vRu, u:B, \Gamma' \To \Delta, w:A}{vRt, t:B, \Gamma' \To \Delta, w:A} \\ w:A, \Pi \To \Sigma}{vRt, t:B, \Gamma', \Pi \To \Delta, \Sigma}}{v: \Dmd B, \Gamma', \Pi \To \Delta, \Sigma}
				            \end{mathpar}
				            con $t \notin \{\Gamma', \Pi. \Delta, \Sigma, v, w\}$.
			      \end{itemize}
			      Tutti gli altri casi sono analoghi.
			\item \emph{La formula di taglio è principale in entrambe le premesse:} dal momento che un atomo relazionale non può essere formula principale, le regole non logiche sono escluse dalla trattazione di questo punto; dunque, ho un caso per ogni connettivo.
			      \begin{itemize}
				      \item $A \equiv B \land C$:
				            \begin{mathpar}
					            \inferrule*[Right=$\latinmath{cut}$]{\inferrule*[Right=$R\land$]{\Gamma \To \Delta, w:B \\ \Gamma \To \Delta, w:C}{\Gamma \To \Delta, w:B \land C} \\ \inferrule*[Right=$L\land$]{w:B, w:C, \Pi \To \Sigma}{w:B \land C, \Pi \To \Sigma}}{\Gamma, \Pi \To \Delta, \Sigma}
				            \end{mathpar}
				            Posso trasformare la derivazione in modo che le istanze di \latinmath{cut} si applichino a formule di lunghezza minore, e siano dunque ammissibili per \latinmath{IH1}; inoltre, devo applicare una o più \hyperlink{amcont}{regole di contrazione}:
				            \begin{mathpar}
					            \inferrule*[Right=$LC$]{\inferrule*[Right=$\latinmath{IH1}$]{\Gamma \To \Delta, w:C \\ \inferrule*[Right=$\latinmath{IH1}$]{\Gamma \To \Delta, w:B \\ w:B, w:C, \Pi \To \Sigma}{w:C, \Gamma, \Pi \To \Delta, \Sigma}}{\Gamma, \Gamma, \Pi \To \Delta, \Delta, \Sigma}}{\Gamma, \Pi \To \Delta, \Sigma}
				            \end{mathpar}
				            Nel primo caso, invece di \latinmath{IH1} avrei potuto usare anche \latinmath{IH2}, avendo diminuito la profondità della derivazione; nel secondo, invece, sono costretto a usare \latinmath{IH1}, perché il \latinmath{cut} è alla stessa profondità (si ricordi che le regole di contrazione sono \hyperlink{ppamet}{\emph{pp}-ammissibili})
				      \item $A \equiv B \lor C$:
				            \begin{mathpar}
					            \inferrule*[Right=$\latinmath{cut}$]{\inferrule*[Right=$R\lor$]{\Gamma \To \Delta, w:B, w:C}{\Gamma \To \Delta, w:B \lor C} \\ \inferrule*[Right=$L\lor$]{w:B, \Pi \To \Sigma \\ w:C, \Pi \To \Sigma}{w:B \lor C, \Pi \To \Sigma}}{\Gamma, \Pi \To \Delta, \Sigma}
				            \end{mathpar}
				            Posso trasformare la derivazione in modo che le istanze di \latinmath{cut} si applichino a formule di lunghezza minore, e siano dunque ammissibili per \latinmath{IH1}; inoltre, devo applicare una o più \hyperlink{amcont}{regole di contrazione}:
				            \begin{mathpar}
					            \inferrule*[Right=$C$]{\inferrule*[Right=$\latinmath{IH1}$]{\inferrule*[Right=$\latinmath{IH1}$]{\Gamma \To \Delta, w:B, w:C \\ w:C, \Pi \To \Sigma}{\Gamma, \Pi \To \Delta, \Sigma, w:B} \\ w:B, \Pi \To \Sigma}{\Gamma, \Pi, \Pi \To \Delta, \Sigma, \Sigma}}{\Gamma, \Pi \To \Delta, \Sigma}
				            \end{mathpar}
				      \item $A \equiv B \to C$:
				            \begin{mathpar}
					            \inferrule*[Right=$\latinmath{cut}$]{\inferrule*[Right=$R\to$]{w:B, \Gamma \To \Delta, w:C}{\Gamma \To \Delta, w:B \to C} \\ \inferrule*[Right=$L\to$]{\Pi \To \Sigma, w:B \\ w:C, \Pi \To \Sigma}{w:B \to C, \Pi \To \Sigma}}{\Gamma, \Pi \To \Delta, \Sigma}
				            \end{mathpar}
				            Posso trasformare la derivazione in modo che le istanze di \latinmath{cut} si applichino a formule di lunghezza minore, e siano dunque ammissibili per \latinmath{IH1}; inoltre, devo applicare una o più \hyperlink{amcont}{regole di contrazione}:
				            \begin{mathpar}
					            \inferrule*[Right=$C$]{\inferrule*[Right=$\latinmath{IH1}$]{\inferrule*[Right=$\latinmath{IH1}$]{\Gamma \To \Delta, w:B \\ w:B, \Pi \To \Sigma, w:C}{\Gamma, \Pi \To \Delta, \Sigma, w:C} \\ w:C, \Pi \To \Sigma}{\Gamma, \Pi, \Pi \To \Delta, \Sigma, \Sigma}}{\Gamma, \Pi \To \Delta, \Sigma}
				            \end{mathpar}
				      \item $A \equiv \Box B$:
				            \begin{mathpar}
					            \inferrule*[Right=$\latinmath{cut}$]{\inferrule*[Right=$R\Box$]{wRu, \Gamma \To \Delta, u:B}{\Gamma \To \Delta, w: \Box B} \\ \inferrule*[Right=$L\Box$]{v:B, wRv, w: \Box B, \Pi' \To \Sigma}{wRv, w: \Box B, \Pi' \To \Sigma}}{wRv, \Gamma, \Pi' \To \Delta, \Sigma}
				            \end{mathpar}
				            con $u \notin \{\Gamma, \Delta, w\}$. \\
				            Non posso trasformare la derivazione direttamente, in quanto:
				            \begin{itemize}
					            \item Ho un'etichetta diversa in ciascuna premessa ($v$ e $u$); applico dunque una \hyperlink{thsoset}{sostituzione} \sfrac{v}{u}.
					            \item La formula $w:\Box B$ è principale in una premessa; per \latinmath{IH2}, la posso eliminare per non farla occorrere nella conclusione.
				            \end{itemize}
				            Poi, per \latinmath{IH1}:
				            \begin{mathpar}
					            \inferrule*[Right=$C$]{\inferrule*[Right=$\latinmath{IH1}$]{\inferrule*[Right={$\left[\sfrac{v}{u}\right]$}]{wRu, \Gamma' \To \Delta, u:B}{wRv, \Gamma' \To \Delta, v:B} \\ \inferrule*[Right=$\latinmath{IH2}$]{\Gamma \To \Delta, w:\Box B \\ v:B, wRv, w:\Box B, \Pi' \To \Sigma}{v:B, wRv, \Gamma, \Pi' \To \Delta, \Sigma}}{wRv, wRv, \Gamma, \Gamma', \Pi' \To \Delta, \Delta, \Sigma}}{wRv, \Gamma', \Pi' \To \Delta, \Sigma}
				            \end{mathpar}
				      \item $A \equiv \Dmd B$:
				            \begin{mathpar}
					            \inferrule*[Right=$\latinmath{cut}$]{\inferrule*[Right=$R\Dmd$]{wRv, \Gamma' \To \Delta, w: \Dmd B, v:B}{wRv, \Gamma' \To \Delta, w: \Dmd B} \\ \inferrule*[Right=$L\Dmd$]{wRu, u:B, \Pi \To \Sigma}{w: \Dmd B, \Pi \To \Sigma}}{wRv, \Gamma', \Pi \To \Delta, \Sigma}
				            \end{mathpar}
				            con $u \notin \{\Pi, \Sigma, w\}$. \\
				            Non posso trasformare la derivazione direttamente, in quanto:
				            \begin{itemize}
					            \item Ho un'etichetta diversa in ciascuna premessa ($v$ e $u$); applico dunque una \hyperlink{thsoset}{sostituzione} \sfrac{v}{u}.
					            \item La formula $w:\Dmd B$ è principale in una premessa; per \latinmath{IH2}, la posso eliminare per non farla occorrere nella conclusione.
				            \end{itemize}
				            Poi, per \latinmath{IH1}:
				            \begin{mathpar}
					            \inferrule*[Right=$C$]{\inferrule*[Right=$\latinmath{IH1}$]{\inferrule*[Right=$\latinmath{IH2}$]{wRv, \Gamma' \To \Delta, w:\Dmd B,v:B \\ w:\Dmd B, \Pi \To \Sigma}{wRv, \Gamma', \Pi \To \Delta, \Sigma, v:B} \\ \inferrule*[Right={$\left[\sfrac{v}{u}\right]$}]{u:B, wRu, \Pi \To \Sigma}{v:B, wRv, \Pi \To \Sigma}}{wRv, wRv, \Gamma', \Pi, \Pi \To \Delta, \Sigma, \Sigma}}{wRv, \Gamma', \Pi \To \Delta, \Sigma}
				            \end{mathpar}
			      \end{itemize}
		\end{enumerate}
	\end{dimo}
	\noindent Possiamo ora proseguire con la dimostrazione per induzione sul numero di \latinmath{cut}:
	\begin{description}
		\item[Base:] 1 \latinmath{cut}: è l'ultimo, dunque segue per il \hyperlink{lemcut}{\textbf{Lemma} precedente}.
		\item[Passo:] $n+1$ \latinmath{cut}: \\
		      \latinmath{IH}: un albero con $n$ \latinmath{cut} è derivabile. \\
		      Posso eliminare il \latinmath{cut} uppermost per il \hyperlink{lemcut}{\textbf{Lemma} precedente}; quindi, mi rimangono $n$ \latinmath{cut}, che sono tutti eliminabili per \latinmath{IH}.
	\end{description}
\end{dimo}

\subsubsubs{Corollario: ammissibilità del \emph{modus ponens}}
\hypertarget{mpamg3l}{}
\noindent \textbf{Corollario.} \emph{La seguente regola $M \! P$ (\emph{modus ponens}) è ammissibile in \latinmath{G3L}:}
\begin{mathpar}
	\inferrule*[Right=$M \! P$]{\To w:A \\ \To w:A \to B}{\To w:B}
\end{mathpar}

\begin{dimo}
	\begin{mathpar}
		\inferrule*[Right=$\latinmath{cut}$]{\To w:A \\ \inferrule*[Right=$\text{Invertibilità}$]{\To w:A \to B}{w:A \To w:B}}{\To w:B}
	\end{mathpar}
\end{dimo}

\vspace{12pt}
\subsubsubs{Corollario: completezza di \latinmath{G3L}}
\noindent \textbf{Corollario.} \emph{\latinmath{G3L} è completo:} se $A$ è valida nella classe delle strutture per \latinmath{L} ($\vdash_{\latinmath{L}} A$), allora $\latinmath{G3L} \vdash \To w:A$.

\begin{dimo}
	[per induzione sulla profondità di $\vdash_{\latinmath{L}} A$] \phantom{ciao}
	\begin{description}
		\item[Base:] $p(D)=0$ \\
		      $A$ è un assioma, cioè un \hyperlink{seqetin}{sequente iniziale} o un'istanza di \hyperlink{Lbotet}{$L\bot$}.
		\item[Passo:] $p(D)=n+1$ \\
		      $A$ è stata derivata tramite necessitazione $N$ o \emph{modus ponens} $M \! P$:
		      \begin{itemize}
			      \item $N$ è \hyperlink{namg3l}{ammissibile in \latinmath{G3L}};
			      \item $M \! P$ è \hyperlink{mpamg3l}{ammissibile in \latinmath{G3L}}.
		      \end{itemize}

	\end{description}
	Per ogni estensione di \latinmath{G3L}, avrò un assioma in più nel caso base.
\end{dimo}

\newpage
\subsection{Validità e completezza in \latinmath{G3L}}
\noindent Le nozioni di validità e completezza possono essere estese ai calcoli \latinmath{G3L} per i sequenti etichettati.

\vspace{4pt}
\subsubsection{Validità}
\noindent \emph{\textbf{Questa sezione è da rivedere.}} \\
Prima di fornire il teorema di validità per i calcoli \latinmath{G3L}, bisogna dare alcune definizioni. \\

\subsubsubs{Definizione di $\mathcal{M}$-realizzazione}
\begin{defin}
	[\emph{$\mathcal{M}$-realizzazione}] Dato un modello $\mathcal{M} = \langle \mathcal{W}, R, I \rangle$, una \emph{$\mathcal{M}$-realizzazione} è una funzione $\sigma$ (o $\tau$) che mappa l'insieme di etichette $\varepsilon$ sull'insieme $\mathcal{W}$ dei mondi:
	$$\sigma : \varepsilon \to \mathcal{W}$$
	D'ora in poi useremo $x, y, z$ come metavariabili per le etichette e $w, v, u$ come metavariabili per i mondi. Ad esempio, avremo che $\sigma(x) = w \in \mathcal{W}$. \\
\end{defin}

\subsubsubs{Definizione di verità di una formula etichettata}
\begin{defin}
	[\emph{Verità di una formula etichettata}] La verità di una \hyperlink{fmet}{formula etichettata} $E$ in un modello $\mathcal{M}$, rispetto a una $\mathcal{M}$-realizzazione $\sigma$ (scriviamo \enquote{$\sigma \vDash E$}, dove \enquote{$\vDash$} significa \enquote{realizza la formula}) è così definita:
	\begin{itemize}
		\item $\sigma \vDash xRy \quad \text{sse} \quad \! \sigma(x)R\sigma(y)$
		\item $\sigma \vDash w:A \;\; \text{sse} \;\; \vDash_{\sigma(w)^{\mathcal{M}} A}$ \\
	\end{itemize}
\end{defin}

\subsubsubs{Definizione di verificazione di un sequente etichettato}
\begin{defin}
	[\emph{Verificazione di un sequente etichettato}] Un \hyperlink{seqet}{sequente etichettato} $\Gamma \To \Delta$ è \emph{verificato} \footnote{Oppure, la $\mathcal{M}$-realizzazione $\sigma$ \emph{realizza} il sequente etichettato $\Gamma \To \Delta$.} da una $\mathcal{M}$-realizzazione $\sigma$ (scriviamo \enquote{$\sigma \vDash \Gamma \To \Delta$}) se e solo se, se $\sigma \vDash A$ per ogni $A \in \Gamma$, allora $\sigma \vDash B$ per qualche $B \in \Delta$. \\
\end{defin}

\subsubsubs{Definizione di validità di un sequente etichettato}
\hypertarget{validseqet}{}
\begin{defin}
	[\emph{Validità di un sequente etichettato}] Un \hyperlink{seqet}{sequente etichettato} $\Gamma \To \Delta$ è \latinmath{L}-valido (cioè valido rispetto alle strutture per \latinmath{L}; scriviamo infatti \enquote{$\mathcal{C}^{\latinmath{L}} \vDash \Gamma \To \Delta$}) se e solo se, per ogni modello $\mathcal{M}$ basato su una struttura per \latinmath{L} (\enquote{$\mathcal{F} \in \mathcal{C}^{\latinmath{L}}$}) e per ogni $\mathcal{M}$-realizzazione $\sigma$, vale che:
	$$\sigma \vDash \Gamma \To \Delta$$
\end{defin}

\vspace{10pt}
\subsubsubs{Teorema: validità in \latinmath{G3L}}
\begin{theo}
	[Validità in \latinmath{G3L}] Se un sequente è derivabile in \latinmath{G3L}, allora è \latinmath{L}-valido:
	$$\text{Se} \; \latinmath{G3L} \vdash \Gamma \To \Delta \text{, allora} \; \mathcal{C}^{\latinmath{L}} \vDash \Gamma \To \Delta$$
\end{theo}
\noindent Va specificato che le formule devono essere appropriate per la classe di riferimento: se ho una struttura non riflessiva, la classe non renderà valida la riflessività. \\
\begin{dimo}
	[per induzione sulla profondità della derivazione $\Gamma \To \Delta$] \phantom{ciao} \\
	Assumendo $\latinmath{G3L} \vdash \Gamma \To \Delta$, va dimostrato $\mathcal{C}^{\latinmath{L}} \vDash \Gamma \To \Delta$; il che, per la definizione di validità di un sequente etichettato, equivale a dimostrare che $\sigma \vDash \Gamma \To \Delta$, dove $\sigma$ è una generica $\mathcal{M}$-realizzazione appropriata per \latinmath{L}. \\
	\begin{description}
		\item[Base:] $p(D) = 0$
		      \begin{itemize}
			      \item $\sigma \vDash x:p, \Gamma' \To \Delta', x:p$ \\
			            \`{E} realizzata, in quanto almeno una formula ($x:p$) è realizzata a destra.
			      \item $\sigma \vDash x:\bot, \Gamma' \To \Delta$ \\
			            \`{E} realizzata, in quanto, essendo un'implicazione, non può esistere $\sigma$ che realizza il falso.
		      \end{itemize}
		\item[Passo:] $p(D) = n = k+1$. Ho un caso per ogni regola (solo alcuni casi).
		      \begin{itemize}
			      \item $L\land$ (va dimostrato $\sigma \vDash x:A \land B, \Gamma' \To \Delta$ \footnote{Dove $\vDash \: \equiv \: \vdash^{k+1}$.}):
			            \begin{mathpar}
				            \inferrule*[Right=$L\land$]{x:A, x:B, \Gamma' \To \Delta}{x:A \land B, \Gamma' \To \Delta}
			            \end{mathpar}
			            Assumo $\sigma \vDash^{\forall} x:A \land B, \Gamma'$ (antecedente dell'implicazione), in particolare, $\sigma \vDash x:A \land B$. Ho quindi che:
			            $$\begin{aligned}
					            \vDash_{\sigma(x)}^{\mathcal{M}} A \land B                                               & \quad \; \text{sse} \quad \; \vDash_{\sigma(x)}^{\mathcal{M}} A \;\; \text{e} \;\; \vDash_{\sigma(x)}^{\mathcal{M}} B \\
					            \vDash_{\sigma(x)}^{\mathcal{M}} A \;\; \text{e} \;\; \vDash_{\sigma(x)}^{\mathcal{M}} B & \quad \; \text{sse} \;\; \quad \sigma \vDash x:A \;\; \text{e} \;\; \sigma \vDash x:B
				            \end{aligned}$$
			            Uso \latinmath{IH}:
			            \begin{mathpar}
				            \inferrule*[Right=$\latinmath{IH}$]{\sigma \vDash \; x:A, x:B, \Gamma' \To \Delta}{\sigma \vDash \; x:A \land B, \Gamma' \To \Delta}
			            \end{mathpar}
			      \item $R\lor$ (va dimostrato $\sigma \vDash \Gamma \To \Delta', x:A \lor B$):
			            \begin{mathpar}
				            \inferrule*[Right=$R\lor$]{\Gamma \To \Delta', x:A \\ \Gamma \To \Delta', x:B}{\Gamma \To \Delta', x:A \lor B}
			            \end{mathpar}
			            Sia $\sigma$ tale che $\sigma \vDash^{\forall} \tau$ (??); per \latinmath{IH}, ho che: $$\sigma \vDash^{\exists} \Delta', x:A$$
			            Assumo $\sigma \vDash^{\exists} \Delta'$ e $\sigma \vDash^{\exists} \Delta', x:B$ (oppure $\sigma \vDash x:A$ e $\sigma \vDash x:A \lor B$); posso concludere:
			            $$\sigma \vDash x:B$$
			      \item $R\to$ (va dimostrato $\sigma \vDash^{\exists} \Gamma \To \Delta, x:A \to B$):
			            \begin{mathpar}
				            \inferrule*[Right=$R\to$]{x:A, \Gamma \To \Delta', x:B}{\Gamma \To \Delta, x:A \to B}
			            \end{mathpar}
			            Assumo $\sigma \vDash^{\forall} \Gamma$. Ho due casi:
			            \begin{enumerate}
				            \item $\sigma \vDash x:A$ \\
				                  Per \latinmath{IH}:
				                  $$\sigma \vDash^{\exists} \Delta', x:B$$
				                  $B$ è realizzata nella conclusione (un'implicazione è sempre vera se è vero il conseguente).
				            \item $\sigma \nvDash x:A$ \\
				                  Ho che: $$\sigma \vDash A$$
				                  Posso quindi concludere:
				                  $$\sigma \vDash A \to B$$
			            \end{enumerate}
			      \item $L\to$ (va dimostrato $\sigma \vDash^{??} x:A \to B, \Gamma' \To \Delta$):
			            \begin{mathpar}
				            \inferrule*[Right=$L\lor$]{\Gamma' \To \Delta, x:A \\ x:B, \Gamma' \To \Delta}{x:A \lor B, \Gamma' \To \Delta}
			            \end{mathpar}
			            Assumo $\sigma \vDash^{\forall} x:A \lor B, \Gamma'$. \\
			            Per \latinmath{IH}, ho che $\sigma \vDash^{\exists} \Delta, x:A$. Quindi, per $M \! P$, ho che:
			            $$\sigma \vDash B$$
			            Per assunzione so che $\sigma \vDash \Gamma'$ e $\sigma \vDash x:B, \Gamma' \To \Delta$; quindi, per $M \! P$, posso concludere:
			            $$\sigma \vDash \Delta$$
			      \item $L\Box$ (va dimostrato $\sigma \vDash^{??} xRy, x: \Box A, \Gamma' \To \Delta$):
			            \begin{mathpar}
				            \inferrule*[Right=$L\Box$]{y:A, xRy, x: \Box A, \Gamma' \To \Delta}{xRy, x: \Box A, \Gamma' \To \Delta}
			            \end{mathpar}
			            Assumo $\sigma \vDash^{\forall} y:A, xRy, x: \Box A, \Gamma'$. Per \latinmath{IH}:
			            $$\sigma \vDash^{\forall} y:A, xRy, x: \Box A, \Gamma' \To \Delta$$
			            Quindi, ho che $\sigma \vDash y:A$, da cui $\sigma \vDash y:A, xRy, x: \Box A$; per $M \! P$, posso quindi concludere:
			            $$\sigma \vDash^{\exists} \Delta$$
			      \item $R\Box$ (va dimostrato $\sigma \vDash \Gamma \To \Delta', x: \Box A$):
			            \begin{mathpar}
				            \inferrule*[Right=$R\Box$]{xRy, \Gamma \To \Delta', y:A}{\Gamma \To \Delta', x: \Box A}
			            \end{mathpar}
			            con $y \notin \{\Gamma, \Delta', x\}$. \\
			            Assumo $\sigma \vDash^{\forall} \Gamma$. \\
			            Ho quindi $\sigma \vDash^{\exists} \Delta'$ (banale) (??), di conseguenza $\sigma \vDash x: \Box A$. Ho due casi:
			            \begin{enumerate}
				            \item $\nexists v (\sigma(x)Rv)$: non c'è nessuno punto accessibile, quindi $\sigma \vDash x: \Box A$.
				            \item $\exists v (\sigma(x)Rv)$: sia $u \in \mathcal{W}$ tale che $\sigma(x)Ru$ ($u$ è un mondo generico, dunque un oggetto semantico, al quale non si applica \latinmath{IH}). Considero $\tau$:
				                  $$\tau = \left\{\begin{aligned}
						                   & \tau (y) = u                                    \\
						                   & \tau (z) = \sigma (z) \; \text{per} \, z \neq y
					                  \end{aligned}
					                  \right.$$
				                  Ho quindi che $\tau \vDash^{\forall} xRy, \Gamma$. Per \latinmath{IH}:
				                  $$\tau \vDash^{\exists} \Delta', y:A$$
				                  So che $\tau \nvDash^{\exists} \Delta'$ (perchè si comporta come $\sigma$), quindi ho che $\tau \vDash^{\exists} y:A$. Posso quindi concludere:
				                  $$\tau \vDash x: \Box A \quad \to \quad \sigma \vDash x: \Box A$$
				                  quando non occorre $u$ (?) (infatti, nella conclusione non occorre).
			            \end{enumerate}
			      \item $\latinmath{Rif}$ (va dimostrato che se $\sigma \vDash \Gamma$ è riflessiva, allora $\sigma \vDash^{\exists} \Delta$):
			            \begin{mathpar}
				            \inferrule*[Right=$\latinmath{Rif}$]{wRw, \Gamma \To \Delta}{\Gamma \To \Delta}
			            \end{mathpar}
			            Se $\mathcal{M}$ è riflessivo allora:
			            $$\sigma (x) R \sigma(x) \quad \to \quad \sigma \vDash xRx$$
			            Per \latinmath{IH}, soddisfa almeno una formula nel conseguente, perché tutti gli antecedenti sono soddisfatti.
			      \item $\latinmath{ConvDeb}$:
			            \begin{mathpar}
				            \inferrule*[Right=$\latinmath{ConvDeb}$]{yRt, zRt, xRy, xRz, \Gamma' \To \Delta}{xRy, xRz, \Gamma' \To \Delta}
			            \end{mathpar}
			            con $t \notin \{\Gamma', \Delta, x, y, z\}$. \\
			            Assumo $\sigma \vDash^{\forall} xRy, xRz, \Gamma'$. Considero $\tau$:
			            $$\tau = \left\{\begin{aligned}
					             & \tau (t) = v          \\
					             & \tau (z) = \sigma (z)
				            \end{aligned}
				            \right.$$
			            Quindi, $\tau$ soddisfa tutte le formule della premessa. \\
			            Per \latinmath{IH} e $M \! P$, ottengo $\tau \vDash^{\exists} \Delta$, quindi posso concludere: $$\sigma \vDash^{\exists} \Delta$$
		      \end{itemize}
	\end{description}
\end{dimo}

\newpage
\subsubsection{Completezza}

\noindent \emph{\textbf{Questa sezione è da rivedere.}} \\
Prima di fornire il teorema di completezza per i calcoli \latinmath{G3L} bisogna dare alcune definizioni.

Innanzitutto, occorre definire una procedura equa per ottenere una derivazione a partire da un arbitrario sequente $\Gamma \To \Delta$, ovvero nella quale tutte le regole che sono applicabili vengono applicate dopo al più un numero finito di passi (in altre parole, non ci devono essere possibili regole la cui applicazione viene posticipata all'infinito). Questo è lo standard per le dimostrazioni con alberi di sequenti. La definizione viene data per induzione:

\subsubsubs{Definizione di procedura per la costruzione di un \latinmath{L}-albero}
\hypertarget{ltree}{}
\begin{defin}
	[\emph{Costruzione di un \latinmath{L}-albero}] \phantom{ciao}
	\begin{description}
		\item[Base:] \phantom{ciao}
		      \begin{itemize}
			      \item $\Gamma \To \Delta$ \\
			            Un sequente è sempre un albero; questo ha solo la radice.
		      \end{itemize}
		\item[Passo:] si hanno 3 casi;
		      \begin{enumerate}
			      \item \emph{Tutte le foglie dell'albero sono assiomi:} la procedura termina e il sequente è derivabile.
			      \item \emph{Non tutte le foglie dell'albero sono assiomi e nessuna regola di \latinmath{G3L} è applicabile ad esse:} si copia ciascuna foglia (a parte quelle che contengono già assiomi) sopra se stessa. In questo caso, si ottengono rami che proseguono all'infinito.
			      \item \emph{Non tutte le foglie dell'albero sono assiomi e alcune regole di \latinmath{G3L} sono applicabiili a certi rami:} si applicano tutte le istanze di regole applicabili; prima si definisce l'ordine delle regole, poi si applicano in parallelo tutte le istanze di una di esse dopo l'altra.

			            Le possibili regole in questo caso sono $10+k$, dove $k$ è il numero di regole non logiche del calcolo \latinmath{G3L} che si sta considerando; vanno analizzate per definire tutti i possibili risultati:
			            \begin{itemize}
				            \item $L\land$: $$w_1 : A_1 \land B_1, \ldots, w_n : A_n \land B_n, \Pi \To \Sigma$$
				                  dove $w_i : A_i \land B_i \notin \{\Pi\}$. \\
				                  Applico $L\land$ un numero $n$ di volte e ottengo:
				                  $$w_1 : A_1, w_1 : B_1, \ldots, w_n : A_n, w_n : B_n, \Pi \To \Sigma$$
				            \item $R\land$: $$\Sigma \To \Pi, w_1 : A_1 \land B_1, \ldots, w_n : A_n \land B_n$$
				                  dove $w_i : A_i \land B_i \notin \{\Pi\}$. \\
				                  Scrivo sopra $2^n$ nuove foglie fatte in questo modo:
				                  $$\Sigma \To \Pi, w_1 : C_1, \ldots, w_n : C_n$$
				                  dove $C_i \notin \{A_i, B_i\}$. \\
				                  Devo inoltre aggiungere la condizione: \emph{ho inserito tutte le combinazioni}. (??)
				            \item $L\lor$ (??): $$w_1 : A_1 \lor B_1, \ldots, w_n : A_n \lor B_n, \Pi \To \Sigma$$
				                  dove $w_i : A_i \lor B_i \notin \{\Pi\}$. \\
				                  Scrivo sopra $2^n$ nuove foglie fatte in questo modo:
				                  $$w_1 : C_1, \ldots, w_n : C_n, \Pi \To \Sigma$$
				                  dove $C_i \notin \{A_i, B_i\}$. \\
				                  Devo inoltre aggiungere la condizione: \emph{ho inserito tutte le combinazioni}. (??)
				            \item $R\lor$ (??): $$\Sigma \To \Pi, w_1 : A_1 \lor B_1, \ldots, w_n : A_n \lor B_n$$
				                  dove $w_i : A_i \lor B_i \notin \{\Pi\}$. \\
				                  Applico $R\lor$ un numero $n$ di volte e ottengo:
				                  $$\Sigma \To \Pi, w_1 : A_1, w_1 : B_1, \ldots, w_n : A_n, w_n : B_n$$
				            \item $L\to$: $$\Sigma \To \Pi, w_1 : A_1 \to B_1, \ldots, w_n : A_n \to B_n$$
				                  dove $w_i : A_i \to B_i \notin \{\Pi\}$. \\
				                  Applico $L\to$ un numero $n$ di volte e ottengo:
				                  $$w_1 : A_1, \ldots, w_n : A_n, \Sigma \To \Pi, w_1 : B_1, \ldots, w_n : B_n,$$
				            \item $R\to$: $$w_1 : A_1 \to B_1, \ldots, w_n : A_n \to B_n, \Pi \To \Sigma$$
				                  dove $w_i : A_i \to B_i \notin \{\Pi\}$. \\
				                  Scrivo sopra $2^n$ nuove foglie (distinte) fatte in questo modo:
				                  $$w_{i_{1}} : B_{i_{1}}, \ldots, w_{i_{k}} : B_{i_{k}}, \Pi \To \Sigma, w_{i_{k+1}} : A_{i_{k+1}}, \ldots, w_{i_{n}} : A_{i_{n}}$$
				                  dove $\{i_1, \ldots, i_k\} \subseteq \{i_1, \l, i_n\}$ e $\{i_{k+1}, \ldots, i_n\} = \{i_1, \l, i_n\} -  \{i_1, \ldots, i_k\}$.
				            \item $L\Box$: sia $\Pi \To \Sigma$ la foglia che stiamo considerando; per ogni coppia $wRv, w:\Box A$ che occorre in $\Pi$, aggiungiamo $v:A$ nell'antecedente della nuova foglia.
				            \item $R\Box$: $$\Sigma \To \Pi, w_1 : \Box A_1, \ldots, w_n : \Box A_n$$
				                  dove $w_i : \Box A_i \notin \{\Pi\}$. \\
				                  Siano $\{v_1, \ldots, v_n\}$ nuove etichette che non occorrono in nessun ramo. Posso quindi applicare $R\Box$ un numero $n$ di volte e ottengo:
				                  $$w_1Rv_1, \ldots, w_nRv_n, \Sigma \To \Pi, v_1 : A_1, \ldots, v_n : A_n$$
				            \item $L\Dmd$: analoga a $R\Box$ (???).
				            \item $R\Dmd$: sia $\Pi \To \Sigma$ la foglia che stiamo considerando; per ogni coppia $wRv \in \Pi$ e $w:\Dmd A \in \Delta$, aggiungiamo $v:A$ nel conseguente della nuova foglia.
				            \item \latinmath{Rif}:
				                  \begin{mathpar}
					                  \inferrule*[Right=$\latinmath{Rif}$]{wRw, \Pi \To \Sigma}{\Pi \To \Sigma}
				                  \end{mathpar}
				                  A causa della mancanza di formule nella conclusione, devo definire la condizione aggiuntiva \enquote{$\forall w \in \{\Pi, \Sigma\}$}. Ottengo quindi:
				                  \begin{mathpar}
					                  \inferrule*[Right=$\latinmath{Rif}$]{w_1Rw_1, \ldots, w_nRw_n, \Pi \To \Sigma}{\Pi \To \Sigma}
				                  \end{mathpar}
				                  dove $\{w_1, \ldots, w_n\} = \{w_1, \ldots, w_n \in \Pi, \Sigma\}$.
				            \item \latinmath{Ser}: procedimento analogo a \latinmath{Rif}, ma con una nuova etichetta $u$.
			            \end{itemize}
			            Le altre regole logiche si comportano come $L\Box$ e $R\Dmd$.
		      \end{enumerate}
	\end{description}
	Questa procedura ha due esiti possibili:
	\begin{enumerate}
		\item genera un albero finito e il sequente è derivabile;
		\item genera un albero infinito e il sequente non è derivabile. \\
	\end{enumerate}
\end{defin}

La procedura appena definita permette di ottenere, in una derivazione per un generico sequente, un \emph{ramo saturo} rispetto alle regole di \latinmath{G3L}, ovvero un ramo la cui foglia non è un sequente iniziale e in cui ogni istanza applicabile di una regola di \latinmath{G3L} è stata applicata. Un ramo \latinmath{L}-saturo, quindi, è così definito: \\

\subsubsubs{Definizione di ramo \latinmath{L}-saturo}
\hypertarget{lsat}{}
\begin{defin}
	[\emph{Ramo \latinmath{L}-saturo}] Siano $\overline{\Gamma}$ e $\overline{\Delta}$ rispettivamente l'unione di tutte le formule occorrenti negli antecedenti e di tutte quelle occorrenti nei conseguenti di un ramo di un albero di sequenti; quel ramo è \latinmath{L}-saturo se è tale che:
	\begin{enumerate}
		\item $w:p \notin \overline{\Gamma} \cap \overline{\Delta}$ (nessuna formula atomica occorre sia in un antecedente che in conseguente:);
		\item $w:\bot \notin \overline{\Gamma}$;
		\item ($L\land$) Se $w:A \land B \in \overline{\Gamma}$, allora $w:A, w:B \in \overline{\Gamma}$;
		\item ($R\land$) Se $w:A \land B \in \overline{\Delta}$, allora $w:A \in \overline{\Delta} \lor w:B \in \overline{\Delta}$;
		\item ($L\lor$) Se $w:A \lor B \in \overline{\Gamma}$, allora $w:A \in \overline{\Gamma} \lor w:B \in \overline{\Gamma}$;
		\item ($R\lor$) Se $w:A \lor B \in \overline{\Delta}$, allora $w:A, w:B \in \overline{\Delta}$;
		\item ($L\to$) Se $w:A \to B \in \overline{\Gamma}$, allora $w:A \in \overline{\Delta} \lor w:B \in \overline{\Gamma}$;
		\item ($R\to$) Se $w:A \to B \in \overline{\Delta}$, allora $w:A \in \overline{\Gamma} \land w:B \in \overline{\Delta}$;
		\item ($L\Box$) Se $w: \Box A, wRv \in \overline{\Gamma}$, allora $v:A \in \overline{\Gamma}$;
		\item ($R\Box$) Se $w: \Box A \in \overline{\Delta}$, allora, per qualche $u$, $wru \in \overline{\Gamma} \land u:A \in \overline{\Delta}$;
		\item ($L\Dmd$) Se $w: \Dmd A \in \overline{\Gamma}$, allora, per qualche $v$, $wrv, v:A \in \overline{\Gamma}$;
		\item ($R\Dmd$) Se $wRv \in \overline{\Gamma} \land w: \Dmd A \in \overline{\Delta}$, allora $v:A \in \overline{\Delta}$;
		\item (\latinmath{Rif}) Se $\latinmath{Rif} \in \latinmath{G3L}$, allora $\forall w \in \{\overline{\Gamma}, \overline{\Delta}\}$, anche $wRw \in \overline{\Gamma}$;
		\item (\latinmath{Ser}) Se $\latinmath{Ser} \in \latinmath{G3L}$, allora $\forall w \in \{\overline{\Gamma}, \overline{\Delta}\}$, per qualche $u$, anche $wRu \in \overline{\Gamma}$;
		\item Se $R$ è una regola non logica e $R \in \latinmath{G3L}$, se le sue formule principali occorrono in $\overline{\Gamma}$, allora anche le sue formule attive occorrono in $\overline{\Gamma}$. \\
	\end{enumerate}
\end{defin}

\subsubsubs{Definizione di modello da un ramo \latinmath{L}-saturo}
\begin{defin}
	[Modello da un ramo \latinmath{L}-saturo] Dato un ramo \latinmath{L}-saturo $B$, il suo modello $\mathcal{M}^{B}$ è così definito:
	$$\mathcal{M}^{B} = \langle \mathcal{W}^{B}, R^B, I^B \rangle$$
	Dove:
	\begin{itemize}
		\item $\mathcal{W}^{B}$ è l'insieme di tutte le etichette contenute nel ramo: $$\mathcal{W}^{B} = \{x:x \in B\}$$
		\item $R^B$ è la relazione di accessibilità propria del ramo: $$xR^By \quad \longleftrightarrow \quad xRy \in B$$
		      che equivale a dire che $xRy \in \overline{\Gamma}$, dal momento che un atomo relazionale non può mai occorrere in $\Delta$.
		\item $I^B$ è la funzione di interpretazione: $$I^B(p) = \{x:x \!:\! p \in \overline{\Gamma}\}$$
		      che afferma la verità delle formule che occorrono in $\overline{\Gamma}$. \\
	\end{itemize}
\end{defin}

\subsubsubs{\emph{Truth Lemma}}
\begin{lem}
	[\emph{Truth Lemma}] Data una $\mathcal{M}$-realizzazione \latinmath{id}, tale che $\latinmath{id}(x)=x$ \footnote{Ovvero, che mappa ogni etichetta su se stessa.}, sia $x:A$ una formula del ramo \latinmath{L}-saturo $B$; è vero che:
	\begin{enumerate}
		\item Se $x:A \in \overline{\Gamma}$, allora $\vDash_{\latinmath{id}(x)}^{\mathcal{M}^{B}} A$.
		\item Se $x:A \in \overline{\Delta}$, allora $\nvDash_{\latinmath{id}(x)}^{\mathcal{M}^{B}} A$.
	\end{enumerate}
\end{lem}
\begin{dimo}
	[per induzione simultanea sulla struttura di $A$] \phantom{ciao} \\
	\latinmath{IH1}: ... \\
	\latinmath{IH2}: ...
	\begin{description}
		\item[Base:] \phantom{ciao}
		      \begin{itemize}
			      \item $A \equiv p$:
			            \begin{enumerate}
				            \item Banale, per costruzione;
				            \item Banale, per \hyperlink{lsat}{definizione di ramo \latinmath{L}-saturo}.
			            \end{enumerate}
			      \item $A \equiv \bot$:
			            \begin{enumerate}
				            \item $\text{Se} \; x:\bot \in \overline{\Gamma} \text{, allora} \; \vDash_{\latinmath{id}(x)}^{\mathcal{M}^{B}} \bot$ \\
				                  Per definizione di ramo \latinmath{L}-saturo, l'antecedente è falso; dunque l'implicazione è vera.
				            \item $\text{Se} \; x:\bot \in \overline{\Delta} \text{, allora} \; \nvDash_{\latinmath{id}(x)}^{\mathcal{M}^{B}} \bot$ \\
				                  Il conseguente è sempre vero; dunque l'implicazione è vera.
			            \end{enumerate}
		      \end{itemize}
		\item[Passo:] ho un caso per ogni connettivo.
		      \begin{itemize}
			      \item $A \equiv B \land C$:
			            \begin{enumerate}
				            \item $x:B \land C \in \overline{\Gamma}$ \\
				                  Per \hyperlink{lsat}{definizione di ramo \latinmath{L}-saturo}, ho che $x:B, x:C \in \overline{\Gamma}$. Per \latinmath{IH}:
				                  $$\vDash_{\latinmath{id}(x)}^{\mathcal{M}^{B}} B \; \text{e} \; \vDash_{\latinmath{id}(x)}^{\mathcal{M}^{B}} C \quad \; \text{sse} \; \quad \vDash_{\latinmath{id}(x)}^{\mathcal{M}^{B}} B \land C$$
				            \item $x:B \land C \in \overline{\Delta}$ \\
				                  Per \hyperlink{lsat}{definizione di ramo \latinmath{L}-saturo}, ho che $x:B \in \overline{\Delta} \lor x:C \in \overline{\Delta}$. Per \latinmath{IH}:
				                  $$\nvDash_{\latinmath{id}(x)}^{\mathcal{M}^{B}} B \; \text{o} \; \nvDash_{\latinmath{id}(x)}^{\mathcal{M}^{B}} C \quad \; \text{sse} \; \quad \nvDash_{\latinmath{id}(x)}^{\mathcal{M}^{B}} B \land C$$
			            \end{enumerate}
			      \item $A \equiv B \lor C$:
			            \begin{enumerate}
				            \item $x:B \lor C \in \overline{\Gamma}$ \\
				                  Per \hyperlink{lsat}{definizione di ramo \latinmath{L}-saturo}, ho che $x:B \in \overline{\Gamma} \lor x:C \in \overline{\Gamma}$. Per \latinmath{IH}:
				                  $$\vDash_{\latinmath{id}(x)}^{\mathcal{M}^{B}} B \; \text{o} \; \vDash_{\latinmath{id}(x)}^{\mathcal{M}^{B}} C \quad \; \text{sse} \; \quad \vDash_{\latinmath{id}(x)}^{\mathcal{M}^{B}} B \lor C$$
				            \item $x:B \lor C \in \overline{\Delta}$ \\
				                  Per \hyperlink{lsat}{definizione di ramo \latinmath{L}-saturo}, ho che $x:B, x:C \in \overline{\Delta}$. Per \latinmath{IH}:
				                  $$\nvDash_{\latinmath{id}(x)}^{\mathcal{M}^{B}} B \; \text{e} \; \nvDash_{\latinmath{id}(x)}^{\mathcal{M}^{B}} C \quad \; \text{sse} \; \quad \nvDash_{\latinmath{id}(x)}^{\mathcal{M}^{B}} B \lor C$$
			            \end{enumerate}
			      \item $A \equiv B \to C$:
			            \begin{enumerate}
				            \item $x:B \to C \in \overline{\Gamma}$ \\
				                  Per \hyperlink{lsat}{definizione di ramo \latinmath{L}-saturo}, ho che $x:B \in \overline{\Delta} \lor x:C \in \overline{\Gamma}$. Per \latinmath{IH}:
				                  $$\nvDash_{\latinmath{id}(x)}^{\mathcal{M}^{B}} B \; \text{o} \; \vDash_{\latinmath{id}(x)}^{\mathcal{M}^{B}} C \quad \; \text{sse} \; \quad \vDash_{\latinmath{id}(x)}^{\mathcal{M}^{B}} B \to C$$
				            \item $x:B \to C \in \overline{\Delta}$ \\
				                  Per \hyperlink{lsat}{definizione di ramo \latinmath{L}-saturo}, ho che $x:B \in \overline{\Gamma} \land x:C \in \overline{\Delta}$. Per \latinmath{IH}:
				                  $$\vDash_{\latinmath{id}(x)}^{\mathcal{M}^{B}} B \; \text{e} \; \nvDash_{\latinmath{id}(x)}^{\mathcal{M}^{B}} C \quad \; \text{sse} \; \quad \nvDash_{\latinmath{id}(x)}^{\mathcal{M}^{B}} B \to C$$
			            \end{enumerate}
			      \item $A \equiv \Box B$:
			            \begin{enumerate}
				            \item $x: \Box B \in \overline{\Gamma}$ \\
				                  So che $\vDash_{\latinmath{id}(x)}^{\mathcal{M}^{B}} \Box B$ e quindi, per definizione di $R$, $\forall y \in \mathcal{W} (xRy \in \overline{\Gamma})$; dunque, per \hyperlink{lsat}{definizione di ramo \latinmath{L}-saturo}, ho che:
				                  $$y:B \in \overline{\Gamma}$$
				                  per ogni $y$ tale che $xRy \in \overline{\Gamma}$. \\
				                  Sia $z$ tale che $xR^Bz$. Quindi, per \latinmath{IH}, posso concludere (??):
				                  $$\vDash_z^{\mathcal{M}^{B}} B$$
				            \item $x: \Box B \in \overline{\Delta}$ \\
				                  So che $\vDash_{\latinmath{id}(x)}^{\mathcal{M}^{B}} \Box B$ e quindi, per \hyperlink{lsat}{definizione di ramo \latinmath{L}-saturo}, $\exists y : xRy \in \overline{\Gamma} \land y:B \in \overline{\Delta}$. Quindi, so che:
				                  $$\exists y \in \mathcal{W}^{B} : xR^By$$
				                  e, per \latinmath{IH}:
				                  $$\nvDash_{\latinmath{id}(y)}^{\mathcal{M}^{B}} B $$
			            \end{enumerate}
			      \item $A \equiv \Dmd B$: analogo a $\Box$ (???).
			            \begin{enumerate}
				            \item $x: \Dmd B \in \overline{\Gamma}$
				            \item $x: \Dmd B \in \overline{\Delta}$ \\
			            \end{enumerate}
		      \end{itemize}
	\end{description}
\end{dimo}

\noindent Possiamo ora enunciare il teorema di completezza. \\

\subsubsubs{Teorema: completezza in \latinmath{G3L}}
\hypertarget{g3lcomp}{}
\begin{theo}
	[Completezza in \latinmath{G3L}] Se un sequente è \latinmath{L}-valido, allora è derivabile in \latinmath{G3L}:
	$$\text{Se} \; \mathcal{C}^{\latinmath{L}} \vDash \Gamma \To \Delta \text{, allora} \; \latinmath{G3L} \vdash \Gamma \To \Delta$$
\end{theo}
\noindent La dimostrazione del teorema di completezza è data per contrapposizione, ovvero si dimostra che è possibile definire una procedura tale che se il sequente in questione non è derivabile nel calcolo \latinmath{G3L}, allora permette di costruire un modello basato su una struttura per \latinmath{L} che falsifica tale sequente. Ci interessano i casi in cui i sequenti non sono derivabili, ovvero generano rami infiniti, che per come li abbiamo definiti sono anche \latinmath{L}-saturi. \\

\begin{dimo}
	[per contrapposizione] Va dimostrato: $$\text{Se} \; \latinmath{G3L} \nvdash \Gamma \To \Delta \text{, allora} \; \mathcal{C}^{\latinmath{L}} \nvDash \Gamma \To \Delta$$
	Sia $\Gamma \To \Delta$ un sequente non derivabile in \latinmath{G3L}, ovvero tale che:
	$$\latinmath{G3L} \nvdash \Gamma \To \Delta$$
	Applicando la \hyperlink{ltree}{definizione di procedura di costruzione di un \latinmath{L}-albero}, si ottiene un albero contenente almeno un ramo \latinmath{L}-saturo (in caso contrario, si sarebbe trovata una derivazione in \latinmath{G3L}). Sia $\mathcal{M}^{B}$ il modello costruito a partire da tale ramo; si avrà che:
	$$\text{Se} \; xRy \in \Gamma \text{, allora} \; \latinmath{id}(x) R^B \latinmath{id}(y)$$
	cioè $\latinmath{id} \vDash xRy$. La $\mathcal{M}$-realizzazione \latinmath{id}, dunque, falsifica il seguente $\Gamma \To \Delta$. \\

	\noindent Questo è sufficiente a provare la completezza in \latinmath{G3K}; per le altre logiche, al fine di concludere che il sequente non è \latinmath{L}-valido, bisogna mostrare che $\mathcal{M}^{B}$ è basato su quella classe di strutture.
\end{dimo}

\newpage
\subsection{Decidibilità di \latinmath{G3L}}
\noindent Avere un calcolo \latinmath{G3L} decidibile significa poterlo usare per la decisione riguardo alla derivabilità e alla soddiisfacibilità di un dato sequente nella logica \latinmath{L}.

In generale la costruzione di un \latinmath{L}-albero non è una procedura finitaria, dato che in alcuni casi essa genera un ramo \latinmath{L}-saturo solamente dopo un numero infinito di passi. In particolare, sono due i motivi per cui la procedura può andare avanti all’infinito:
\begin{enumerate}
	\item Se ad un dato passo è possibile applicare la regola $L\Box$ o $R\Dmd$, allora tale regola deve essere riapplicata anche ad ogni passo successivo, dato che le formule principali vengono ripetute nella premessa;
	\item Se un’istanza di una regola non logica è applicabile a una foglia ottenuta ad un dato passo, allora essa rimarrà applicabile su tale ramo ad ogni passo successivo, dato che le formule principali vengono ripetute nella premessa \footnote{Inoltre, le regole \latinmath{Rif} e \latinmath{Ser} non hanno formule principali e, dunque, sono sempre applicabili.}.
\end{enumerate}
Per alcuni casi, però, si può mostrare che la procedura di costruzione di un \latinmath{L}-albero può essere modificata in modo che termini necessariamente in un numero finito di passi. Non verranno considerati i calcoli che includono la regola \latinmath{Trans}, dal momento che essa aggiunge troppa complessità alle procedure \footnote{Esistono dimostrazioni della decidibilità di logiche che includono \latinmath{Trans}, ma sono, appunto, molto complesse.}. Per fare ciò, bisogna fornire alcune definizioni. \\

\subsubsubs{Definizione di derivazione minimale in \latinmath{G3L}}
\begin{defin}
	[\emph{Derivazione minimale in \latinmath{G3L}}] Sia $\Gamma \To \Delta$ un sequente derivabile in \latinmath{G3L}; una derivazione \emph{minimale} in \latinmath{G3L} di $\Gamma \To \Delta$ è la derivazione che ha profondità $p$ minore o uguale a quella di ogni altra sua derivazione. \\
\end{defin}
Si noti che non è importante l'ordine in cui le regole vengono applicate, purché siano applicate tutte e sole quelle che servono alla derivazione.

Le derivazioni minimali godono della seguente proprietà: \\

\subsubsubs{Lemma: proprietà del sottotermine}
\begin{lem}
	[Proprietà del sottotermine] Ogni etichetta che occorre in una derivazione minimale di un sequente $\Gamma \To \Delta$ ricade in uno di questi casi:
	\begin{itemize}
		\item Occorre nella conclusione.
		\item \`{E} introdotta da una regola con restrizioni sulle variabili. \\
	\end{itemize}
\end{lem}

Questo lemma è però insufficiente per mostrare la decidibilità di \latinmath{G3K}; per fare ciò, va mostrato che, se in un ramo di una derivazione la regola $L\Box$ (o $R\Dmd$) è applicata due volte alle stesse formule principali, allora quelle due applicazioni della regola possono esser rese consecutive. In altre parole, va mostrato che una derivazione del tipo:
\begin{mathpar}
	\inferrule*[Right=$L\Box$]{v:A, wRv, w:\Box A, \Gamma' \To \Delta'}{wRv, w:\Box A, \Gamma' \To \Delta'}
\end{mathpar}
\vspace*{-20pt}
$$\phantom{(D^{\ast})} \; \vdots \; (D^{\ast}) $$
\begin{mathpar}
	\inferrule*[Right=$L\Box$]{v:A, wRv, w:\Box A, \Gamma \To \Delta}{wRv, w:\Box A, \Gamma \To \Delta}
\end{mathpar}
non è mai minimale. Dimostriamo il seguente lemma e un suo corollario: \\

\subsubsubs{Lemma: permutazione in \latinmath{G3K}}
\hypertarget{lemperm}{}
\begin{lem}
	[Permutazione in \latinmath{G3K}] \`{E} possibile permutare verso il basso un'istanza delle regole $L\Box$ o $R\Dmd$ avente $wRv$ come formula principale rispetto a ogni istanza di una regola qualsiasi, eccetto per le istanze di $R\Box$ o $L\Dmd$ con $wRv$ attiva.
\end{lem}
\begin{dimo}
	Per ogni regola, si mostra che in una derivazione con quella regola applicata più in basso $L\Box$ (o $R\Dmd$) si può invertire l'ordine di applicazione mantenendo stessa radice e stessa foglia (o stesse foglie, nel caso di regole a due premesse).
	\begin{itemize}
		\item $R\Box$:
		      \begin{mathpar}
			      \inferrule*[Right=$R\Box$]{\inferrule*[Right=$L\Box$]{uRt, v:A, wRv, w:\Box A, \Pi \To \Sigma, t:B}{uRt, wRv, w:\Box A, \Pi \To \Sigma, t:B}}{wRv, w:\Box A, \Pi \To \Sigma, u: \Box B}
		      \end{mathpar}
		      con $t \notin \{\Pi, \Sigma, w, v, u\}$. Inverto l'ordine delle regole:
		      \begin{mathpar}
			      \inferrule*[Right=$L\Box$]{\inferrule*[Right=$R\Box$]{uRt, v:A, wRv, w:\Box A, \Pi \To \Sigma, t:B}{v:A, wRv, w:\Box A, \Pi \To \Sigma, u:\Box B}}{wRv, w:\Box A, \Pi \To \Sigma, u: \Box B}
		      \end{mathpar}
		      con $t \notin \{\Pi, \Sigma, w, v, u\}$.
		\item $L\to$:
		      \begin{mathpar}
			      \inferrule*[Right=$L\to$]{\inferrule*[Right=$L\Box$]{v:A, wRv, w:\Box A, \Pi \To \Sigma, u:B}{wRv, w:\Box A, \Pi \To \Sigma, u:B} \\ u:C, wRv, w:\Box A, \Pi \To \Sigma}{u:B \to C, wRv, w:\Box A, \Pi \To \Sigma}
		      \end{mathpar}
		      Non posso invertire direttamente l'ordine delle regole, quindi uso $W$:
		      \begin{mathpar}
			      \inferrule*[Right=$L\Box$]{\inferrule*[Right=$L\to$]{v:A, wRv, w:\Box A, \Pi \To \Sigma, u:B \\ \inferrule*[Right=$W$]{u:C, wRv, w:\Box A, \Pi \To \Sigma}{u:C, v:A, wRv, w:\Box A, \Pi \To \Sigma}}{u:B \to C, v:A, wRv, w:\Box A, \Pi \To \Sigma}}{u:B \to C, wRv, w:\Box A, \Pi \To \Sigma}
		      \end{mathpar}
	\end{itemize}
	Tutti gli altri casi sono analoghi. \\
\end{dimo}

\subsubsubs{Corollario: formule principali di regole con restrizioni in una derivazione minimale}
\hypertarget{corperm}{}
\noindent \textbf{Corollario.} \emph{Se un sequente $\Gamma \To \Delta$ è derivabile in \latinmath{G3K}, allora in una sua derivazione minimale ciascuna coppia di formule del tipo $\{wRv, w:\Box A\}$ è principale in al più un'istanza di $L\Box$.}

\noindent Il corollario può essere enunciato analogamente per $R\Dmd$.
\begin{dimo}
	Supponiamo di avere due istanze differenti di $L\Box$ in una derivazione minimale di un sequente $\Gamma \To \Delta$, e che $wRv, w:\Box A$ siano le formule principali in entrambe:
	\begin{mathpar}
		\inferrule*[Right=$L\Box$]{v:A, wRv, w:\Box A, \Gamma' \To \Delta'}{wRv, w:\Box A, \Gamma' \To \Delta'}
	\end{mathpar}
	\vspace*{-20pt}
	$$\phantom{(D^{\ast})} \; \vdots \; (D^{\ast}) $$
	\begin{mathpar}
		\inferrule*[Right=$L\Box$]{v:A, wRv, w:\Box A, \Gamma \To \Delta}{wRv, w:\Box A, \Gamma \To \Delta}
	\end{mathpar}
	Per il \hyperlink{lemperm}{\textbf{Lemma} precedente}, posso permutare verso il basso l'istanza superiore di $L\Box$ fino a renderla immediatamente successiva a quella inferiore:
	$$\phantom{(D^{\ast})} \; \vdots \; (D^{\ast}) $$
	\vspace*{-20pt}
	\begin{mathpar}
		\inferrule*[Right=$L\Box$]{\inferrule*[Right=$L\Box$]{v:A, v:A, wRv, w:\Box A, \Gamma \To \Delta}{v:A, wRv, w:\Box A, \Gamma \To \Delta}}{wRv, w:\Box A, \Gamma \To \Delta}
	\end{mathpar}
	A questo punto, posso rimpiazzare l'istanza superiore di una regola $L\Box$ con la regola $LC$ e mantenere la stessa foglia:
	\begin{mathpar}
		\inferrule*[Right=$L\Box$]{\inferrule*[Right=$LC$]{v:A, v:A, wRv, w:\Box A, \Gamma \To \Delta}{v:A, wRv, w:\Box A, \Gamma \To \Delta}}{wRv, w:\Box A, \Gamma \To \Delta}
	\end{mathpar}
	Questo è in contraddizione con l'ipotesi iniziale che la derivazione in questione fosse minimale; si è mostrato che in una derivazione minimale non è mai necessario applicare due istanze di $L\Box$ (o $R\Dmd$) con le stesse formule principali. \\
\end{dimo}

\noindent Possiamo ora fornire la dimostrazione della decidibilità di \latinmath{G3K}. \\

\subsubsubs{Teorema: decidibilità di \latinmath{G3K}}
\begin{theo}
	Il calcolo \latinmath{G3K} è decidibile rispetto alla derivabilità e alla soddisfacibilità di un sequente.
\end{theo}
\begin{dimo}
	Va mostrato che è possibile rendere la \hyperlink{ltree}{procedura di costruzione di un \latinmath{L}-albero} tale da fornire una derivazione minimale di un \latinmath{K}-albero. Questo può essere fatto aggiungendo la seguente condizione:
	\begin{itemize}
		\item Ad un dato passo della procedura di costruzione di un \latinmath{K}-albero, un’istanza della regola $L\Box$ (o $R\Dmd$) con formule principali del tipo $wRv, w:\Box A$ è applicabile solo se in nessun passo sottostante è stata applicata un’istanza di tale regola avente le stesse formule principali.
	\end{itemize}
	So che questa condizione non interferisce con la dimostrazione del \hyperlink{g3lcomp}{teorema di completezza} dal \hyperlink{corperm}{\textbf{Corollario} precedente}.

	Inoltre, essa assicura che nella procedura di costruzione di \latinmath{K}-albero non sia possibile generare rami infiniti. Se in un certo ramo arrivo a non poter scrivere più nulla e non avere un sequente iniziale, allora posso costruire un contromodello (invece di ricopiare le formule sopra come accadeva nella \hyperlink{ltree}{procedura di costruzione di un \latinmath{L}-albero}).

	Possiamo perciò concludere che la procedura di costruzione di un \latinmath{K}-albero in un numero finito di passi genera o una derivazione o un contromodello finito per il sequente alla sua radice. \\
\end{dimo}

\subsubsubs{Corollario: \latinmath{G3K} gode della proprietà del modello finito}
\noindent \textbf{Corollario.} La classe di strutture $\mathcal{C}^{\latinmath{K}}$ (ovvero le strutture per \latinmath{K}) gode della \emph{proprietà del modello finito}:
\begin{itemize}
	\item Se una formula è soddisfacile in \latinmath{G3K}, è soddisfacibile in un modello $\mathcal{M}$ finito, ovvero in cui $\mathcal{W}$ è un insieme finito di mondi. \\
\end{itemize}

\noindent Per estendere il teorema di decidibilità ad altri calcoli \latinmath{G3L}, sarà sufficiente dimostrare il \hyperlink{lemperm}{\textbf{Lemma} di permutazione} e il suo \hyperlink{corperm}{\textbf{Corollario}} per le regole non logiche di quel calcolo.

\end{document}
